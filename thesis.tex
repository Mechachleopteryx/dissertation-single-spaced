%!TEX TS-program = xelatex
% vim:syntax=none

%\documentclass[onecolumn, 11pt, oneside, openright, dvipdfm]{book}
%\documentclass[onecolumn, 11pt, oneside, openright, dvipdfmx]{book}
\documentclass[onecolumn, 11pt, oneside, openright]{book}

%\documentclass[onecolumn, 11pt, twosides, openright, dvipdfm]{book}

%\usepackage[dvipdfm,CJKbookmarks,bookmarks=true,bookmarksopen=true]{hyperref}
%\usepackage[dvipdfmx,CJKbookmarks,bookmarks=true,bookmarksopen=true]{hyperref}

\newcommand{\thesisauthor}[0]{William E. Byrd}
\newcommand{\thesistitle}[0]{Relational Programming in miniKanren: \\ Techniques, Applications, and Implementations}
% [Dan thinks 'Limitations' may be better than obstacles]
\newcommand{\thesiskeywords}[0]{Kwd1, Kwd2, Kwd3}
\newcommand{\thesismonth}[0]{August}
\newcommand{\thesisyear}[0]{2009}
%\newcommand{\thesisdate}[0]{\today}
\newcommand{\thesisdate}[0]{May 7, 2009}


\newcommand{\thesiscommitteemembers}[0]{%
\cmember{Daniel P. Friedman, Ph.D. \\ (Principal Advisor)}
\cmember{Amr Sabry, Ph.D.}
\cmember{Christopher T. Haynes, Ph.D.}
\cmember{Lawrence S. Moss, Ph.D.}
}



\usepackage{microtype}

\usepackage{natbib}
\usepackage{hyphenat}
\usepackage{amsmath}
\usepackage{slatex}
\usepackage{alltt}
\usepackage{amssymb}
\usepackage{amsfonts}

\usepackage{verbatim}
\usepackage{url}
\usepackage{float}
\usepackage{DraTex}
\usepackage[compact]{titlesec}
\usepackage[bbgreekl]{mathbbol}
%%% the cspex option loads the infamous stmaryrd font

\usepackage{comment}
%\usepackage{mathabx}

\def\schemecodehook{\setlength{\baselineskip}{0.5\baselineskip}}

\addtolength{\textwidth}{.5in}
\addtolength{\textheight}{.5in}
\setlength{\topmargin}{.25in}
\usepackage{fontspec}
\usepackage{hanging}
\usepackage{xltxtra}
\setlength{\oddsidemargin}{.75in}
\setlength{\evensidemargin}{.25in}

%\defaultfontfeatures{Scale=MatchLowercase} 
%\setmainfont[Mapping=tex-text,Scale=1.0]{Charis SIL}
%\setsansfont[Mapping=tex-text,Scale=0.9]{Geneva}
%\setmonofont[Scale=0.88]{Monaco}

\usepackage{makeidx}


\usepackage[compact]{fancyvrb1}
\usepackage{fancyvrb}
%\fvset{xleftmargin=\parindent,baseline=t}
\DefineVerbatimEnvironment{code}{Verbatim}{commandchars=\\\{\}}
\DefineVerbatimEnvironment{Code}{Verbatim} 
 {baselinestretch=1.1,
  frame=single,
  framerule=0.5pt,
  commandchars=\\\{\}}
\DefineVerbatimEnvironment{Transcript}{Verbatim}
 {baselinestretch=1.1,
  frame=lines,
  xleftmargin=4em,
  xrightmargin=4em,
  framerule=0.5pt,
  numbers=left,
  numbersep=.5em,
  commandchars=\\\{\}}

\usepackage{rotating}
\usepackage{multicol,ragged2e}


%% FIXME - get hyperlinks working, and uncomment 'phantomsection' below
%% \usepackage[xetex,CJKbookmarks,bookmarks=true,bookmarksopen=true]{hyperref}
%% \hypersetup{
%%     pdftitle={\thesistitle},
%%     pdfauthor={\thesisauthor},
%%     pdfkeywords={\thesiskeywords},
%%     bookmarksnumbered=true,
%% %    pagebackref=true,
%%     breaklinks=true,
%%     urlcolor=blue,
%%     colorlinks=true,
%%     citecolor=blue,          %citeref's color
%%     linkcolor=blue,
%% }


\def\turnstilesymbol{{$\vdash$}}
\setspecialsymbol{!-}{{\turnstilesymbol}}

\def\divosymbol{{$\div^{o}$}}
\setspecialsymbol{/o}{{\it {\divosymbol}}}

\def\cinsymbol{{c$_{\mathrm{in}}$}}
\setspecialsymbol{cin}{{\it {\cinsymbol}}}

\def\coutsymbol{{c$_{\mathrm{out}}$}}
\setspecialsymbol{cout}{{\it {\coutsymbol}}}

\def\mfosymbol{{mf$^{\thinspace o}$}}
\setspecialsymbol{mfo}{{\it {\mfosymbol}}}

\def\fosymbol{{f$^{\thinspace o}$}}
\setspecialsymbol{fo}{{\it {\fosymbol}}}

\def\gosymbol{{g$^{\thinspace o}$}}
\setspecialsymbol{go}{{\it {\gosymbol}}}

\def\pathosymbol{{path$^{\thinspace o}$}}
\setspecialsymbol{patho}{{\it {\pathosymbol}}}

\def\arcosymbol{{arc$^{\thinspace o}$}}
\setspecialsymbol{arco}{{\it {\arcosymbol}}}

\def\varosymbol{{var$^{\thinspace o}$}}
\setspecialsymbol{varo}{{\it {\varosymbol}}}

\def\genosymbol{{gen$^{\thinspace o}$}}
\setspecialsymbol{geno}{{\it {\genosymbol}}}

\def\genuosymbol{{genu$^{\thinspace o}$}}
\setspecialsymbol{genuo}{{\it {\genuosymbol}}}

\def\genbosymbol{{genb$^{\thinspace o}$}}
\setspecialsymbol{genbo}{{\it {\genbosymbol}}}

\def\zeroosymbol{{zero$^{\thinspace o}$}}
\setspecialsymbol{zeroo}{{\it {\zeroosymbol}}}

\def\splitosymbol{{split$^{\thinspace o}$}}
\setspecialsymbol{splito}{{\it {\splitosymbol}}}

\def\pososymbol{{pos$^{\thinspace o}$}}
\setspecialsymbol{poso}{{\it {\pososymbol}}}

\def\semimulosymbol{{semimul$^{\thinspace o}$}}
\setspecialsymbol{semimulo}{{\it {\semimulosymbol}}}

\def\lessosymbol{{less$^{\thinspace o}$}}
\setspecialsymbol{lesso}{{\it {\lessosymbol}}}

\def\lesslosymbol{{lessl$^{\thinspace o}$}}
\setspecialsymbol{lesslo}{{\it {\lesslosymbol}}}

\def\samelosymbol{{samel$^{\thinspace o}$}}
\setspecialsymbol{samelo}{{\it {\samelosymbol}}}

\def\lesslthreeosymbol{{lessl3$^{\thinspace o}$}}
\setspecialsymbol{lessl3o}{{\it {\lesslthreeosymbol}}}

\def\mulosymbol{{mul$^{\thinspace o}$}}
\setspecialsymbol{mulo}{{\it {\mulosymbol}}}

\def\oddmulosymbol{{odd-mul$^{\thinspace o}$}}
\setspecialsymbol{odd-mulo}{{\it {\oddmulosymbol}}}

\def\boundmulosymbol{{bound-mul$^{\thinspace o}$}}
\setspecialsymbol{bound-mulo}{{\it {\boundmulosymbol}}}

\def\onceosymbol{{once$^{\thinspace o}$}}
\setspecialsymbol{onceo}{{\it {\onceosymbol}}}

\def\repeatedmulosymbol{{repeated-mul$^{\thinspace o}$}}
\setspecialsymbol{repeated-mulo}{{\it {\repeatedmulosymbol}}}

\def\muloneosymbol{{mul1$^{\thinspace o}$}}
\setspecialsymbol{mul1o}{{\it {\muloneosymbol}}}

\def\multwoosymbol{{mul2$^{\thinspace o}$}}
\setspecialsymbol{mul2o}{{\it {\multwoosymbol}}}

\def\multhreeosymbol{{mul3$^{\thinspace o}$}}
\setspecialsymbol{mul3o}{{\it {\multhreeosymbol}}}

\def\gtlosymbol{{gtl$^{\thinspace o}$}}
\setspecialsymbol{gtlo}{{\it {\gtlosymbol}}}

\def\addosymbol{{add$^{\thinspace o}$}}
\setspecialsymbol{addo}{{\it {\addosymbol}}}

\def\minusosymbol{{minus$^{\thinspace o}$}}
\setspecialsymbol{minuso}{{\it {\minusosymbol}}}

\def\newgtoneosymbol{{$>\!$1$^{o}$}}
\setspecialsymbol{>1o}{{\bf{\newgtoneosymbol}}}

\def\genadderosymbol{{gen-adder$^{\thinspace o}$}}
\setspecialsymbol{gen-addero}{{\it {\genadderosymbol}}}

\def\fulladderosymbol{{full-adder$^{\thinspace o}$}}
\setspecialsymbol{full-addero}{{\it {\fulladderosymbol}}}

\def\halfadderosymbol{{half-adder$^{\thinspace o}$}}
\setspecialsymbol{half-addero}{{\it {\halfadderosymbol}}}

\def\fulloneadderosymbol{{full1-adder$^{\thinspace o}$}}
\setspecialsymbol{full1-addero}{{\it {\fulloneadderosymbol}}}

\def\fullnadderosymbol{{fulln-adder$^{\thinspace o}$}}
\setspecialsymbol{fulln-addero}{{\it {\fullnadderosymbol}}}

\def\adderosymbol{{adder$^{\thinspace o}$}}
\setspecialsymbol{addero}{{\it {\adderosymbol}}}

\def\subosymbol{{sub$^{\thinspace o}$}}
\setspecialsymbol{subo}{{\it {\subosymbol}}}

\def\divosymbol{{div$^{\thinspace o}$}}
\setspecialsymbol{divo}{{\it {\divosymbol}}}

\def\logosymbol{{log$^{\thinspace o}$}}
\setspecialsymbol{logo}{{\it {\logosymbol}}}

\def\exposymbol{{exp$^{\thinspace o}$}}
\setspecialsymbol{expo}{{\it {\exposymbol}}}

\def\exptwoosymbol{{exp2$^{\thinspace o}$}}
\setspecialsymbol{exp2o}{{\it {\exptwoosymbol}}}

\def\sfusymbol{{\sf u}}
\setspecialsymbol{sfu}{{\sfusymbol}}

\setspecialsymbol{bn0}{{\bf 0}}
\setspecialsymbol{bn1}{{\bf 1}}
\setspecialsymbol{bn2}{{\bf 2}}
\setspecialsymbol{bn3}{{\bf 3}}
\setspecialsymbol{bn4}{{\bf 4}}
\setspecialsymbol{bn5}{{\bf 5}}
\setspecialsymbol{bn6}{{\bf 6}}
\setspecialsymbol{bn7}{{\bf 7}}

\setspecialsymbol{bnminus1}{{\bf -1}}



\def\lambdasymbol{{$\lambda$}}
\setspecialsymbol{lambda}{{\bf {\lambdasymbol}}}

\def\ulambdasymbol{{u$\lambda$}}

\setspecialsymbol{ulambda}{{\bf {\ulambdasymbol}}}
\setspecialsymbol{fake-ulambda}{{\bf {\ulambdasymbol}}}

\setspecialsymbol{ulambda}{{\bf {\ulambdasymbol}}}



\def\umatchsymbol{{umatch}}
\setspecialsymbol{umatch}{{\bf {\umatchsymbol}}}
\setspecialsymbol{fake-umatch}{{\bf {\umatchsymbol}}}

\def\umatchauxsymbol{{umatch-aux}}
\setspecialsymbol{umatch-aux}{{\bf {\umatchauxsymbol}}}



\def\lambdaesymbol{{$\lambda^{e}$}}
\setspecialsymbol{lambda-e}{{\bf {\lambdaesymbol}}}
\setspecialsymbol{lambdae}{{\bf {\lambdaesymbol}}}

%\def\lambdawsymbol{{$\lambda^{w}$}}
%\setspecialsymbol{lambda-w}{{\bf {\lambdawsymbol}}}
%\setspecialsymbol{lambdaw}{{\bf {\lambdawsymbol}}}

\def\lambdaasymbol{{$\lambda^{a}$}}
\setspecialsymbol{lambda-a}{{\bf {\lambdaasymbol}}}
\setspecialsymbol{lambdaa}{{\bf {\lambdaasymbol}}}

\def\lambdausymbol{{$\lambda^{u}$}}
\setspecialsymbol{lambda-u}{{\bf {\lambdausymbol}}}
\setspecialsymbol{lambdau}{{\bf {\lambdausymbol}}}

\def\lpatsymbol{{lpat}}
\setspecialsymbol{lpat}{{\bf {\lpatsymbol}}}



\def\matchesymbol{{match$^{e}$}}
\setspecialsymbol{match-e}{{\bf {\matchesymbol}}}
\setspecialsymbol{matche}{{\bf {\matchesymbol}}}

\def\matchasymbol{{match$^{a}$}}
\setspecialsymbol{match-a}{{\bf {\matchasymbol}}}
\setspecialsymbol{matcha}{{\bf {\matchasymbol}}}

\def\matchusymbol{{match$^{u}$}}
\setspecialsymbol{match-u}{{\bf {\matchusymbol}}}
\setspecialsymbol{matchu}{{\bf {\matchusymbol}}}

\def\mpatsymbol{{mpat}}
\setspecialsymbol{mpat}{{\bf {\mpatsymbol}}}


\def\ssstarsymbol{{ss$^{*}$}}
\setspecialsymbol{ss*}{{\it {\ssstarsymbol}}}


\def\ifstarsymbol{{if$^{\thinspace *}$}}
\setspecialsymbol{if*}{{\it {\ifstarsymbol}}}

\def\plusstarosymbol{{plus$^{\thinspace *o}$}}
\setspecialsymbol{plus*o}{{\it {\plusstarosymbol}}}

\def\foldlosymbol{{foldl$^{\thinspace o}$}}
\setspecialsymbol{foldlo}{{\it {\foldlosymbol}}}

\def\foldrosymbol{{foldr$^{\thinspace o}$}}
\setspecialsymbol{foldro}{{\it {\foldrosymbol}}}

\setspecialsymbol{proveo}{$prove^o$}

\setspecialsymbol{body1}{$body_1$}

\setspecialsymbol{existsbad}{{\textit{exists}}}

\setspecialsymbol{anom}{{\textit{a}}}
\setspecialsymbol{bnom}{{\textit{b}}}

\setspecialsymbol{Nvar}{{\textit{n}}}
\setspecialsymbol{Mvar}{{\textit{m}}}
\setspecialsymbol{Xvar}{{\textit{x}}}
\setspecialsymbol{Yvar}{{\textit{y}}}

\setspecialsymbol{unexp1}{$unexp_1$}

\setspecialsymbol{prf1}{$pr\!f_1$}
\setspecialsymbol{prf2}{$pr\!f_2$}

\setspecialsymbol{lit1}{$lit_1$}

\setspecialsymbol{tm*}{$tm^*$}

\setspecialsymbol{ansv*}{$ansv^*$}
\setspecialsymbol{cache-ansv*}{$cache$-$ansv^*$}
%\setspecialsymbol{cache-ansv*-set!}{$cache$-$ansv^*$-$set!$}
\setspecialsymbol{cache-ansv*-set!}{{\it cache-ansv$^*$\!-set!}}
\setspecialsymbol{ss-ansv*}{$ss$-$ansv^*$}

\def\truesymbol{{\tt \#}{\sf t}}
\def\falsesymbol{{\tt \#}{\sf f}}

\setspecialsymbol{#t}{{\truesymbol}}
\setspecialsymbol{#f}{{\falsesymbol}}

\def\failsymbol{{\tt \#}{\sf u}}
\def\succeedsymbol{{\tt \#}{\sf s}}

\setspecialsymbol{#u}{{\failsymbol}}
\setspecialsymbol{#s}{{\succeedsymbol}}

\setspecialsymbol{...}{{$\ldots$}}

\def\quotelitsymbol{{(\bf{{quote}} $lit$)}}
\setspecialsymbol{quotelit}{{{\quotelitsymbol}}}

\def\walkstarsymbol{walk$^{*}$}
\setspecialsymbol{walk*}{{\it{\walkstarsymbol}}}

\setspecialsymbol{lambdag@}{{\bf{$\lambda$$_{\raisebox{-1pt}{\mbox{\tiny \sffamily \bfseries G}}}$}}}
\setspecialsymbol{lambdaf@}{{\bf{$\lambda$$_{\raisebox{-1pt}{\mbox{\tiny \sffamily \bfseries F}}}$}}}
\setspecialsymbol{lambdap@}{{\bf{$\lambda$$_{\raisebox{-1pt}{\mbox{\tiny \sffamily \bfseries P}}}$}}}

\def\symboltostringsymbol{{\it{symbol$\rightarrow$string}}}
\setspecialsymbol{symbol->string}{{\symboltostringsymbol}}

\def\stringtosymbolsymbol{{\it{string$\rightarrow$symbol}}}
\setspecialsymbol{string->symbol}{{\stringtosymbolsymbol}}

\def\stringtonumbersymbol{{\it{string$\rightarrow$number}}}
\setspecialsymbol{string->number}{{\stringtonumbersymbol}}

\def\stringtolistsymbol{{\it{string$\rightarrow$list}}}
\setspecialsymbol{string->list}{{\stringtolistsymbol}}

\def\listtostringsymbol{{\it{list$\rightarrow$string}}}
\setspecialsymbol{list->string}{{\listtostringsymbol}}

\def\numbertostringsymbol{{\it{number$\rightarrow$string}}}
\setspecialsymbol{number->string}{{\numbertostringsymbol}}

\def\bindstarsymbol{{bind$^{*}$}}
\setspecialsymbol{bind*}{{\bf {\bindstarsymbol}}}

\def\mplusstarsymbol{{mplus$^{*}$}}
\setspecialsymbol{mplus*}{{\bf {\mplusstarsymbol}}}

\def\incsymbol{{inc}}
\setspecialsymbol{inc}{{\bf {\incsymbol}}}

\def\choosesymbol{{choose}}
\setspecialsymbol{choose}{{\bf{\choosesymbol}}}

\def\choicesymbol{{choice}}
\setspecialsymbol{choice}{{\bf{\choicesymbol}}}

\def\ainfsymbol{a{\tiny $^{^{\infty}}$}}
\setspecialsymbol{a-inf}{{\it{\ainfsymbol}}}

\def\pinfsymbol{p{\tiny $^{^{\infty}}$}}
\setspecialsymbol{p-inf}{{\it{\pinfsymbol}}}

\def\caseinfsymbol{case{\tiny $^{^{\infty}}$}}
\setspecialsymbol{case-inf}{{\bf {\caseinfsymbol}}}

\def\sigmasymbol{{$\sigma$}}
\setspecialsymbol{sigma}{{\it {\sigmasymbol}}}

\def\tausymbol{{$\tau$}}
\setspecialsymbol{tau}{{\it {\tausymbol}}}

\def\nablasymbol{{$\nabla$}}
\setspecialsymbol{nabla}{{\it {\nablasymbol}}}

\def\deltasymbol{{$\delta$}}
\setspecialsymbol{delta}{{\it {\deltasymbol}}}

\def\deltahatsymbol{{$\hat{\delta}$}}
\setspecialsymbol{delta^}{{\it {\deltahatsymbol}}}

\def\eqnssymbol{{$\epsilon$}}
\setspecialsymbol{eqns}{{\it {\eqnssymbol}}}

\def\applysigmarulessymbol{{apply-$\sigma$\!-rules}}
\setspecialsymbol{apply-sigma-rules}{{\it {\applysigmarulessymbol}}}

\def\applypisymbol{{apply-$\pi$}}
\setspecialsymbol{apply-pi}{{\it {\applypisymbol}}}

\def\mvletsymbol{{mv-let}}
\setspecialsymbol{mv-let}{{\bf {\mvletsymbol}}}

\def\pmatchsymbol{{pmatch}}
\setspecialsymbol{pmatch}{{\bf {\pmatchsymbol}}}

\def\ppatsymbol{{ppat}}
\setspecialsymbol{ppat}{{\bf {\ppatsymbol}}}

\def\guardsymbol{{guard}}
\setspecialsymbol{guard}{{\bf {\guardsymbol}}}

\def\bfconssymbol{{cons}}
\setspecialsymbol{bf-cons}{{\bf {\bfconssymbol}}}

\def\bfexpandsymbol{{expand}}
\setspecialsymbol{bf-expand}{{\bf {\bfexpandsymbol}}}




\def\unifyhashsymbol{{unify$\#$}}
\setspecialsymbol{unifyhash}{{\it {\unifyhashsymbol}}}

\def\sigmahatsymbol{{$\hat{\sigma}$}}
\setspecialsymbol{sigma^}{{\it {\sigmahatsymbol}}}

\def\deltaunionsymbol{{$\delta$-union}}
\setspecialsymbol{delta-union}{{\it {\deltaunionsymbol}}}

\def\applysubstsymbol{{apply-subst}}
\setspecialsymbol{apply-subst}{{\it {\applysubstsymbol}}}

\def\composesubstsymbol{{compose-subst}}
\setspecialsymbol{compose-subst}{{\it {\composesubstsymbol}}}

\def\applynablarulessymbol{{apply-\!$\nabla$\!-rules}}
\setspecialsymbol{apply-nabla-rules}{{\it {\applynablarulessymbol}}}

\def\sigmarulessymbol{{$\sigma$\!-rules}}
\setspecialsymbol{sigma-rules}{{\it {\sigmarulessymbol}}}

\def\emptysigmasymbol{{empty-$\sigma$}}
\setspecialsymbol{empty-sigma}{{\it {\emptysigmasymbol}}}

\def\emptynablasymbol{{empty-\!$\nabla$}}
\setspecialsymbol{empty-nabla}{{\it {\emptynablasymbol}}}

\def\emptydeltasymbol{{empty-$\delta$}}
\setspecialsymbol{empty-delta}{{\it {\emptydeltasymbol}}}

\def\nablarulessymbol{{$\nabla$\!-rules}}
\setspecialsymbol{nabla-rules}{{\it {\nablarulessymbol}}}

\def\pisymbol{{$\pi$}}
\setspecialsymbol{pi}{{\bf {\pisymbol}}}

\def\pihatsymbol{{$\hat{\pi}$}}
\setspecialsymbol{pi^}{{\bf {\pihatsymbol}}}


\def\cstarhatsymbol{{$\hat{c}^{*}$}}
\setspecialsymbol{c*^}{{\cstarhatsymbol}}

\def\cstarofsymbol{{$c^{*}$-of}}
\setspecialsymbol{c*-of}{{\it \cstarofsymbol}}

\def\emptycstarsymbol{{empty-$c^{*}$}}
\setspecialsymbol{empty-c*}{{\it \emptycstarsymbol}}


\newcommand{\RfiveRS}{$R^{5}\!RS$}
\newcommand{\RfiveRSsp}{$R^{5}\!RS$ }

\newcommand{\RsixRS}{$R^{6}\!RS$}
\newcommand{\RsixRSsp}{$R^{6}\!RS$ }

\newcommand{\one}{$\textcircled{\raisebox{-.8pt}{1}}\;$}
\newcommand{\two}{$\textcircled{\raisebox{-.8pt}{2}}\;$}

\newcommand{\onet}{$\textcircled{\raisebox{-.8pt}{1}}$}
\newcommand{\twot}{$\textcircled{\raisebox{-.8pt}{2}}$}

\setspecialsymbol{one}{\one} \setspecialsymbol{two}{\two}
\setspecialsymbol{backquotefootnote}{\footnotemark}

\newcommand{\copyterm}{{\tt copy\_term/2}}
\newcommand{\copytermo}{$copy\mbox{-}term^o$}
\setspecialsymbol{copy-termo}{$copy\mbox{-}term^o$}

\setspecialsymbol{subst-lito}{$subst\mbox{-}lit^o$}
\setspecialsymbol{subst-termo}{$subst\mbox{-}term^o$}
\setspecialsymbol{subst-term*}{$subst\mbox{-}term^{*o}$}

\newcommand{\conda}{\textbf{cond}$^a$}

\newcommand{\lambdaprolog}{$\lambda$Pro\-log}
\newcommand{\lambdaprologsp}{$\lambda$Pro\-log }

\newcommand{\alphaprolog}{$\alpha$Pro\-log}
\newcommand{\alphaprologsp}{$\alpha$Pro\-log }
\newcommand{\caml}{C$\alpha$ML}

\newcommand{\alphatap}{\mbox{$\alpha${\sf lean}\-$T\!\!A\!P$}}
\newcommand{\alphatapsp}{\mbox{$\alpha${\sf lean}\-$T\!\!A\!P\;$}}

\newcommand{\alphakanren}{$\alpha$Kan\-ren}
\newcommand{\alphakanrensp}{$\alpha$Kan\-ren }

\newcommand{\leantap}{\mbox{{\sf lean}\-$T\!\!A\!P\!$}}
\newcommand{\leantapsp}{\mbox{{\sf lean}\-$T\!\!A\!P\;$}}

\def\mk{miniKanren}

\def\wspace{\vspace{0.2cm}}
\def\tspace{\vspace{0.1cm}}
\def\nspace{\vspace{-0.08cm}}

\setspecialsymbol{run}{{\textbf{run}}}

\def\runonesymbol{run{$^{\bf{1}}$}}
\def\runtwosymbol{run{$^{\bf{2}}$}}
\def\runthreesymbol{run{$^{\bf{3}}$}}
\def\runfoursymbol{run{$^{\bf{4}}$}}
\def\runfivesymbol{run{$^{\bf{5}}$}}
\def\runsixsymbol{run{$^{\bf{6}}$}}
\def\runsevensymbol{run{$^{\bf{7}}$}}
\def\runeightsymbol{run{$^{\bf{8}}$}}
\def\runninesymbol{run{$^{\bf{9}}$}}
\def\runtensymbol{run{$^{\bf{10}}$}}
\def\runelevensymbol{run{$^{\bf{11}}$}}
\def\runtwelvesymbol{run{$^{\bf{12}}$}}
\def\runthirteensymbol{run{$^{\bf{13}}$}}
\def\runfourteensymbol{run{$^{\bf{14}}$}}
\def\runfifteensymbol{run{$^{\bf{15}}$}}
\def\runsixteensymbol{run{$^{\bf{16}}$}}
\def\runseventeensymbol{run{$^{\bf{17}}$}}
\def\runeighteensymbol{run{$^{\bf{18}}$}}
\def\runnineteensymbol{run{$^{\bf{19}}$}}
\def\runtwentysymbol{run{$^{\bf{20}}$}}
\def\runtwentyonesymbol{run{$^{\bf{21}}$}}
\def\runtwentytwosymbol{run{$^{\bf{22}}$}}
\def\runtwentythreesymbol{run{$^{\bf{23}}$}}
\def\runtwentyfoursymbol{run{$^{\bf{24}}$}}

\setspecialsymbol{run1}{{\bf{\runonesymbol}}}
\setspecialsymbol{run2}{{\bf{\runtwosymbol}}}
\setspecialsymbol{run3}{{\bf{\runthreesymbol}}}
\setspecialsymbol{run4}{{\bf{\runfoursymbol}}}
\setspecialsymbol{run5}{{\bf{\runfivesymbol}}}
\setspecialsymbol{run6}{{\bf{\runsixsymbol}}}
\setspecialsymbol{run7}{{\bf{\runsevensymbol}}}
\setspecialsymbol{run8}{{\bf{\runeightsymbol}}}
\setspecialsymbol{run9}{{\bf{\runninesymbol}}}
\setspecialsymbol{run10}{{\bf{\runtensymbol}}}
\setspecialsymbol{run11}{{\bf{\runelevensymbol}}}
\setspecialsymbol{run12}{{\bf{\runtwelvesymbol}}}
\setspecialsymbol{run13}{{\bf{\runthirteensymbol}}}
\setspecialsymbol{run14}{{\bf{\runfourteensymbol}}}
\setspecialsymbol{run15}{{\bf{\runfifteensymbol}}}
\setspecialsymbol{run16}{{\bf{\runsixteensymbol}}}
\setspecialsymbol{run17}{{\bf{\runseventeensymbol}}}
\setspecialsymbol{run18}{{\bf{\runeighteensymbol}}}
\setspecialsymbol{run19}{{\bf{\runnineteensymbol}}}
\setspecialsymbol{run20}{{\bf{\runtwentysymbol}}}
\setspecialsymbol{run21}{{\bf{\runtwentyonesymbol}}}
\setspecialsymbol{run22}{{\bf{\runtwentytwosymbol}}}
\setspecialsymbol{run23}{{\bf{\runtwentythreesymbol}}}
\setspecialsymbol{run24}{{\bf{\runtwentyfoursymbol}}}

\setspecialsymbol{run*}{{\textbf{run}$^{\bf *}$}}

\setspecialsymbol{tie}{$\bowtie$}
\setspecialsymbol{==}{$\equiv$}
\def\equalequalnochecksymbol{{$\equiv$-no-check}}
\setspecialsymbol{==-no-check}{{\it \equalequalnochecksymbol}}
\def\neverequalnochecksymbol{{$\neq$-no-check}}
\setspecialsymbol{=/=-no-check}{{\it \neverequalnochecksymbol}}
\def\equalequalverifysymbol{{$\equiv$-verify}}
\setspecialsymbol{==-verify}{{\it \equalequalverifysymbol}}
\setspecialsymbol{=/=}{$\neq$}
\def\neverequalverifysymbol{{$\neq$-verify}}
\setspecialsymbol{=/=-verify}{{\it \neverequalverifysymbol}}
\def\neverequalcheckosymbol{{$\neq^{\protect\surd}$}}
\setspecialsymbol{=/=-check}{{\neverequalcheckosymbol}}
\setspecialsymbol{hash}{{\textit{\#}}}
\setspecialsymbol{fresh}{{\textbf{fresh}}}
\setspecialsymbol{exist}{{\textbf{exist}}}

\def\verifycstarsymbol{{verify-c$^{*}$}}
\setspecialsymbol{verify-c*}{{\it \verifycstarsymbol}}

\def\unifystarsymbol{{unify$^{*}$}}
\setspecialsymbol{unify*}{{\it \unifystarsymbol}}


\setspecialsymbol{exist*}{{\textbf{exist$^{\bf *}$}}}
\setspecialsymbol{fresh*}{{\textbf{fresh$^{\bf *}$}}}

\setspecialsymbol{ext-s-check}{{\it{\extendschecksymbol}}}
\setspecialsymbol{unify-check}{{\it{\unifychecksymbol}}}
\setspecialsymbol{occurs-check}{{\it{\occurschecksymbol}}}
\setspecialsymbol{==-check}{{\it{\equalequalchecksymbol}}}

\def\extendschecksymbol{{ext-s$^{\protect\surd}$}}
\def\unifychecksymbol{{unify$^{\protect\surd}$}}
\def\occurschecksymbol{{occurs$^{\protect\surd}$}}
\def\equalequalchecksymbol{{$\equiv^{\protect\surd}$}}

\setspecialsymbol{conde}{{\textbf{cond}$^e$}}
\setspecialsymbol{conda}{{\textbf{cond}$^a$}}
\setspecialsymbol{condu}{{\textbf{cond}$^u$}}
\setspecialsymbol{project}{{\textbf{project}}}

\def\substosymbol{{subst$^{\thinspace o}$}}
\setspecialsymbol{substo}{{\it {\substosymbol}}}

\def\valueosymbol{{value$^{\thinspace o}$}}
\setspecialsymbol{valueo}{{\it {\valueosymbol}}}

\def\expressionosymbol{{expression$^{\thinspace o}$}}
\setspecialsymbol{expressiono}{{\it {\expressionosymbol}}}

\def\contextosymbol{{context$^{\thinspace o}$}}
\setspecialsymbol{contexto}{{\it {\contextosymbol}}}

\def\onesteposymbol{{one-step$^{\thinspace o}$}}
\setspecialsymbol{one-stepo}{{\it {\onesteposymbol}}}

\def\fullyreduceosymbol{{fully-reduce$^{\thinspace o}$}}
\setspecialsymbol{fully-reduceo}{{\it {\fullyreduceosymbol}}}

\def\safefullyreduceosymbol{{safe-fully-reduce$^{\thinspace o}$}}
\setspecialsymbol{safe-fully-reduceo}{{\it {\safefullyreduceosymbol}}}

\def\inholeosymbol{{in-hole$^{\thinspace o}$}}
\setspecialsymbol{in-holeo}{{\it {\inholeosymbol}}}

\def\lookuposymbol{{lookup$^{\thinspace o}$}}
\setspecialsymbol{lookupo}{{\it {\lookuposymbol}}}

\def\bitxorosymbol{{bit-xor$^{\thinspace o}$}}
\setspecialsymbol{bit-xoro}{{\it {\bitxorosymbol}}}

\def\bitandosymbol{{bit-and$^{\thinspace o}$}}
\setspecialsymbol{bit-ando}{{\it {\bitandosymbol}}}


\def\plusosymbol{{plus$^{\thinspace o}$}}
\setspecialsymbol{pluso}{{\it {\plusosymbol}}}

\setspecialsymbol{negateo}{$negate^o$}

\setspecialsymbol{membero}{$member^o$}

\def\rembersymbol{{$rember$}}
\setspecialsymbol{rember}{{\rembersymbol}}

\def\remberosymbol{{$rember^o$}}
\setspecialsymbol{rembero}{{\remberosymbol}}

\setspecialsymbol{swappendo}{$swappend^o$}
\setspecialsymbol{appendo}{$append^o$}

\setspecialsymbol{pairo}{$pair^o$}

\def\typosymbol{{typ$^{\thinspace o}$}}
\setspecialsymbol{typo}{{\it {\typosymbol}}}

\def\anyosymbol{{any$^{\thinspace o}$}}
\setspecialsymbol{anyo}{{\it {\anyosymbol}}}

\def\alwaysosymbol{{always$^{\thinspace o}$}}
\setspecialsymbol{alwayso}{{\it {\alwaysosymbol}}}

\def\neverosymbol{{never$^{\thinspace o}$}}
\setspecialsymbol{nevero}{{\it {\neverosymbol}}}

\def\rightarrowsymbol{{$\rightarrow$}}
\setspecialsymbol{rightarrowsymbol}{{\it {\rightarrowsymbol}}}
\setspecialsymbol{-->}{{\it {\rightarrowsymbol}}}

\def\bindertagsymbol{{tie}}
\setspecialsymbol{tie-tag}{{\sf {\bindertagsymbol}}}

\def\susptagsymbol{{susp}}
\setspecialsymbol{susp-tag}{{\sf {\susptagsymbol}}}

\def\nomtagsymbol{{nom}}
\setspecialsymbol{nom-tag}{{\sf {\nomtagsymbol}}}

\def\vartag{{\sf var}}
\setspecialsymbol{vartag}{{\vartag}}

\def\asubnsymbol{{$a_{n}$}}
\setspecialsymbol{a_n}{{\it {\asubnsymbol}}}

\def\bsubnsymbol{{$b_{n}$}}
\setspecialsymbol{b_n}{{\it {\bsubnsymbol}}}

\def\asubzerosymbol{{$a_{0}$}}
\setspecialsymbol{a_0}{{\it {\asubzerosymbol}}}

\def\bsubzerosymbol{{$b_{0}$}}
\setspecialsymbol{b_0}{{\it {\bsubzerosymbol}}}

\def\asubonesymbol{{$a_{1}$}}
\setspecialsymbol{a_1}{{\it {\asubonesymbol}}}

\def\bsubonesymbol{{$b_{1}$}}
\setspecialsymbol{b_1}{{\it {\bsubonesymbol}}}

\def\asubtwosymbol{{$a_{2}$}}
\setspecialsymbol{a_2}{{\it {\asubtwosymbol}}}

\def\bsubtwosymbol{{$b_{2}$}}
\setspecialsymbol{b_2}{{\it {\bsubtwosymbol}}}

\setspecialsymbol{=>}{$\Rightarrow$}

\def\ahatsymbol{{$\hat{a}$}}
\def\bhatsymbol{{$\hat{b}$}}
\def\chatsymbol{{$\hat{c}$}}
\def\dhatsymbol{{$\hat{d}$}}
\def\ehatsymbol{{$\hat{e}$}}
\def\fhatsymbol{{$\hat{f}$}}
\def\ghatsymbol{{$\hat{g}$}}
\def\hhatsymbol{{$\hat{h}$}}
\def\ihatsymbol{{$\hat{i}$}}
\def\jhatsymbol{{$\hat{j}$}}
\def\khatsymbol{{$\hat{k}$}}
\def\lhatsymbol{{$\hat{l}$}}
\def\mhatsymbol{{$\hat{m}$}}
\def\nhatsymbol{{$\hat{n}$}}
\def\ohatsymbol{{$\hat{o}$}}
\def\phatsymbol{{$\hat{p}$}}
\def\qhatsymbol{{$\hat{q}$}}
\def\rhatsymbol{{$\hat{r}$}}
\def\shatsymbol{{$\hat{s}$}}
\def\thatsymbol{{$\hat{t}$}}
\def\uhatsymbol{{$\hat{u}$}}
\def\vhatsymbol{{$\hat{v}$}}
\def\whatsymbol{{$\hat{w}$}}
\def\xhatsymbol{{$\hat{x}$}}
\def\yhatsymbol{{$\hat{y}$}}
\def\zhatsymbol{{$\hat{z}$}}

\def\txhatsymbol{{$t\hat{x}$}}
\def\tehatsymbol{{$t\hat{e}$}}

\setspecialsymbol{a^}{{\ahatsymbol}}
\setspecialsymbol{b^}{{\bhatsymbol}}
\setspecialsymbol{c^}{{\chatsymbol}}
\setspecialsymbol{d^}{{\dhatsymbol}}
\setspecialsymbol{e^}{{\ehatsymbol}}
\setspecialsymbol{f^}{{\fhatsymbol}}
\setspecialsymbol{g^}{{\ghatsymbol}}
\setspecialsymbol{h^}{{\hhatsymbol}}
\setspecialsymbol{i^}{{\ihatsymbol}}
\setspecialsymbol{j^}{{\jhatsymbol}}
\setspecialsymbol{k^}{{\khatsymbol}}
\setspecialsymbol{l^}{{\lhatsymbol}}
\setspecialsymbol{m^}{{\mhatsymbol}}
\setspecialsymbol{n^}{{\nhatsymbol}}
\setspecialsymbol{o^}{{\ohatsymbol}}
\setspecialsymbol{p^}{{\phatsymbol}}
\setspecialsymbol{q^}{{\qhatsymbol}}
\setspecialsymbol{r^}{{\rhatsymbol}}
\setspecialsymbol{s^}{{\shatsymbol}}
\setspecialsymbol{t^}{{\thatsymbol}}
\setspecialsymbol{u^}{{\uhatsymbol}}
\setspecialsymbol{v^}{{\vhatsymbol}}
\setspecialsymbol{w^}{{\whatsymbol}}
\setspecialsymbol{x^}{{\xhatsymbol}}
\setspecialsymbol{y^}{{\yhatsymbol}}
\setspecialsymbol{z^}{{\zhatsymbol}}

\setspecialsymbol{tx^}{{\txhatsymbol}}
\setspecialsymbol{te^}{{\tehatsymbol}}

\def\tonehatsymbol{{$\hat{t}_{1}$}}
\setspecialsymbol{t1^}{{\bf {\tonehatsymbol}}}

\def\ttwohatsymbol{{$\hat{t}_{2}$}}
\setspecialsymbol{t2^}{{\bf {\ttwohatsymbol}}}

\setspecialsymbol{a0}{$a_{0}$}
\setspecialsymbol{b0}{$b_{0}$}
\setspecialsymbol{c0}{$c_{0}$}
\setspecialsymbol{d0}{$d_{0}$}
\setspecialsymbol{e0}{$e_{0}$}
\setspecialsymbol{f0}{$f_{0}$}
\setspecialsymbol{g0}{$g_{0}$}
\setspecialsymbol{h0}{$h_{0}$}
\setspecialsymbol{i0}{$i_{0}$}
\setspecialsymbol{j0}{$j_{0}$}
\setspecialsymbol{k0}{$k_{0}$}
\setspecialsymbol{l0}{$l_{0}$}
\setspecialsymbol{m0}{$m_{0}$}
\setspecialsymbol{n0}{$n_{0}$}
\setspecialsymbol{o0}{$o_{0}$}
\setspecialsymbol{p0}{$p_{0}$}
\setspecialsymbol{q0}{$q_{0}$}
\setspecialsymbol{r0}{$r_{0}$}
\setspecialsymbol{s0}{$s_{0}$}
\setspecialsymbol{t0}{$t_{0}$}
\setspecialsymbol{u0}{$u_{0}$}
\setspecialsymbol{v0}{$v_{0}$}
\setspecialsymbol{w0}{$w_{0}$}
\setspecialsymbol{x0}{$x_{0}$}
\setspecialsymbol{y0}{$y_{0}$}
\setspecialsymbol{z0}{$z_{0}$}

\setspecialsymbol{a1}{$a_{1}$}
\setspecialsymbol{b1}{$b_{1}$}
\setspecialsymbol{c1}{$c_{1}$}
\setspecialsymbol{d1}{$d_{1}$}
\setspecialsymbol{e1}{$e_{1}$}
\setspecialsymbol{f1}{$f_{1}$}
\setspecialsymbol{g1}{$g_{1}$}
\setspecialsymbol{h1}{$h_{1}$}
\setspecialsymbol{i1}{$i_{1}$}
\setspecialsymbol{j1}{$j_{1}$}
\setspecialsymbol{k1}{$k_{1}$}
\setspecialsymbol{l1}{$l_{1}$}
\setspecialsymbol{m1}{$m_{1}$}
\setspecialsymbol{n1}{$n_{1}$}
\setspecialsymbol{o1}{$o_{1}$}
\setspecialsymbol{p1}{$p_{1}$}
\setspecialsymbol{q1}{$q_{1}$}
\setspecialsymbol{r1}{$r_{1}$}
\setspecialsymbol{s1}{$s_{1}$}
\setspecialsymbol{t1}{$t_{1}$}
\setspecialsymbol{u1}{$u_{1}$}
\setspecialsymbol{v1}{$v_{1}$}
\setspecialsymbol{w1}{$w_{1}$}
\setspecialsymbol{x1}{$x_{1}$}
\setspecialsymbol{y1}{$y_{1}$}
\setspecialsymbol{z1}{$z_{1}$}

\setspecialsymbol{a2}{$a_{2}$}
\setspecialsymbol{b2}{$b_{2}$}
\setspecialsymbol{c2}{$c_{2}$}
\setspecialsymbol{d2}{$d_{2}$}
\setspecialsymbol{e2}{$e_{2}$}
\setspecialsymbol{f2}{$f_{2}$}
\setspecialsymbol{g2}{$g_{2}$}
\setspecialsymbol{g3}{$g_{3}$}
\setspecialsymbol{h2}{$h_{2}$}
\setspecialsymbol{i2}{$i_{2}$}
\setspecialsymbol{j2}{$j_{2}$}
\setspecialsymbol{k2}{$k_{2}$}
\setspecialsymbol{l2}{$l_{2}$}
\setspecialsymbol{m2}{$m_{2}$}
\setspecialsymbol{n2}{$n_{2}$}
\setspecialsymbol{o2}{$o_{2}$}
\setspecialsymbol{p2}{$p_{2}$}
\setspecialsymbol{q2}{$q_{2}$}
\setspecialsymbol{r2}{$r_{2}$}
\setspecialsymbol{s2}{$s_{2}$}
\setspecialsymbol{t2}{$t_{2}$}
\setspecialsymbol{u2}{$u_{2}$}
\setspecialsymbol{v2}{$v_{2}$}
\setspecialsymbol{w2}{$w_{2}$}
\setspecialsymbol{x2}{$x_{2}$}
\setspecialsymbol{y2}{$y_{2}$}
\setspecialsymbol{z2}{$z_{2}$}

\setspecialsymbol{a3}{$a_{3}$}
\setspecialsymbol{b3}{$b_{3}$}
\setspecialsymbol{c3}{$c_{3}$}
\setspecialsymbol{d3}{$d_{3}$}
\setspecialsymbol{e3}{$e_{3}$}
\setspecialsymbol{f3}{$f_{3}$}
\setspecialsymbol{g3}{$g_{3}$}
\setspecialsymbol{g3}{$g_{3}$}
\setspecialsymbol{h3}{$h_{2}$}
\setspecialsymbol{i3}{$i_{3}$}
\setspecialsymbol{j3}{$j_{3}$}
\setspecialsymbol{k3}{$k_{3}$}
\setspecialsymbol{l3}{$l_{3}$}
\setspecialsymbol{m3}{$m_{3}$}
\setspecialsymbol{n3}{$n_{3}$}
\setspecialsymbol{o3}{$o_{3}$}
\setspecialsymbol{p3}{$p_{3}$}
\setspecialsymbol{q3}{$q_{3}$}
\setspecialsymbol{r3}{$r_{3}$}
\setspecialsymbol{s3}{$s_{3}$}
\setspecialsymbol{t3}{$t_{3}$}
\setspecialsymbol{u3}{$u_{3}$}
\setspecialsymbol{v3}{$v_{3}$}
\setspecialsymbol{w3}{$w_{3}$}
\setspecialsymbol{x3}{$x_{3}$}
\setspecialsymbol{y3}{$y_{3}$}
\setspecialsymbol{z3}{$z_{3}$}

\setspecialsymbol{g4}{$g_{4}$}

\setspecialsymbol{tn}{$t_{n}$}
\setspecialsymbol{tm}{$t_{m}$}

\setspecialsymbol{xn}{$x_{n}$}

\setspecialsymbol{t11}{$t_{11}$}
\setspecialsymbol{t1n}{$t_{1n}$}

\setspecialsymbol{t21}{$t_{21}$}
\setspecialsymbol{t2n}{$t_{2n}$}

\setspecialsymbol{q2m}{$q_{2m}$}
\setspecialsymbol{q2md}{$q_{2md}$}

\setspecialsymbol{aa}{$a_{a}$}
\setspecialsymbol{ba}{$b_{a}$}
\setspecialsymbol{ca}{$c_{a}$}
\setspecialsymbol{ma}{$m_{a}$}
\setspecialsymbol{na}{$n_{a}$}
\setspecialsymbol{pa}{$p_{a}$}
\setspecialsymbol{ra}{$r_{a}$}

\setspecialsymbol{ad}{$a_{d}$}
\setspecialsymbol{bd}{$b_{d}$}
\setspecialsymbol{cd}{$c_{d}$}
\setspecialsymbol{md}{$m_{d}$}
\setspecialsymbol{nd}{$n_{d}$}
\setspecialsymbol{pd}{$p_{d}$}
\setspecialsymbol{rd}{$r_{d}$}

\setspecialsymbol{p1d}{$p_{1d}$}

\def\gstarsymbol{g$^{*}$}
\def\fstarsymbol{f$^{\thinspace *}$}
\def\cstarsymbol{c$^{*}$}
\def\vstarsymbol{v$^{*}$}
\def\pstarsymbol{p$^{*}$}
\def\tstarsymbol{t$^{*}$}
\def\wstarsymbol{w$^{*}$}

\setspecialsymbol{g*}{{\it{\gstarsymbol}}}
\setspecialsymbol{f*}{{\it{\fstarsymbol}}}
\setspecialsymbol{w*}{{\it{\wstarsymbol}}}
\setspecialsymbol{c*}{{\it{\cstarsymbol}}}
\setspecialsymbol{v*}{{\it{\vstarsymbol}}}
\setspecialsymbol{p*}{{\it{\pstarsymbol}}}
\setspecialsymbol{t*}{{\it{\tstarsymbol}}}

\setspecialsymbol{ratorres}{$rator$-$res$}
\setspecialsymbol{randres}{$rand$-$res$}
\setspecialsymbol{bodyres}{$body$-$res$}

\setspecialsymbol{e1res}{$e{_{1}}$-$res$}
\setspecialsymbol{e2res}{$e{_{2}}$-$res$}
\setspecialsymbol{e3res}{$e{_{3}}$-$res$}

\def\strstarsymbol{str$^{*}$}
\setspecialsymbol{str*}{{\it{\strstarsymbol}}}

\def\ifasymbol{{if$^{\thinspace a}$}}
\setspecialsymbol{ifa}{{\bf {\ifasymbol}}}

\def\ifusymbol{{if$^{\thinspace u}$}}
\setspecialsymbol{ifu}{{\bf {\ifusymbol}}}

\setspecialsymbol{n-const}{{\bf n}}
\setspecialsymbol{m-const}{{\bf m}}

\setspecialsymbol{__}{{\bf \_}}

\setspecialsymbol{_.0}{$\__{_{0}}$}
\setspecialsymbol{_.1}{$\__{_{1}}$}
\setspecialsymbol{_.2}{$\__{_{2}}$}
\setspecialsymbol{_.3}{$\__{_{3}}$}
\setspecialsymbol{_.4}{$\__{_{4}}$}

\hyphenation{mi-ni-Kan-ren}


% arithm
\hyphenation{co-type}

% representation of a binary numeral
\newcommand{\bn}[1]{\ensuremath{\mathbf{#1}}}

\newcommand{\add}[3]{\ensuremath{\mathtt{add}(#1,#2,#3)}}
\newcommand{\mul}[4][]{\ensuremath{\mathtt{mul#1}(#2,#3,#4)}}
\newcommand{\semimul}[3]{\ensuremath{\mathtt{semimul}(#1,#2,#3)}}
\newcommand{\denot}[1]{[\![#1]\!]}
\newcommand{\set}[1]{\{\, #1 \,\}}
\providecommand{\norm}[1]{\lVert#1\rVert}
\providecommand{\floor}[1]{\lfloor#1\rfloor}

\DeclareMathOperator{\dom}{dom}
\DeclareMathOperator{\UU}{\mathtt{u}}
\DeclareMathOperator{\BO}{\mathtt{o}}
\renewcommand{\o}[0]{\BO}
\DeclareMathOperator{\BL}{\mathtt{l}}
\renewcommand{\l}[0]{\BL}
% \DeclareMathOperator{\fv}{fv}

\newtheorem{definition}{Definition}
\newtheorem{proposition}{Proposition}

\newcommand{\cons}[2]{\ensuremath{[{#1}\mathop{\mid}{#2}]}}

%\newtheorem{definition}{Definition} % [section]
%\newtheorem{example}{Example} % [section]
%\newtheorem{proposition}{Proposition} % [section]
%\newtheorem{corollary}{Corollary} % [section]
%\let\from=\leftarrow

%\newcommand{\oleg}[1]{{\it [Oleg says: #1]}}
%\newcommand{\ccshan}[1]{{\it [Ken says: #1]}}
%\newcommand{\will}[1]{{\it [Will says: #1]}}
%\newcommand{\dan}[1]{{\it [Dan says: #1]}}
\newcommand{\omitit}[1]{\ignorespaces}

%\newcommand{\oleg}[1]{\ignorespaces}
%\newcommand{\ccshan}[1]{\ignorespaces}
%\newcommand{\will}[1]{\ignorespaces}
%\newcommand{\dan}[1]{\ignorespaces}


\setspecialsymbol{mzero}{{\textbf{mzero}}}
\setspecialsymbol{unit}{{\textbf{unit}}}
\setspecialsymbol{var}{{\textbf{var}}}
\setspecialsymbol{var?}{{\textbf{var?}}}

\setspecialsymbol{ak-var}{{\it{var}}}
\setspecialsymbol{ak-var?}{{\it{var?}}}

%%%%%%%%%% ferns

\setspecialsymbol{ints-bottom}{{\textit{ints-from}$_\perp$}} % $
\setspecialsymbol{always-five-bottom}{{\textit{always-five$_\perp$}}}
\setspecialsymbol{map-bottom}{{\it{map$_\perp$}}}
\setspecialsymbol{append-bottom}{{\it{append$_\perp$}}}
\setspecialsymbol{unit-bottom}{{\textit{unit}$_\perp$}}
\setspecialsymbol{mzero-bottom}{{\textit{mzero}$_\perp$}}
\setspecialsymbol{bind-bottom}{{\textit{bind}$_\perp$}}
\setspecialsymbol{mplus-bottom}{{\textbf{{mplus$_\perp$}}}}
\setspecialsymbol{mplus-func-bottom}{{\textit{mplus}$_\perp$}}
\setspecialsymbol{mplus-fn-bottom}{{\textit{mplus-aux}$_\perp$}}
\setspecialsymbol{==-bottom}{{$\equiv$$_\perp$}}
\setspecialsymbol{or-bottom}{{\textit{or}$_\perp$}}
\setspecialsymbol{disj-bottom}{{\textbf{{disj$_\perp$}}}}
\setspecialsymbol{conj-bottom}{{\textbf{{conj$_\perp$}}}}
\setspecialsymbol{run-bottom}{{\textit{run}$_\perp$}}
\setspecialsymbol{choose-bottom}{{\textbf{{choose$_\perp$}}}}
\setspecialsymbol{choose-aux-bottom}{{\textit{choose-aux}$_\perp$}}

\def\lessthanequalolsymbol{{$\leqslant${\it{l}}$^{\thinspace o}$}}
\def\lessthanosymbol{{$<^{o}$}}
\def\lessthanequalosymbol{{$\leqslant^{o}$}}
\def\lessthanolsymbol{{$<${\it{l}}$^{\thinspace o}$}}
\def\equalolsymbol{{$=${\it{l}}$^{\thinspace o}$}}

\setspecialsymbol{<=lo}{{\lessthanequalolsymbol}}
\setspecialsymbol{<lo}{{\lessthanolsymbol}}
\setspecialsymbol{=lo}{{\equalolsymbol}}
\setspecialsymbol{<o}{{\lessthanosymbol}}
\setspecialsymbol{<=o}{{\lessthanequalosymbol}}

\setspecialsymbol{mplusdollar}{{\textbf{{mplus$_s$}}}} %$
\setspecialsymbol{mplusdollar-fn}{{\textit{mplus-aux}$_s$}} % $
\setspecialsymbol{always-five$}{{\textit{always-five$_s$}}} %$
\setspecialsymbol{disjdollar}{{\textbf{{disj$_s$}}}}
\setspecialsymbol{conjdollar}{{\textbf{{conj$_s$}}}}
\setspecialsymbol{list$}{{\textbf{{list}$_s$}}} % $
\setspecialsymbol{cons$}{{\textbf{{cons}$_s$}}} % $
\setspecialsymbol{car$}{{\textit{car}$_s$}} % $
\setspecialsymbol{cdr$}{{\textit{cdr}$_s$}} % $
\setspecialsymbol{cadr$}{{\textit{cadr}$_s$}} % $
\setspecialsymbol{caar$}{{\textit{caar}$_s$}} % $
\setspecialsymbol{cdar$}{{\textit{cdar}$_s$}} % $
\setspecialsymbol{caddr$}{{\textit{caddr}$_s$}} % $
\setspecialsymbol{listdollar}{{\textbf{{list}$_s$}}} % $
\setspecialsymbol{consdollar}{{\textbf{{cons}$_s$}}} % $
\setspecialsymbol{cardollar}{{\textit{car}$\!_s$}} % $
\setspecialsymbol{cdrdollar}{{\textit{cdr}$\!_s$}} % $
\setspecialsymbol{cadrdollar}{{\textit{cadr}$\!_s$}} % $
\setspecialsymbol{caardollar}{{\textit{caar}$\!_s$}} % $
\setspecialsymbol{cdardollar}{{\textit{cdar}$\!_s$}} % $
\setspecialsymbol{caddrdollar}{{\textit{caddr}$\!_s$}} % $
\setspecialsymbol{ints$}{{\textit{ints-from}$_s$}} % $
\setspecialsymbol{binddollar}{{\textit{bind}$_s$}} % $
\setspecialsymbol{take-bottom}{{\it{take$_\perp$}}}
\setspecialsymbol{product-bottom}{{\it{product$_\perp$}}}
\setspecialsymbol{Cartesian-product-bottom}{{\it{Cartesian-product$_\perp$}}}

\setspecialsymbol{conj-standard}{{\textbf{{conj}}}}
\setspecialsymbol{disj-standard}{{\textbf{{disj}}}}
\def\fronssymbol{{\textbf{{cons}$_\perp$}}}
\setspecialsymbol{frons}{{\fronssymbol}}

\def\fcarsymbol{{\textit{car}$_\perp$}}
\setspecialsymbol{fcar}{{\fcarsymbol}}

\def\fcdrsymbol{{\textit{cdr}$_\perp$}}
\setspecialsymbol{fcdr}{{\fcdrsymbol}}

\setspecialsymbol{fcaar}{{\textit{caar}$_\perp$}}
\setspecialsymbol{fcadr}{{\it{cadr$_\perp$}}}
\setspecialsymbol{fcdar}{{\it{cdar$_\perp$}}}
\setspecialsymbol{fcddr}{{\it{cddr$_\perp$}}}
\setspecialsymbol{fcaddr}{{\it{caddr$_\perp$}}}

\setspecialsymbol{fern}{{\textbf{{list$_\perp$}}}}
%\setspecialsymbol{list}{{\textbf{{list}}}}
%%% \setspecialsymbol{run}{{\textit{{run$_\perp$}}}}


\setspecialsymbol{standard-bind}{{\textit{bind}}}
\setspecialsymbol{standard-mplus}{{\textbf{{mplus}}}}

\setspecialsymbol{seventwenty}{{\scriptsize{720}}}
\setspecialsymbol{onetwenty}{{\scriptsize{120}}}
\setspecialsymbol{six}{{\scriptsize{6}}}
\setspecialsymbol{one}{{\scriptsize{1}}}
\setspecialsymbol{two}{{\scriptsize{2}}}
\setspecialsymbol{append_s}{{\it{append$_s$}}}

\setspecialsymbol{~>}{$\rightsquigarrow$}
\setspecialsymbol{<-}{$\Leftarrow$} %% $\sim$ preceded it
\setspecialsymbol{let*}{{\textbf{{let$^*$}}}}
\setspecialsymbol{timed-lambda}{$\lambda_t$}
\setspecialsymbol{timed-let}{{\textbf{{let}$_t$}}}
\setspecialsymbol{gamma}{$\gamma$}
\setspecialsymbol{beta}{$\beta$}
\setspecialsymbol{alpha}{$\alpha$}
\setspecialsymbol{add1}{{\textit{add1}}}

\setspecialsymbol{p_start}{{\textit{p}$_{start}$}}
\setspecialsymbol{p_finish}{{\textit{p}$_{finish}$}}


\setspecialsymbol{car}{{\textit{car}}} % $
\setspecialsymbol{cdr}{{\textit{cdr}}} % $
\setspecialsymbol{caar}{{\textit{caar}}} % $
\setspecialsymbol{cadr}{{\textit{cadr}$_p$}} % $
\setspecialsymbol{cadr}{{\textit{cadr}}} % $

%\setspecialsymbol{step}{{\textit{coaxer}}}
%\setspecialsymbol{step/car}{{\textit{coax}$_a$}}
%\setspecialsymbol{step/cdr}{{\textit{coax}$_d$}}

%\setspecialsymbol{coaxer}{{\textit{coaxera}}}
\setspecialsymbol{coax-a}{{\textit{coax}$_a$}}
\setspecialsymbol{coax-d}{{\textit{coax}$_d$}}

\setspecialsymbol{expo}{{\textit{exp}$^{\thinspace o}$}}

\setspecialsymbol{mplus}{{\textit{{mplus}}}}
\setspecialsymbol{pick-fn}{{\textit{pick-aux}}}
\setspecialsymbol{to-cons!}{{\textit{to-cons!}}}
\setspecialsymbol{e-a}{{\textit{e}$_a$}}
\setspecialsymbol{e-d}{{\textit{e}$_d$}}
\setspecialsymbol{sk/car}{{\textit{sk}$_a$}}
\setspecialsymbol{sk/cdr}{{\textit{sk}$_d$}}
\setspecialsymbol{promote!}{{\textit{promote}}}
\setspecialsymbol{e*}{{\textit{e}$^*$}}
\setspecialsymbol{body*}{{\textit{body}$^*$}}
\setspecialsymbol{exp1}{{\textit{exp}$_1$}}
\setspecialsymbol{exp2}{{\textit{exp}$_2$}}
\setspecialsymbol{race-car}{{\textit{race}$_a$}}
\setspecialsymbol{race-cdr}{{\textit{race}$_d$}}
\setspecialsymbol{stepsd}{{\textit{steps}$_d$}}
\setspecialsymbol{stepd}{{\textit{step!}$_d$}}
\setspecialsymbol{engine-tag-L-car}{{\textit{L}$_a$\hspace{-1.5pt}\textit{?}}}
\setspecialsymbol{engine-tag-U-car}{{\textit{U}$\hspace{-1pt}\!_a$\hspace{-1.5pt}\textit{?}}}
\setspecialsymbol{engine-tag-L-cdr}{{\textit{L}$_d$\hspace{-1.5pt}\textit{?}}}
\setspecialsymbol{engine-tag-U-cdr}{{\textit{U}$\hspace{-1pt}\!_d$\hspace{-1.5pt}\textit{?}}}
\setspecialsymbol{unlocked-car?}{{\textit{unlocked}$_a?$}}
\setspecialsymbol{unlocked-cdr?}{{\textit{unlocked}$_d?$}}
\setspecialsymbol{locked-car?}{{\textit{locked}$_a?$}}
\setspecialsymbol{locked-cdr?}{{\textit{locked}$_d?$}}
\setspecialsymbol{bx-d}{{\textit{bx}$_d$}}
\setspecialsymbol{tag-a}{{\textit{tag}$_a$}}
\setspecialsymbol{tag-d}{{\textit{tag}$_d$}}
\setspecialsymbol{V}{{\textsf{V}}}
\setspecialsymbol{U}{{\textsf{U}}}
\setspecialsymbol{L}{{\textsf{L}}}
\setspecialsymbol{bottom}{{$\perp$}}
\setspecialsymbol{converges}{{$\!\downarrow$}}
%\setspecialsymbol{ext-s}{{\textit{ext-}$\sigma$}}
%\setspecialsymbol{empty-s}{{\textit{empty-}$\sigma$}}
\setspecialsymbol{subst}{{$\sigma$}}

\setspecialsymbol{!}{{$!$}}
\setspecialsymbol{!!}{{$!\ $}}
\setspecialsymbol{stepbang}{{\textit{step!}}}

\setspecialsymbol{engine}{{\textbf{{engine}}}}
\setspecialsymbol{pick}{{\textbf{{pick}}}}

\setspecialsymbol{tabled}{{\textbf{{tabled}}}}

\setspecialsymbol{define-record-type}{{\textbf{{define-record-type}}}}
\setspecialsymbol{define-syntax}{{\textbf{{define-syntax}}}}
\setspecialsymbol{syntax-rules}{{\textbf{{syntax-rules}}}}
\setspecialsymbol{quote}{{\textbf{{quote}}}}
\setspecialsymbol{unquote}{{\textbf{{unquote}}}}
\setspecialsymbol{and}{{\textbf{{and}}}}
\setspecialsymbol{or}{{\textbf{{or}}}}
\setspecialsymbol{when}{{\textbf{{when}}}}
\setspecialsymbol{case}{{\textbf{{case}}}}
\setspecialsymbol{cond}{{\textbf{{cond}}}}
\setspecialsymbol{if}{{\textbf{{if}}}}
\setspecialsymbol{begin}{{\textbf{{begin}}}}
\setspecialsymbol{letrec}{{\textbf{{letrec}}}}
\setspecialsymbol{let}{{\textbf{{let}}}}
\setspecialsymbol{define}{{\textbf{{define}}}}
\setspecialsymbol{set!}{{\textbf{{set!}}}}
\setspecialsymbol{else}{{\textbf{{else}}}}
\setspecialsymbol{set-car!}{{\textit{set-car!}}}
\setspecialsymbol{set-cdr!}{{\textit{set-cdr!}}}

\setspecialsymbol{and-tag}{{\sf and}}
\setspecialsymbol{or-tag}{{\sf or}}
\setspecialsymbol{var-tag}{{\sf var}}
\setspecialsymbol{if-tag}{{\sf if}}

\def\boldleftparen{{\textbf{(}\!\!\hspace*{-1pt}\textbf{(}}}
\def\boldrightparen{{\textbf{)}\!\!\hspace*{-1pt}\textbf{)}}}



\defschememathescape{$} % $

\def\qqmagicoff{}
\def\qqmagicon{}

%\def\schemedisplayspace{\vspace{0.5cm}}
\def\schemedisplayspace{\vspace{0.2cm}}


%%%% Nominal Biography
% http://www4.in.tum.de/~urbanc/NomBib/nominal.html

% Uncomment to use author-date citation format
\bibpunct();A{},
\let\cite=\citep

\bibliographystyle{abbrvnat}
%\bibliographystyle{plainnat}
%\bibliographystyle{acm}
%\bibliographystyle{plain}

\VerbatimFootnotes

\makeindex


\begin{schemeregion}

\begin{document}

\frontmatter

\title{\thesistitle{}}

\author{\thesisauthor{}}

\thispagestyle{empty}

\mbox{}
\vspace{1in}

%\newcommand{\fstpagefont}[0]
%{\fontspec{Hoefler Text}}

\begin{center}
{ 
%  \fstpagefont{}
  \fontsize{24}{24}
  \noindent \scshape \thesistitle{} \\
}
\vspace{2in}
{\fontsize{16}{16} % \fstpagefont{} 
\scshape \thesisauthor{}}

\vspace{1in}
{ \fontsize{14}{14} % \fstpagefont{}
\scshape
  Submitted to the faculty of the \\
  University Graduate School \\
  in partial fulfillment of the requirements \\
  for the degree \\
  Doctor of Philosophy \\
  in the Department of Computer Science, \\
  Indiana University \\
  \vspace{.5in}
  \thesismonth{}, \thesisyear{}}
\end{center}

\newpage
%\thispagestyle{empty}

\newcommand{\cmember}[1]{

\vspace{1in}
\hfill 
\begin{minipage}{3.5in}
\begin{center}
\rule{3.5in}{1pt}
#1 
\end{center}
\end{minipage}
}

{
  \fontsize{12}{12}
\noindent Accepted by the faculty of the University Graduate School,
Indiana University, in partial fulfillment of the requirements for the
degree Doctor of Philosophy\vspace{.5in}

\thesiscommitteemembers{}

\vfill{}
\noindent Bloomington, Indiana\\
\thesisdate{}
}

\newpage

{
  \fontsize{12}{12}
  \mbox{}
  \vfill

  \hrule

  \begin{center}
    Copyright \copyright{} \thesisyear{}, \thesisauthor{}\\ 
    All rights reserved
    \end{center}
}

%\setlength{\baselineskip}{2\baselineskip}
%\setlength{\parskip}{0\baselineskip}

\titlespacing*{\section}{0pt}{*4}{*1.5}
\titlespacing*{\subsection}{0pt}{*4}{*1.5}
\titlespacing*{\subsubsection}{0pt}{*4}{*1.5}

%\chapter*{Dedication}
%\addcontentsline{toc}{chapter}{Dedication}

\begin{center}

For my parents.
\end{center}
\ \\
\begin{flushright}
\textit{Hold, p'ease!}

---Brian Byrd

\end{flushright}


%\newpage

\chapter*{Acknowledgments}

I first want to acknowledge my mother---without her I wouldn't be here
writing this dissertation.  My father taught me a love of science and
nature, which led to my interest in computers and programming
languages---without him I wouldn't be here, either.  I dedicate my
dissertation to them, with love and admiration.

I also thank my brother Brian, my sister Mary, and their spouses,
Claudia D\'{i}az-Byrd and Donald Stevens.  I can't imagine better
siblings or in-laws.

Renzhong Chen has been part of the Byrd family for twenty years.  I
thank him for his friendship, and for the incredible tour of China that
was a highlight of my time in grad school. I am also grateful for the
hospitality of Renzhong's wife, Lea~Li.

Dan Friedman has been my teacher, mentor, boss, hacking buddy,
coauthor, and friend during my six years at Indiana University.  It
was Dan who introduced me to logic programming, and to his language
that eventually evolved into miniKanren.

I cannot thank Dan without also thanking his wife Mary and the rest of
the Friedmans, who have been a second family to me in Bloomington.  I
am especially grateful to Sa{\tt \{nd|mm\}}i Friedman for hours of
amusement.

My committee members, Dan, Amr, Chris, and Larry, have been
unfailingly cheerful, patient, and supportive.  This is fortunate,
since their intellects might otherwise be intimidating.  I am
especially gratified that all of these scholars are deeply committed
to the art of teaching.  Thank you all!

Oleg Kiselyov taught me an unbelievable amount about logic
programming, especially the benefits of purity, and the dangers of
impure operators like \scheme|conda| and \scheme|condu|.  It was Oleg
who had the incredibly clever idea of implementing a relational
arithmetic system inspired by hardware half-adders and full-adders
(see Chapter~\ref{arithchapter}).  Oleg has also had a tremendous
influence on the development and evolution of miniKanren and its
predecessor, Kanren.

Chung-chieh (Ken) Shan has also greatly influenced the evolution, and
especially the implementation, of miniKanren.  Much of the brevity and
elegance of the core miniKanren implementation in
Chapter~\ref{mkimplchapter} is due to Ken, who designed the critical
\scheme|case-inf| macro.

Some of the ideas, implementation code, and example programs in this
dissertation were first presented, often in a slightly different form,
in {\em The Reasoned Schemer}~\cite{trs}.  Chapter~\ref{arithchapter}
is based on chapters seven and eight of {\em The Reasoned Schemer}
and, to a lesser extent,
\citet{conf/flops/KiselyovBFS08}. Chapters~\ref{akchapter} and
\ref{akimplchapter} are adapted from~\citet{alphamk}.
Chapter~\ref{alphatapchapter} is adapted
from~\citet{DBLP:conf/iclp/NearBF08}.  A paper containing the contents
of Chapter~\ref{fernschapter} and the nestable engines code in
Appendix~\ref{nestable-engines} will be presented at Mitch Wand's
Festschrift.  Many thanks to all of my coauthors.  Please see the
acknowledgments sections of these papers for additional credits.

The first tabling implementation for miniKanren was designed by the
author and Ramana Kumar, and inspired by the Dynamic Reordering of
Alternatives (DRA) approach to
tabling~\cite{dra09,simpleimplementingtabling}.
Ramana implemented the design, with debugging assistance from the
author.  The tabling implementation presented in Chapter~\ref{tablingchapter}
is a slightly modified version of Ramana's second, and improved,
tabling implementation.  Most importantly, the new implementation is
based on streams rather than success and failure continuations, which
means answers are produced in the same order as in the core miniKanren
implementation of Chapter~\ref{mkimplchapter}.

The nominal unifier using triangular substitutions in
section~\ref{triangularsection} is due to Joe Near.  Ramana Kumar
has implemented a faster but very different triangular unifier.

As described in section~\ref{walkexpensive}, Abdulaziz Ghuloum, David
Bender, and Lindsey Kuper have explored which purely functional data
structures are best for representing triangular substitutions.

The \scheme|pmatch| pattern-matching macro in Appendix~\ref{pmatch}
was written by Oleg Kiselyov.  The \scheme|matche| and
\scheme|lambdae| pattern-matching macros in Appendix~\ref{matche} were
originally designed by the author and implemented by Ramana Kumar with
the help of Dan Friedman.  Andy Keep, Michael Adams, and Lindsey Kuper
worked with us to implement an optimized version of \scheme|matche|
and \scheme|lambdae|, which will be presented at the 2009 Scheme
Workshop.

Visits to Bloomington from Christian Urban, Matt Lakin, and Gopal
Gupta greatly aided my research.  I also benefited from the 3rd
International Compulog/ALP Summer School on Logic Programming and
Computational Logic at New Mexico State University, organized by
Enrico Pontelli, Inna Pivkina, and Son Cao Tran.  I enjoyed many
stimulating conversations with visiting scholars Juliana Vizzotto,
Katja Grace, Dave Herman, and Sourav Mukherjee.

Indiana University's PL Wonks lecture series, organized by Roshan
James, has been most stimulating.  I thank Roshan, Michael Adams, Andy
Keep, Jeremiah Willcock, Ron Garcia, Jeremy Siek, Steve Ganz, Larisse
Voufo, and all the other Wonks for many interesting conversations
about programming languages.  The PL Wonks also benefited from special
visits by Jeffrey Siskind and Robby Findler.  I look forward to a
relaxing conversation with Olivier Danvy that does not require either
of us to use our first-responder skills.

For the past eleven semesters I have had the great pleasure of being
the associate instructor for Indiana University's undergraduate (C311)
and graduate (B521) introductory programming languages courses, under
the enthusiastic leadership of Dan Friedman.  The material in this
dissertation was greatly improved by the comments and corrections of
our students.  I am grateful to you all, especially those former
students who have conducted summer research with us: Dave Bender,
Jordan Brown, Adam Foltzer, Adam Hinz, Andy Keep, Jiho Kim, Ramana
Kumar, Lindsey Kuper, Micah Linnemeier, and Joe Near.

Special thanks goes to former C311 student Jeremiah Penery, who
discovered and corrected a subtle error in the definition of
\scheme|logo| from the first printing of {\em The Reasoned Schemer}.

Dan and I have had the great fortune to work with exceptional graduate
and undergraduate associate instructors: David Mack, Alex Platte, Kyle
Blocher, Joe Near, Ramana Kumar, and Lindsey Kuper.  Their hard
work has made C311 and B521 such a success.  Joe and Ramana also read
the final draft of this dissertation, and provided many insightful
comments and corrections---thank you!

I learned a great deal from teaching the honors section of IU's
introductory programming course (C211) under supreme Scheme hacker
Kent Dybvig.  Several years later I had the great pleasure of teaching
C211 with another Scheme master, Aziz Ghuloum, as my associate
instructor.  Every teacher should be so lucky.

I thank Olin Shivers for writing an exemplary dissertation, the
structure and organization of which I shamelessly ripped off.  Once
again I relied on Dorai Sitaram's Scheme typesetting program, S\LaTeX.

Whenever I was floundering in grad school Mitch Wand seemed to
magically appear, pulling me aside to see how I was doing, and
offering much appreciated advice and encouragement.  Although I did
not always follow his advice (to my detriment, I am sure), I will
always be grateful for his support.

In addition to sharing his teaching and programming expertise, Kent
Dybvig also offered invaluable advice about navigating the pitfalls of
grad school.  I didn't heed all of Kent's advice, though I learned to
pay special attention to anything following the catchphrase, ``You'll
be committing academic suicide.''

Lucy Battersby, Rob Henderson, and the rest of the Indiana University
Computer Science Department staff provided expert help and an
outstanding work environment.

The friendly staff of the east-side Bloomington Quiznos, Sunny Bal,
Kae Lunde, Caitlyn Muncy, Alisha Findley, Tylla Carlisle, and Meagan
Perry, kept me rolling in veggie subs and ``om-nom-nom-nom''-worthy
cookies.

Caitlin Coar, Cortney Packett, and Alisha Stout, formerly of Cold
Stone Creamery, supplied me with delicious and nourishing PB\&C
milkshakes.

I wrote most of this dissertation at the east-side Starbucks in
Bloomington, where the generous, hilarious, and slightly-unhinged
baristas provided a tasty setting for extended writing sessions.  Many
thanks to:

\begin{itemize}
\item Amanda Buck for all the fish stories;
\item Andrea Jerabek for always giving me a hard time;
\item Ben Canary for inventing the infamous and delectable ``Christmas Cookie'';
\item Brian Ibison for adoring Olivia Munn;
\item Christina Liwski for never giving me a hard time;
\item Ciera Brannon for not believing in cells (\emph{No cells, no mercy!});
\item Eric ``Big Eric'' Martin for pointing out that UFO over Starbucks;
\item Erin Dobias for being a smart chick;
\item Gabby Baehl for living up to her first name;
\item Megan Traxinger-James for putting up with Eric and Brian;
\item and Phil Wood for his Twitter-powered news reports.
\end{itemize}

\noindent I also thank former baristas Jessica Fugate and Shannon
Pilrose for hanging out with me when I should have been writing.

% Chili's ???  Marnie, Todd, Sara,

I've been fortunate to have made many close friends in Bloomington.
In particular, Aziz Ghuloum, Larisse Voufo, Ron Garcia, Suzanna Crage,
Andy Keep, Lindsey Kuper, Lindsay and Ahmed Hamed, and Anne and Mike
Faber helped keep me sane while I wrote this dissertation.

Marc Muher visited Bloomington for Dig Dug \emph{battle royale}.
Leslie Cuevas helped keep Jennifer Fitzgerald in line, while Ada
Brunstein offered end game encouragement.  As always, my childhood
friends Mike, Daryl, and Bobby helped relieve the tension with the
occasional game.

It has been my honor to know the Miller family for many years, and am
grateful for their friendship.

Cisco Nochera, former director of Camp Greentop, has been my friend
and mentor for almost two decades.

During the spring of 2006 I visited Chile for a week, along with my
brother Brian and my sister-in-law Claudia.  Claudia's parents, Hugo
D\'{i}az and Nancy Zu\~{n}iga de D\'{i}az, warmly welcomed us to their
home in the beautiful countryside on the outskirts of Casablanca,
Chile.  I thank them for their never-ending hospitality and goodwill.

I have had many incredible teachers in my life, but a few stand out.
My absolutely amazing 11$^{th}$ grade Spanish teacher, Tom Rahauser,
taught me to never fear Se\~{n}or Subjunctivo.  Richard Saenz's class
on special relatively completely blew my mind.  Tom Anastasio taught
me LISP, cleverly disguised as C.  Alan Sherman prepared me for
graduate school.  Dan Friedman taught me that you don't understand
your code until it fits on a 3x5 card.

\chapter*{Abstract}

%WILLIAM E. BYRD  \underline{Relational Programming in miniKanren: Techniques, Applications, and Implementations}

\noindent The promise of logic programming is that programs can be
written {\em relationally}, without distinguishing between input and
output arguments.  Relational programs are remarkably flexible---for
example, a relational type-inferencer also performs type checking and
type inhabitation, while a relational theorem prover generates
theorems as well as proofs and can even be used as a simple proof
assistant.

Unfortunately, writing relational programs is difficult, and requires
many interesting and unusual tools and techniques.  For example, a
relational interpreter for a subset of Scheme might use nominal
unification to support variable binding and scope, Constraint Logic
Programming over Finite Domains (CLP(FD)) to implement relational
arithmetic, and tabling to improve termination behavior.

In this dissertation I present {\em miniKanren}, a family of languages
specifically designed for relational programming, and which supports a
variety of relational idioms and techniques.  I show how miniKanren
can be used to write interesting relational programs, including an
extremely flexible lean tableau theorem prover and a novel
constraint-free binary arithmetic system with strong termination
guarantees.  I also present interesting and practical techniques used
to implement miniKanren, including a nominal unifier that uses
triangular rather than idempotent substitutions and a novel
``walk''-based algorithm for variable lookup in triangular
substitutions.

The result of this research is a family of languages that supports a
variety of relational idioms and techniques, making it feasible and
useful to write interesting programs as relations.

% \noindent [check abstracts of my papers]

% \noindent [describe disparity between promise and reality of logic programming]

% \noindent [brief explanation of and motivation for relational programming]

% \noindent [describe the work this dissertation presents]


{
\setcounter{tocdepth}{1}
\setlength{\parskip}{0pt}
\tableofcontents
}

\newpage

\mainmatter

\chapter{Introduction}\label{introchapter}

\section{My Thesis}
\begin{quotation}
\begin{flushright}
\textit{A beginning is a very delicate time.}\\
---Princess Irulan
\end{flushright}
\end{quotation}


\noindent miniKanren supports a variety of relational idioms and
techniques, making it feasible and useful to write interesting
programs as relations.

The promise of logic programming is that programs can be written {\em
  relationally}, without distinguishing between input and output
arguments.  Each relation produces meaningful answers, even when all
of its arguments are unbound logic variables.  Relational programs are
remarkably flexible---for example, a relational type inferencer can
also perform type checking and type inhabitation.  Similarly, a
relational theorem prover can also be used as a proof checker, proof
generator, theorem generator, and even as a primitive proof assistant.

% Unfortunately, writing remarkably flexible relational programs is
% remarkably difficult.  Relational programming requires a variety of
% unusual and advanced tools and techniques: for example, a relational
% interpreter for a subset of Scheme might use nominal unification to
% support variable binding and scope, Constraint Logic Programming over
% Finite Domains (CLP(FD)) to implement relational arithmetic, and
% tabling to improve termination behavior.  In addition, relational
% programmers must give up many standard but non-declarative programming
% constructs used in traditional logic programming, such as Prolog's
% infamous cut operator\footnote{Prolog's cut operator ({\tt !}) prunes
%   the program's search tree, which can greatly improve efficiency and
%   even avoid divergence for infinite search trees.  Unfortunately,
%   careless use of cut can prune nodes that contain answers, resulting
%   in an unsound program.}.  As a result of these difficulties,
% non-trivial Prolog programs are rarely written as relations.  Newer
% languages such as Curry and Mercury attempt to solve this problem by
% using only declative language features.  While these languages offer
% many interesting and useful declarative constructs, 

% potentially diverging when used with logic variables rather than
% ground terms.

% Thus, the promise of
% logic programming remains largely unfulfilled.  What is needed is a
% language for {\em relational programming}, as opposed to logic
% programming or even declarative programming.

Unfortunately, writing remarkably flexible relational programs is
remarkably difficult, and requires a variety of unusual and advanced
tools and techniques.  For example, a relational interpreter for a
subset of Scheme might use nominal unification to support variable
binding and scope, Constraint Logic Programming over Finite Domains
(CLP(FD)) to implement relational arithmetic, and tabling to improve
termination behavior.

This dissertation presents {\em miniKanren}, a family of languages
specifically designed for relational programming, and which supports a
variety of relational idioms and techniques.  We show how miniKanren
can be used to write interesting relational programs, including an
extremely flexible lean tableau theorem prover and a novel
constraint-free binary arithmetic system with strong termination
guarantees.  We also present interesting and practical techniques used
to implement miniKanren, including a nominal unifier that uses
triangular rather than idempotent substitutions and a novel
``walk''-based algorithm for variable lookup in triangular
substitutions.

Chapter~\ref{mkintrochapter} presents the core miniKanren language,
which we then extend with disequality constraints
(Chapter~\ref{diseqchapter}), nominal logic (Chapter~\ref{akchapter}),
tabling (Chapter~\ref{tablingchapter}), and expression-level
divergence avoidance using ferns (Chapter~\ref{fernschapter}).  We
provide implementations of all of these language extensions in
Chapters~\ref{mkimplchapter}, \ref{walkimpl}, \ref{diseqimplchapter},
\ref{akimplchapter}, \ref{tablingimplchapter}, and \ref{fernsimpl}.
Together, these chapters establish the first half of my thesis:
miniKanren supports a variety of relational idioms and techniques.

To illustrate the use of these techniques, we present two
non-trivial miniKanren applications.  The constraint-free relational
arithmetic system of Chapter~\ref{arithchapter} and the theorem prover
of Chapter~\ref{alphatapchapter}
% and the term reducer of Chapter~\ref{reducerchapter} 
establish the second half of my thesis:
it is feasible and useful to write interesting programs as relations
in miniKanren, using these idioms and techniques.

\section{Structure of this Dissertation}

With the exception of two early chapters
(Chapters~\ref{mkintrochapter} and~\ref{divergencechapter}), each
technical chapter in this dissertation is divided into one of three
categories: techniques, applications, or
implementations\footnote{Hence the title of this dissertation:
  \emph{Relational Programming in miniKanren: Techniques,
    Applications, and Implementations}.}.  \emph{Technique chapters}
describe language features and idioms for writing relations, such as
disequality constraints (Chapter~\ref{diseqchapter}) and nominal logic
(Chapter~\ref{akchapter}).  \emph{Application chapters} demonstrate
how to write interesting, non-trivial relations in miniKanren; these
applications demonstrate the use of many of the language forms and
idioms presented in the technique chapters.  \emph{Implementation
  chapters} show how to implement the language extensions presented in
the technique chapters.

At a higher level, the dissertation is divided into six parts, which
are organized by theme:

\begin{itemize}

\item Part~\ref{coremkpart} presents the core miniKanren language,
  which we will extend in the latter parts of the dissertation.
  Chapter~\ref{mkintrochapter} introduces the core language, along
  with a few simple examples, while Chapter~\ref{mkimplchapter}
  presents the implementation of the core language.  These two
  chapters are especially important, since they form the foundation
  for the advanced techniques and implementations that follow.  In
  Chapter~\ref{walkimpl} we optimize the \scheme|walk| algorithm
  presented in Chapter~\ref{mkimplchapter}, which is the heart of
  miniKanren's unifier.  Chapter~\ref{divergencechapter} attempts to
  categorize the many ways miniKanren programs can diverge, and
  describes techniques that can be used to avoid each type of
  divergence.  Avoiding divergence while maintaining declarativeness is
  what makes relational programming so fascinating, yet so
  challenging.  Chapter~\ref{arithchapter} presents a non-trivial
  application of core miniKanren: a constraint-free arithmetic system
  with strong termination guarantees.
  
\item Part~\ref{diseqpart} extends core miniKanren with disequality
  constraints, which allow us to express that two terms are different,
  and can never be unified.  Disequality constraints express a very
  limited form of negation, and can be seen as a very simple kind of
  constraint logic programming.  Chapter~\ref{diseqchapter} describes
  disequality constraints from the perspective of the user, while
  Chapter~\ref{diseqimplchapter} shows how we can use unification in a
  clever way to simply and efficiently implement the constraints.  We
  give special attention to \emph{constraint reification}---the
  process of displaying constraints in a human-friendly manner.

\item Part~\ref{nominallogicpart} extends core miniKanren with
  operators for expressing nominal logic; we call the resulting
  language \alphakanren.  Nominal logic allows us to easily express
  notions of scope and binding, which is useful when writing
  declarative interpreters, type inferencers, and many other relations
  that deal with variables.  Chapter~\ref{akchapter} introduces
  nominal logic, explains \alphakanren's new language constructs, and
  provides a few simple example programs.
  Chapter
  \ref{alphatapchapter} presents a non-trivial application of
  \alphakanren: a relational theorem
  prover.  In Chapter~\ref{akimplchapter} we present our
  implementation of \alphakanren, including two different
  implementations of nominal unification.

\item Part~\ref{tablingpart} adds tabling to our implementation of
  core miniKanren.  Tabling is a form of memoization: the answers
  produced by a tabled relation are ``remembered'' (that is, stored in
  a table), so that subsequent calls to the relation can avoid
  recomputing the answers.  Tabling allows our programs to run more
  efficiently in many cases; more importantly, many programs that
  would otherwise diverge terminate when using tabling.
  Chapter~\ref{tablingchapter} introduces the notion of tabling, and
  explains which programs benefit from tabling.
%   Chapter~\ref{reducerchapter} presents a non-trivial miniKanren
%   application, a relational term reducer, and shows how tabling
%   improves the termination behavior of the reducer.
  Chapter~\ref{tablingimplchapter} presents our streams-based
  implementation of tabling, which demonstrates the advantage of
  embedding miniKanren in a language with higher-order functions.

\item Part~\ref{fernspart} presents a bottom-avoiding data structure
  called a \emph{fern}, and shows how ferns can be used to avoid
  expression-level divergence.  Chapter~\ref{fernschapter} introduces
  the fern data structure and implements a simple, miniKanren-like
  language using ferns.  Chapter~\ref{fernsimpl} presents our
  embedding of ferns in Scheme.

\item Part~\ref{contextpart} provides context and conclusions for the
  work in this dissertation.  Chapter~\ref{relatedworkchapter}
  describes related work, while Chapter~\ref{futureworkchapter}
  proposes future research.  We offer our final conclusions in
  Chapter~\ref{conclusionchapter}.

\end{itemize}

The dissertation also includes four appendices.
Appendix~\ref{helpers} contains several generic helper functions that
could be part of any standard Scheme library.  Appendix~\ref{pmatch}
describes and defines \scheme|pmatch|, a simple pattern matching macro
for Scheme programs.  Appendix~\ref{matche} describes and defines
\scheme|matche| and \scheme|lambdae|, pattern matching macros for
writing concise miniKanren relations.  Appendix~\ref{nestable-engines}
contains our implementation of nestable engines, which are used in our
embedding of ferns.



% For example, the theme of
% Part~\ref{nominallogicpart} is nominal logic;
% Part~\ref{nominallogicpart} contains one technique chapter
% (Chapter~\ref{akchapter}), two application chapters
% (Chapters~\ref{typeinferencerchapter} and
% ~\ref{alphatapchapter}), and one implementation chapter
% (Chapter~\ref{akimplchapter}).


%This dissertation has the following structure:

%\noindent [bullet list roadmap with chunking of chapters, illustrating how the
%chapters fit together]

\section{Relational Programming}

Relational programming is a discipline of logic programming in which
every goal is written as a ``pure'' relation.  Each relation produces
meaningful answers, even when all of its arguments are unbound logic
variables.  For example, Chapter~\ref{arithchapter} presents
\scheme|pluso|, which performs addition over natural numbers.
\mbox{\scheme|(pluso bn1 bn2 bn3)|}\footnote{\scheme|bn1|, 
\scheme|bn2|, and \scheme|bn3| are shorthand for the little-endian binary 
lists representing the numbers 1, 2, and 3---see Chapter~\ref{arithchapter} for details.}
 succeeds, since $1+2=3$---that is,
the triple $(1,2,3)$ is in the ternary addition relation.  We can use
\scheme|pluso| to add two numbers: \mbox{\scheme|(pluso bn1 bn2 z)|}
associates the logic variable \scheme|z| with \scheme|3|.  We can also
subtract numbers using \scheme|pluso|: \mbox{\scheme|(pluso bn1 y bn3)|} 
associates \scheme|y| with 2, since $3-1=2$.  We can even
call \scheme|pluso| with only logic variables: \mbox{\scheme|(pluso x y z)|} 
produces an infinite number of answers in which the natural
numbers associated with \scheme|x|, \scheme|y|, and \scheme|z|
satisfy \scheme|x+y=z|.  For example, one such answer associates
\scheme|x| with 3, \scheme|y| with 4, and \scheme|z| with 7.

To write relational goals, programmers must avoid a variety of
powerful logic programming constructs, such as Prolog's
cut ({\tt !}), {\tt var/1}, and \mbox{{\tt copy\_term/2}} operators.  These
operators inhibit relational programming, since their proper use is
dependent upon the groundness or non-groundness of terms\footnote{A
  term is \emph{ground} if it does not contain unassociated logic
  variables.}.
%, and therefore upon the mode of the relation in which
%they appear.  
Programmers who wish to write relations must avoid these
constructs, and instead use language features compatible with the
relational paradigm.

A critical aspect of relational programming is the desire for
relations to terminate whenever possible.  Writing a goal without mode
restrictions is not very interesting if the goal diverges when passed
one or more fresh variables.  In particular, we desire the
\emph{finite failure} property for our goals---if a goal is asked to
produce an answer, yet no answer exists, that goal should fail in a
finite amount of time.  Although G\"{o}del and Turing showed that it
is impossible to guarantee termination for all goals we might wish to
write, the use of clever data encoding, nominal unification, tabling,
and the derivation of bounds on the maximum size of terms allows a
careful miniKanren programmer to write surprisingly sophisticated
programs that exhibit finite failure.

Our emphasis on both pure relations and finite failure leads to
different design choices than those of more established logic
programming languages such as
Prolog~\cite{ISO:1995:IIIe,ISO:2000:IIIf},
Mercury~\cite{Somogyi95mercury}, and
Curry~\cite{Hanus95curry:a,Hanus06}.  For example, unlike Prolog,
miniKanren uses a complete (interleaving) search strategy by default.
Unlike Mercury, miniKanren uses full unification, required to
implement goals that take only fresh logic variables as their
arguments\footnote{Mercury is statically typed, and requires
  programmers to specify ``mode
  annotations''~\cite{Apt94reasoningabout} indicating whether each
  argument to a goal is an ``input'' (that is, fully ground) or an
  ``output'' (that is, an unassociated logic variable).  Programmers
  also specify whether each goal can produce one, finitely many, or
  infinitely many answers.  Given all this information, the Mercury
  compiler can generate multiple specialized functions that perform
  the work of a single goal.  For example, a ternary goal that
  expresses addition (similar to the \scheme|pluso| function described
  above) might be compiled into separate functions that perform
  addition or subtraction; at runtime, the appropriate function will
  be called depending on which arguments are ground.  In fact,
  compiled Mercury programs do not use logic variables or unification,
  and are therefore extremely efficient.  Unfortunately, this lack of
  unification means it is not possible to write Mercury goals that
  take only ``output'' variables.}.  And our desire for termination
prevents us from adapting Curry's
residuation\footnote{\emph{Residuation}~\cite{hanus:jlp95} suspends
  certain operations on non-ground terms, until those terms become
  ground.  For example, we could use residuation to express addition
  using Scheme's built-in \scheme|+| procedure.  If we try to add
  \scheme|x| and 5, and \scheme|x| is an unassociated logic variable,
  we suspend the addition, and instead try running another goal.
  Hopefully this goal will associate \scheme|x| with a number; when
  that happens, we can perform the addition.  However, if \scheme|x|
  never becomes ground, we will be unable to perform the addition, and
  we will never produce an answer.}.

\section{miniKanren}

This dissertation presents miniKanren, a language designed for
relational programming, along with various language extensions that
add expressive power without sacrificing the ability to write
relations.

% The premise of this work is that:
% \begin{enumerate}
% \item Logic programmers should strive to write their programs as
%   relations, without mode restrictions,
% \end{enumerate}
% and
% \begin{enumerate}
% \item[2.] miniKanren's combination of relational language constructs,
%   along with its easily modifiable, purely functional implementation,
%   makes it uniquely suited for relational programming.
% \end{enumerate}

miniKanren is implemented as an embedding in Scheme, using only a
handful of special forms and functions.  The concise and purely
functional implementation of the core operators makes the language
easy to extend.  miniKanren programmers have access to all of Scheme,
including higher-order functions, first-class continuations, and
Scheme's unique and powerful hygienic macro system.  Having access to
Scheme's features makes it easy for implementers to extend miniKanren;
for example, from a single figure explaining XSB-style OLDT resolution
we were able to design and implement a tabling system for miniKanren
in under a week.

This thesis presents complete Scheme implementations of core
miniKanren and its extensions, including two versions of nominal
unification, a simple constraint system, a streams-based tabling
system, and a minimal implementation of a miniKanren-like language
using the bottom-avoiding fern data-structure.  Our implementation of
core miniKanren is purely functional, and is designed to be easily
modifiable, encouraging readers to experiment with and extend
miniKanren.

\section{Typographical Conventions}

% [META This section is likely to grow, and perhaps turn into a table.]

The code in this dissertation uses the following typographic conventions.  Lexical variables are in \scheme{italic}, forms are in {\bf boldface}, and quoted symbols are in {\sf sans serif}.
Quotes, quasiquotes, and unquotes are suppressed,
and quoted or quasiquoted lists appear with bold parentheses---for example \scheme{`()} and \mbox{\scheme|`(x . ,x)|} are entered as
\qqmagicoff 
\texttt{'()} and \mbox{\texttt{`(x . \!,x)}}, respectively.
\qqmagicon 
By our convention, names of relations end with a superscript $o$---for example \scheme{substo}, which is entered as {\tt substo}.  Relational operators do not follow this convention:  \scheme{==} (entered as {\tt ==}), \scheme{conde} (entered as {\tt conde}), and \scheme{exist}.  Chapter~\ref{diseqchapter} introduces the relational operator \scheme|=/=| (entered as {\tt =/=}), while Chapter~\ref{akchapter} introduces \scheme{fresh}, \scheme|hash| (entered as {\tt hash}), and the term constructor \scheme|tie| (entered as {\tt tie}).  Similarly, \mbox{\scheme|(run5 (q) body)|} and \mbox{\scheme|(run* (q) body)|} are entered as \mbox{{\tt (run 5 (q) body)}} and \mbox{{\tt (run* (q) body)}}, respectively.

\scheme|lambda| is entered as {\tt lambda}.
\scheme|lambdae| from Appendix~\ref{matche} is entered as {\tt lambdae}.
The arithmetic relations \scheme|<=lo| and \scheme|<=o| from
Chapter~\ref{arithchapter} are entered as {\tt <=lo} and {\tt <=o},
respectively. \scheme|occurs-check| from Chapter~\ref{mkimplchapter} 
is entered as {\tt occurs-check}.

% [TODO point out how to enter \scheme|>1o|, \scheme|<o|, 
% =lo, <lo, <=lo, <=o,
% and other
% arithmetic names.  Say something about R5RS compatibility.
% lambda, lambdae, matche,
% ]


%   \section{Where to Find the Source Code}

%   [various implementations, arithmetic system, type inferencer,
%   alphaleanTAP, etc.]

%   \section{Logic Programming and Relations}

%     \subsection{Logic Programming}

%     \noindent [Dan says I should do something like this: for
%     convenience, in the sequel logic programming is just referred to
%     as ``programming'']

%     \subsection{Relations}

%     \subsection{The Problem}

%     \noindent [perhaps give a simple example or two showing difficulty
%     of writing relations: perhaps rembero, demonstrating the need for
%     constraints [or something based on lambda calculus, demonstrating
%     need for nominal logic], and an example showing divergence,
%     demonstrating the need for bounds or constraints]


\part{Core miniKanren}\label{coremkpart}

% Part~\ref{coremkpart} presents the core miniKanren language, which we
% will extend in the latter parts of the dissertation.
% Chapter~\ref{mkintrochapter} introduces the core language, along with
% a few simple examples, while chapter~\ref{mkimplchapter} presents the
% implementation of the core language.  These two chapters are
% especially important, since they form the foundation for the advanced
% techniques and implementations that follow.  In Chapter~\ref{walkimpl}
% we optimize the \scheme|walk| algorithm presented in
% Chapter~\ref{mkimplchapter}, which is the heart of miniKanren's
% unifier.  Chapter~\ref{divergencechapter} attempts to categorize the
% many ways miniKanren programs can diverge, and describes techniques
% that can be used to avoid each type of divergence.  Avoiding
% divergence while maintaining declarativeness is what makes relational
% programming so fascinating, yet so challenging.
% Chapter~\ref{arithchapter} presents a non-trivial application of core
% miniKanren: a constraint-free arithemtic system with strong
% termination guarantees.

\chapter{Introduction to Core miniKanren}\label{mkintrochapter}

This chapter introduces the core miniKanren language, provides several
short example programs, and shows how to translate a simple Scheme
function into a miniKanren relation.

This chapter is organized as follows.  Section~\ref{coremksection}
introduces the core miniKanren language.  In
section~\ref{transappendosection} we show how to translate the
standard Scheme \scheme|append| function into a miniKanren relation.
Section~\ref{impureoperatorssection} describes several ``impure''
operators that, while not part of the pure miniKanren core language,
are useful when trying to model Prolog programs.

\section{Core miniKanren}\label{coremksection}

miniKanren extends Scheme with three operators: \scheme|==|,
\scheme|conde|, and \scheme|exist|. There is also \scheme|run|, which
serves as an interface between Scheme and miniKanren, and whose value
is a list.

\scheme|exist|, which syntactically looks like \scheme|lambda|,
introduces new variables into its scope; \scheme|==| unifies two
values.  Thus

\wspace

\noindent\scheme|(exist (x y z) (== x z) (== 3 y))|

\wspace

\noindent would associate \scheme|x| with \scheme|z| and \scheme|y|
with \scheme|3|.  This, however, is not a legal miniKanren
program---we must wrap a \scheme|run| around the entire expression.

\wspace

\noindent\scheme|(run1 (q) (exist (x y z) (== x z) (== 3 y)))| $\Rightarrow$ \begin{schemeresponsebox}(_.0)\end{schemeresponsebox}

\wspace

\noindent The value returned is a list containing the single value \schemeresult|_.0|;
we say that \schemeresult|_.0| is the \emph{reified value} of the fresh variable \scheme|q|.  \scheme|q| also remains fresh in

\wspace

\noindent\scheme|(run1 (q) (exist (x y) (== x q) (== 3 y)))| $\Rightarrow$ \begin{schemeresponsebox}(_.0)\end{schemeresponsebox}

\wspace

We can get back other values, of course.

\wspace

\begin{schemebox}
(run1 (y)
  (exist (x z) 
    (== x z)
    (== 3 y)))
$\ $
\end{schemebox}
\begin{schemebox}
(run1 (q)
  (exist (x z)
    (== x z)
    (== 3 z)
    (== q x)))
\end{schemebox}
\begin{schemebox}
(run1 (y)
  (exist (x y)
    (== 4 x)
    (== x y))
  (== 3 y))
\end{schemebox}

\wspace

\noindent Each of these examples returns \scheme|`(3)|; in the rightmost example, the \scheme|y| introduced by \scheme|exist| is different from the \scheme|y| introduced by \scheme|run| because the variables are lexically scoped.  \scheme|run| can also return the empty list, indicating that there are no values.

\wspace

\noindent\scheme|(run1 (x) (== 4 3))| $\Rightarrow$ \schemeresult|`()|

\wspace

We use \scheme|conde| to get several values---syntactically,
\scheme|conde| looks like \scheme|cond| but without \scheme|=>|
or \scheme|else|.  For example,

\schemedisplayspace
\begin{schemedisplay}
(run2 (q)
  (exist (x y z)
    (conde
      ((== `(,x ,y ,z ,x) q))
      ((== `(,z ,y ,x ,z) q))))) $\Rightarrow$ 
\end{schemedisplay}
\nspace
\begin{schemeresponse}
((_.0 _.1 _.2 _.0) (_.0 _.1 _.2 _.0))
\end{schemeresponse}

\noindent Although the two \scheme|conde|-clauses are different, the
values returned are identical.  This is because distinct reified fresh
variables are assigned distinct numbers, increasing from left to
right---the numbering starts over again from zero within each value,
which is why the reified value of \scheme|x|
is \schemeresult|_.0| in the first
value but \schemeresult|_.2| in the
second value.

Here is a simpler example using \scheme|conde|.

\schemedisplayspace
\begin{schemedisplay}
(run5 (q)
  (exist (x y z)
    (conde
      ((== 'a x) (== 1 y) (== 'd z))
      ((== 2 y) (== 'b x) (== 'e z))
      ((== 'f z) (== 'c x) (== 3 y)))
    (== `(,x ,y ,z) q))) $\Rightarrow$
\end{schemedisplay}
\nspace
\begin{schemeresponse}
((a 1 d) (b 2 e) (c 3 f))
\end{schemeresponse}

\noindent The superscript \scheme|5| denotes the maximum length of the resultant
list.  If the superscript \scheme|*| is used, then there is no maximum
imposed.  This can easily lead to infinite loops:

\schemedisplayspace
\begin{schemedisplay}
(run* (q)
  (let loop ()
    (conde
      ((== #f q))
      ((== #t q))
      ((loop)))))
\end{schemedisplay}

\noindent Had the \scheme|*| been replaced by a non-negative integer \scheme|n|, then
a list of \scheme|n| alternating \scheme|#f|'s and \scheme|#t|'s would be returned.  
The \scheme|conde| succeeds while associating \scheme|q| with \scheme|#f|, which 
accounts for the first value.  When getting the second value, 
the second \scheme|conde|-clause is tried, and the association made between \scheme|q| and \scheme|#f| is forgotten---we
say that \scheme|q| has been \emph{refreshed}.  In the third 
\scheme|conde|-clause, 
\scheme|q| is refreshed once again.

We now look at several interesting examples that rely on \scheme|anyo|.

\schemedisplayspace
\begin{schemedisplay}
(define anyo 
  (lambda (g)
    (conde
      (g)
      ((anyo g)))))
\end{schemedisplay}

\noindent \scheme|anyo| tries \scheme|g| an unbounded number of times.
Here is our first example using \scheme|anyo|.

\schemedisplayspace
\begin{schemedisplay}
(run* (q)
  (conde
    ((anyo (== #f q)))
    ((== #t q))))
\end{schemedisplay}

\noindent This example does not terminate, because the call to
\scheme|anyo| succeeds an unbounded number of times.
If \scheme|*| is replaced by \scheme|5|, then instead we get
 \mbox{\schemeresult|`(#t #f #f #f #f)|}.
(The user should not be concerned with the order in which values are returned.)

Now consider

\schemedisplayspace
\begin{schemedisplay}
(run10 (q)
  (anyo 
    (conde
      ((== 1 q))
      ((== 2 q))
      ((== 3 q))))) $\Rightarrow$
\end{schemedisplay}
\nspace
\begin{schemeresponse}
(1 2 3 1 2 3 1 2 3 1)
\end{schemeresponse}


\noindent Here the values \scheme|1|, \scheme|2|, and \scheme|3| are interleaved;
our use of \scheme|anyo| ensures that this sequence will be repeated indefinitely.

Here is \scheme|alwayso|,

\wspace

\noindent\mbox{\scheme|(define alwayso (anyo (== #f #f)))|}

\wspace

\noindent along with two \scheme|run| expressions that use it.

\schemedisplayspace
\begin{schemebox}
(run1 (x)
  (== #t x)
  alwayso
  (== #f x))
$\ $
$\ $
\end{schemebox}
\hspace{2cm}
\begin{schemebox}
(run5 (x)
  (conde
    ((== #t x))
    ((== #f x)))
  alwayso
  (== #f x))
\end{schemebox}

\wspace

\noindent The left-hand expression diverges---this is because 
\scheme|alwayso| succeeds an unbounded number of times, 
and because \mbox{\scheme|(== #f x)|} fails each of those times.

The right-hand expression returns a list of five \scheme|#f|'s.  This
is because both \scheme|conde|-clauses are tried, and both succeed.
However, only the second \scheme|conde|-clause contributes to the
values returned. Nothing changes if we swap the two
\scheme|conde|-clauses.  If we change the last expression to
\mbox{\scheme|(== #t x)|}, we instead get a list of five
\scheme|#t|'s.

Even if some \scheme|conde|-clauses loop indefinitely, other
\scheme|conde|-clauses can contribute to the values returned by a
\scheme|run| expression.
% (We are not concerned with
% Scheme expressions looping indefinitely, however.)  
For example,

\schemedisplayspace
\begin{schemedisplay}
(run3 (q)
  (let ((nevero (anyo (== #f #t))))
    (conde
      ((== 1 q))
      (nevero)
      ((conde
         ((== 2 q))
         (nevero)
         ((== 3 q)))))))
\end{schemedisplay}

\noindent returns \schemeresult|`(1 2 3)|; replacing \scheme|run3|
with \scheme|run4| causes divergence, however, since there are only
three values, and since \scheme|nevero| loops indefinitely.

\section{Translating Scheme Code to miniKanren}\label{transappendosection}

In this section we translate the standard Scheme function
\scheme|append| to the equivalent miniKanren relation,
\scheme|appendo|.  \scheme|append| takes two lists as arguments, and
returns the appended list.

\wspace

\noindent\mbox{\scheme|(append '(a b c) '(d e)) => |}\mbox{\schemeresult|`(a b c d e)|}

\wspace

Here is the definition of \mbox{\scheme|append|}.

\schemedisplayspace
\begin{schemedisplay}
(define append
  (lambda (l s)
    (cond
      ((null? l) s)
      (else (cons (car l) (append (cdr l) s))))))
\end{schemedisplay}

Rather than translate the Scheme definition directly to miniKanren, we
will massage the Scheme code to make it closer in spirit to a
miniKanren relation. Only after we have performed several
Scheme-to-Scheme transformations will we translate to
miniKanren\footnote{This approach differs from that of~\cite{trs},
  which translates Scheme functions directly to miniKanren.}.

First we replace the always-true \scheme|else| test with an explicit
\scheme|pair?| test, making the \scheme|cond| clauses
\emph{non-overlapping}\footnote{The concept of non-overlapping clauses
  is revisited in section~\ref{reconsiderrember}.}.

\newpage

%\schemedisplayspace
\begin{schemedisplay}
(define append
  (lambda (l s)
    (cond
      ((null? l) s)
      ((pair? l) (cons (car l) (append (cdr l) s))))))
\end{schemedisplay}

Next we replace \scheme|cond| with the \scheme|pmatch|
pattern-matching macro from Appendix~\ref{pmatch}.  The use of pattern
matching is close in spirit to unification, and lets us easily
translate the code to use \scheme|matche| or \scheme|lambdae| from
Appendix~\ref{matche}.

\schemedisplayspace
\begin{schemedisplay}
(define append
  (lambda (l s)
    (pmatch `(,l ,s)
      (`(() ,s) s)
      (`((,a . ,d) ,s)
       (cons a (append d s))))))
\end{schemedisplay}

We then perform an \emph{unnesting} step reminiscent of the
Continuation-Passing Style (CPS) transformation\footnote{More correctly, the unnested program is
  similar to one in A-Normal Form (ANF)~\cite{essencecompiling}.}
 (see, for example, \citet{eopl3}): we unnest
any nested calls, introducing \scheme|let|-bound variables
where necessary\footnote{Unlike in the CPS transformation we must
  unnest \emph{every} call, even those guaranteed to terminate.
  For example, unnesting \mbox{\scheme|(cons (cons 1 2) 3)|} results
  in \mbox{\scheme|(let ((tmp (cons 1 2))) (cons tmp 3))|}.}.

\schemedisplayspace
\begin{schemedisplay}
(define append
  (lambda (l s)
    (pmatch `(,l ,s)
      (`(() ,s) s)
      (`((,a . ,d) ,s)
       (let ((res (append d s)))
         (cons a res))))))
\end{schemedisplay}

After unnesting, we are ready to translate the Scheme function into a
miniKanren relation.  We add a superscript $o$ to the name, to
indicate the new function is a relation.  We add an ``output''
argument\footnote{When translating a Scheme predicate to a miniKanren
  relation we do not add an ``output'' argument. This is because success
  or failure of a call to the relation is equivalent to the Scheme
  predicate returning \scheme|#t| or \scheme|#f|, respectively.} and
change \scheme|pmatch| to \scheme|matche|.  We add the output argument
to the list of values being matched against by \scheme|matche|, and the
individual patterns.  Any value that would have previously been returned
must now be unified with the \scheme|out| argument, either explicitly
using \scheme|==| or implicitly using pattern matching.  We also
change the \scheme|let| to \scheme|exist| introducing a ``temporary''
logic variable.

\newpage

%\schemedisplayspace
\begin{schemedisplay}
(define appendo
  (lambda (l s out)
    (match-e `(,l ,s ,out)
      (`(() ,s ,s))
      (`((,a . ,d) ,s ,out)
       (exist (res)
         (appendo d s res)
         (== (cons a res) out))))))
\end{schemedisplay}

Since we are matching against all the arguments of \scheme|appendo|,
we can use \scheme|lambdae| rather than \scheme|matche|.
Also, we may wish to replace \mbox{\scheme|(cons a res)|} with
\mbox{\scheme|`(,a . ,res)|} to reflect our use of unification as
pattern matching.

\schemedisplayspace
\begin{schemedisplay}
(define appendo
  (lambda-e (l s out)
    (`(() ,s ,s))
    (`((,a . ,d) ,s ,out)
     (exist (res)
       (appendo d s res)
       (== `(,a . ,res) out)))))
\end{schemedisplay}

If we do not wish to use the \scheme|matche| or \scheme|lambdae|
pattern matching macros, we can rewrite \scheme|appendo| in core
miniKanren.

\schemedisplayspace
\begin{schemedisplay}
(define appendo
  (lambda (l s out)
    (conde
      ((== '() l) (== s out))
      ((exist (a d)
         (== `(,a . ,d) l)
         (exist (res)
           (appendo d s res)
           (== `(,a . ,res) out)))))))
\end{schemedisplay}

Of course we can use the \scheme|appendo| relation to append two
lists.

\noindent\mbox{\scheme|(run* (q) (appendo '(a b c) '(d e) q)) => |}\mbox{\schemeresult|`((a b c d e))|}

\noindent But we can also find all pairs of lists that, when
appended, produce \mbox{\scheme|'(a b c d e)|}.

\schemedisplayspace
\begin{schemedisplay}
(run6 (q)
  (exist (l s)
    (appendo l s '(a b c d e))
    (== `(,l ,s) q))) $\Rightarrow$ 
\end{schemedisplay}
\nspace
\begin{schemeresponse}
((() (a b c d e))
 ((a) (b c d e))
 ((a b) (c d e))
 ((a b c) (d e))
 ((a b c d) (e))
 ((a b c d e) ()))
\end{schemeresponse}

\noindent Unfortunately, replacing \scheme|run6| with \scheme|run7|
results in divergence, for reasons explained in
Chapter~\ref{divergencechapter}.  We can avoid this problem if we swap
the last two lines of \scheme|appendo|.

\schemedisplayspace
\begin{schemedisplay}
(define appendo
  (lambda (l s out)
    (conde
      ((== '() l) (== s out))
      ((exist (a d)
         (== `(,a . ,d) l)
         (exist (res)
           (== `(,a . ,res) out)
           (appendo d s res)))))))
\end{schemedisplay}

\noindent This final version of \scheme|appendo| illustrates an
important principle: unifications should always come before recursive
calls, or calls to other ``serious'' relations.

%%% From diseq chapter

% Here is \mbox{\scheme|rember|}

% \schemedisplayspace
% \begin{schemedisplay}
% (define rember
%   (lambda (x ls)
%     (cond
%       ((null? ls) '())
%       ((eq? (car ls) x) (cdr ls))
%       (else (cons (car ls) (rember x (cdr ls)))))))
% \end{schemedisplay}

% To translate \mbox{\scheme|rember|} into the miniKanren relation
% \mbox{\scheme|rembero|} we add a third argument \mbox{\scheme|out|},
% change \mbox{\scheme|cond|} to \mbox{\scheme|conde|}, and replace uses
% of \mbox{\scheme|null?|}, \mbox{\scheme|eq?|}, \mbox{\scheme|cons|},
% \mbox{\scheme|car|}, and \mbox{\scheme|cdr|} with calls to
% \mbox{\scheme|==|}.  We also unnest the recursive call, using a
% temporary variable \mbox{\scheme|res|} to hold the ``output'' value of
% the recursive call.
% \newpage
% \schemedisplayspace
% \begin{schemedisplay}
% (define rembero
%   (lambda (x ls out)
%     (conde
%       ((== '() ls) (== '() out))
%       ((exist (a d)
%          (== `(,a . ,d) ls)
%          (== a x)
%          (== d out)))
%       ((exist (a d res)
%          (== `(,a . ,d) ls)
%          (== `(,a . ,res) out)
%          (rembero x d res))))))
% \end{schemedisplay}



\section{Impure Operators}\label{impureoperatorssection}

In this section we include several \emph{impure} operators that appear
in earlier work on miniKanren, notably~\citet{trs} and
\citet{DBLP:conf/iclp/NearBF08}: \scheme|project|, \scheme|conda|,
\scheme|condu|, \scheme|onceo|, and \scheme|copy-termo|.  These
operators are not considered part of core miniKanren, and are
inherently non-relational since they may not work correctly for every
goal ordering of a program; also, it is not legal to pass only fresh
variables to some of these operators, namely \scheme|onceo| and
\scheme|copy-termo|.  As a result we only use these operators to
demonstrate impure Prolog-like features, for example in
Chapter~\ref{alphatapchapter} during translation of the \leantapsp
theorem prover from Prolog to miniKanren.  Importantly, the final
version of the translated prover does not use any impure operators.

\scheme|project| can be used to access the values associated with
logic variables.  For example, the expression

\schemedisplayspace
\begin{schemedisplay}
(run* (q)
  (exist (x)
    (== 5 x)
    (== (* x x) q)))
\end{schemedisplay}

\noindent has no value, since Scheme's multiplication function
operates only on numbers, not logic variables associated with numbers.
We can solve this problem by projecting \scheme|x|: within the body of
the \scheme|project| form, \scheme|x| is a lexical variable bound to
5.

\schemedisplayspace
\begin{schemedisplay}
(run* (q)
  (exist (x)
    (== 5 x)
    (project (x)
      (== (* x x) q)))) $\Rightarrow$ 
\end{schemedisplay}
\nspace
\begin{schemeresponse}
(25)
\end{schemeresponse}

\noindent Unfortunately, the expression

\schemedisplayspace
\begin{schemedisplay}
(run* (q)
  (exist (x)
    (project (x)
      (== (* x x) q))
    (== 5 x)))
\end{schemedisplay}

\noindent has no value, since \scheme|x| is unassociated when
\mbox{\scheme|(* x x)|} is evaluated.  This example demonstrates that
\scheme|project| is not a relational operator\footnote{We explore a
  relational approach to arithmetic in Chapter~\ref{arithchapter}.}.

\scheme|conda| and \scheme|condu| are used to prune a program's search
tree, and can be used in place of Prolog's \emph{cut} ({\tt !})\footnote{More specifically, 
\scheme|conda| corresponds to a \emph{soft-cut}~\cite{ClocksinClause}, while \scheme|condu|
  corresponds to Mercury's \emph{committed-choice}~\cite{Fergus-Henderson:1996uq,Naish:1995pr}.}.
The examples from chapter 10 of \emph{The Reasoned Schemer}~\cite{trs}
demonstrate uses of \scheme|conda| and \scheme|condu|, and
the pitfalls that await the unsuspecting programmer.

\scheme|conda| and \scheme|condu| differ from \scheme|conde| in that
at most one clause can succeed.  Furthermore, the clauses are tried in
order, from top to bottom.  Also, the first goal in each clause is
treated specially, as a ``test'' goal that determines whether to
commit to that clause; in this way, \scheme|conda| and \scheme|condu|
are reminiscent of \scheme|cond|.

For example,

\schemedisplayspace
\begin{schemedisplay}
(run* (x)
  (conda
    ((== 'olive x))
    ((== 'oil x))))
\end{schemedisplay}

\noindent returns \scheme|'(olive)| since \scheme|conda| commits to
the first clause when \mbox{\scheme|(== 'olive x)|} succeeds.
However,

\schemedisplayspace
\begin{schemedisplay}
(run* (x)
  (conda
    ((== 'virgin x) (== #t #f))
    ((== 'olive x))
    ((== 'oil x))))
\end{schemedisplay}

\noindent returns \scheme|'()| since \mbox{\scheme|(== #t #f)|} fails,
and since \scheme|conda| committed to the first clause once
\mbox{\scheme|(== 'virgin x)|} succeeded.
The expression

\schemedisplayspace
\begin{schemedisplay}
(run* (q)
  (conda
    ((== #t #f))
    (alwayso))
  (== #t q))
\end{schemedisplay}

\noindent diverges.  The ``test'' goal for the first clause,
\mbox{\scheme|(#t #f)|}, fails.  The test goal for the second clause,
\scheme|alwayso|, succeeds; therefore \scheme|conda| commits to this
clause.  Since \scheme|alwayso| can succeed an unbounded number of
times, the \scheme|run*| expression diverges.

However, if we replace \scheme|conda| with \scheme|condu|, the
resulting expression

\schemedisplayspace
\begin{schemedisplay}
(run* (q)
  (condu
    ((== #t #f))
    (alwayso))
  (== #t q))
\end{schemedisplay}

\noindent returns \scheme|'(#t)|.  This is because the test goal of a
\scheme|condu| clause can succeed at most once, which is the only
difference between \scheme|conda| and \scheme|condu|.

The next impure operator, \scheme|onceo|, can be trivially defined
using \scheme|condu|.  \scheme|onceo| takes a single argument, which
must be a goal; \scheme|onceo| ensures that when the goal is run it
produces at most a single answer.

\schemedisplayspace
\begin{schemedisplay}
(define onceo
  (lambda (g)
    (condu
      (g))))
\end{schemedisplay}

\noindent For example, \mbox{\scheme|(run* (q) (onceo alwayso))|} produces \scheme|'(_.0)|.


\scheme|copy-termo| creates a copy of its first argument, consistently
replacing unassociated logic variables with new variables; the
resulting copy is then associated with the second argument.

\schemedisplayspace
\begin{schemedisplay}
(run* (q)
  (exist (w x y z)
    (== `(a ,x 5 ,y ,x) w)
    (copy-termo w z)
    (== `(,w ,z) q))) $\Rightarrow$
\end{schemedisplay}
\nspace
\begin{schemeresponse}
(((a _.0 5 _.1 _.0) (a _.2 5 _.3 _.2)))
\end{schemeresponse}

\noindent A major theme of Chapter~\ref{alphatapchapter} is how
\scheme|copy-termo| can be replaced with a relational combination of
nominal unification and tagging, at least in certain cases.


\chapter{Implementation I:  Core miniKanren}\label{mkimplchapter}

In this chapter we present the implementation of the core miniKanren
operators described in Chapter~\ref{mkintrochapter}.  Later chapters
describe additions or modifications to this core implementation; unless
otherwise stated, these later chapters only present the definitions
that differ from those of the core implementation.

This chapter is organized as follows.  In section~\ref{mkunif} we
describe our representation of variables and substitutions, and define
the \scheme|unify| function, which uses the \scheme|walk| function to
look up variables in a triangular substitution.
Section~\ref{mkreification} presents our reification algorithm, which
converts miniKanren terms into regular Scheme values without logic
variables.  Finally, in section~\ref{goalconstructors}, we discuss
miniKanren goals, which map substitutions to (potentially infinite)
streams of substitutions.  We then define the core miniKanren goal
constructors \scheme|==|, \scheme|exist|, and \scheme|conde|, along
with the interface operator \scheme|run|.

% miniKanren's implementation comprises three kinds of operators:
% functions such as \scheme{unify} and \scheme{reify}, which take
% substitutions explicitly; goal constructors \scheme{==},
% \scheme{conde}, and \scheme{exist}, which take substitutions
% implicitly; and the interface operator \scheme{run}.  We represent
% substitutions as association lists associating variables with values.


%\section{Unification}
\section{Variables, Substitutions, and Unification}\label{mkunif}

We represent logic variables as vectors of length one\footnote{R6RS
  Scheme supports records, which arguably provide a better abstraction
  for logic variables.  We use vectors for compatibility with R5RS
  Scheme---one consequence is that vectors should not appear in
  arguments passed to \scheme|unify|.}.


\schemedisplayspace
\begin{schemedisplay}
(define-syntax var
  (syntax-rules ()
    ((_ x) (vector x))))

(define-syntax var?
  (syntax-rules ()
    ((_ x) (vector? x))))
\end{schemedisplay}

\noindent The single argument to the \scheme|var| constructor is a symbol
representing the name of the variable\footnote{This name is useful for
  debugging.  More importantly, we must ensure that the vectors
  created with \scheme|var| are non-empty. This is because we use
  Scheme's \scheme|eq?| test to distinguish between variables, and
  \scheme|eq?| is not guaranteed to distinguish between two non-empty
  vectors.}.

A \emph{substitution} \scheme|s| is a mapping between logic
variables and values (also called \emph{terms}).  We represent a
substitution as an \emph{association list}, which is a list of pairs
associating vectors to values; we construct an empty substitution using \scheme|empty-s|

\schemedisplayspace
\begin{schemedisplay}
(define empty-s '())
\end{schemedisplay}

\noindent and extend an existing substitution \scheme|s| with a new
association between a variable \scheme|x| and a value \scheme|v| using
\scheme|ext-s-no-check|

\schemedisplayspace
\begin{schemedisplay}
(define ext-s-no-check (lambda (x v s) (cons `(,x . ,v) s)))
\end{schemedisplay}

\noindent If \scheme|x|, \scheme|y|, and \scheme|z| are logic
variables constructed using \scheme|var|, then the association list
\mbox{\scheme|`((,x . 5) (,y . #t))|} represents a substitution that
associates \scheme|x| with \scheme|5|, \scheme|y| with \scheme|#t|,
and leaves \scheme|z| unassociated.

The right-hand-side (\emph{rhs}) of an association may itself be a
logic variable.  In the substitution \mbox{\scheme|`((,y . 5) (,x
  . ,y))|}, \scheme|x| is associated with \scheme|y|, which in turn is
associated with \scheme|5|.  Thus, both \scheme|x| and \scheme|y| are
associated with \scheme|5|.  This representation is known as a
``triangular'' substitution, as opposed to the more common
``idempotent'' representation\footnote{In an \emph{idempotent}
  substitution, a variable that appears on the left-hand-side
  of an association never appears on the rhs.} of \mbox{\scheme|`((,y . 5) (,x . 5))|}.  
(See \citet{FBaade01} for more on substitutions.)
One advantage of triangular substitutions is that they can be easily
extended using \scheme|cons|, without side-effecting or rebuilding the
substitution.  This lack of side-effects permits sharing of
substitutions, while substitution extension remains a constant-time
operation.  This sharing, in turn, gives us backtracking for free---we
just ``forget'' irrelevant associations by using an older version of
the substitution, which is always a suffix of the current
substitution.

Triangular substitution representation is well-suited for functional
implementations of logic programming, since it allows sharing of
substitutions.  Unfortunately, there are several significant
disadvantages to the triangular representation.  The major
disadvantage is that variable lookup is both more complicated and more
expensive\footnote{In Chapter~\ref{walkimpl} we will explore several
  ways to improve the efficiency of variable lookup when using
  triangular substitutions.} than with idempotent substitutions.
With idempotent substitutions, variable lookup can be defined as
follows,
where \scheme|rhs|\footnote{\scheme|rhs| is just defined to
  be \scheme|cdr|.} returns the right-hand-side of an association.

\newpage

%\schemedisplayspace
\begin{schemedisplay}
(define lookup
  (lambda (v s)
    (cond
      ((var? v)
       (let ((a (assq v s)))
         (cond
           (a (rhs a))
           (else v))))
      (else v))))
\end{schemedisplay}

\noindent 
If \scheme|v| is an unassociated variable, or a non-variable term,
\scheme|lookup|$\footnote{For fans of syntactic sugar,
this definition can be shortened using \scheme|cond|'s arrow notation.

\begin{schemedisplay}
(define lookup
  (lambda (v s)
    (cond
      ((and (var? v) (assq v s)) => rhs)
      (else v))))
\end{schemedisplay}}$ just returns \scheme|v|.

When looking up a variable in a triangular substitution, we must
instead use the more complicated \scheme|walk| function.

\schemedisplayspace
\begin{schemedisplay}
(define walk
  (lambda (v s)
    (cond
      ((var? v)
       (let ((a (assq v s)))
         (cond
           (a (walk (rhs a) s))
           (else v))))
      (else v))))
\end{schemedisplay}

If, when walking a variable \scheme|x| in a substitution \scheme|s|,
we find that \scheme|x| is bound to another variable \scheme|y|, we
must then walk \scheme|y| in the original substitution \scheme|s|.
\scheme|walk| is therefore not primitive recursive~\cite{Kleene:meta}---in fact, 
\scheme|walk| can diverge if used
on a substitution containing a circularity; for example, when walking
\scheme|x| in either the substitution \mbox{\scheme|`((,x . ,x))|} or
\mbox{\scheme|`((,y . ,x) (,x . ,y))|}.  miniKanren's unification
function, \scheme|unify|, ensures that these kinds of circularities
are never introduced into a substitution.  In addition, \scheme|unify|
prohibits circularities of the form \mbox{\scheme|`((,x . (,x)))|}
from being added to the substitution.  Although this circularity will
not cause \scheme|walk| to diverge, it can cause divergence during
reification (described in section~\ref{mkreification}).  To prevent
circularities from being introduced, we extend the substitution using
\scheme|ext-s| rather than \scheme|ext-s-no-check|.

\schemedisplayspace
\begin{schemedisplay}
(define ext-s
  (lambda (x v s)                                   
    (cond
      ((occurs-check x v s) #f)
      (else (ext-s-no-check x v s)))))

(define occurs-check
  (lambda (x v s)
    (let ((v (walk v s)))
      (cond
        ((var? v) (eq? v x))
        ((pair? v) (or (occurs-check x (car v) s) (occurs-check x (cdr v) s)))
        (else #f)))))
\end{schemedisplay}

\noindent \scheme|ext-s| calls the \scheme|occurs-check| predicate, which
returns \scheme|#t| if adding an association between \scheme|x| and
\scheme|v| would introduce a circularity.  If so, \scheme|ext-s|
returns \scheme|#f| instead of an extended substitution, indicating
that unification has failed.

\scheme|unify| unifies two terms \scheme|u| and \scheme|v| with
respect to a substitution \scheme|s|, returning a (potentially
extended) substitution if unification succeeds, and returning
\scheme|#f| if unification fails or would introduce a
circularity\footnote{Observe that \scheme|unify| calls
\scheme|ext-s-no-check| rather than \scheme|ext-s| if \scheme|u| and
\scheme|v| are distinct unassociated variables, thereby avoiding an
unnecessary call to \scheme|walk| from inside
\scheme|occurs-check|.}.

\schemedisplayspace
\begin{schemedisplay}
(define unify
  (lambda (u v s)
    (let ((u (walk u s))
          (v (walk v s)))
      (cond
        ((eq? u v) s)
        ((var? u) 
         (cond
           ((var? v) (ext-s-no-check u v s))
           (else (ext-s u v s))))
        ((var? v) (ext-s v u s))
        ((and (pair? u) (pair? v))
         (let ((s (unify (car u) (car v) s)))
           (and s (unify (cdr u) (cdr v) s))))
        ((equal? u v) s)
        (else #f)))))
\end{schemedisplay}

The call to \scheme|occurs-check| from within \scheme|ext-s| is
potentially expensive, since it must perform a complete tree walk on
its second argument.  Therefore, we also define \scheme|unify-no-check|, which
performs \emph{unsound} unification but is more efficient than
\scheme|unify|\footnote{\citet{occurcheck} point out that, in
  practice, omission of the occurs check is usually not a problem.
  However, the type inferencer presented in
  section~\ref{aktypeinf} requires sound unification to
  prevent self-application from typechecking.}.

\newpage

%\schemedisplayspace
\begin{schemedisplay}
(define unify-no-check
  (lambda (u v s)
    (let ((u (walk u s))
          (v (walk v s)))
      (cond
        ((eq? u v) s)
        ((var? u) (ext-s-no-check u v s))
        ((var? v) (ext-s-no-check v u s))
        ((and (pair? u) (pair? v))
         (let ((s (unify-no-check (car u) (car v) s)))
           (and s (unify-no-check (cdr u) (cdr v) s))))
        ((equal? u v) s)
        (else #f)))))                                                    
\end{schemedisplay}


% \scheme{unify} is based on the triangular model of substitutions (See
% Baader and Snyder \cite{FBaade01}, for
% example.). Vectors should not occur in arguments passed to \scheme{unify},
% since we represent variables as vectors.


\section{Reification}\label{mkreification}
\enlargethispage{1em}

\emph{Reification} is the process of turning a miniKanren term into a
Scheme value that does not contain logic variables.
The \scheme{reify} function takes a substitution \scheme|s| and an arbitrary value \scheme|v|,
perhaps containing variables, and returns the reified value of \scheme|v|. 

\schemedisplayspace
\begin{schemedisplay}
(define reify
  (lambda (v s)
    (let ((v (walk* v s)))
      (walk* v (reify-s v empty-s)))))
\end{schemedisplay}

\noindent For example, \mbox{\scheme|(reify `(5 ,x (#t ,y ,x) ,z) empty-s)|}
returns \mbox{\scheme|`(5 _.0 (#t _.1 _.0) _.2)|}.

% \scheme{reify} first uses \scheme{walk*}
% to apply the substitution to a value and then methodically replaces
% any variables with reified names.

\scheme|reify| uses \scheme|walk*| to deeply walk a term with respect
to a substitution.  If \scheme|s| is the substitution
\mbox{\scheme|`((,z . 6) (,y . 5) (,x . (,y ,z)))|}, then
\mbox{\scheme|(walk x s)|} returns \mbox{\scheme|`(,y ,z)|} while
\mbox{\scheme|(walk* x s)|} returns \mbox{\scheme|`(5 6)|}\footnote{If \scheme|s| is idempotent, \scheme|walk*| is equivalent to \scheme|walk|.}.

\schemedisplayspace
\begin{schemedisplay}
(define walk*
  (lambda (v s)
    (let ((v (walk v s)))
      (cond
        ((var? v) v)
        ((pair? v) (cons (walk* (car v) s) (walk* (cdr v) s)))
        (else v)))))
\end{schemedisplay}

\scheme|reify| also calls \scheme|reify-s|, which is the heart of the
reification algorithm. 

\schemedisplayspace
\begin{schemedisplay}
(define reify-s
  (lambda (v s)
    (let ((v (walk v s)))
      (cond
        ((var? v) (ext-s v (reify-name (length s)) s))
        ((pair? v) (reify-s (cdr v) (reify-s (car v) s)))
        (else s)))))
\end{schemedisplay}

\noindent \mbox{\scheme|reify-s|} takes a \scheme|walk*|ed term
as its first argument; its second argument starts out as
\scheme|empty-s|.  The result of invoking \scheme|reify-s| is a
\emph{reified name} substitution, associating logic variables to
distinct symbols of the form $\__{_{n}}$.

\scheme|reify-s| in turn relies on \scheme|reify-name| to produce the actual symbol.

\schemedisplayspace
\begin{schemedisplay}
(define reify-name
  (lambda (n)
    (string->symbol (string-append "_." (number->string n)))))
\end{schemedisplay}

\section{Goals and Goal Constructors}\label{goalconstructors}

A goal \scheme{g} is a function that maps a substitution~\scheme{s} to
an ordered sequence of zero or more values---these values are almost
always substitutions.  (For clarity, we notate \scheme{lambda} as
\scheme{lambdag@} when creating such a function~\scheme{g}.)  Because
the sequence of values may be infinite, we represent it not as a list
but as a special kind of stream, \mbox{\scheme|a-inf|}.

Such streams contain either zero, one, or more values
\cite{backtracking,CombinatorsforLP%,hinze2000,Wadler85,conf/icfp/WandV04
}.  We use
\mbox{\scheme|(mzero)|} to represent the empty stream of
values. If \scheme{a} is a value, then
\mbox{\scheme|(unit a)|} represents the stream containing
just \scheme{a}.  To represent a non-empty stream
we use \mbox{\scheme|(choice a f)|}, where \scheme{a}
is the first value in the stream, and where \scheme{f} is a
function of zero arguments.  (For clarity, we notate \scheme{lambda}
as \scheme{lambdaf@} when creating such a function~\scheme{f}.)
Invoking the function \scheme{f} produces the remainder of the
stream.
\mbox{\scheme|(unit a)|} can be represented as
\mbox{\scheme|(choice a (lambdaf@ () (mzero)))|}, but the
\scheme{unit} constructor avoids the cost of building and taking apart
pairs and invoking functions, since many goals return 
only singleton streams.
To represent an incomplete stream, we create an \scheme{f} using \mbox{\scheme|(inc e)|}, 
where \scheme{e} is an \emph{expression} that evaluates to an \mbox{\scheme|a-inf|}.

\schemedisplayspace
\begin{schemedisplay}
(define-syntax mzero 
  (syntax-rules () 
    ((_) #f)))

(define-syntax unit
  (syntax-rules ()
    ((_ a) a)))

(define-syntax choice
  (syntax-rules ()
    ((_ a f) (cons a f))))

(define-syntax inc
  (syntax-rules ()
    ((_ e) (lambdaf@ () e))))
\end{schemedisplay}

To ensure that streams produced by these four \mbox{\scheme|a-inf|}
constructors can be distinguished, 
we assume that a singleton \mbox{\scheme|a-inf|} is
never \schemeresult{#f}, a function, or a pair whose \scheme{cdr} is a
function.  To discriminate among these four cases, we define
\mbox{\scheme|case-inf|}.

\schemedisplayspace
\begin{schemedisplay}
(define-syntax case-inf
  (syntax-rules ()
    ((_ e (() e0) ((f^) e1) ((a^) e2) ((a f) e3))
     (let ((a-inf e))
       (cond
         ((not a-inf) e0)
         ((procedure? a-inf)  (let ((f^ a-inf)) e1))
         ((and (pair? a-inf) (procedure? (cdr a-inf)))
          (let ((a (car a-inf)) (f (cdr a-inf))) e3))
         (else (let ((a^ a-inf)) e2)))))))
\end{schemedisplay}

The simplest goal constructor is \scheme{==}, which returns either a
singleton stream or an empty stream, depending on whether the
arguments unify with the implicit substitution.  As with the other
goal constructors, \scheme{==} always expands to a goal, even if an
argument diverges.  % We avoid the use of \scheme{unit} and
% \scheme{mzero} in the definition of \scheme{==}, since \scheme{unify}
% returns either a substitution (a singleton
% stream) or \scheme{#f} (our representation of the empty stream).

\schemedisplayspace
\begin{schemedisplay}
(define-syntax ==
  (syntax-rules ()
    ((_ u v)
     (lambdag@ (a)
       (cond
         ((unify u v a) => (lambda (a) (unit a)))
         (else (mzero)))))))
\end{schemedisplay}

We can also define \scheme|==-no-check|, which performs unsound
unification without the occurs check.

\schemedisplayspace
\begin{schemedisplay}
(define-syntax ==-no-check
  (syntax-rules ()
    ((_ u v) 
     (lambdag@ (a)
       (cond
         ((unify-no-check u v a) => (lambda (a) (unit a)))
         (else (mzero)))))))
\end{schemedisplay}

\scheme{conde} is a goal constructor that combines successive
\scheme{conde}-clauses using \scheme{mplus*}.  To avoid unwanted
divergence, we treat the \scheme{conde}-clauses as a single
\scheme{inc} stream.  Also, we use the same implicit substitution
for each \scheme{conde}-clause.  \scheme{mplus*} relies on
\scheme{mplus}, which takes an \scheme{a-inf} and an \scheme{f}
and combines them (a kind of \scheme{append}).  Using
\scheme{inc}, however, allows an argument to \emph{become} a
stream, thus leading to a relative fairness because all of the
stream values will be interleaved.  

\schemedisplayspace
\begin{schemedisplay}
(define-syntax conde
  (syntax-rules ()
    ((_ (g0 g ...) (g1 g^ ...) ...)
     (lambdag@ (a) 
       (inc 
         (mplus* (bind* (g0 a) g ...) (bind* (g1 a) g^ ...) ...))))))
\end{schemedisplay}

\begin{schemedisplay}
(define-syntax mplus*
  (syntax-rules ()
    ((_ e) e)
    ((_ e0 e ...) (mplus e0 (lambdaf@ () (mplus* e ...))))))

(define mplus
  (lambda (a-inf f)
    (case-inf a-inf
      (() (f))
      ((f^) (inc (mplus (f) f^)))
      ((a) (choice a f))
      ((a f^) (choice a (lambdaf@ () (mplus (f) f^)))))))
\end{schemedisplay}
\vspace{4.3pt}
\noindent If the body of \scheme{conde} were just the \scheme{mplus*}
expression, then the \scheme{inc} clauses of \scheme{mplus},
\scheme{bind}, and \scheme{take} (defined below) would never be
reached, and there would be no interleaving of values.

\scheme{exist} is a goal
constructor that first lexically binds its variables (created by
\scheme{var}) and then, using \scheme{bind*}, combines successive goals.
\scheme{bind*} is short-circuiting: since the empty stream
\mbox{\scheme|(mzero)|} is represented by \mbox{\scheme|#f|}, any
failed goal causes \scheme{bind*} to immediately return \scheme{#f}.
\scheme{bind*} relies on \scheme{bind} \cite{moggi91notions,Wadler92},
which applies the goal \scheme{g} to each element in 
\scheme{a-inf}.  These \scheme{a-inf}'s are then merged together with
\scheme{mplus} yielding an \scheme{a-inf}.  (\scheme{bind} is
similar to Lisp's \scheme{mapcan}, with the arguments reversed.)

\schemedisplayspace
\begin{schemedisplay}
(define-syntax exist
  (syntax-rules ()
    ((_ (x ...) g0 g ...)
     (lambdag@ (a)
       (inc
         (let ((x (var 'x)) ...)
           (bind* (g0 a) g ...)))))))
\end{schemedisplay}

\begin{schemedisplay}
(define-syntax bind*
  (syntax-rules ()
    ((_ e) e)
    ((_ e g0 g ...) (bind* (bind e g0) g ...))))

(define bind
  (lambda (a-inf g)
    (case-inf a-inf
      (() (mzero))
      ((f) (inc (bind (f) g)))
      ((a) (g a))
      ((a f) (mplus (g a) (lambdaf@ () (bind (f) g)))))))
\end{schemedisplay}

To minimize heap allocation we create a single
\scheme{lambdag@} closure for each goal constructor,
and we define \scheme{bind*} and \scheme{mplus*} to
manage sequences, not lists, of goal-like expressions.  

\scheme{run}, and therefore \scheme{take}, converts an \scheme{f} to a
list.  

\schemedisplayspace
\begin{schemedisplay}
(define-syntax run
  (syntax-rules ()
    ((_ n (x) g0 g ...)
     (take n
       (lambdaf@ ()
         ((exist (x) g0 g ... 
            (lambdag@ (a)
              (cons (reify x a) '())))
          empty-s))))))

(define take
  (lambda (n f)
    (if (and n (zero? n)) 
      '()
      (case-inf (f)
        (() '())
        ((f) (take n f))
        ((a) a)
        ((a f) (cons (car a) (take (and n (- n 1)) f)))))))
\end{schemedisplay}

\noindent
We wrap the result of \mbox{\scheme|(reify x s)|} in a list 
so that the \scheme{case-inf} in \scheme{take} can distinguish a
singleton \scheme{a-inf} from the other three \scheme{a-inf} types.
We could simplify \scheme{run} by using \scheme{var} to create
the unassociated variable \scheme{x}, but we prefer that \scheme{exist} be the
only operator that calls \scheme{var}\footnote{This becomes important in 
Chapter~\ref{walkimpl}, when we redefine the way \scheme|exist| uses \scheme|var|.}.
If the first argument to \scheme{take} is \scheme{#f}, we get the
behavior of \scheme{run*}.  It is trivial to write a read-eval-print
loop that uses \scheme{run*}'s interface by redefining \scheme{take}.

\section{Impure Operators}
\enlargethispage{1em}

We conclude this chapter by defining the impure operators introduced
in section~\ref{impureoperatorssection}: \scheme{project}, which can
be used to access the values of variables, \scheme{conda} and
\scheme{condu}, which can be used to prune the search tree of a
program, and \scheme|copy-termo|, which copies a term, consistently
replacing unassociated logic variables with new variables.

\scheme|project| applies the implicit substitution to zero or more
lexical variables, rebinds those variables to the values returned, and
then evaluates the goal expressions in its body.  The body of a
\scheme|project| typically includes at least one \scheme|begin|
expression---any expression is a goal expression if its value is a
miniKanren goal.

\schemedisplayspace
\begin{schemedisplay}
(define-syntax project 
  (syntax-rules ()                                                
    ((_ (x ...) g g* ...)  
     (lambdag@ (s)
       (let ((x (walk* x s)) ...)
         ((exist () g g* ...) s))))))
\end{schemedisplay}


\scheme|copy-termo| creates a copy of its first argument, consistently
replacing unassociated variables with new logic variables in the copy.
The term \scheme|u| is projected before copying, to avoid accidentally
replacing associated variables with new variables.

\schemedisplayspace
\begin{schemedisplay}
(define copy-termo
  (lambda (u v)
    (project (u)
      (== (walk* u (build-s u '()) v))))

(define build-s
  (lambda (u s)
    (cond
      ((var? u) (if (assq u s) s (cons (cons u (var 'ignore)) s)))
      ((pair? u) (build-s (cdr u) (build-s (car u) s)))
      (else s))))
\end{schemedisplay}

Unlike \scheme{conde}, only one \scheme{conda}-clause or
\scheme{condu}-clause can return an \scheme{a-inf}: the first clause
whose first goal succeeds.  With \scheme{conda}, the entire stream
returned by the first goal is passed to \scheme{bind*}.  With
\scheme{condu}, a singleton stream is passed to \scheme{bind*}---this
stream contains the first value of the stream returned by the first
goal.

\schemedisplayspace
\begin{schemedisplay}
(define-syntax conda
  (syntax-rules ()
    ((_ (g0 g ...) (g1 g^ ...) ...)
     (lambdag@ (a)
       (inc
         (ifa ((g0 a) g ...)
              ((g1 a) g^ ...) ...))))))

(define-syntax condu
  (syntax-rules ()
    ((_ (g0 g ...) (g1 g^ ...) ...)
     (lambdag@ (a)
       (inc
         (ifu ((g0 a) g ...)
              ((g1 a) g^ ...) ...))))))

(define-syntax ifa
  (syntax-rules ()
    ((_) (mzero))
    ((_ (e g ...) b ...)
     (let loop ((a-inf e))
       (case-inf a-inf
         (() (ifa b ...))
         ((f) (inc (loop (f))))
         ((a) (bind* a-inf g ...))
         ((a f) (bind* a-inf g ...)))))))
\end{schemedisplay}
\newpage
\begin{schemedisplay}  
(define-syntax ifu
  (syntax-rules ()
    ((_) (mzero))
    ((_ (e g ...) b ...)
     (let loop ((a-inf e))
       (case-inf a-inf
         (() (ifu b ...))
         ((f) (inc (loop (f))))
         ((a) (bind* a-inf g ...))
         ((a f) (bind* (unit a) g ...)))))))
\end{schemedisplay}

% Joe's defn for alphaleanTAP:

% \begin{schemedisplay}
% (define var
%   (lambda (n)
%     (make-var n '())))

% (define copy-walk
%   (lambda (v s)
%     (cond
%       ((var? v)
%        (let ((a (assq v s)))
%          (cond
%            (a (copy-walk (cdr a) s))
%            (else v))))
%       (else v))))

% (define copy-walk*
%   (lambda (w s)
%     (let ((v (copy-walk w s)))
%       (cond
%         ((var? v) v)
%         ((pair? v)
%          (cons
%            (copy-walk* (car v) s)
%            (copy-walk* (cdr v) s)))
%         (else v)))))

% (define refresh-s
%   (lambda (v s)
%     (let ((v (copy-walk v s)))
%       (cond
%         ((var? v)
%          (cons (cons v (var 'copy)) s))
%         ((pair? v) (refresh-s (cdr v)
%                      (refresh-s (car v) s)))
%         (else s)))))

% (define copy-termo
%   (lambda (t1 t2)
%     (project (t1)
%       (== (copy-walk* t1 (refresh-s t1 '())) t2))))
% \end{schemedisplay}


% [TODO explain that \scheme|var?| is a dirty operator--if we expose it
% to the user, they can cause all manner of trouble.  \scheme|varo| is
% the goal version of \scheme|var?| if we give the user \scheme|conda|
% or \scheme|condu|, they can define \scheme|varo| on their own]


\chapter{Implementation II:  Optimizing {\em walk}}\label{walkimpl}

In this chapter we examine the efficiency of the \mbox{\scheme|walk|}
algorithm presented in Chapter~\ref{mkimplchapter}, which is the heart
of the unification algorithm.  We present various optimizations to
\mbox{\scheme|walk|}, which significantly improve performance of
unification, and indeed the entire miniKanren implementation.

This chapter is organized as follows.  In section~\ref{walkexpensive}
we examine the worst-case performance of the \mbox{\scheme|walk|}
algorithm, with emphasis on the cost of looking up an unassociated
variable.  Section~\ref{birthrecords} introduces an optimization using
\emph{birth records}, which can greatly increase the speed of looking
up an unassociated variable.  In section~\ref{elimassq} we look at an
additional optimization that requires we rewrite \mbox{\scheme|walk|}
using explicit recursion instead of \mbox{\scheme|assq|}.  Finally,
section~\ref{storesubst} shows how we can further improve on the
birth-records optimization by storing the current substitution in a
variable when it is first introduced.
% Finally, section~\ref{smartwalkbenchmark} demonstrates that the
% optimizations work in practice by benchmarking the performance of
% several miniKanren applications using the various walks.

\section{Why {\em walk} is Expensive}\label{walkexpensive}
In the worst case, the number of cdrs and tests
performed by \mbox{\scheme|walk|} is quadratic in the length of the
substitution.  This happens when looking up a variable at the
beginning of a long ``unification chain''---for example, when looking
up \mbox{\scheme|v|} in the ``perfectly triangular'' substitution
\mbox{\scheme|`((,y . ,z) (,x . ,y) (,w . ,x) (,v . ,w))|}.  Contrast
this with the linear cost of looking up \mbox{\scheme|v|} in the
equivalent idempotent substitution 
\mbox{\scheme|`((,y . ,z) (,x . ,z) (,w . ,z) (,v . ,z))|}.

Fortunately, extremely long unification chains rarely occur in real
logic programs.  Rather, the major cost of variable lookups is in
walking unassociated variables.  When using triangular substitutions
(or even idempotent substitutions), the entire substitution must be
examined to determine that a variable in unassociated\footnote{Prolog
  implementations based on the Warren Abstract Machine~\cite{wamtutorial} do
  not use explicit substitutions to represent variable associations.
  Instead, they represent each variable as a mutable box, and
  side-effect the box during unification.  This makes variable lookup
  extremely fast, but requires remembering and undoing these
  side-effects during backtracking.  In addition, this simple model
  assumes a depth-first search strategy, whereas our purely functional
  representation can be used with interleaving search without
  modification.}.

One solution to this problem is to use a more sophisticated data
structure to represent triangular substitutions---for example, we
might use a trie~\cite{triememory} instead of a list, to ensure logarithmic
cost when looking up an unassociated variable\footnote{Abdulaziz
  Ghuloum has implemented miniKanren using a trie-based representation
  of triangular substitutions.  David Bender and Lindsey Kuper have
  extended this work, using a variety of purely functional data
  structures to represent triangular substitutions.  These more
  sophisticated representations of substitutions can result in much
  faster walking of variables, which can greatly speed up many
  miniKanren programs.  The best performance for their benchmarks
  was achieved using a skew binary number representation within a
  random access list~\cite{randomaccesslists}.}.  For simplicity we
will retain our association list representation of substitutions.
Instead of changing the substitution representation, we will use a
trick to determine if a variable is unassociated without having to
look at the entire substitution.

\section{Birth Records}\label{birthrecords}

To avoid examining the entire substitution when walking an
unassociated variable, we will add a \emph{birth record} to the
substitution whenever we introduce a variable using
\mbox{\scheme|exist|}.  For example, to run the goal
\mbox{\scheme|(exist (x y) (== 5 x))|} we would add the birth records
\mbox{\scheme|`(,x . ,x)|} and
\mbox{\scheme|`(,y . ,y)|} to the current substitution, then run
\mbox{\scheme|(== 5 x)|} in the extended substitution.  
Unifying \mbox{\scheme|x|} with \mbox{\scheme|5|} requires us to walk \mbox{\scheme|x|}:
when we do so, we immediately encounter the birth record \mbox{\scheme|`(,x . ,x)|},
indicating \mbox{\scheme|x|} is unassociated.  Unification then succeeds, adding the association
\mbox{\scheme|`(,x . 5)|} to the substitution to produce \mbox{\scheme|`((,x . 5) (,x . ,x) (,y . ,y) ...)|}.

Here are \mbox{\scheme|exist|} and \mbox{\scheme|walk|}, modified to use birth records.

\schemedisplayspace
\begin{schemedisplay}
(define-syntax exist
  (syntax-rules ()
    ((_ (x ...) g0 g ...)
     (lambdag@ (s)
       (inc
         (let ((x (var 'x)) ...)
           (let* ((s (ext-s x x s))
                  ...)
             (bind* (g0 s) g ...))))))))
\end{schemedisplay}
\newpage
\begin{schemedisplay}    
(define walk
  (lambda (v s)
    (cond
      ((var? v)
       (let ((a (assq v s)))
         (cond
           (a (if (eq? (rhs a) v) v (walk (rhs a) s)))
           (else v))))
      (else v))))
\end{schemedisplay}

Technically, birth records ensure that we need not examine the entire
substitution to determine a variable is unassociated.  However, in the
worst case our situation has not improved\footnote{Indeed, the
  situation is even worse, since the birth records more than double
  the length of the substitution that must be walked.}: if a variable
is introduced at the beginning of a program, but is not unified until
the end of the program, the birth record will occur at the very end of
the substitution, and lookup will still take linear time.
Fortunately, in most real-world programs variables are unified shortly
after they have been introduced.  This locality of reference means
that, in practice, birth records significantly reduce the cost of
walking unassociated variables.
% (see section~\ref{smartwalkbenchmark}).

\section{Eliminating {\em assq} and Checking the {\em rhs}}\label{elimassq}

We can optimize \mbox{\scheme|walk|} in another way, although we will
need to eliminate our call to \mbox{\scheme|assq|}, and introduce a
recursion using ``named'' \scheme|let|
\footnote{This chapter assumes miniKanren is
  run under an optimizing compiler, such as Ikarus Scheme
  or Chez Scheme.  When run under an interpreter, the
  ``named''-\mbox{\scheme|let|} based \mbox{\scheme|walk|} described in this
  section may run much slower than the \mbox{\scheme|assq|}-based
  version, since \mbox{\scheme|assq|} is often hand-coded in C.  When
  running under an interpreter, the \mbox{\scheme|assq|}-based
  \mbox{\scheme|walk|} with birth records will probably be fastest.}.
Here is the standard \mbox{\scheme|walk|}, without birth records.

\schemedisplayspace
\begin{schemedisplay}
(define walk
  (lambda (v s^)
    (let loop ((s s^))
      (cond
        ((var? v)
         (cond
           ((null? s) v)
           ((eq? v (lhs (car s))) (walk (rhs (car s)) s^))
           (else (loop (cdr s)))))
        (else v)))))
\end{schemedisplay}

We can optimize \mbox{\scheme|walk|} by exploiting an important
property of the triangular substitutions produced by
\mbox{\scheme|unify|}: in the substitution \mbox{\scheme|`((,x . ,y) . ,s^)|},
the variable \mbox{\scheme|y|} will never appear in the
left-hand-side (lhs) of any binding in \mbox{\scheme|s^|}.  Therefore,
when walking a variable \mbox{\scheme|y|} we can look for
\mbox{\scheme|y|} in both the lhs and rhs of each association.  If
\mbox{\scheme|y|} is the lhs, we found the variable we are looking
for, and need to walk the rhs in the original substitution.  However,
if we find \mbox{\scheme|y|} in the rhs of an association, we know
that \mbox{\scheme|y|} is unassociated.

Here is the optimized version of \mbox{\scheme|walk|}

\schemedisplayspace
\begin{schemedisplay}
(define walk
  (lambda (v s^)
    (let loop ((s s^))
      (cond
        ((var? v)
         (cond
           ((null? s) v)
           ((eq? v (rhs (car s))) v)
           ((eq? v (lhs (car s))) (walk (rhs (car s)) s^))
           (else (loop (cdr s)))))
        (else v)))))
\end{schemedisplay}

\noindent where \scheme|lhs| and \scheme|rhs|\footnote{\scheme|lhs| is
just defined to be \scheme|car|; \scheme|rhs| is just defined to be
\scheme|cdr|.} return the left-hand-side and right-hand-side of an
association, respectively\footnote{By checking the rhs before
the lhs, we ensure that \scheme|walk| always terminates, even
with substitutions that contain circularities.  If the substitution
contains a circularity of the form \mbox{\scheme|`(,x . ,x)|} (a birth
record), then walking \scheme|x| clearly terminates, since the
\scheme|rhs| test will find \scheme|x| before performing the
recursion.  If the substitution contains associations 
\mbox{\scheme|`(,x . ,y)|} and \mbox{\scheme|`(,y . ,x)|},
walking \scheme|x| still terminates despite the circularity.
Assume \mbox{\scheme|`(,y . ,x)|} appears after 
\mbox{\scheme|`(,x . ,y)|}
(which will never happen for substitutions returned by
\scheme|unify|); then when we walk \scheme|x|, we will end up walking
\scheme|y| in the recursion.  But we will then find \scheme|y| on the
rhs of \mbox{\scheme|`(,x . ,y)|}, which will end the walk.  The
only other possibility is that \mbox{\scheme|`(,y . ,x)|} appears before \mbox{\scheme|`(,x . ,y)|}.  In this case, walking \scheme|x| does not
result in a recursive call, since we find \scheme|x| on the
rhs of \mbox{\scheme|`(,y . ,x)|}.  Similar reasoning applies for
arbitrarily complicated circularity chains.}.

Once we make a recursive call to \scheme|walk|, the
\scheme|null?| test becomes superfluous, so we
redefine \scheme|walk| using the \scheme|step| helper function.

\schemedisplayspace
\begin{schemedisplay}
(define walk
  (lambda (v s^)
    (let loop ((s s^))
      (cond
        ((var? v)
         (cond
           ((null? s) v)
           ((eq? v (rhs (car s))) v)
           ((eq? v (lhs (car s))) (step (rhs (car s)) s^))
           (else (loop (cdr s)))))
        (else v)))))
\end{schemedisplay}
\newpage
\begin{schemedisplay}
(define step
  (lambda (v s^)
    (let loop ((s s^))
      (cond
        ((var? v)
         (cond
           ((eq? v (rhs (car s))) v)
           ((eq? v (lhs (car s))) (step (rhs (car s)) s^))
           (else (loop (cdr s)))))
        (else v)))))
\end{schemedisplay}


\section{Storing the Substitution in the Variable}\label{storesubst}

We now combine the birth records optimization presented in
section~\ref{birthrecords} with checking for the walked variable in
the rhs of each association, described in section~\ref{elimassq}.
However, we wish to avoid polluting the substitution with birth
records, which not only lengthen the substitution but also violate
important invariants of our substitution
representation\footnote{Namely, that a variable never appears on the
  lhs of more than one association, and that substitutions never
  contain circularities of the form \mbox{\scheme|`(,x . ,x)|}.}.
Instead of adding birth records to the substitution, we will add a
``birth substitution'' to each variable by storing the current
substitution in the variable when it is created.

\schemedisplayspace
\begin{schemedisplay}
(define-syntax exist
  (syntax-rules ()
    ((_ (x ...) g0 g ...)
     (lambdag@ (s)
       (inc
         (let ((x (var s)) ...)
           (bind* (g0 s) g ...)))))))
\end{schemedisplay}

Now, instead of checking for the birth records as we walk down the
substitution, we check if the entire substitution is
\mbox{\scheme|eq?|} to the substitution stored in the walked variable;
if so, we know the variable is unassociated\footnote{It should be
  noted that none of these optimizations avoid the $n+1$ passes that
  might be required when looking up a variable in a perfectly
  triangular substitution of length $n$.}.

\newpage

Here, then, is the most efficient definition of
\mbox{\scheme|walk|}\footnote{Exercise for the reader: show that this
  definition of \mbox{\scheme|walk|} works correctly on the renaming
  substitution used in reification (section~\ref{mkreification})}.

\schemedisplayspace
\begin{schemedisplay}
(define walk
  (lambda (v s^)
    (let loop ((s s^))
      (cond
        ((var? v)
         (cond
           ((eq? (vector-ref v 0) s) v)
           ((eq? v (rhs (car s))) v)
           ((eq? v (lhs (car s))) (step (rhs (car s)) s^))
           (else (loop (cdr s)))))
        (else v)))))
\end{schemedisplay}

% \section{Benchmarking {\em walk}}\label{smartwalkbenchmark}

% [TODO Time the arithemtic system using the different walks.  From
% previous informal benchmarks I know these optimizations are effective
% for a wide variety of programs.]


















% [check our various implementation for variants of smart walk code]

% [objective for this chapter--improve performance of walk algorithm]

% [start with standard definition of exist and walk (non-interleaving
% for exist; interleaving variant is obvious---just wrap an inc around
% the body of the lambdag@)]

% \begin{schemedisplay}
% (define-syntax exist
%   (syntax-rules ()
%     ((_ (x ...) g0 g ...)
%      (lambdag@ (s)
%        (inc
%          (let ((x (var 'x)) ...)
%            (bind* (g0 s) g ...)))))))

% (define walk
%   (lambda (v s)
%     (cond
%       ((var? v)
%        (let ((a (assq v s)))
%          (cond
%            (a (walk (rhs a) s))
%            (else v))))
%       (else v))))
% \end{schemedisplay}

% [may also want to include unify-check and ext-s defintions]


% \section{Birth Records}

% [discuss performance implications.  Ikarus and Chez vs. MzScheme, for
% example.  birthrecords vs. our more efficient approach.  what if we
% were using Haskell, and wanted to avoid side effects?]

% [an advantage of birth records: can use normal walk inside of
% walk*---don't need a second, naive version of walk for reification]

% If we find the association \mbox{\scheme|`(,x . ,x)|} when walking
% \scheme|x| in a substitution \scheme|s|, we know that \scheme|x| is
% not associated with any value---there is no need to continue walking
% \scheme|x|.


% \section{Eliminating {\em assq}}

% We can define the naive version of \scheme|walk| without \scheme|assq|:
% \newpage

% Since we walk both \scheme|u| and \scheme|v| at the beginning of
% \scheme|unify|, we know that \scheme|u| and \scheme|v| do not appear
% as the {\em lhs} of any association.
% If when walking \scheme|u| in a substitution \scheme|s| 
% we find the association \mbox{\scheme|`(,u . ,v)|}, then we know
% \scheme|v| does not have an association to the right of 
% \mbox{\scheme|`(,u . ,v)|}.  Hence we need to walk \scheme|v|
% only up to its first occurrence as a {\em rhs}, and thus we can 
% redefine \scheme|step|.



% \section{Storing Substitutions in Variables}

% [disadvantage of this approach: walk* must use naive walk for
% reification---need two definitions of walk]

% [present variants of walk in increasing efficiency]

% [Mitch's optimization]

% [code from /iu/mk/alphatap/optimized/mk.scm is the smartest walk
% version in TRS (second edition?)]

% A vector passed in as one of the first two arguments to \scheme|unify|
% is treated as a variable.  In the definition of \scheme|exist|, we
% create each vector that represents a variable with a symbol. Here, we
% create each vector with the current substitution.


% If the stored substitution of the variable \scheme|x| is encountered
% (with \scheme|eq?|), then we know that the variable \scheme|x| that we
% are walking cannot be a \scheme|lhs| of any association.  Thus, we can
% redefine \scheme|step| for the last time.


% \section{Alternative Substitution Representations}

% [Aziz's trie-based representation---look in /Users/wilja/iu/mk/alphatap/aziz tree-subst (slow)]

% [advantages and disadvantages]

% [can we still use the clever approach of implementing disequality
% constraints with this representation?]

% [this is Aziz's code.  make sure he is okay with including the code in
% my dissertation]

% \begin{schemedisplay}
% (library (mk)
%   (export run == ==-check exist conde conda condu project var)
%   (import (ikarus) (rnrs) (tree))

%   (define-syntax lambdag@
%     (syntax-rules ()
%       ((_ (s) e) (lambda (s) e))))

%   (define-syntax lambdaf@
%     (syntax-rules ()
%       ((_ () e) (lambda () e))))

%   (define-syntax rhs
%     (syntax-rules ()
%       ((_ x) (cdr x))))

%   (define-syntax lhs
%     (syntax-rules ()
%       ((_ x) (car x))))

%   (define-syntax size-s
%     (syntax-rules ()
%       ((_ x) (t:size x))))

%   (define counter -1)
  
%   (define var
%     (lambda (x)
%       (set! counter (add1 counter))
%       (vector x counter)))
  
%   (define-syntax var?
%     (syntax-rules ()
%       ((_ x) (vector? x))))
  
%   (define empty-s '())

%   (define var-idx
%     (lambda (x)
%       (vector-ref x 1)))

%   (define walk
%     (lambda (v s)
%       (cond
%         ((var? v)
%          (cond
%            ((t:lookup (var-idx v) s) =>
%             (lambda (a)
%               (walk (t:binding-value a) s)))
%            (else v)))
%         (else v))))

%   (define ext-s
%     (lambda (x v s)
%       (t:bind (var-idx x) v s)))

%   (define unify
%     (lambda (v w s)
%       (let ((v (walk v s))
%             (w (walk w s)))
%         (cond
%           ((eq? v w) s)
%           ((var? v) (ext-s v w s))
%           ((var? w) (ext-s w v s))
%           ((and (pair? v) (pair? w))
%            (let ((s (unify (car v) (car w) s)))
%              (and s (unify (cdr v) (cdr w) s))))
%           ((equal? v w) s)
%           (else #f)))))

%   (define unify-check
%     (lambda (u v s)
%       (let ((u (walk u s))
%             (v (walk v s)))
%         (cond
%           ((eq? u v) s)
%           ((var? u) (ext-s-check u v s))
%           ((var? v) (ext-s-check v u s))
%           ((and (pair? u) (pair? v))
%            (let ((s (unify-check 
%                       (car u) (car v) s)))
%              (and s (unify-check 
%                       (cdr u) (cdr v) s))))
%           ((equal? u v) s)
%           (else #f)))))

%   (define ext-s-check
%     (lambda (x v s)
%       (cond
%         ((occurs-check x v s) #f)
%         (else (ext-s x v s)))))

%   (define occurs-check
%     (lambda (x v s)
%       (let ((v (walk v s)))
%         (cond
%           ((var? v) (eq? v x))
%           ((pair? v) 
%            (or 
%              (occurs-check x (car v) s)
%              (occurs-check x (cdr v) s)))
%           (else #f)))))

%   (define walk*
%     (lambda (w s)
%       (let ((v (walk w s)))
%         (cond
%           ((var? v) v)
%           ((pair? v)
%            (cons
%              (walk* (car v) s)
%              (walk* (cdr v) s)))
%           (else v)))))

%   (define reify-s
%     (lambda (v s)
%       (let ((v (walk v s)))
%         (cond
%           ((var? v)
%            (ext-s v (reify-name (size-s s)) s))
%           ((pair? v) (reify-s (cdr v)
%                        (reify-s (car v) s)))
%           (else s)))))

%   (define reify-name
%     (lambda (n)
%       (string->symbol
%         (string-append "_" "." (number->string n)))))

%   (define reify
%     (lambda (v s)
%       (let ((v (walk* v s)))
%         (walk* v (reify-s v empty-s)))))

%   (define ==-check
%     (lambda (v w)
%       (lambdag@ (s)
%         (unify-check v w s))))

%   (define-syntax mzero 
%     (syntax-rules () ((_) #f)))
%   (define-syntax inc 
%     (syntax-rules () ((_ e) (lambdaf@ () e))))
%   (define-syntax unit 
%     (syntax-rules () ((_ a) a)))
%   (define-syntax choice 
%     (syntax-rules () ((_ a f) (cons a f))))

%   (define-syntax case-inf
%     (syntax-rules ()
%       ((_ e (() e0) ((f^) e1) ((a^) e2) ((a f) e3))
%        (let ((a-inf e))
%          (cond
%            ((not a-inf) e0)
%            ((procedure? a-inf)  (let ((f^ a-inf)) e1))
%            ((not (and (pair? a-inf)
%                       (procedure? (cdr a-inf))))
%             (let ((a^ a-inf)) e2))
%            (else (let ((a (car a-inf)) (f (cdr a-inf))) 
%                    e3)))))))

%   (define-syntax run
%     (syntax-rules ()
%       ((_ n (x) g0 g ...)
%        (take n
%          (lambdaf@ ()
%            ((exist (x) g0 g ... 
%               (lambdag@ (s)
%                 (cons (reify x s) '())))
%             empty-s))))))

%   (define take
%     (lambda (n f)
%       (if (and n (zero? n)) 
%         '()
%         (case-inf (f)
%           (() '())
%           ((f) (take n f))
%           ((a) a)
%           ((a f)
%            (cons (car a)
%              (take (and n (- n 1)) f)))))))

%   (define ==
%     (lambda (u v)
%       (lambdag@ (s)
%         (unify u v s))))

%   (define-syntax exist
%     (syntax-rules ()
%       ((_ (x ...) g0 g ...)
%        (lambdag@ (s)
%          (inc
%            (let ((x (var 'x)) ...)
%              (bind* (g0 s) g ...)))))))

%   (define-syntax bind*
%     (syntax-rules ()
%       ((_ e) e)
%       ((_ e g0 g ...) (bind* (bind e g0) g ...))))

%   (define bind
%     (lambda (a-inf g)
%       (case-inf a-inf
%         (() (mzero))
%         ((f) (inc (bind (f) g)))
%         ((a) (g a))
%         ((a f) (mplus (g a) (lambdaf@ () (bind (f) g)))))))

%   (define-syntax conde
%     (syntax-rules ()
%       ((_ (g0 g ...) (g1 g^ ...) ...)
%        (lambdag@ (s) 
%          (inc 
%            (mplus* 
%              (bind* (g0 s) g ...)
%              (bind* (g1 s) g^ ...) ...))))))

%   (define-syntax mplus*
%     (syntax-rules ()
%       ((_ e) e)
%       ((_ e0 e ...) (mplus e0 
%                       (lambdaf@ () (mplus* e ...))))))

%   (define mplus
%     (lambda (a-inf f)
%       (case-inf a-inf
%         (() (f))
%         ((f^) (inc (mplus (f) f^)))
%         ((a) (choice a f))
%         ((a f^) (choice a (lambdaf@ () (mplus (f) f^)))))))


%   (define-syntax conda
%     (syntax-rules ()
%       ((_ (g0 g ...) (g1 g^ ...) ...)
%        (lambdag@ (s)
%          (inc
%            (ifa ((g0 s) g ...)
%                 ((g1 s) g^ ...) ...))))))

%   (define-syntax ifa
%     (syntax-rules ()
%       ((_) (mzero))
%       ((_ (e g ...) b ...)
%        (let loop ((a-inf e))
%          (case-inf a-inf
%            (() (ifa b ...))
%            ((f) (inc (loop (f))))
%            ((a) (bind* a-inf g ...))
%            ((a f) (bind* a-inf g ...)))))))

%   (define-syntax condu
%     (syntax-rules ()
%       ((_ (g0 g ...) (g1 g^ ...) ...)
%        (lambdag@ (s)
%          (inc
%            (ifu ((g0 s) g ...)
%                 ((g1 s) g^ ...) ...))))))

%   (define-syntax ifu
%     (syntax-rules ()
%       ((_) (mzero))
%       ((_ (e g ...) b ...)
%        (let loop ((a-inf e))
%          (case-inf a-inf
%            (() (ifu b ...))
%            ((f) (inc (loop (f))))
%            ((a) (bind* a-inf g ...))
%            ((a f) (bind* (unit a) g ...)))))))

%   (define-syntax project 
%     (syntax-rules ()                                                              
%       ((_ (x ...) g g* ...)  
%        (lambdag@ (s)
%          (let ((x (walk* x s)) ...)
%            ((exist () g g* ...) s))))))

% )
% \end{schemedisplay}

% \begin{schemedisplay}
% (library (tree)
%   (export t:bind t:unbind t:lookup t:binding-value t:size)
%   (import (rnrs) (rnrs records syntactic) (ikarus))
  
% ;  (import scheme)
%   ;;; subst ::= (empty)
%   ;;;         | (node even odd)
%   ;;;         | (data idx val)

%   (define-record-type node (fields e o))

%   (define-record-type data (fields idx val))
 
%   (define shift (lambda (n) (fxsra n 1)))

%   (define unshift (lambda (n i) (fx+ (fxsll n 1) i)))

%   (define t:size
%     (lambda (x) (size x)))
  
%   (define t:bind
%     (lambda (xi v s)
%       (unless (and (fixnum? xi) (>= xi 0))
%         (error 't:bind "index must be a fixnum, got ~s" xi))
%       (bind xi v s)))
  
%   (define t:unbind
%     (lambda (xi s)
%       (unless (and (fixnum? xi) (>= xi 0))
%         (error 't:unbind "index must be a fixnum, got ~s" xi))
%       (unbind xi s)))
  
%   (define t:lookup
%     (lambda (xi s)
%       (unless (and (fixnum? xi) (>= xi 0))
%         (error 't:lookup "index must be a fixnum, got ~s" xi))
%       (lookup xi s)))
  
%   (define t:binding-value
%     (lambda (s)
%       (unless (data? s)
%         (error 't:binding-value "not a binding ~s" s))
%       (data-val s)))
  
%   (define push
%     (lambda (xi vi xj vj)
%       (if (fxeven? xi)
%           (if (fxeven? xj)
%               (make-node (push (shift xi) vi (shift xj) vj) '())
%               (make-node (make-data (shift xi) vi) (make-data (shift xj) vj)))
%           (if (fxeven? xj)
%               (make-node (make-data (shift xj) vj) (make-data (shift xi) vi))
%               (make-node '() (push (shift xi) vi (shift xj) vj))))))

%   (define bind
%     (lambda (xi vi s*)
%       (cond
%        [(node? s*)
%         (if (fxeven? xi)
%             (make-node (bind (shift xi) vi (node-e s*)) (node-o s*))
%             (make-node (node-e s*) (bind (shift xi) vi (node-o s*))))]
%        [(data? s*)
%         (let ([xj (data-idx s*)] [vj (data-val s*)])
%           (if (fx= xi xj)
%               (make-data xi vi)
%               (push xi vi xj vj)))]
%        [else (make-data xi vi)])))

%   (define lookup
%     (lambda (xi s*)
%       (cond
%        [(node? s*)
%         (if (fxeven? xi)
%             (lookup (shift xi) (node-e s*))
%             (lookup (shift xi) (node-o s*)))]
%        [(data? s*)
%         (if (fx= (data-idx s*) xi)
%             s*
%             #f)]
%        [else #f])))

%   (define size
%     (lambda (s*)
%       (cond
%        [(node? s*) (fx+ (size (node-e s*)) (size (node-o s*)))]
%        [(data? s*) 1]
%        [else 0])))
  
%   (define cons^
%     (lambda (e o)
%       (cond
%        [(or (node? e) (node? o)) (make-node e o)]
%        [(data? e)
%         (make-data (unshift (data-idx e) 0) (data-val e))]
%        [(data? o)
%         (make-data (unshift (data-idx o) 1) (data-val o))]
%        [else '()])))

%   (define unbind
%     (lambda (xi s*)
%       (cond
%        [(node? s*)
%         (if (fxeven? xi)
%             (cons^ (unbind (shift xi) (node-e s*)) (node-o s*))
%             (cons^ (node-e s*) (unbind (shift xi) (node-o s*))))]
%        [(and (data? s*) (fx= (data-idx s*) xi)) '()]
%        [else s*])))
% )
% \end{schemedisplay}

% \section{Performance}

% [compare mk's performance using different walk algorithms, with
% benchmarks based on applications in the dissertation]

% [do I want an entire chapter on performance?  if so, might need to
% incorporate performance sections on walk, alphatap, and alphaleantap]


\chapter{A Slight Divergence}\label{divergencechapter}

In this chapter we explore the divergence of relational programs.  We
present several divergent miniKanren programs; for each program we
consider different techniques that can be used to make the program
terminate.

By their very nature, relational programs are prone to divergence.  As
relational programmers, we may ask for an infinite number of answers
from a program, or we may look for a non-existent answer in an
infinite search tree.  In fact, miniKanren programs can (and do!)
diverge for a variety of reasons.  A frustration common to beginning
miniKanren programmers is that of carefully writing or deriving a
program, only to have it diverge on even simple test cases.  Learning
to recognize the sources of divergence in a program, and which
techniques can be used to achieve termination, is a critical stage in
the evolution of every relational programmer.

To help miniKanren programmers write relations that terminate, this
chapter presents several divergent example programs; for each program,
we discuss why it diverges, and how the divergence can be avoided.

It is important to remember that a single relational program may
contain multiple, and completely different, causes of divergence; such
programs may require a variety of techniques in order to
terminate\footnote{Challenge for the reader: construct a single
miniKanren program that contains every cause of divergence discussed
in this chapter.  Then use the techniques from this chapter to ``fix''
the program.}.  Also, a single technique may be useful for avoiding
multiple causes of divergence, as will be made clear in the examples
below.  miniKanren does not currently support all of these
techniques (such as operators on cyclic terms)---unsupported
techniques are clearly identified in the text.  Even techniques not
yet supported by miniKanren are of value, however, since they may be
supported by other programming languages.

We now present the divergent example programs, along with techniques
for avoiding divergence.  


\section*{Example 1}

Consider the divergent \scheme|run*| expression

\schemedisplayspace
\begin{schemedisplay}
(run* (q)
  (exist (x y z)
    (pluso x y z)
    (== `(,x ,y ,z) q)))
\end{schemedisplay}

\noindent where \scheme|pluso| is the ternary addition relation defined
in Chapter~\ref{arithchapter}.  This expression diverges because
\mbox{\scheme|(pluso x y z)|} succeeds an unbounded number of times;
therefore, the \scheme|run*| never stops producing answers.  Although
it could be argued that this is a ``good'' infinite loop, and that we
got what we asked for, presumably we want to see some of these
answers.  Also, the user has no way of knowing that the system is
producing any answers, since the divergence might be due to one of the
other causes described below.  (Not to mention that, in general, the
user cannot tell whether the program is diverging or merely taking a
very long time to produce an answer.)

We can avoid this divergence in several different ways:

\begin{enumerate}

\item We could replace the \scheme|run*| with \scheme|run|$^n$, where
$n$ is some positive integer.  This will return the $n$ answers,
although miniKanren's interleaving search makes the order in which
answers are produced difficult to predict.

\item Instead of using the \scheme|run| interface, we could directly
manipulate the answer stream passed as the second argument to
\scheme|take| (Chapter~\ref{mkimplchapter}), and examine the answers
one at a time.  This the ``read-eval-print loop'' approach is used by
Prolog systems, and is trivial to implement in miniKanren by redefining 
\scheme|take|.

\item We can use \scheme|onceo| or \scheme|condu| to ensure that goals
that might succeed an unbounded number of times succeed only once.  Of
course, these operators are non-declarative, so we reject this
approach.  Instead, it would be better to use a \scheme|run1|.

\item A more sophisticated approach is to represent infinitely many
answers as a single answer by using constraints.  For example, one way
to express that \scheme|x| is a natural number other than $2$ is to
associate \scheme|x| with $0, 1, 3, \ldots$.  Clearly, there are
infinitely many such associations, and enumerating them can lead to an
unbounded number of answers.  Instead, we might represent the same
information using the single disequality constraint \mbox{\scheme|(=/= 2 x)|}.

Similarly, we might use a clever data representation rather than a
constraint to represent infinitely many answers as a single term.  For
example, using the little-endian binary representation of natural
numbers presented in Chapter~\ref{arithchapter}, the term
\mbox{\scheme|`(1 . ,x)|} represents any one of the infinitely many
odd naturals.

Using this technique, programs that previously produced infinitely
many answers may fail finitely, proving that no more answers exist.
Unfortunately, it is not always possible to find a constraint or data
representation to concisely represent infinitely many terms.  For
example, although the data representation from
Chapter~\ref{arithchapter} makes it easy to express every odd natural
as a single term, there is no little-endian binary list that succinctly
represents every prime number.  Similarly, disequality constraints are
not sufficient to concisely express that some term does not appear in
an uninstantiated tree\footnote{However, the freshness
constraint (\scheme|hash|) described in Chapter~\ref{akchapter} allows
us to express a similar constraint.}.

\end{enumerate}



\section*{Example 2}

Consider the divergent \scheme|run1| expression

\wspace

\mbox{\scheme|(run1 (q) (==-no-check `(,q) q))|}

\wspace

\noindent The unification of \scheme|q| with \scheme|`(,q)| results in
a substitution containing a circularity\footnote{The \scheme|=/=-no-check|
disequality operator (Chapter~\ref{diseqchapter}) suffers from the
same problem, since it can add circularities to the constraint
store.}: \mbox{\scheme|`((q . (,q)))|}.  However, it is not
unification that diverges, or subsequent calls to \scheme|walk|.
Rather, the reification of \scheme|q| at the end of the computation
calls \scheme|walk*| (Chapter~\ref{mkimplchapter}), which
diverges\footnote{The non-logical operator \scheme|project| also calls
\scheme|walk*|, and can therefore diverge on circular substitutions.}.

We can avoid this divergence in several different ways:

\begin{enumerate}

\item We can use \scheme|==| rather than \scheme|==-no-check| to
  perform sound unification with the occurs check.  The goal
  \mbox{\scheme|(== `(,q) q)|} violates the occurs check and therefore
  fails; hence, \mbox{\scheme|(run1 (q) (== `(,q) q))|} returns
  \mbox{\schemeresult|`()|} rather than diverging\footnote{Similarly,
    we can use \scheme|=/=| rather than \scheme|=/=-no-check| when
    introducing disequality constraints.}.  Since the occurs check can
  be expensive, we may wish to restrict \scheme|==| to only those
  unifications that might introduce a circularity, such as in the
  application line of a type inferencer; this requires reasoning about
  the program.  Alternatively, we can always be safe by using only
  \scheme|==| rather than \scheme|==-no-check|\footnote{As pointed out
    by~\citet{occurcheck} this approach may be overly conservative.
    However, since our primary interest is in avoiding divergence, this
    approach seems reasonable.}.

\item Since the reification of \scheme|q| causes divergence in
this example, the \scheme|run| expression will terminate if we do not
reify the variable associated with the circularity.  For example,

\mbox{\scheme|(run1 (q) (exist (x) (==-no-check `(,x) x)))|}

\noindent returns \mbox{\schemeresult|`(_.0)|}.  Although the
\scheme|run| expression terminates, the resulting substitution is
still circular: \mbox{\scheme|`((,x . (,x)))|}.
However, unless we allow infinite terms,
the unification \mbox{\scheme|(==-no-check `(,x) x)|} is \emph{unsound}.  This
is a problem for the type inferencers based on the simply typed $\lambda$-calculus, 
for example, since self-applications such
as \mbox{\scheme|(f f)|} should not type check (see the inferencer in section~\ref{aktypeinf}).  
If we do not perform
the occurs check, and the circular term
is not reified, the type inference will succeed instead of failing.
Clearly this is not an acceptable way to avoid divergence.  However,
it is important to understand why the program above terminates,
since it is possible to unintentionally write programs that abuse
unsound unification, unless we use \scheme|==| everywhere.

\item Since reification is the cause of divergence in this
example, we can just avoid reification entirely and return the raw
substitution.  The user must determine which associations in the
substitution are of interest; furthermore, the user must check the
substitution for circularities introduced by unsound unification.
There is one more problem with both this approach and the previous
one: the occurs check can prevent divergence by making the program
fail early, which may avoid an unbounded number of successes or a
futile search for a non-existent answer in an infinite search space.

\item Another approach to avoiding divergence is to allow infinite (or
  \emph{cyclic}) terms, as introduced by Prolog
  II~\cite{prologtenfigs,DBLP:conf/fgcs/Colmerauer84,Colmerauer82}.
  Then the unification \mbox{\scheme|(==-no-check `(,q) q)|} is sound,
  even though it returns a circular substitution.  miniKanren does not
  currently support infinite terms; however, it would not be difficult
  to extend the reifier to handle cyclic terms, just as many Scheme
  implementations can print circular lists.

\end{enumerate}




\section*{Example 3}

Consider the divergent \scheme|run1| expression\footnote{Recall that
\scheme|alwayso| was defined in Chapter~\ref{mkintrochapter} as
\mbox{\scheme|(define alwayso (anyo (== #f #f)))|}.  However, for the
purposes of this chapter we define \scheme|alwayso| as

\begin{schemedisplay}
(define alwayso
  (letrec ((alwayso (lambda ()
                      (conde
                        ((== #f #f))
                        ((alwayso))))))
    (alwayso)))
\end{schemedisplay}

\noindent This is because tabling (Chapters~\ref{tablingchapter} and
\ref{tablingimplchapter}) uses reification to determine if a call is a
variant of a previously tabled call.  Since all procedures have the
same reified form (\schemeresult|#<procedure>| under Chez Scheme, for
example), and since \scheme|anyo| takes a goal (a procedure) as its
argument, tabling \scheme|anyo| can lead to unsound behavior.}

\wspace

\mbox{\scheme|(run1 (q) alwayso fail)|}

\wspace

\noindent where \scheme|fail| is defined as \mbox{\scheme|(== #t #f)|}.
Recall that the body of a \scheme|run| is an implicit
conjunction\footnote{\mbox{\scheme|(run1 (q) g1 g2)|} expands into an
expression containing \mbox{\scheme|(exist () g1 g2)|}.}.  In order for
the \scheme|run| expression to succeed, both \scheme|alwayso| and
\scheme|fail| must succeed.  First, \scheme|alwayso| succeeds, then
\scheme|fail| fails.  We then backtrack into \scheme|alwayso|, which
succeeds again, followed once again by failure of the \scheme|fail|
goal.  Since \scheme|alwayso| succeeds an unbounded number of times,
we repeat the cycle forever, resulting in divergence.


We can avoid this divergence in several different ways:

\begin{enumerate}

\item We could simply reorder the goals: \mbox{\scheme|(run1 (q) fail alwayso)|}.
This expression returns \mbox{\schemeresult|`()|} rather
than diverging, since \scheme|fail| fails before \scheme|alwayso| is
even tried.  miniKanren's conjunction operator (\scheme|exist|) is
commutative, but only if an answer exists.  If no answer exists, then
reordering goals within an \scheme|exist| may result in divergence
rather than failure\footnote{We say that conjunction is commutative,
  \emph{modulo divergence versus failure}.}.

However, reordering goals has its disadvantages.  For many programs,
no ordering of goals will result in finite failure (see the remaining
example in this chapter).  Also, by committing to a certain goal
ordering we are giving up on the declarative nature of relational
programming: we are specifying \emph{how} the program computes, rather
than only \emph{what} it computes.  For these reasons, we should
consider alternative solutions.

\item We may be able to use constraints or clever data structures to
represent infinitely many terms as a single term (as described in
Example~1).  If we can use these techniques to make all the conjuncts
succeed finitely many times, then the program will terminate
regardless of goal ordering.

\item Another approach to making the conjuncts succeed finitely many
times is to use tabling, described in Chapter~\ref{tablingchapter}.
Tabling is a form of memoization---we remember every distinct call to
the tabled goal, along with the answers produced.  When a tabled goal
is called, we check whether the goal has previously been called with
similar arguments---if so, we use the tabled answers.

In addition to potentially making goals more efficient by avoiding
duplicate work, tabling can improve termination behavior by cutting
off infinite recursions.  For example, the tabled version of
\scheme|alwayso| succeeds exactly once rather than an unbounded number
of times.  Therefore, \mbox{\scheme|(run1 (q) alwayso fail)|}
returns \mbox{\schemeresult|`()|} rather than diverging when
\scheme|alwayso| is tabled.

Unfortunately, tabling has a major disadvantage: it does not work if
one or more of the arguments to a tabled goal changes with each
recursive call\footnote{As demonstrated by the \scheme|geno| example
in a later footnote.}.

\item We could perform a \emph{dependency analysis} on the
conjuncts---if the goals do not share any logic variables, they cannot
affect each other.  Therefore we can run the goals in parallel,
passing the original substitution to each goal.  If either goal fails,
the entire conjunction fails.  If both goals succeed, we take the
Cartesian product of answers from the goals, and use those new
associations to extend the original substitution.

miniKanren does not currently support this technique; however,
miniKanren's interleaving search should make it straightforward to run
conjuncts in parallel.  A run-time dependency analysis would also be
easy to implement\footnote{Ciao Prolog~\cite{Hermenegildo95strictand}
  performs dependency analysis of conjuncts, along with many other
  analyses, to support efficient parallel logic programming.}.

\item We could address the problem directly by trying to make our
conjunction operator commutative.  For example, we could run both goal
orderings in parallel\footnote{We might
do this by wrapping the goals in a
fern (Chapter~\ref{fernschapter}).}, \mbox{\scheme|(exist () alwayso fail)|} and
\mbox{\scheme|(exist () alwayso fail)|}, and see if either ordering
converges.  If so, we could commit to this goal ordering.
Unfortunately, this commitment may be premature, since the goal
ordering we picked might diverge when we ask for a second answer,
while the other ordering may fail finitely after producing a single
answer.

We could try \emph{all} possible goal orderings, but this is
prohibitively expensive for all but the simplest programs.  In
particular, recursive goals containing conjunctions will result in an
exponential explosion in the number of orderings.

For these reasons, miniKanren does not currently provide a
commutative conjunction operator.  However, future versions of
miniKanren may include an operator that \emph{simulates} full
commutative conjunction using a combination of tabling, parallel goal
evaluation, and continuations (see the Future Work chapter).

\end{enumerate}


\section*{Example 4}

Consider the \scheme|run1| expression \mbox{\scheme|(run1 (x) (pluso bn2 x bn1))|}.
If \scheme|pluso| represents the ternary addition relation
over natural numbers, there is no value for \scheme|q| that satisfies
\mbox{\scheme|(pluso bn2 x bn1)|} (since $2 + x = 1$ has no solution in
the naturals).  Ideally, the \scheme|run1| expression will return
\mbox{\schemeresult|`()|}.  However, a naive implementation of
\scheme|pluso| that enumerates values for \scheme|x| will diverge,
since it will keep associating \scheme|x| with larger numbers without
bound.  Since \scheme|x| grows with each recursive call, tabling
\scheme|pluso| will not help.

We can avoid this divergence in several different ways:

\begin{enumerate}

\item We can relax the domain of \scheme|x| to include negative
integers---then the \scheme|run1| expression will return
\mbox{\schemeresult|`(bnminus1)|}.  However, changing \scheme|run1| to
\scheme|run2| still results in divergence, since $2 + x = 1$ has only
a single solution in the integers.

\item We could use a domain-specific constraint system.  For example,
instead of writing an addition goal, we could use Constraint Logic
Programming over the integers (also known as ``CLP(Z)'').  If we restrict
the sizes of our numbers, we could use CLP(FD) (Constraint Logic
Programming over finite domains).

Alas, no single constraint system can express every interesting
relation in a non-trivial application.  We could try to create a
custom constraint system for each application we write, but this may
be a very difficult task, especially since constraints may interact
with other language features in complex ways.

miniKanren currently supports four kinds of constraints: unification
and disunification constraints using \scheme|==| and
\scheme|=/=| (Chapters~\ref{mkintrochapter} and \ref{diseqchapter});
$\alpha$-equivalence constraints using nominal
unification (Chapter~\ref{akchapter}); and freshness constraints using
\scheme|hash| (Chapter~\ref{akchapter})\footnote{Some non-published
versions of miniKanren have also supported \scheme|pa/ir| constraints:
\mbox{\scheme|(pa/ir x)|} expresses that \scheme|x| can never be
instantiated as a pair.  Uses of \scheme|pa/ir| can typically be
removed through careful use of tagging, however, so we do not include
the constraint in this dissertation.}.  Future versions of miniKanren
will likely support more sophisticated constraints.

\item Another approach is to bound the size of the terms in the
recursive calls to \scheme|pluso|.  For example, if we represent numbers
as binary lists, we know that the lengths of the first two arguments
to \scheme|pluso| (the summands) should never exceed the length of the
third argument (the sum).  By encoding these bounds on term size in
our \scheme|pluso| relation, the call \mbox{\scheme|(pluso bn2 x bn1)|}
will fail finitely.  We use exactly this technique when defining
\scheme|pluso| in Chapter~\ref{arithchapter}.

Bounding term sizes is a very powerful technique, as is demonstrated
in the relational arithmetic chapter of this dissertation.  But as
with the other techniques presented in this chapter, it has its
limitations.  Establishing relationships between argument sizes may
require considerable insight into the relation being expressed.  In
fact, the arithmetic definitions in Chapter~\ref{arithchapter},
including the bounds on term size, were derived from mathematical
equations; this code would be almost impossible to write otherwise\footnote{For example, see the definition of \scheme|logo| in section~\ref{arithexplog}.}.

Furthermore, overly-eager bounds on term size can themselves cause
divergence.  For example, assume that we know arguments \scheme|x| and
\scheme|y| represent lists, which must be of the same length.  We
might be tempted to first determine the length of \scheme|x|, then
determine the length of \scheme|y|, and finally compare the result.
However, if \scheme|x| is an unassociated logic variable, it has no
fixed length: we could cdr down \scheme|x| forever, inadvertently
lengthening \scheme|x| as we go.  Instead, we must
\emph{simultaneously} compare the lengths of \scheme|x| and
\scheme|y|.  To make the task more difficult, we want to enforce the
bounds while we are performing the primary computation of the relation
(for example, while performing addition in the case of \scheme|pluso|).
In fact, lazily enforcing complex bounds between multiple arguments is
likely to be more difficult than writing the underlying relation.

Another problem with bounds on term sizes is that they may not help
when arguments share logic variables.  For example, consider the
\scheme|lesslo| relation: \mbox{\scheme|(lesslo x y)|} succeeds if
\scheme|x| and \scheme|y| are lists, and \scheme|y| is longer than
\scheme|x|.  We can easily implement \scheme|lesslo| by simultaneously
cdring down \scheme|x| and \scheme|y|:

\schemedisplayspace
\begin{schemebox}
(define lesslo
  (lambda-e (x y)
    (`(() (_ . _)))
    (`((_ . ,xd) (_ . ,yd))
     (lesslo xd yd))))
\end{schemebox}

However, consider the call \mbox{\scheme|(lesslo x x)|}.  The
first \scheme|lambda-e| clause fails, while the second clause results
in a recursive call where both arguments are the same uninstantiated
variable.  Therefore \mbox{\scheme|(lesslo x x)|} diverges.

If we were to table \scheme|lesslo|, \mbox{\scheme|(lesslo x x)|}
would fail instead of diverging.  Unfortunately, sharing of arguments
in more complicated relations may result in arguments growing with
each recursive call, which would defeat tabling.

\end{enumerate}

In this section we have examined several divergent miniKanren
programs, investigated the causes of their
divergence\footnote{miniKanren's interleaving search avoids some forms
  of divergence that afflict Prolog, which uses an incomplete search
  strategy equivalent to depth-first search.  For example, the
  left-recursive \scheme|swappendo| relation from
  Chapter~\ref{mkintrochapter} is equivalent to the standard
  \scheme|appendo| relation in miniKanren.  In Prolog, however,
  \scheme|swappendo| diverges in many cases that \scheme|appendo|
  terminates, even when answers exist.  (Although tabling can be used
  to avoid divergence for left-recursive Prolog goals---indeed, this
  is one of the main reasons for including tabling in a Prolog
  implementation.)}, and considered techniques we can use to make
these programs converge.  As miniKanren programmers, divergence, and
how to avoid it, should never be far from our minds.  Indeed, every
extension to the core miniKanren language can be viewed as a new
technique for avoiding divergence\footnote{For example, the freshness
  constraints of nominal logic allow us to express that a nom
  \scheme|a| does not occur free within a variable \scheme|x|.
  Without such a constraint, we would need to instantiate \scheme|x|
  to a potentially unbounded number of ground terms to establish that
  \scheme|a| does not appear in the term.}.

In the next chapter we present a relational arithmetic system that
uses bounds on term size to establish strong termination guarantees.


% [examples that converge in miniKanren, but might diverge in Prolog, other LP systems, or simplified variants of mk]

% \schemedisplayspace
% \begin{schemedisplay}
% (define appendo
%   (lambda (l s out)
%     (conde
%       ((exist (a d res)
%          (== `(,a . ,d) l)
%          (== `(,a . ,res) out)
%          (appendo d s res)))
%       ((== '() l) (== s out)))))
% \end{schemedisplay}

% \schemedisplayspace
% \begin{schemedisplay}
% (define appendo
%   (lambda (l s out)
%     (conde
%       ((== '() l) (== s out))
%       ((exist (a d res)
%          (== `(,a . ,d) l)
%          (== `(,a . ,res) out)
%          (appendo d s res))))))
% \end{schemedisplay}



% \schemedisplayspace
% \begin{schemedisplay}
% (define geno
%   (lambda (n x)
%     (conde
%       ((== n x))
%       ((geno (add1 n) x)))))

% (run1 (q)
%   (exist (x y)
%     (geno 0 x)
%     (geno 0 y)
%     (== 1 x)))
% \end{schemedisplay}


\chapter{Applications I:  Pure Binary Arithmetic}\label{arithchapter}

This chapter presents relations for arithmetic over the non-negative
integers: addition, subtraction, multiplication, division,
exponentiation, and logarithm.  Importantly, these relations are
refutationally complete---if an individual arithmetic relation is
called with arguments that do not satisfy the relation, the relation
will fail in finite time rather than diverge. The conjunction of two
or more arithmetic relations may not fail finitely, however. This is
because the conjunction of arithmetic relations can express
Diophantine equations; were such conjunctions guaranteed to terminate,
we would be able to solve Hilbert's 10$^{th}$ problem, which is
undecidable~\cite{hilbertstenth}.  We also do not guarantee
termination if the goal's arguments share variables, since sharing can
express the conjunction of sharing-free relations.

\citet{conf/flops/KiselyovBFS08} gives proofs of refutational
completeness for these relations.  \citet{trs} and
\citet{conf/flops/KiselyovBFS08} give additional examples and
exposition of these arithmetic relations\footnote{The definition of
  \scheme|logo| in the first printing of \citet{trs} contains an
  error, which has been corrected in the second printing and in
  section~\ref{arithexplog}.}.

This chapter is organized as follows.  Section~\ref{arithrep}
describes our representation of numbers.  In section~\ref{arithnaive}
we present a naive implementation of addition and show its
limitations.  Section~\ref{arithrevisited} presents a more
sophisticated implementation of addition, inspired by the half-adders
and full-adders of digital hardware.  Sections~\ref{arithmult} and
\ref{arithdivision} present the multiplication and division relations,
respectively.  Finally in section~\ref{arithexplog} we define
relations for logarithm and exponentiation.


\section{Representation of Numbers}\label{arithrep}

Before we can write our arithmetic relations, we must decide how we
will represent numbers.  For simplicity, we restrict the domain of our
arithmetic relations to non-negative integers\footnote{We could extend
  our treatment to negative integers by adding a sign tag to each
  number.}.  We might be tempted to use Scheme's built-in numbers for
our arithmetic relations.  Unfortunately, unification cannot decompose
Scheme numbers.  Instead, we need an inductively defined
representation of numbers that can be constructed and deconstructed
using unification.  We will therefore represent numbers as lists.

The simplest approach would be to use a unary
representation\footnote{Even when using unary numbers, defining
  refutationally complete arithmetic relations is non-trivial, as
  demonstrated by~\citet{conf/flops/KiselyovBFS08}.}; however, for
efficiency we will represent numbers as lists of binary digits.  Our
lists of binary digits are \emph{little-endian}: the \scheme|car| of
the list contains the least-significant-bit, which is convenient when
performing arithmetic.  We can define the \scheme|build-num| helper
function, which constructs binary little-endian lists from Scheme
numbers.

\schemedisplayspace
\begin{schemedisplay}
(define build-num
  (lambda (n)
    (cond
      ((zero? n) '())
      ((and (not (zero? n)) (even? n))
       (cons 0 (build-num (quotient n 2))))
      ((odd? n)
       (cons 1 (build-num (quotient (- n 1) 2)))))))
\end{schemedisplay}

\noindent For example \mbox{\scheme|(build-num 6)|} returns
\mbox{\scheme|`(0 1 1)|}, while \mbox{\scheme|(build-num 19)|} returns
\mbox{\scheme|`(1 1 0 0 1)|}.

To ensure there is a unique representation of every number, we suppress
trailing $0$'s. Thus \mbox{\scheme|`(0 1)|} is the unique
representation of the number two; both \mbox{\scheme|`(0 1 0)|} and
\mbox{\scheme|`(0 1 0 0)|} are illegal.  Similarly, \scheme|'()| is
the unique representation of zero; \mbox{\scheme|`(0)|} is illegal.
Lists representing numbers may be partially instantiated:
\mbox{\scheme|`(1 . ,x)|} represents any odd integer, while
\mbox{\scheme|`(0 . ,y)|} represents any \emph{positive} even number.
We must ensure that our relations never instantiate variables
representing numbers to illegal values---in these examples, \scheme|x|
can be instantiated to any legal number, while \scheme|y| can be
instantiated to any number \emph{other} than zero to avoid creating
the illegal value \mbox{\scheme|`(0)|}.

We can now define the simplest useful arithmetic relations,
\scheme|poso| and \scheme|>1o|.  The \scheme|poso| relation is
satisfied if its argument represents a positive
integer.

\schemedisplayspace
\begin{schemedisplay}
(define poso
  (lambda-e (n)
    (`((,a . ,d)))))
\end{schemedisplay}

\noindent The \scheme|>1o| relation is satisfied if its argument
represents an integer greater than one.

\newpage

%\schemedisplayspace
\begin{schemedisplay}
(define >1o
  (lambda-e (n)
    (`((,a ,b . ,d)))))
\end{schemedisplay}

\noindent We will use \scheme|poso| and \scheme|>1o| in more
sophisticated arithmetic relations, starting with addition.

\section{Naive Addition}\label{arithnaive}

Now that we have decided on a representation for numbers, we can
define the addition relation, \scheme|pluso|.  

\schemedisplayspace
\begin{schemedisplay}
(define pluso
  (lambda-e (n m s)
    (`(,x () ,x))
    (`(() ,y ,y))
    (`((0 . ,x) (,b . ,y) (,b . ,res))
     (pluso x y res))    
    (`((,b . ,x) (0 . ,y) (,b . ,res))
     (pluso x y res))
    (`((1 . ,x) (1 . ,y) (0 . ,res))
     (exist (res-1)
       (pluso x y res-1)
       (pluso '(1) res-1 res)))))
\end{schemedisplay}

\noindent The first two clauses handle when \scheme|n| or \scheme|m|
is zero.  The next two clauses handle when both \scheme|n| and
\scheme|m| are positive integers, at least one of which is even.  The
final clause handles when \scheme|n| and \scheme|m| are both positive
odd integers.

At first glance, our definition of \scheme|pluso| seems to work fine.

\wspace

\noindent\scheme|(run1 (q) (pluso '(1 1) '(0 1 1) q))| $\Rightarrow$ \scheme|`((1 0 0 1))|

\wspace

\noindent As expected, adding three and six yields nine.  However,
replacing \scheme|run1| with \scheme|run*| results in the answer
\mbox{\scheme|((1 0 0 1) (1 0 0 1))|}.  The duplicate value is due to
the overlapping of clauses in \scheme|pluso|---for example, both of
the first two clauses succeed when \scheme|n|, \scheme|m|, and
\scheme|s| are all zero.  Even worse, \mbox{\scheme|(run* (q) (pluso '(0 1) q '(1 0 1)))|}
 returns \mbox{\scheme|`((1 1) (1 1) (1 1 0) (1 1 0))|}. The
last two values are not even legal representations of a number, since
the most-significant bit is zero.

We can fix these problems by making the clauses of \scheme|pluso|
non-overlapping, and by adding calls to \scheme|poso| to ensure the
most-significant bit of a positive number is never instantiated to
zero.

\newpage

%\schemedisplayspace
\begin{schemedisplay}
(define pluso
  (lambda-e (n m k)
    (`(,x () ,x))
    (`(() (,x . ,y) (,x . ,y)))
    (`((0 . ,x) (0 . ,y) (0 . ,res)) (poso x) (poso y)
     (pluso x y res))
    (`((0 . ,x) (1 . ,y) (1 . ,res)) (poso x)
     (pluso x y res))
    (`((1 . ,x) (0 . ,y) (1 . ,res)) (poso y)
     (pluso x y res))
    (`((1 . ,x) (1 . ,y) (0 . ,res))
     (exist (res-1)
       (pluso x y res-1)
       (pluso '(1) res-1 res)))))
\end{schemedisplay}

\enlargethispage{1em}

\noindent We separated the third clause of the original \scheme|pluso|
into two clauses, so we can use \scheme|poso| to avoid illegal
instantiations of numbers.

The improved definition of \scheme|pluso| no longer produces duplicate
or illegal values.

\wspace

\noindent\scheme|(run* (q) (pluso '(1 1) '(0 1 1) q))| $\Rightarrow$ \scheme|`((1 0 0 1))|

\noindent\scheme|(run* (q) (pluso '(0 1) q '(1 0 1)))| $\Rightarrow$ \scheme|`((1 1))|

\wspace

It may appear that our new \scheme|pluso| is refutationally complete,
since attempting to add eight to some number \scheme|q| to produce six
fails finitely:

\wspace

\noindent\scheme|(run* (q) (pluso '(0 0 0 1) q '(0 1 1)))| $\Rightarrow$ \scheme|'()|

\wspace

\noindent Unfortunately, this example is misleading---\scheme|pluso|
is not refutationally complete.  The expression
\mbox{\scheme|(run1 (q) (pluso q '(1 0 1) '(0 0 0 1)))|} 
returns \mbox{\scheme|'((1 1))|} as expected, but replacing
\scheme|run1| with \scheme|run2| results in divergence.  Similarly,

\schemedisplayspace
\begin{schemedisplay}
(run6 (q)
  (exist (x y)
    (pluso x y '(1 0 1))
    (== `(,x ,y) q)))
\end{schemedisplay}

\noindent returns

\schemedisplayspace
\begin{schemeresponse}
(((1 0 1) ())
 (() (1 0 1))
 ((0 0 1) (1))
 ((1) (0 0 1))
 ((0 1) (1 1))
 ((1 1) (0 1)))
\end{schemeresponse}

\noindent but \scheme|run7| diverges. If we were to swap the recursive
calls in last clause of \scheme|pluso|, the previous expressions would
converge when using \scheme|run*|; unfortunately, many previously
convergent expressions would then diverge\footnote{These examples
  demonstrate why an efficient implementation (or simulation) of
  commutative conjunction would be useful.}.  If we want
\scheme|pluso| to be refutationally complete, we must reconsider our
approach.


\section{Arithmetic Revisited}\label{arithrevisited}

In this section we develop a refutationally complete definition of
\scheme|pluso|, inspired by the half-adders and full-adders of digital
logic\footnote{See \citet{hennessy-computer} for a description of
  hardware adders.}.

We first define \scheme|half-addero|, which, when given the binary
digits \scheme|x|, \scheme|y|, \scheme|r|, and \scheme|c|, satisfies
the equation $x + y = r + 2 \cdot c$.  

\schemedisplayspace
\begin{schemedisplay}
(define half-addero
  (lambda (x y r c)
    (exist ()
      (bit-xoro x y r)
      (bit-ando x y c))))
\end{schemedisplay}

\scheme|half-addero| is defined using bit-wise relations for logical
{\tt and} and {\tt exclusive-or}.

\schemedisplayspace
\begin{schemedisplay}
(define bit-ando
  (lambda-e (x y r)
    (`(0 0 0))
    (`(1 0 0))
    (`(0 1 0))
    (`(1 1 1))))
\end{schemedisplay}

\begin{schemedisplay}
(define bit-xoro
  (lambda-e (x y r)
    (`(0 0 0))
    (`(0 1 1))
    (`(1 0 1))
    (`(1 1 0))))
\end{schemedisplay}

Now that we have defined \scheme|half-addero|, we can define
\scheme|full-addero|.  \scheme|full-addero| is similar to
\scheme|half-addero|, but takes a carry-in bit \scheme|b|; given bits
\scheme|b|, \scheme|x|, \scheme|y|, \scheme|r|, and \scheme|c|,
\scheme|full-addero| satisfies $b + x + y = r + 2 \cdot c$.

\schemedisplayspace
\begin{schemedisplay}
(define full-addero
  (lambda (b x y r c)
    (exist (w xy wz)
      (half-addero x y w xy)
      (half-addero w b r wz)
      (bit-xoro xy wz c))))
\end{schemedisplay}

\scheme|half-addero| and \scheme|full-addero| add individual bits.  We
now define \scheme|addero| in terms of \scheme|full-addero|;
\scheme|addero| adds a carry-in bit \scheme|d| to arbitrarily large
numbers \scheme|n| and \scheme|m| to produce a number \scheme|r|.

\newpage

%\schemedisplayspace
\begin{schemedisplay}
(define addero
  (lambda (d n m r)
    (match-e `(,d ,n ,m)
      (`(0 __ ()) (== n r))
      (`(0 () __) (== m r) (poso m))
      (`(1 __ ())
       (addero 0 n '(1) r))
      (`(1 () __)
       (poso m)
       (addero 0 '(1) m r))
      (`(__ (1) (1))
       (exist (a c)
         (== `(,a ,c) r)
         (full-addero d 1 1 a c)))
      (`(__ (1) __)
       (gen-addero d n m r))
      (`(__ __ (1))
       (>1o n) (>1o r)
       (addero d '(1) n r))
      (`(__ __ __) 
       (>1o n) 
       (gen-addero d n m r)))))
\end{schemedisplay}

The last clause of \scheme|addero| calls \scheme|gen-addero|; given
the bit \scheme|d| and numbers \scheme|n|, \scheme|m|, and \scheme|r|,
\scheme|gen-addero| satisfies $d + n + m = r$, provided that
\scheme|m| and \scheme|r| are greater than one and \scheme|n| is
positive.

\schemedisplayspace
\begin{schemedisplay}
(define gen-addero
  (lambda (d n m r)
    (match-e `(,n ,m ,r)
      (`((,a . ,x) (,b . ,y) (,c . ,z))
       (exist (e)
         (poso y) (poso z)
         (full-addero d a b c e)
         (addero e x y z))))))
\end{schemedisplay}

We are finally ready to redefine \scheme|pluso|.

\schemedisplayspace
\begin{schemedisplay}
(define pluso (lambda (n m k) (addero 0 n m k)))
\end{schemedisplay}

\noindent As proved by~\citet{conf/flops/KiselyovBFS08}, this
definition of \scheme|pluso| is refutationally complete.  Using the
new \scheme|pluso| all the addition examples from the previous section
terminate, even when using \scheme|run*|.  We can also generate
triples of numbers, where the sum of the first two numbers equals the
third.

\newpage

%\schemedisplayspace
\begin{schemedisplay}
(run9 (q)
  (exist (x y r)
    (pluso x y r)
    (== `(,x ,y ,r) q))) $\Rightarrow$
\end{schemedisplay}
\nspace
\begin{schemeresponse}
((_.0 () _.0)
 (() (_.0 . _.1) (_.0 . _.1))
 ((1) (1) (0 1))
 ((1) (0 _.0 . _.1) (1 _.0 . _.1))
 ((1) (1 1) (0 0 1))
 ((0 _.0 . _.1) (1) (1 _.0 . _.1))
 ((1) (1 0 _.0 . _.1) (0 1 _.0 . _.1))
 ((0 1) (0 1) (0 0 1))
 ((1) (1 1 1) (0 0 0 1)))
\end{schemeresponse}
% \begin{schemeresponse}
% ((_.0 () _.0)
%  (() (_.0 . _.1) (_.0 . _.1))
%  ((1) (1) (0 1))
%  ((1) (0 _.0 . _.1) (1 _.0 . _.1))
%  ((1) (1 1) (0 0 1))
%  ((0 _.0 . _.1) (1) (1 _.0 . _.1))
%  ((1) (1 0 _.0 . _.1) (0 1 _.0 . _.1))
%  ((0 1) (0 1) (0 0 1))
%  ((1) (1 1 1) (0 0 0 1))
%  ((1 1) (1) (0 0 1))
%  ((1) (1 1 0 _.0 . _.1) (0 0 1 _.0 . _.1))
%  ((1 1) (0 1) (1 0 1))
%  ((1) (1 1 1 1) (0 0 0 0 1))
%  ((1 0 _.0 . _.1) (1) (0 1 _.0 . _.1))
%  ((1) (1 1 1 0 _.0 . _.1) (0 0 0 1 _.0 . _.1)))
% \end{schemeresponse}

We can take advantage of the flexibility of the relational approach by
defining subtraction in terms of addition.

\schemedisplayspace
\begin{schemedisplay}
(define minuso (lambda (n m k) (pluso m k n)))
\end{schemedisplay}

\noindent \scheme|minuso| works as expected: 

\wspace

\noindent\scheme|(run* (q) (minuso '(0 0 0 1) '(1 0 1) q))| $\Rightarrow$ \scheme|`((1 1))|

\wspace

\noindent eight minus five is indeed three. \scheme|minuso| is also
refutationally complete:

\wspace

\noindent\scheme|(run* (q) (minuso '(0 1 1) q '(0 0 0 1)))| $\Rightarrow$ \scheme|`()|

\wspace

\noindent there is no non-negative integer \scheme|q| that, when
subtracted from six, produces eight.

\section{Multiplication}\label{arithmult}
\enlargethispage{2em}

Next we define the multiplication relation \scheme|mulo|, which
satisfies $n \cdot m = p$.

\schemedisplayspace
\begin{schemedisplay}
(define mulo
  (lambda (n m p)
    (match-e `(,n ,m)
      (`(() __) (== () p))
      (`(__ ()) (== () p) (poso n))  
      (`((1) __) (== m p) (poso m))   
      (`(__ (1)) (== n p) (>1o n))
      (`((0 . ,x) __)
       (exist (z)
         (== `(0 . ,z) p)
         (poso x) (poso z) (>1o m)
         (mulo x m z)))
      (`((1 . ,x) (0 . ,y))
       (poso x) (poso y)
       (mulo m n p))
      (`((1 . ,x) (1 . ,y))
       (poso x) (poso y)
       (odd-mulo x n m p)))))
\end{schemedisplay}

\noindent \scheme|mulo| is defined in terms of the helper relation
\scheme|odd-mulo|.

\schemedisplayspace
\begin{schemedisplay}
(define odd-mulo
  (lambda (x n m p)
    (exist (q)
      (bound-mulo q p n m)
      (mulo x m q)
      (pluso `(0 . ,q) m p))))
\end{schemedisplay}

\noindent For detailed descriptions of \scheme|mulo| and
\scheme|odd-mulo|, see~\cite{trs} and \cite{conf/flops/KiselyovBFS08}.
From a refutational-completeness perspective, the definition of
\scheme|bound-mulo| is most interesting.

\scheme|bound-mulo| ensures that the product of \scheme|n| and
\scheme|m| is no larger than \scheme|p| by enforcing that the
length\footnote{More correctly, the length of the list representing
  the number.}  of \scheme|n| plus the length of \scheme|m| is an
upper bound for the length of \scheme|p|.  In the process of enforcing
this bound, \scheme|bound-mulo| length-instantiates \scheme|q|---that
is, \scheme|q| becomes a list of fixed length containing
uninstantiated variables representing binary digits.  The length of
\scheme|q|, written $\norm{q}$, satisfies $\norm{q} <
\min(\norm{p},\penalty\binoppenalty \norm{n} + \norm{m} + 1)$.

\schemedisplayspace
\begin{schemedisplay}
(define bound-mulo
  (lambda (q p n m)
    (match-e `(,q ,p)
      (`(() (__ . __)))
      (`((__ . ,x) (__ . ,y))
       (exist (a z)
         (conde
           ((== '() n)
            (== `(,a . ,z) m)
            (bound-mulo x y z '()))
           ((== `(,a . ,z) n)
            (bound-mulo x y z m))))))))
\end{schemedisplay}

\scheme|mulo| works as expected:

\wspace

\noindent\scheme|(run* (p) (mulo '(1 0 1) '(1 1) p))| $\Rightarrow$ \scheme|`(1 1 1 1)|

\wspace

\noindent multiplying five by three yields fifteen. Thanks to the
bounds on term sizes enforced by \scheme|bound-mulo|, \scheme|mulo| is
refutationally complete:

\wspace

\noindent\scheme|(run* (q) (mulo '(0 1) q '(1 1)))| $\Rightarrow$ \scheme|`()|

\wspace

\noindent there exists no non-negative integer \scheme|q| that, when
multiplied by two, yields three.

As we expect of all our relations, \scheme|mulo| is flexible---it can
even be used to factor numbers.  For example, this \scheme|run*|
expression returns all the factors of twelve.

\newpage

%\schemedisplayspace
\begin{schemedisplay}
(run* (q) 
  (exist (m)
    (mulo q m '(0 0 1 1)))) $\Rightarrow$
\end{schemedisplay}
\nspace
\begin{schemeresponse}
((1) (0 0 1 1) (0 1) (0 0 1) (1 1) (0 1 1))
\end{schemeresponse}

% From FLOPS paper.
%
% We guarantee termination only for stand-alone base arithmetic goals but
% not their conjunctions (see \S\ref{s:solution-set}). This
% non-compositionality is expected, since conjunctions of arithmetic
% goals can express Diophantine equations; were such conjunctions
% guaranteed to terminate, we would be able to solve Hilbert's 10th
% problem, which is undecidable~\cite{hilbertstenth}.  We also do not
% guarantee termination if the goal's arguments share variables.  Such a
% goal can be expressed by conjoining a sharing-free base goal and
% equalities.

% [TODO discuss Presberger Arithmetic versus Peano Arithmetic; Hilbert's
% 10th problem; Diophantine equations; conjunction of arithmetic
% relations, perhaps by sharing variables.

% Should be able to lift wording for this.]

\section{Division}\label{arithdivision}

Next we define a relation that performs division with remainder.  We
will need additional bounds on term sizes to define division (and
logarithm in section~\ref{arithexplog}).

The relation \scheme|=lo| ensures that the lists representing the numbers
\scheme|n| and \scheme|m| are the same length.  As before, we must
take care to avoid instantiating either number to an illegal value
like \scheme|`(0)|.

\schemedisplayspace
\begin{schemedisplay}
(define =lo
  (lambda-e (n m)
    (`(() ()))
    (`((1) (1)))
    (`((,a . ,x) (,b . ,y)) (poso x) (poso y)
     (=lo x y))))
\end{schemedisplay}

\scheme|<lo| ensures that the length of the list representing \scheme|n| is
less than that of \scheme|m|.

\schemedisplayspace
\begin{schemedisplay}
(define <lo
  (lambda-e (n m)
    (`(() __) (poso m))
    (`((1) __) (>1o m))
    (`((,a . ,x) (,b . ,y)) (poso x) (poso y)
     (<lo x y))))
\end{schemedisplay}

We can now define \scheme|<=lo| by combining \scheme|=lo| and \scheme|<lo|.

\schemedisplayspace
\begin{schemedisplay}
(define <=lo
  (lambda (n m)
    (conde
      ((=lo n m))
      ((<lo n m)))))
\end{schemedisplay}

Using \scheme|<lo| and \scheme|=lo| we can define \scheme|<o|, which
ensures that the value of \scheme|n| is less than that of \scheme|m|.

\schemedisplayspace
\begin{schemedisplay}
(define <o
  (lambda (n m)
    (conde
      ((<lo n m))
      ((=lo n m)
       (exist (x)
         (poso x)
         (pluso n x m))))))
\end{schemedisplay}

Combining \scheme|<o| and \scheme|==| leads to the definition of
\scheme|<=o|.

\schemedisplayspace
\begin{schemedisplay}
(define <=o
  (lambda (n m)
    (conde
      ((== n m))
      ((<o n m)))))
\end{schemedisplay}

With the bounds relations in place, we can define division with
remainder.  The \scheme|divo| relation takes numbers \scheme|n|,
\scheme|m|, \scheme|q|, and \scheme|r|, and satisfies $n = m \cdot q +
r$, with $0 \leq r < m$; this is equivalent to the equation $\frac{n}{m} = q$
with remainder $r$, with $0 \leq r < m$.
A simple definition of \scheme|divo| is 

\schemedisplayspace
\begin{schemedisplay}
(define divo
  (lambda (n m q r)
    (exist (mq)
      (<o r m)
      (<=lo mq n)
      (mulo m q mq)
      (pluso mq r n))))
\end{schemedisplay}

\noindent Unfortunately, 
\mbox{\scheme|(run* (m) (exist (r) (divo '(1 0 1) m '(1 1 1) r)))|}
diverges. Because we want refutational completeness,
we instead use the more sophisticated definition

\schemedisplayspace
\begin{schemedisplay}
(define divo
  (lambda (n m q r)
    (match-e q 
      (`() (== r n) (<o n m))
      (`(1) (=lo n m) (pluso r m n) (<o r m))
      (__ (<lo m n) (<o r m) (poso q)
       (exist (nh nl qh ql qlm qlmr rr rh)
         (splito n r nl nh)
         (splito q r ql qh)
         (conde
           ((== '() nh)
            (== '() qh)
            (minuso nl r qlm)
            (mulo ql m qlm))
           ((poso nh)
            (mulo ql m qlm)
            (pluso qlm r qlmr)
            (minuso qlmr nl rr)
            (splito rr r '() rh)
            (divo nh m qh rh))))))))
\end{schemedisplay}

\noindent The refutational completeness of \scheme|divo| is largely due to the
use of \scheme|<o|, \scheme|<lo|, and \scheme|=lo| to establish bounds
on term sizes. \scheme|divo| is described in detail in~\citet{trs}.

\scheme|divo| relies on the relation \scheme|splito| to `split' a
binary numeral at a given length: \mbox{\scheme|(splito n r l h)|}
holds if $n = 2^{s+1} \cdot l + h$ where $s = \norm{r}$ and $h < 2^{s+1}$.
\scheme|splito| can construct \scheme|n| by combining the lower-order
bits\footnote{The lowest bit of a positive number \mbox{\scheme|n|} is
  the car of \mbox{\scheme|n|}.} of \scheme|l| with
the higher-order bits of \scheme|h|, inserting \emph{padding} bits as
specified by the length of \scheme|r|---\scheme|splito| is essentially
a specialized version of \scheme|appendo|.  \scheme|splito| ensures
that illegal values like \scheme|'(0)| are not constructed by removing
the rightmost zeros after splitting the number \scheme|n| into its
lower-order bits and its higher-order bits.

\schemedisplayspace
\begin{schemedisplay}
(define splito
  (lambda-e (n r l h)
    (`(() __ () ()))
    (`((0 ,b . ,n^) () () (,b . ,n^)))
    (`((1 . ,n^) () (1) ,n^))
    (`((0 ,b . ,n^) (,a . ,r^) () __)
     (splito `(,b . ,n^) r^ '() h))
    (`((1 . ,n^) (,a . ,r^) (1) __)
     (splito n^ r^ '() h))
    (`((,b . ,n^) (,a . ,r^) (,b . ,l^) __)
     (poso l^)
     (splito n^ r^ l^ h))))
\end{schemedisplay}

% Second, the definition of the \scheme|splito| helper
% relation is so complicated because it must avoid instantiating numbers
% to illegal values.

% \scheme|splito|'s definition is so complicated
% because \scheme|splito| must not allow the list \scheme|'(0)| to
% represent a number.  For example, \mbox{\scheme|(splito '(0 0 1) '()
%   '() '(0 1))|} should succeed, but \mbox{\scheme|(splito '(0 0 1) '()
%   '(0) '(0 1))|} should fail.

% \scheme|l| contains the lowest bits of \scheme|n| (if any), while \scheme|h|
% contains \scheme|n|'s highest bits.  

% The \scheme|splito| relation is similar to the ternary
% \scheme|appendo| relation: appending \scheme|l|

% $\norm{r} - \norm{l} + 1$ padding bits

%  and \scheme|h| yields \scheme|n|, where \scheme|l|,
% \scheme|h|, and \scheme|n| represent non-negative integers using our
% little-endian list scheme.  \scheme|l| contains the lowest bits of
% \scheme|n|, while \scheme|h| contains \scheme|n|'s highest bits.

% The call \mbox{\scheme|(splito n '() l h)|} \emph{moves} the lowest
% bit\footnote{The lowest bit of a positive number \mbox{\scheme|n|} is
%   the \mbox{\scheme|car|} of \mbox{\scheme|n|}.} of \scheme|n|, if
% any, into \scheme|l|, and moves the remaining bits of \scheme|n| into
% \scheme|h|; \mbox{\scheme|(splito n '(1) l h)|} moves the two lowest
% bits of \scheme|n| into \scheme|l| and moves the remaining bits of
% \scheme|n| into \scheme|h|; and \mbox{\scheme|(splito n '(1 1 1 1) l h)|}, 
% \mbox{\scheme|(splito n '(0 1 1 1) l h)|}, or
% \mbox{\scheme|(splito n '(0 0 0 1) l h)|} move the five lowest bits of
% \scheme|n| into \scheme|l| and move the remaining bits into
% \scheme|h|; and so on.

% Since \scheme|splito| is a relation, it can construct \scheme|n| by
% combining the lower-order bits of \scheme|l| with the higher-order
% bits of \scheme|h|, inserting \emph{padding} bits as specified by the
% length of \scheme|r|.

% \scheme|splito|'s definition is so complicated
% because \scheme|splito| must not allow the list \scheme|'(0)| to
% represent a number.  For example, 
% \mbox{\scheme|(splito '(0 0 1) '() '() '(0 1))|} should succeed, but
% \mbox{\scheme|(splito '(0 0 1) '() '(0) '(0 1))|} should fail.
% \scheme|splito| ensures that illegal values like \scheme|'(0)| are not
% constructed by removing the rightmost zeros after splitting the number
% \scheme|n| into its lower-order bits and its higher-order bits.



\section{Logarithm and Exponentiation}\label{arithexplog}

We end this chapter by defining relations for logarithm with remainder
and exponentiation.

\newpage

%\schemedisplayspace
\begin{schemedisplay}
(define logo
  (lambda-e (n b q r)
    (`((1) __ () ()) (poso b))
    (`(__ __ () __) (<o n b) (pluso r '(1) n))
    (`(__ __ (1) __) (>1o b) (=lo n b) (pluso r b n))
    (`(__ (1) __ __) (poso q) (pluso r '(1) n))
    (`(__ () __ __) (poso q) (== r n))
    (`((,a ,b^ . ,dd) (0 1) __ __) (poso dd)
     (exp2o n '() q)
     (exist (s) (splito n dd r s)))
    (`(__ __ __ __)
     (exist (a b^ add ddd)
       (conde
         ((== '(1 1) b))
         ((== `(,a ,b^ ,add . ,ddd) b))))
     (<lo b n)
     (exist (bw1 bw nw nw1 ql1 ql s)
       (exp2o b '() bw1)
       (pluso bw1 '(1) bw)
       (<lo q n)
       (exist (q^ bwq1)
         (pluso q '(1) q^)
         (mulo bw q^ bwq1)
         (<o nw1 bwq1))
       (exp2o n '() nw1)
       (pluso nw1 '(1) nw)
       (divo nw bw ql1 s)
       (pluso ql '(1) ql1)
       (<=lo ql q)
       (exist (bql qh s qdh qd)
         (repeated-mulo b ql bql)
         (divo nw bw1 qh s)
         (pluso ql qdh qh)
         (pluso ql qd q)
         (<=o qd qdh)
         (exist (bqd bq1 bq)
           (repeated-mulo b qd bqd)
           (mulo bql bqd bq)
           (mulo b bq bq1)
           (pluso bq r n)
           (<o n bq1)))))))
\end{schemedisplay}

Given numbers \scheme|n|, \scheme|b|, \scheme|q|, and \scheme|r|,
\scheme|logo| satisfies $n = b^q + r$, where $0 \leq n$ and where $q$
is the largest number that satisfies the equation.  The \scheme|logo|
definition is similar to \scheme|divo|, but uses exponentiation rather
than multiplication\footnote{A line-by-line description of the Prolog
version of \scheme|logo| and its helper relations can be found at
\url{http://okmij.org/ftp/Prolog/Arithm/pure-bin-arithm.prl}}.

%% From Oleg:
% When b = 2, exponentiation and discrete logarithm are easier to obtain
% n = 2^q + r, 0<= 2*r < n
% Here, we just relate n and q.
%    exp2 n b q
% holds if: n = (|b|+1)^q + r, q is the largest such number, and
% (|b|+1) is a power of two.
%
% Side condition: (|b|+1) is a power of two and b is L-instantiated.
% To obtain the binary exp/log relation, invoke the relation as
%  (exp2 n '() q)
% Properties: if n is L-instantiated, one solution, q is fully instantiated.
% If q is fully instantiated: one solution, n is L-instantiated.
% In any event, q is always fully instantiated in any solution
% and n is L-instantiated.
% We depend on the properties of split.

\scheme|logo| relies on helpers \scheme|exp2o| and
\scheme|repeated-mulo|.  \scheme|exp2o| is a simplified version of
exponentiation; given our binary representation of numbers,
exponentiation using base two is particularly simple.
\mbox{\scheme|(exp2o n '() q)|} satisfies $n = 2^q$; the more general
\mbox{\scheme|(exp2o n b q)|} satisfies $n = (\norm{b}+1)^q + r$ for
some \scheme|r|, where \scheme|q| is the largest such number and $0
\leq 2 \cdot r < n$, provided that \scheme|b| is length-instantiated
and $\norm{b}+1$ is a power of two.

\schemedisplayspace
\begin{schemedisplay}
(define exp2o
  (lambda (n b q)
    (match-e `(,n ,q)
      (`((1) ()))
      (`(__ (1))
       (>1o n)
       (exist (s)
         (splito n b s '(1))))
      (`(__ (0 . ,q^))
       (exist (b^)
         (poso q^)
         (<lo b n)
         (appendo b `(1 . ,b) b^)
         (exp2o n b^ q^)))
      (`(__ (1 . ,q^))
       (exist (nh b^ s)
         (poso q^)
         (poso nh)
         (splito n b s nh)
         (appendo b `(1 . ,b) b^)
         (exp2o nh b^ q^))))))
\end{schemedisplay}

%% From Oleg:
% nq = n^q where n is L-instantiated and q is fully instantiated

\mbox{\scheme|(repeated-mulo n q nq)|} satisfies $nq = n^q$ provided
\scheme|n| is length-instantiated and \scheme|q| is fully
instantiated.

\schemedisplayspace
\begin{schemedisplay}
(define repeated-mulo
  (lambda (n q nq)
    (match-e q
      (`() (== (1) nq) (poso n))
      (`(1) (== n nq))
      (__
        (>1o q)
        (exist (q^ nq1)
          (pluso q^ '(1) q)
          (repeated-mulo n q^ nq1)
          (mulo nq1 n nq))))))
\end{schemedisplay}

This simple \scheme|logo| example shows that $14 = 2^3 + 6$.

\wspace

\noindent\scheme|(run* (q) (logo '(0 1 1 1) '(0 1) '(1 1) q))| $\Rightarrow$ \scheme|`(0 1 1)|

\wspace

A more sophisticated example of \scheme|logo| is

\schemedisplayspace
\begin{schemedisplay}
(run9 (s)
  (exist (b q r)
    (logo '(0 0 1 0 0 0 1) b q r)
    (>1o q)
    (== `(,b ,q ,r) s))) $\Rightarrow$
\end{schemedisplay}
\nspace
\begin{schemeresponse}
((() (_.0 _.1 . _.2) (0 0 1 0 0 0 1))
 ((1) (_.0  _.1 . _.2) (1 1 0 0 0 0 1))
 ((0 1) (0 1 1) (0 0 1))
 ((1 1) (1 1) (1 0 0 1 0 1))
 ((0 0 1) (1 1) (0 0 1))
 ((0 0 0 1) (0 1) (0 0 1))
 ((1 0 1) (0 1) (1 1 0 1 0 1))
 ((0 1 1) (0 1) (0 0 0 0 0 1))
 ((1 1 1) (0 1) (1 1 0 0 1))),
\end{schemeresponse}

\noindent which shows that:

\noindent$68 = 0^n + 68$ where $n$ is greater than one,

\nspace

\noindent$68 = 1^n + 67$ where $n$ is greater than one,

\nspace

\noindent$68 = 2^6 + 4$,

\nspace

\noindent$68 = 3^3 + 59$,

\nspace

\noindent$68 = 4^3 + 4$,

\nspace

\noindent$68 = 8^2 + 4$, 

\nspace

\noindent$68 = 5^2 + 43$, 

\nspace

\noindent$68 = 6^2 + 32$, and

\nspace

\noindent$68 = 7^2 + 19$.

We can define the exponentiation relation in terms of \scheme|logo|.

\schemedisplayspace
\begin{schemedisplay}
(define expo (lambda (b q n) (logo n b q '())))
\end{schemedisplay}

\noindent We can use \scheme|expo| to show that three to the fifth
power is $243$:

\wspace

\noindent \scheme|(run* (q) (expo '(1 1) '(1 0 1) q))| $\Rightarrow$ \scheme|`(1 1 0 0 1 1 1 1)|.

\wspace

The code in this chapter demonstrates the difficulty of achieving
refutational completeness, even for relatively simple relations.
Bounding the sizes of terms is a very powerful technique for ensuring
termination, but can be tricky to apply.  The definitions in this
chapter were derived from equations defining arithmetic operators, and
from the design of hardware half-adders and full-adders.  It would
have been extremely difficult to write this code from first
principles.



\part{Disequality Constraints}\label{diseqpart}

% Part~\ref{diseqpart} extends core miniKanren with disequality
% constraints, which allow us to express that two terms are different,
% and can never be unified.  Disequality constraints express a very
% limited form of negation, and can be seen as a simple kind of
% constraint logic programming.  Chapter~\ref{diseqchapter} describes
% disequality constraints from the perspective of the user, while
% Chapter~\ref{diseqimplchapter} shows how we can use unification in a
% clever way to simply and efficiently implement the constraints.  We
% give special attention to \emph{constraint reification}---the process
% of displaying constraints in a human-friendly manner.

\chapter{Techniques I:  Disequality Constraints}\label{diseqchapter}

In this chapter we naively translate a Scheme program to miniKanren,
and observe that the miniKanren relation exhibits undesirable
behavior.  This behavior is due to our inability to express negation
in core miniKanren.  We improve our miniKanren relation through the use of
disequality constraints, which can express a limited form of negation.

This chapter is organized as follows.  In
section~\ref{translaterember} we translate the Scheme function 
\mbox{\scheme|rember|} into the miniKanren relation \mbox{\scheme|rembero|}.
In section~\ref{remberotrouble} we observe that
\mbox{\scheme|rembero|} produces unexpected answers that do not
correspond to answers produced by \mbox{\scheme|rember|}.  In
section~\ref{reconsiderrember} we show that the unexpected answers are
due to our failure to translate implicit tests in the
\mbox{\scheme|rember|} function.  Section~\ref{diseqsection}
introduces disequality constraints, which allow us to express a
limited form of negation.  In section~\ref{fixingrembero} we fix our
definition of \mbox{\scheme|rembero|} by adding a disequality
constraint, thereby eliminating the unexpected answers.  Finally in
section~\ref{diseqlimits} we point out several disadvantages of
disequality constraints, and discuss when these constraints should be
used.

\section{Translating {\rembersymbol} into miniKanren}\label{translaterember}

We begin by naively translating the \scheme|rember| function into
miniKanren.  \mbox{\scheme|rember|} takes two arguments: a symbol \scheme|x| and a
list of symbols \scheme|ls|, and removes the first occurrence of
\scheme|x| from \scheme|ls|.

\wspace

\noindent\mbox{\scheme|(rember 'b '(a b c b d)) => |}\mbox{\schemeresult|`(a c b d)|}

\wspace

\noindent\mbox{\scheme|(rember 'd '(a b c)) => |}\mbox{\schemeresult|`(a b c)|}

%\wspace

\newpage

Here is \mbox{\scheme|rember|}

\schemedisplayspace
\begin{schemedisplay}
(define rember
  (lambda (x ls)
    (cond
      ((null? ls) '())
      ((eq? (car ls) x) (cdr ls))
      (else (cons (car ls) (rember x (cdr ls)))))))
\end{schemedisplay}

To translate \mbox{\scheme|rember|} into the miniKanren relation
\mbox{\scheme|rembero|} we add a third argument \mbox{\scheme|out|},
change \mbox{\scheme|cond|} to \mbox{\scheme|conde|}, and replace uses
of \mbox{\scheme|null?|}, \mbox{\scheme|eq?|}, \mbox{\scheme|cons|},
\mbox{\scheme|car|}, and \mbox{\scheme|cdr|} with calls to
\mbox{\scheme|==|}.  We also unnest the recursive call, using a
temporary variable \mbox{\scheme|res|} to hold the ``output'' value of
the recursive call.

\schemedisplayspace
\begin{schemedisplay}
(define rembero
  (lambda (x ls out)
    (conde
      ((== '() ls) (== '() out))
      ((exist (a d)
         (== `(,a . ,d) ls)
         (== a x)
         (== d out)))
      ((exist (a d res)
         (== `(,a . ,d) ls)
         (== `(,a . ,res) out)
         (rembero x d res))))))
\end{schemedisplay}

\section{The Trouble with {\remberosymbol}}\label{remberotrouble}

For simple tests, it may seem that \mbox{\scheme|rembero|} works as
expected, mimicking the behavior of \mbox{\scheme|rember|}.

\wspace

\noindent\mbox{\scheme|(run1 (q) (rembero 'b '(a b c b d) q)) => |} \mbox{\schemeresult|`((a c b d))|}

\wspace

\noindent\mbox{\scheme|(run1 (q) (rembero 'd '(a b c) q)) => |} \mbox{\schemeresult|`((a b c))|}

\wspace

\noindent However, we notice a problem if we replace the \mbox{\scheme|run1|}
with \mbox{\scheme|run*|}.

\wspace

\noindent\mbox{\scheme|(run* (q) (rembero 'b '(a b c b d) q)) => |} \mbox{\schemeresult|`((a c b d) (a b c d) (a b c b d))|}

\wspace

\noindent Now there are multiple answers.  The first answer is expected, but in
the second answer \mbox{\scheme|rembero|} removes the second
occurrence of \mbox{\scheme|`b|} rather than the first occurrence.  The
last answer is even worse---\mbox{\scheme|rembero|} does not remove
either \mbox{\scheme|`b|}, as is evidenced by the \mbox{\scheme|run|}
expression

\wspace

\noindent\mbox{\scheme|(run* (q) (rembero 'b '(b) '(b))) => |} \mbox{\schemeresult|`(_.0)|}

\section{Reconsidering {\rembersymbol}}\label{reconsiderrember}

Where did we go wrong?  Is our miniKanren translation not faithful to
the original Scheme program?

Not quite. The problem is that \mbox{\scheme|cond|} tries its clauses
in order, stopping at the first clause whose test evaluates to a true
value, while \mbox{\scheme|conde|} tries every possible clause.  But
isn't there only one \mbox{\scheme|cond|} clause that matches any
given values of \mbox{\scheme|x|} and \mbox{\scheme|ls|}?  Actually,
no.

Let us examine the definition of \mbox{\scheme|rember|} once again.

\schemedisplayspace
\begin{schemedisplay}
(define rember
  (lambda (x ls)
    (cond
      ((null? ls) '())
      ((eq? (car ls) x) (cdr ls))
      (else (cons (car ls) (rember x (cdr ls)))))))
\end{schemedisplay}

\noindent Consider the call \mbox{\scheme|(rember 'a '(a b c))|}.
Clearly the \mbox{\scheme|null?|} test keeps the first clause from
returning an answer, while the \mbox{\scheme|eq?|} test allows the
second clause to produce an answer.  But the test of the final clause,
the ``always-true'' \mbox{\scheme|else|} keyword, 
is equivalent to the trivial \mbox{\scheme|#t|} test.

\schemedisplayspace
\begin{schemedisplay}
(define rember
  (lambda (x ls)
    (cond
      ((null? ls) '())
      ((eq? (car ls) x) (cdr ls))
      (#t (cons (car ls) (rember x (cdr ls)))))))
\end{schemedisplay}

\noindent If it were not for the second clause, the third clause would produce
an answer for the call \mbox{\scheme|(rember 'a '(a b c))|}.  In fact,
if we swap the last two clauses

\schemedisplayspace
\begin{schemedisplay}
(define rember
  (lambda (x ls)
    (cond
      ((null? ls) '())
      (#t (cons (car ls) (rember x (cdr ls))))
      ((eq? (car ls) x) (cdr ls)))))
\end{schemedisplay}

\noindent the call \mbox{\scheme|(rember 'a '(a b c))|} returns
\mbox{\scheme|`(a b c)|} rather than \mbox{\scheme|`(b c)|}.

What does the \mbox{\scheme|else|} test really mean in the original
definition of \scheme|rember|?  It means that the tests in all the above
clauses must evaluate to \mbox{\scheme|#f|}.  Similar reasoning holds
for the \mbox{\scheme|eq?|} test of the second clause---the test
implies that the \mbox{\scheme|null?|} test in the first clause
returned \mbox{\scheme|#f|}.  We can therefore redefine
\mbox{\scheme|rember|} to make the implicit tests explicit.

\newpage

%\schemedisplayspace
\begin{schemedisplay}
(define rember
  (lambda (x ls)
    (cond
      ((null? ls) '())
      ((and (not (null? ls)) (eq? (car ls) x))
       (cdr ls))
      ((and (not (null? ls)) (not (eq? (car ls) x)))
       (cons (car ls) (rember x (cdr ls)))))))
\end{schemedisplay}

\mbox{\scheme|rember|} now produces the same answers no matter how we
reorder the clauses; the clauses are now \emph{non-overlapping},
since only a single clause can produce an answer for any specific call
to \mbox{\scheme|rember|}\footnote{Throughout this dissertation we strive to write programs that adhere to the \emph{non-overlapping principle}, to avoid duplicate or misleading answers.  Such programs are similar to the guarded command programs described in \citet{guardedcommands,disciplineprog}.}.

\schemedisplayspace
\begin{schemedisplay}
(define rember
  (lambda (x ls)
    (cond
      ((and (not (null? ls)) (not (eq? (car ls) x)))
       (cons (car ls) (rember x (cdr ls))))
      ((and (not (null? ls)) (eq? (car ls) x))
       (cdr ls))
      ((null? ls) '()))))
\end{schemedisplay}

\noindent Even though we have reordered the \scheme|cond| clauses,
\scheme|rember| works as expected.

\wspace

\noindent\mbox{\scheme|(rember 'a '(a b c)) => |}\mbox{\schemeresult|`(b c)|}

\section{Disequality Constraints}\label{diseqsection}

Now we can reconsider our definition of \mbox{\scheme|rembero|},
adding the equivalent of the explicit tests to make our
\mbox{\scheme|conde|} clauses non-overlapping\footnote{More than one
  \mbox{\scheme|conde|} clause may succeed if \mbox{\scheme|rembero|}
  is passed fresh variables.  However, only one clause will succeed if
  the first two arguments to \mbox{\scheme|rembero|} are fully
  ground.}.

Unfortunately, we do not have a way to express negation in core
miniKanren\footnote{The impure operators \scheme|conda| and
  \scheme|condu| from section~\ref{impureoperatorssection} can be used to express
  ``negation as failure'', as is commonly done in Prolog programs, but
  we eschew this non-declarative approach.}.  However, we do not need
full negation to express the test \mbox{\scheme|(not (null? ls))|},
since if \mbox{\scheme|ls|} is not null it must be a
pair\footnote{This assumes, of course, that the second argument to
  \mbox{\scheme|rembero|} can be unified with a proper list.  Passing
  in \mbox{\scheme|5|} as the \mbox{\scheme|ls|} argument makes no
  more sense for \mbox{\scheme|rembero|} than it does for
  \mbox{\scheme|rember|}.}.  In fact, we are already expressing the
\mbox{\scheme|(not (null? ls))|} test implicitly, through the
unification \mbox{\scheme|(== `(,a . ,d) ls)|} that appears in the
last two \mbox{\scheme|conde|} clauses.

The only remaining test is \mbox{\scheme|(not (eq? (car ls) x))|} in
the last clause.  How might we express that the car of
\mbox{\scheme|ls|} is not \mbox{\scheme|x|}?  We could attempt to
unify the car of \mbox{\scheme|ls|} with every symbol other than
\mbox{\scheme|x|}.  Even if \mbox{\scheme|x|} were instantiated, to the
symbol \mbox{\scheme|`a|} for example, we would have to unify
\mbox{\scheme|x|} with every symbol \emph{other} than
\mbox{\scheme|`a|}, of which there are infinitely many.  Clearly this
is problematic: enumerating an infinite domain can easily lead to
divergent behavior\footnote{It is possible to enumerate some infinite
  domains using a finite number of cases, through the use of clever
  data representation.  For example, using the binary list notation
  from Chapter~\ref{arithchapter} we can express that a natural number
  \mbox{\scheme|x|} is not \mbox{\scheme|5|} by unifying
  \mbox{\scheme|x|} with the patterns \mbox{\scheme|`()|},
  \mbox{\scheme|`(1)|}, \mbox{\scheme|`(,a 1)|}, \mbox{\scheme|`(0 ,a 1)|}, 
  \mbox{\scheme|`(1 1 1)|}, and \mbox{\scheme|`(,a ,b ,c ,d . ,rest)|}.  Although this
  approach avoids divergence, it requires us to know the domain and
  representation of \mbox{\scheme|x|}.  Furthermore, this approach may result in
  duplicate answers even for programs that adhere to the
  non-overlapping principle, which can be a problem even when
  enumerating finite domains.}.

Compare the tests in the second and third \mbox{\scheme|rember|} clauses: \mbox{\scheme|(eq? (car ls) x)|} and \mbox{\scheme|(not (eq? (car ls) x))|}.  We use \mbox{\scheme|(== a x)|} to express that
the car of \mbox{\scheme|ls|} (which is \mbox{\scheme|a|}) is equal to \mbox{\scheme|x|}.  What we need is the ability
to express the \emph{disequality constraint}\footnote{As opposed to an \emph{equality constraint}, such as \mbox{\scheme|(== a x)|}.  Disequality is also known as \emph{disunification}.} \mbox{\scheme|(=/= a x)|}\footnote{We may also wish to introduce an operator \mbox{\scheme|=/=-no-check|} that performs \emph{unsound} disunification, to avoid the cost of the occurs check.}, which
asserts that \mbox{\scheme|a|} and \mbox{\scheme|x|} are not equal, and can never be made equal
through unification.

Before we add a disequality constraint to \mbox{\scheme|rembero|}, let us examine some
simple uses of \mbox{\scheme|=/=|}.
In the first example, we unify \mbox{\scheme|q|} with
\mbox{\scheme|5|}, then specify that \mbox{\scheme|q|} can never be
\mbox{\scheme|5|}.  As expected, the call to \mbox{\scheme|=/=|}
fails.

\wspace

\noindent\mbox{\scheme|(run* (q) (== 5 q) (=/= 5 q)) => |} \mbox{\schemeresult|`()|}

\wspace

If we swap the goals, the program behaves the same.

\wspace

\noindent\mbox{\scheme|(run* (q) (=/= 5 q) (== 5 q)) => |} \mbox{\schemeresult|`()|}

\wspace

\mbox{\scheme|=/=|} can take arbitrary expressions, as shown in the next two examples.

\wspace

\noindent\mbox{\scheme|(run* (q) (=/= (+ 2 3) 5)) => |} \mbox{\schemeresult|`()|}

\wspace

\noindent\mbox{\scheme|(run* (q) (=/= (* 2 3) 5)) => |} \mbox{\schemeresult|`(_.0)|}

\wspace

In this \mbox{\scheme|run*|} expression we assert that
\mbox{\scheme|q|} can never be \mbox{\scheme|5|} or \mbox{\scheme|6|}.
We express the latter constraint indirectly, by constraining
\mbox{\scheme|x|}.

\schemedisplayspace
\begin{schemedisplay}
(run* (q)
  (exist (x)
    (=/= 5 q)
    (== x q)
    (=/= 6 x))) =>
\end{schemedisplay}
\nspace
\begin{schemeresponse}
`((_.0 : (never-equal ((_.0 . 5)) ((_.0 . 6)))))
\end{schemeresponse}

\noindent The answer includes two reified constraints indicating that the output
variable (\mbox{\scheme|q|}) can never be \mbox{\scheme|5|} or \mbox{\scheme|6|}.

\newpage

Consider this \mbox{\scheme|run*|} expression.

\schemedisplayspace
\begin{schemedisplay}
(run* (q)
  (exist (y z)
    (=/= `(,y . ,z) q))) =>
\end{schemedisplay}
\nspace
\begin{schemeresponse}
`(_.0)
\end{schemeresponse}

\noindent It may seem that the constraint on \mbox{\scheme|q|} should
be reified.  However, this constraint can only be violated if
\mbox{\scheme|q|} is unified with \mbox{\scheme|`(,y . ,z)|}.  Since
\mbox{\scheme|y|} and \mbox{\scheme|z|} are not reified, the
constraint is not \emph{relevant} and is therefore not reified.

To reify a constraint, we must reify all of the variables involved in the constraint.

\schemedisplayspace
\begin{schemedisplay}
(run* (q)
  (exist (x y z)
    (=/= `(,y . ,z) x)
    (== `(,x ,y ,z) q))) =>
\end{schemedisplay}
\nspace
\begin{schemeresponse}
`(((_.0 _.1 _.2) : (never-equal ((_.0 _.1 . _.2)))))
\end{schemeresponse}

\noindent The constraint is easier to interpret if we remember that 
\mbox{\scheme|`(never-equal ((_.0 _.1 . _.2)))|} is equivalent to 
\mbox{\scheme|`(never-equal ((_.0 . (_.1 . _.2))))|}.

Here is a slightly more complicated example of \mbox{\scheme|=/=|}.

\schemedisplayspace
\begin{schemedisplay}
(run* (q)
  (exist (x y z)
    (== `(,y . ,z) x)
    (=/= '(5 . 6) x)
    (== 5 y)
    (== `(,x ,y ,z) q))) =>
\end{schemedisplay}
\nspace
\begin{schemeresponse}
`((((5 . _.0) 5 _.0) : (never-equal ((_.0 . 6)))))
\end{schemeresponse}

Here is the same program, but with \mbox{\scheme|(== 6 y)|} instead of \mbox{\scheme|(== 5 y)|}.

\schemedisplayspace
\begin{schemedisplay}
(run* (q)
  (exist (x y z)
    (== `(,y . ,z) x)
    (=/= '(5 . 6) x)
    (== 6 y)
    (== `(,x ,y ,z) q))) =>
\end{schemedisplay}
\nspace
\begin{schemeresponse}
`(((6 . _.0) 6 _.0))
\end{schemeresponse}

\noindent Since \mbox{\scheme|y|} cannot be \mbox{\scheme|5|}, \mbox{\scheme|(=/= '(5 . 6) x)|} cannot be violated and is
therefore discarded.

We end this section with a final example, to demonstrate how to interpret more complicated reified constraints.

\newpage

%\schemedisplayspace
\begin{schemedisplay}
(run* (q)
  (exist (x y z)
    (=/= 5 x)
    (=/= 6 x)
    (=/= `(,y 1) `(2 ,z))
    (== `(,x ,y ,z) q))) =>
\end{schemedisplay}
\nspace
\begin{schemeresponse}
`(((_.0 _.1 _.2) : (never-equal ((_.1 . 2) (_.2 . 1)) ((_.0 . 6)) ((_.0 . 5)))))
\end{schemeresponse}

\noindent The constraints \mbox{\scheme|`((_.0 . 6))|} and
\mbox{\scheme|`((_.0 . 5))|} are independent of each other, and
indicate that \mbox{\scheme|x|} can never be \mbox{\scheme|5|} or
\mbox{\scheme|6|}. However, \mbox{\scheme|`((_.1 . 2) (_.2 . 1))|}
represents a \emph{single} constraint, indicating that
\mbox{\scheme|y|} cannot be \mbox{\scheme|2|} if \mbox{\scheme|z|} is
\mbox{\scheme|1|}\footnote{Reifying constraints in a friendly manner
  is non-trivial, as we will see in Chapter~\ref{diseqimplchapter}.}.

\section{Fixing {\remberosymbol}}\label{fixingrembero}
\enlargethispage{2em}


Now that we understand \mbox{\scheme|=/=|}, and how to interpret reified constraints,
we are ready to add the disequality constraint \mbox{\scheme|(=/= a x)|} to the last
clause of \mbox{\scheme|rembero|}.

\schemedisplayspace
\begin{schemedisplay}
(define rembero
  (lambda (x ls out)
    (conde
      ((== '() ls) (== '() out))
      ((exist (a d)
         (== `(,a . ,d) ls)
         (== a x)
         (== d out)))
      ((exist (a d res)
         (== `(,a . ,d) ls)
         (=/= a x)
         (== `(,a . ,res) out)
         (rembero x d res))))))
\end{schemedisplay}

If we re-run the programs from section~\ref{remberotrouble} we see
that \mbox{\scheme|rembero|}'s behavior is consistent with that of \mbox{\scheme|rember|}.

\wspace

\noindent\mbox{\scheme|(run* (q) (rembero 'b '(a b c b d) q)) => |} \mbox{\schemeresult|`((a c b d))|}

\wspace

\noindent\mbox{\scheme|(run* (q) (rembero 'b '(b) '(b))) => |} \mbox{\schemeresult|`()|}

\wspace

Of course, \mbox{\scheme|rembero|} is more flexible than \mbox{\scheme|rember|}.

\schemedisplayspace
\begin{schemedisplay}
(run* (q)
  (exist (x out)
    (rembero x '(a b c) out)
    (== `(,x ,out) q))) =>
\end{schemedisplay}
\nspace
\begin{schemeresponse}
`((a (b c))
  (b (a c))
  (c (a b))
  ((_.0 (a b c)) : (never-equal ((_.0 . c)) ((_.0 . b)) ((_.0 . a)))))
\end{schemeresponse}

The final answer indicates that removing a symbol \mbox{\scheme|x|}
from the list \mbox{\scheme|`(a b c)|} results in the original list,
provided that \mbox{\scheme|x|} is not \mbox{\scheme|`a|},
\mbox{\scheme|`b|}, or \mbox{\scheme|`c|}.


\section{Limitations of Disequality Constraints}\label{diseqlimits}

Disequality constraints add expressive power to core
miniKanren\footnote{It seems that disequality constraints were present
  in a very early version of Prolog~\cite{birthofprolog}, although
  they were apparently removed after several years.  Prolog
  II~\cite{prologtenfigs} reintroduced disequality constraints, which
  are now standard in most Prolog systems.}, allowing us to express a
limited form of negation.  However, disequality constraints have
several limitations and disadvantages.

First, the \mbox{\scheme|=/=|} operator can only express that two
terms are never the same.  This is much more limited than the ability
to express full negation.  For example, consider the test
\mbox{\scheme|(and (not (null? ls)) (not (eq? (car ls) x)))|} from the
version of \mbox{\scheme|rember|} in section~\ref{reconsiderrember}.
By de Morgan's law, this test is logically equivalent to
\mbox{\scheme|(not (or (null? ls) (eq? (car ls) x)))|}.  We can use
disequality constraints to express the first version of the test, but
not the second.

Answers containing reified disequality constraints can be more
difficult to interpret than answers without constraints.  Also, it is
not always obvious why a constraint was \emph{not} reified (whether it
was not relevant or could not be violated).

Disequality constraints also complicate the implementation of the
unifier, and especially the reifier.  Disequality constraints can also
be expensive, since every constraint must be checked after each
successful unification.

Because of these disadvantages, it is preferable to use
\mbox{\scheme|==|} rather than \mbox{\scheme|=/=|} whenever practical.
For example, it is better to express the test \mbox{\scheme|(not (null? ls))|} 
as \mbox{\scheme|(== `(,a . ,d) ls)|} rather than as
\mbox{\scheme|(=/= '() ls)|}.

Still, disequality constraints add expressive power to core miniKanren, and are
generally preferable to enumerating infinite (or even finite) domains.






% [membero example]

% [rembero  (Scheme -> mk translation)
% duplicate answers

% simplest example of CLP

% implicit negation in cond lines

% Dijkstra guard

% must be able to swap cond clauses in Scheme

% tagging (for application expressions in lambda calculus, for example)]

% [surpriseo]

% [an arithmetic example---some number is not 5]

% can represent infinitely many terms using a single constraint---useful
% in avoiding divergence

% only relevant constraints are printed

% pairs

% constraint simplification

% [look at test programs from =/= test cases]

% [all-different]


\chapter{Implementation III:  Disequality Constraints}\label{diseqimplchapter}

In this chapter we implement the \mbox{\scheme|=/=|} disequality
constraint operator described in Chapter~\ref{diseqchapter}.  We
implement disequality constraints using unification, which results in
remarkably concise and elegant code.  The mathematics of this approach
were described by \citeauthor{Comon91disunification:a} in the
1980's\footnote{See \citet{Comon91disunification:a} and
\citet{ComonEquational1989}.}---to our knowledge, our implementation
is the first to use this technique, for which triangular
substitutions (section~\ref{mkunif}) are a perfect match.  We also
present a sophisticated reifier that removes irrelevant and redundant
constraints.

This chapter is organized as follows.  In section~\ref{diseqrep} we
describe our representation of the constraint store, which is passed
to every goal as part of a \emph{package} that also contains the
substitution.  Section~\ref{solvediseq} presents the constraint
solving algorithm, which is based on unification, while
section~\ref{neverequaloimplsection} defines the \scheme|=/=| and
\scheme|==| operators and related helpers.  Finally in
section~\ref{diseqreify} we present a sophisticated reifier that
produces human-friendly representations of constraints.

\section{Constraints, Constraint Lists, and Packages}\label{diseqrep}

We represent a constraint \scheme|c| as a list of pairs associating variables with terms.
For example, the constraint \mbox{\scheme|(=/= 5 x)|} would be represented as
\mbox{\scheme|`((,x . 5))|}, while the constraint \mbox{\scheme|(=/= `(5 6) `(,y ,z))|}
would be represented as \mbox{\scheme|`((,y . 5) (,z . 6))|}.  In fact, our representation
of disequality constraints is identical to our representation of substitutions---indeed, a
constraint can be viewed as a mini-substitution that indicates which simultaneous variable
associations would violate the constraint.

% Point out that a
% constraint looks exactly like a substitution, and possesses the same
% properties (for example, each variable appears on the lhs at most
% once).

% [constraint store \scheme|c*| is a list of constraints (a list of substitutions)]

A program can introduce many constraints, which requires that we
introduce the notion of a \emph{constraint store} that will be passed to
every goal, along with the substitution.  We represent our constraint
store \scheme|c*| as a list of constraints (that is, a list of
substitutions).  For example, after running the goal

\wspace

\noindent
\mbox{\scheme|(exist (x y z) (=/= 5 x) (=/= `(5 6) `(,y ,z)))|}

\wspace

\noindent
the constraint store would be \mbox{\scheme|`(((,y . 5) (,z . 6)) ((,x . 5)))|}.

We define \scheme|empty-c*| to be the empty list: \mbox{\scheme|(define empty-c* '())|}.
 We extend \scheme|c*| using \scheme|cons|.

We must pass the constraint store to every goal.  We could add an
extra \scheme|c*| argument to each goal, but instead we pass around
the substitution and constraint store as a single value, which we call
a \emph{package}.  Most goal constructors just pass around the
substitution---their definitions need not change.  We only need to
modify goal constructors that extend or inspect the substitution (such
as \scheme|==|).
(We will use the package abstraction whenever we need to pass around
constraint information, such as the freshness constraints of nominal
logic in Chapter~\ref{akimplchapter}.)

Here are our package constructors and deconstructors\footnote{The
\scheme|s-of| deconstructor, which returns a package's substitution,
is all we need to update our definition of the impure operator
\scheme|project|.
\begin{schemedisplay}
(define-syntax project 
  (syntax-rules ()                                                              
    ((_ (x ...) g g* ...)
     (lambdag@ (a)
       (let ((s (s-of a)))
         (let ((x (walk* x s)) ...)
           ((exist () g g* ...) a)))))))
\end{schemedisplay}}.

\wspace

\begin{schemedisplay}
(define make-a (lambda (s c*) (cons s c*)))
(define s-of (lambda (a) (car a)))
(define c*-of (lambda (a) (cdr a)))
(define empty-a (make-a empty-s empty-c*))
\end{schemedisplay}






\section{Solving Disequality Constraints}\label{solvediseq}

In this section we will use unification in a clever way to solve
disequality constraints after a call to \scheme|=/=| or \scheme|==|,
and to keep these constraints in simplified form.  First, observe that
unifying terms \scheme|t1| and \scheme|t2| in a substitution
\scheme|s| has three possible outcomes:

\begin{enumerate}
\item unification can fail, indicating there is no extension to
  \scheme|s| that will make \scheme|t1| and \scheme|t2| equal;
\item unification can succeed {\em without} extending
  \scheme|s|---this implies that \scheme|t1| and \scheme|t2| are
  already equal;
\item unification can succeed, returning an extended substitution
  containing new associations---in this case, the
  ``mini-substitution'' \scheme|s^| containing only these new
  associations represents the most general substitution that makes
  \scheme|t1| and \scheme|t2| equal\footnote{The technical term for
    this substitution is the \scheme{most general unifier} or
    \emph{mgu}.}.
\end{enumerate}

Now let us consider disequality constraints: instead of determining if
\scheme|t1| and \scheme|t2| can be made equal, we wish to determine if
\scheme|t1| and \scheme|t2| can be made \emph{disequal} with respect
to \scheme|s|.  Fortunately, this requires only a slight change in
perspective.  We unify \scheme|t1| and \scheme|t2| with respect to
\scheme|s|, but we interpret result of the unification differently:

\begin{enumerate}
\item if unification fails, \scheme|t1| and \scheme|t2| can never be
  made equal, and the disequality constraint can never be
  violated---therefore, we can throw the constraint away;
\item if unification succeeds {\em without} extending \scheme|s|, then
  \scheme|t1| and \scheme|t2| are already equal---the disequality
  constraint has been violated;
\item if unification succeeds and returns an extended substitution
  containing new associations, then the constraint has not been
  violated, but could still be violated through future calls to
  \scheme|==|---in this case, the ``mini-substitution'' \scheme|s^|
  that contains the new associations represents the updated
  disequality constraint in simplified form.
\end{enumerate}

A few examples should clarify how unification can be used to solve
disequality constraints.  

\begin{enumerate}
\item Running the goal \mbox{\scheme|(=/= 5 6)|} corresponds to the
  first case above: \scheme|5| and \scheme|6| fail to unify in any
  substitution, which means the constraint can never be violated.
  Therefore \mbox{\scheme|(=/= 5 6)|} succeeds, without extending the
  constraint store.

\item The goal \mbox{\scheme|(=/= 5 5)|} corresponds to the second
  case above: \scheme|5| unifies with itself, without extending the
  current substitution, which means the disequality constraint has
  been violated.  Therefore \mbox{\scheme|(=/= 5 5)|} fails.

\item The goal \mbox{\scheme|(=/= `(5 6) `(,x ,y))|} corresponds to
  the third case above: \mbox{\scheme|`(5 6)|} and \mbox{\scheme|`(,x ,y)|} 
  unify in the empty substitution (let's say), resulting in a
  substitution extended with the associations \mbox{\scheme|`(,x . 5)|} 
  and \mbox{\scheme|`(,y . 6)|}.  This means the
  constraint was not violated, but could be violated in the future (if 
  \scheme|x| is unified with \scheme|5| and \scheme|y| with \scheme|6|).
  Therefore \mbox{\scheme|(=/= `(5 6) `(,x ,y))|} succeeds, extending the 
  constraint store with the simplified constraint \mbox{\scheme|`((,x . 5) (,y . 6))|}.
\end{enumerate}


Let us consider a final, more complicated example that uses both
\scheme|=/=| and \scheme|==|.

\schemedisplayspace
\begin{schemedisplay}
(exist (p x y)
  (=/= `(5 6) p)
  (== `(,x ,y) p)
  (== 5 x)
  (== 7 y))
\end{schemedisplay}

\noindent Let us assume that we run this goal in the empty package, containing
the empty substitution \mbox{\scheme|s = `()|} and the empty constraint
store \mbox{\scheme|c* = `()|}.  First we run the goal \mbox{\scheme|(=/= `(5 6) p)|};
\mbox{\scheme|p|} unifies with
\mbox{\scheme|`(5 6)|} in the empty substitution, extending the substitution with the association
\mbox{\scheme|`((,p . (5 6)))|}.  Therefore \mbox{\scheme|(=/= `(5 6) p)|} succeeds, returning a
package with \mbox{\scheme|s = ()|} and \mbox{\scheme|c* =  `(((,p . (5 6))))|}.

Next we run \mbox{\scheme|(== `(,x ,y) p)|};
\mbox{\scheme|p|} unifies with \mbox{\scheme|`(,x ,y)|} in the empty substitution,
returning the extended substitution
\mbox{\scheme|s = `((,p . (,x ,y)))|}.  But after the successful unification we must verify
all of the constraints in the constraint store.  We have only the single constraint
\mbox{\scheme|`((,p . (5 6)))|}, which we verify by unifying \mbox{\scheme|p|} and
\mbox{\scheme|`(5 6)|} in the new substitution
\mbox{\scheme|`((,p . (,x ,y)))|}.  This unification succeeds, extending the substitution with the
associations \mbox{\scheme|`(,x . 5)|} and \mbox{\scheme|`(,y . 6)|}.  Therefore
\mbox{\scheme|(== `(,x ,y) p)|} succeeds, returning
a new package with \mbox{\scheme|s = `((,p . (,x ,y)))|} and
\mbox{\scheme|c* = `(((,x . 5) (,y . 6)))|}.

Next we run \mbox{\scheme|(== 5 x)|}; \mbox{\scheme|x|} unifies with \mbox{\scheme|5|}
in the substitution \mbox{\scheme|`((,p . (,x ,y)))|},
returning the extended substitution \mbox{\scheme|s = `((,x . 5) (,p . (,x ,y)))|}.
Since the unification
was successful, we must verify our constraints.
We still have only a single constraint, \mbox{\scheme|`((,x . 5) (,y . 6))|}, which we verify by
simultaneously unifying \mbox{\scheme|x|} with \mbox{\scheme|5|} and
\mbox{\scheme|y|} with \mbox{\scheme|6|} in the new substitution
\mbox{\scheme|s = `((,x . 5) (,p . (,x ,y)))|}. This unification succeeds, extending
\mbox{\scheme|s|} with the association
\mbox{\scheme|`(,y . 6)|}.  Therefore \mbox{\scheme|(== 5 x)|} succeeds,
returning a new package with
\mbox{\scheme|s = `((,x . 5) (,p . (,x ,y)))|} and \mbox{\scheme|c* = `(((,y . 6)))|}.

Finally we run \mbox{\scheme|(== 7 y)|}; \mbox{\scheme|y|} unifies with \mbox{\scheme|7|}
in the substitution \mbox{\scheme|`((,x . 5) (,p . (,x ,y)))|},
returning the extended substitution \mbox{\scheme|s = `((,y . 7) (,x . 5) (,p . (,x ,y)))|}.
We then check the constraint \mbox{\scheme|`((,y . 6))|} by unifying \mbox{\scheme|y|} and
\mbox{\scheme|6|} in the new substitution; this unification fails,
indicating that the constraint can never be violated, and can therefore be discarded.
The goal \mbox{\scheme|(== 7 y)|} succeeds, as does the entire \mbox{\scheme|exist|},
returning the package
\mbox{\scheme|s = `((,y . 7) (,x . 5) (,p . (,x ,y)))|} and \mbox{\scheme|c* = ()|}.

Had we replaced the final goal \mbox{\scheme|(== 7 y)|}
with \mbox{\scheme|(== 6 y)|}, \mbox{\scheme|y|} and \mbox{\scheme|6|}
would have succeeded without extending the substitution; the constraint
would therefore have been violated, and the entire \scheme|exist| would fail.


\section{Implementing $\neq$ and $\equiv$}\label{neverequaloimplsection}

Now that we understand how to solve disequality constraints using
unification, we are ready to define \scheme|=/=|.
\scheme|=/=| just unifies its arguments in the current
substitution, then passes the result of the unification, along with
original package, to \scheme|=/=-verify|.

\schemedisplayspace
\begin{schemedisplay}
(define-syntax =/=
  (syntax-rules ()
    ((_ u v)
     (lambdag@ (a)
       (=/=-verify (unify u v (s-of a)) a)))))
\end{schemedisplay}

\scheme|=/=-verify| performs a case analysis on the result of the
unification, \scheme|s^|, as described in section~\ref{solvediseq}.
If unification failed, the constraint cannot be violated; therefore
\scheme|=/=| succeeds, and just returns the package passed to it.
Since we are using triangular substitutions, we can use a single
\scheme|eq?| test to determine if unification succeeded without
extending the substitution (the second \scheme|cond| clause); if so,
the constraint has been violated, and \scheme|=/=| returns \scheme|(mzero)|
to indicate failure.  Otherwise, unification returned an extended
substitution.  We therefore call the \scheme|prefix-s| helper (below),
which returns a mini-substitution \scheme|c| containing only the new
associations added during unification.  We then construct a new
package containing both the extended substitution \scheme|s^| and the
simplified constraint \scheme|c|.

\schemedisplayspace
\begin{schemedisplay}
(define =/=-verify
  (lambda (s^ a)
    (cond
      ((not s^) (unit a))
      ((eq? (s-of a) s^) (mzero))
      (else (let ((c (prefix-s s^ (s-of a))))
              (unit (make-a (s-of a) (cons c (c*-of a)))))))))
\end{schemedisplay}

Here is \scheme|prefix-s|, which returns the new associations in
\scheme|s| that do not occur in the older substitution \scheme|<s|.
Our use of triangular substitutions makes it trivial to define
\scheme|prefix-s|, since the new substitutions always form a prefix of
\scheme|s|.

\schemedisplayspace
\begin{schemedisplay}
(define prefix-s
  (lambda (s <s)
    (cond
      ((eq? s <s) empty-s)
      (else (cons (car s) (prefix-s (cdr s) <s))))))
\end{schemedisplay}

We can now define \scheme|==|, which must check every constraint in
the constraint store after a successful unification.  Constraint
checking also ensures the constraints are kept in simplified form,
making future constraint checking more efficient.  This simplified
form also simplifies reification\footnote{We keep each individual
  constraint in simplified form.  However, the constraint store itself
  is not simplified, and may contain redundant constraints.
  Determining if a constraint subsumes another is expensive, so we
  only remove redundant constraints at reification time.}.

\schemedisplayspace
\begin{schemedisplay}
(define-syntax ==
  (syntax-rules ()
    ((_ u v)
     (lambdag@ (a)
       (==-verify (unify u v (s-of a)) a)))))
\end{schemedisplay}

\scheme|==-verify| is similar to, but slightly more complicated than
\scheme|=/=-verify|, since upon successful unification we need to
verify all the constraints in \scheme|c*|.

\newpage

%\schemedisplayspace
\begin{schemedisplay}
(define ==-verify
  (lambda (s^ a)
    (cond
      ((not s^) (mzero))
      ((eq? (s-of a) s^) (unit a))
      ((verify-c* (c*-of a) empty-c* s^)
       => (lambda (c*) (unit (make-a s^ c*))))
      (else (mzero)))))
\end{schemedisplay}

\scheme|verify-c*| verifies all the constraints in \scheme|c*| with
respect to the current substitution \scheme|s|, accumulating the
verified (and simplified) constraints in \scheme|c*^|.
\scheme|verify-c*| uses \scheme|unify*| (below) to simultaneously
unify the left- and right-hand-sides of all the associations within a
given constraint.

\schemedisplayspace
\begin{schemedisplay}
(define verify-c*
  (lambda (c* c*^ s)
    (cond
      ((null? c*) c*^)
      ((unify* (car c*) s)
       => (lambda (s^)
            (cond
              ((eq? s s^) #f)
              (else (let ((c (prefix-s s^ s)))
                      (verify-c* (cdr c*) (cons c c*^) s))))))
      (else (verify-c* (cdr c*) c*^ s)))))

(define unify*
  (lambda (p* s)
    (cond
      ((null? p*) s)
      ((unify (lhs (car p*)) (rhs (car p*)) s)
       => (lambda (s) (unify* (cdr p*) s)))
      (else #f))))
\end{schemedisplay}

\enlargethispage{1em}

For completeness, here is \scheme|==-no-check|\footnote{We can also
define \scheme|=/=-no-check|, which performs \emph{unsound disunification},
allowing circular constraints such as \mbox{\scheme|`((,x . (,x)))|}.
\begin{schemedisplay}
(define-syntax =/=-no-check
  (syntax-rules ()
    ((_ u v)
     (lambdag@ (a)
       (=/=-verify (unify-no-check u v (s-of a)) a)))))
\end{schemedisplay}
Reifying a circular constraint introduced by \scheme|=/=-no-check| can result in divergence.}.
\schemedisplayspace
\begin{schemedisplay}
(define-syntax ==-no-check
  (syntax-rules ()
    ((_ u v)
     (lambdag@ (a)
       (==-verify (unify-no-check u v (s-of a)) a)))))
\end{schemedisplay}


\section{Reification}\label{diseqreify}

We want our reified constraints to be as concise and readable as possible;
we therefore eliminate \emph{irrelevant}
constraints, which contain one or more variables
that are not themselves reified (see section~\ref{diseqsection}).  We
also remove redundant constraints that are subsumed by other reified
constraints.  Our subsumption check uses unification and is
potentially expensive, so we perform this check only during
reification.

A relevant constraint contains no unreified variables.
\scheme|purify| takes the constraint store \scheme|c*| and the reified
name substitution \scheme|r| (section~\ref{mkreification}), and
returns a constraint store containing only relevant constraints.

\schemedisplayspace
\begin{schemedisplay}
(define purify
  (lambda (c* r)
    (cond
      ((null? c*) empty-c*)
      ((anyvar? (car c*) r)
       (purify (cdr c*) r))
      (else (cons (car c*)
              (purify (cdr c*) r))))))
\end{schemedisplay}

\scheme|purify| calls \scheme|anyvar?| on each constraint, which
returns \scheme|#t| if the constraint contains a variable that is
unassociated in the reified name substitution.
(The constraint store is \scheme|walk*|ed in the package's normal
substitution before purification, so that variables associated with
ground terms do not affect purification.)

\schemedisplayspace
\begin{schemedisplay}
(define anyvar?
  (lambda (v r)
    (cond
      ((var? v) (var? (walk v r)))
      ((pair? v) (or (anyvar? (car v) r) (anyvar? (cdr v) r)))
      (else #f))))
\end{schemedisplay}

In addition to removing irrelevant constraints, we also want to remove
any constraint that is subsumed by another reified constraint.  For example,
after running the goal \mbox{\scheme|(exist (x y) (=/= `(5 6) `(,x ,y)) (=/= 5 x))|}
the constraint store will be \mbox{\scheme|`(((,x . 5)) ((,x . 5) (,y . 6)))|}.
Although the individual constraints are simplified, the constraint \mbox{\scheme|`((,x . 5))|}
subsumes the constraint \mbox{\scheme|`((,x . 5) (,y . 6))|}
(since it is not possible to violate the latter constraint without
also violating the former).

We can determine if a constraint \scheme|c| is subsumed by another
constraint \scheme|c^| through yet another clever use of unification.
We use \scheme|unify*| to perform simultaneous unification of the left-
and right-hand-sides of all the associations in \scheme|c^|, with
respect to the ``substitution'' \scheme|c| (see section~\ref{neverequaloimplsection}); 
if \scheme|unify*|
succeeds without extending the substitution, then \scheme|c| is
subsumed by \scheme|c^|.  For example, to determine if the constraint
\mbox{\scheme|c = `((,x . 5) (,y . 6))|} is subsumed by
\mbox{\scheme|c^ = `((,x . 5))|}, we unify \scheme|x| and \scheme|5|
in the substitution \mbox{\scheme|`((,x . 5) (,y . 6))|}.  This
unification succeeds without extending \scheme|c|: therefore,
\mbox{\scheme|`((,x . 5) (,y . 6))|} is subsumed by
\mbox{\scheme|`((,x . 5))|}, and can be discarded.

The \scheme|subsumed?| predicate returns \scheme|#t| if the constraint
\scheme|c| is subsumed by any constraint in \scheme|c*|.

\schemedisplayspace
\begin{schemedisplay}
(define subsumed?
  (lambda (c c*)
    (and (not (null? c*))
         (or (eq? (unify* (car c*) c) c)
             (subsumed? c (cdr c*))))))
\end{schemedisplay}

\scheme|rem-subsumed| takes a list of \emph{unseen} constraints
\scheme|c*| and previously seen constraints \scheme|c*^| (initially
empty), and returns a new constraint store containing independent
constraints, none of which are subsumed by any other.  As
\scheme|rem-subsumed| cdrs down \scheme|c*|, it checks if the car of \scheme|c*| is
subsumed by any of the other constraints, either in the rest of the
unseen constraints in \scheme|c*|, or the already seen constraints accumulated
in \scheme|c*^|.  If so, the car of \scheme|c*| is thrown away;
otherwise, it is added to the list of already seen constraints.

\schemedisplayspace
\begin{schemedisplay}
(define rem-subsumed
  (lambda (c* c*^)
    (cond
      ((null? c*) c*^)
      ((or (subsumed? (car c*) c*^) (subsumed? (car c*) (cdr c*)))
       (rem-subsumed (cdr c*) c*^))
      (else (rem-subsumed (cdr c*) (cons (car c*) c*^))))))
\end{schemedisplay}

Here is the updated definition of \scheme|reify|, which
\scheme|walk*|s the constraint store in the package's substitution
before calling \scheme|purify| and \scheme|rem-subsumed|.
\scheme|reify| returns only the reified value if there are no relevant
constraints; otherwise, \scheme|reify| returns a list containing the
reified value, followed by a tagged list of relevant, and independent,
reified constraints.

\schemedisplayspace
\begin{schemedisplay}
(define reify
  (lambda (v a)
    (let ((s (s-of a)))
      (let ((v (walk* v s))
            (c* (walk* (c*-of a) s)))
        (let ((r (reify-s v empty-s)))
          (let ((v (walk* v r))
                (c* (walk* (rem-subsumed (purify c* r) empty-c*) r)))
            (cond
              ((null? c*) v)
              (else `(,v : (never-equal . ,c*))))))))))
\end{schemedisplay}
























%[unification based-implementation, showing advantage of triangular substitutions]

%[need to reference H. Comon paper]

%[I discovered algorithm independently, in the context of term equality
%for our initial, clumsier implementation of disequality constraints]

%[explain algorithm using case analysis and examples]


%[code taken from /iu/c311/2008\_II/infer/mkneverequalo.scm]

%[relies on standard definitions of conde, etc.]

%[reification]

%[only reify "relevant" constraints]



\part{Nominal Logic}\label{nominallogicpart}

% Part~\ref{nominallogicpart} extends core miniKanren with operators for
% expressing nominal logic; we call the resulting language \alphakanren.
% Nominal logic allows us to easily express notions of scope and
% binding, which is useful when writing declarative interpreters, type
% inferencers, and many other relations that deal with variables.
% Chapter~\ref{akchapter} introduces nominal logic, explains
% \alphakanren's new language constructs, and provides a few simple
% example programs.  Chapter~\ref{alphatapchapter} presents a
% non-trivial application of \alphakanren: a relational theorem prover.
% In Chapter~\ref{akimplchapter} we present our implementation of
% \alphakanren, including two different implementations of nominal
% unification.

\chapter{Techniques II:  Nominal Logic}\label{akchapter}

In this chapter we introduce \alphakanren, which extends core miniKanren with
operators for \emph{nominal logic programming}.  \alphakanrensp was inspired by
\alphaprolog\ \cite{CheneyThesis,CheneyU04} and MLSOS \cite{lakin2007},
and their use of nominal logic \cite{Pitts03} to solve a class of
problems more elegantly than is possible with conventional logic
programming.

Like \alphaprolog\ and MLSOS, \alphakanrensp allows programmers to explicitly manage
variable names and bindings, making it easier to write interpreters, type
inferencers, and other programs that must reason about scope.  \alphakanrensp
also eases the burden of implementing a language from its structural
operational semantics, since the requisite side-conditions 
can often be trivially encoded in nominal logic.

A standard class of such side conditions is to state that a certain
variable name cannot occur free in a particular expression.  It is a
simple matter to check for free occurrences of a variable name in a
fully-instantiated term, but in a logic program the term might contain
unbound logic variables.  At a later point in the program those
variables might be instantiated to terms containing the variable name
in question.  Also, when the writer of semantics employs the equality
symbol, what they really mean is that the two terms are the same \emph{up
  to $\alpha$-equivalence}, as in the variable hygiene convention
popularized by \citet{barendregt84}.  As functional programmers, we
would never quibble with the statement: \mbox{$\lambda x.x$ $=$
  $\lambda y.y$}, yet without the implicit assumption that one can
rename variables using $\alpha$-conversion, we would have to forgo
this obvious equality.  And again, if either expression contains an
unbound logic variable, it is impossible to perform a full parallel
tree walk to determine if the two expressions are $\alpha$-equivalent:
at least part of the tree walk must be deferred until one or both
expressions are fully instantiated.

This chapter is organized as follows.  Section~\ref{akintro}
introduces the \alphakanrensp operators, and provides trivial examples
of their use.  Section~\ref{aksubst} provides a concise but useful
\alphakanrensp program that performs capture-avoiding substitution.
Section~\ref{aktypeinf} presents a second \alphakanrensp program: a
type inferencer for a subset of Scheme.

\section{Introduction to \alphakanren}\label{akintro}
\alphakanrensp extends miniKanren with two additional operators, \scheme|fresh| and \scheme|hash| (entered as {\tt hash}), and one term constructor, \scheme|tie| (entered as {\tt tie}).

\scheme|fresh|, which syntactically looks like \scheme|exist|,
introduces new \emph{noms} into its scope.  (Noms are also called
``names'' or ``atoms'', overloaded terminology which we avoid.)
Conceptually, a nom represents a variable name\footnote{Less commonly,
  a nom may represent a non-variable entity.  For example, a nom may
  represent a channel name in the $\pi$-calculus---see
  \citet{CheneyThesis} for details.}; however, a nom behaves more like
a constant than a variable, since it only unifies with itself or with
an unassociated variable.

\wspace

\noindent\scheme|(run* (q) (fresh (a) (== a a)))| $\Rightarrow$ \begin{schemeresponsebox}(_$_{_{0}}$)\end{schemeresponsebox}

\tspace

\noindent\scheme|(run* (q) (fresh (a) (== a 5)))| $\Rightarrow$ \begin{schemeresponsebox}()\end{schemeresponsebox}

\tspace

\noindent\scheme|(run* (q) (fresh (a b) (== a b)))| $\Rightarrow$ \begin{schemeresponsebox}()\end{schemeresponsebox}

\tspace

\noindent\scheme|(run* (q) (fresh (b) (== b q)))| $\Rightarrow$ \begin{schemeresponsebox}(a$_{_{0}}$)\end{schemeresponsebox}

\wspace

A reified nom is subscripted in the same fashion as a reified
variable, but \schemeresult|a| is used instead
of an underscore (\scheme|_|)---hence
the \begin{schemeresponsebox}(a$_{_{0}}$)\end{schemeresponsebox} in
the final example above.  \scheme|fresh| forms can be nested, which
may result in noms being shadowed.

\schemedisplayspace
\begin{schemedisplay}
(run* (q)
  (exist (x y z)
    (fresh (a)
      (== x a)
      (fresh (a b)
        (== y a)
        (== `(,x ,y ,z ,a ,b) q))))) $\Rightarrow$ 
\end{schemedisplay}
\nspace
\begin{schemeresponse}
((a$_{_{0}}$ a$_{_{1}}$ _$_{_{0}}$ a$_{_{1}}$ a$_{_{2}}$))
\end{schemeresponse}

\noindent Here \schemeresult|a$_{_{0}}$|,
\schemeresult|a$_{_{1}}$|, and
\schemeresult|a$_{_{2}}$|
represent different noms, which will not unify with each other.


\scheme|tie| is a \emph{term constructor} used to limit the scope
of a nom within a term.

\schemedisplayspace
\begin{schemedisplay}
(define-syntax tie
  (syntax-rules ()
    ((_ a t) `(tie-tag ,a ,t))))
\end{schemedisplay}
\noindent Terms constructed using \scheme|tie| are called \emph{binders}.
In the term created by the expression
\mbox{\scheme|(tie a t)|}, all occurrences of the nom \scheme|a|
within term \scheme|t| are considered bound.
We refer to the term \scheme|t| as the \emph{body} of \mbox{\scheme|(tie a t)|}, and to the nom \scheme|a| 
as being in \emph{binding position}.
The \scheme|tie| constructor does not create noms; 
rather, it delimits the scope of noms, 
already introduced using \scheme|fresh|.

\newpage

For example, consider this \scheme|run*| expression.

\schemedisplayspace
\begin{schemedisplay}
(run* (q)
  (fresh (a b)
    (== (tie a `(foo ,a 3 ,b)) q))) $\Rightarrow$
\end{schemedisplay}
\nspace
\begin{schemeresponse}
((tie-tag a$_{_{0}}$ (foo a$_{_{0}}$ 3 a$_{_{1}}$)))
\end{schemeresponse}

\noindent The tagged list \begin{schemeresponsebox}(tie-tag a$_{_{0}}$ (foo a$_{_{0}}$ 3 a$_{_{1}}$))\end{schemeresponsebox} is the reified value of the term constructed using \scheme|tie|.  (The tag name \scheme|tie-tag| is a pun---the bowtie \scheme|tie| is the ``tie that binds.'')  The nom whose reified value is 
\schemeresult|a$_{_{0}}$|
occurs bound within the term
\begin{schemeresponsebox}(tie-tag a$_{_{0}}$ (foo a$_{_{0}}$ 3 a$_{_{1}}$))\end{schemeresponsebox}
while \schemeresult|a$_{_{1}}$| occurs free in that same term.

\scheme|hash|~introduces a \emph{freshness constraint} (henceforth referred to as simply a \emph{constraint}).  The expression \mbox{\scheme|(hash a t)|}
asserts that the nom \scheme|a| does \emph{not} occur free in term
\scheme|t|---if \scheme|a| occurs free in \scheme|t|, then
\mbox{\scheme|(hash a t)|} fails.  Furthermore, if \scheme|t| contains
an unbound variable \scheme|x|, and some later unification 
involving \scheme|x| results in
\scheme|a| occurring free in \scheme|t|, then that unification fails.

\wspace

\noindent\scheme|(run* (q) (fresh (a) (== `(3 ,a #t) q) (hash a q)))| $\Rightarrow$ \begin{schemeresponsebox}()\end{schemeresponsebox}

\tspace

\noindent\scheme|(run* (q) (fresh (a) (hash a q) (== `(3 ,a #t) q)))| $\Rightarrow$ \begin{schemeresponsebox}()\end{schemeresponsebox}

\tspace

\noindent\scheme|(run* (q) (fresh (a b) (hash a (tie b a))))| $\Rightarrow$ \begin{schemeresponsebox}()\end{schemeresponsebox}

\tspace

\noindent\scheme|(run* (q) (fresh (a) (hash a (tie a a))))| $\Rightarrow$ \begin{schemeresponsebox}(_$_{_{0}}$)\end{schemeresponsebox}

%\tspace

\begin{schemedisplay}
(run* (q)
  (exist (x y z)
    (fresh (a)      
      (hash a x)
      (== `(,y ,z) x)
      (== `(,x ,a) q)))) $\Rightarrow$
\end{schemedisplay}
\nspace
\begin{schemeresponse}
((((_$_{_{0}}$ _$_{_{1}}$) a$_{_{0}}$) : ((a$_{_{0}}$ . _$_{_{0}}$) (a$_{_{0}}$ . _$_{_{1}}$))))
\end{schemeresponse}

\noindent In the fourth example, the constraint \mbox{\scheme|(hash a (tie a a))|} is not violated because \scheme|a| does not occur free in \mbox{\scheme|(tie a a)|}.  
In the final example, the partial instantiation of \scheme|x| causes the constraint introduced by
\mbox{\scheme|(hash a x)|} to be ``pushed down'' onto the unbound variables
\scheme|y| and \scheme|z|.  The
answer comprises two parts, separated by a colon and enclosed in an
extra set of parentheses: the reified value of \mbox{\scheme|`((,y ,z) ,a)|}
and a list of reified constraints indicating that
\scheme|a| cannot occur free in either \scheme|y| or \scheme|z|.

The notion of a constraint is prominent in the standard definition of $\alpha$-equivalence \cite{stoy}:

\wspace

\centerline{$\lambda a.M$ $\equiv _{\alpha}$ $\lambda b.$\scheme|[b/a]|$M$
where \scheme|b| does not occur free in \scheme|M|.}

\wspace

\noindent In \alphakanrensp this constraint is expressed as \mbox{\scheme|(# b M)|}.
We shall revisit the connection between constraints and $\alpha$-equivalence shortly.

We now extend the standard notion of unification to that of \emph{nominal unification} \cite{Urban-Pitts-Gabbay/04}, which equates $\alpha$-equivalent binders.  Consider this \scheme|run*| expression:
\mbox{\scheme|(run* (q) (fresh (a b) (== (tie a a) (tie b b))))| $\Rightarrow$ \schemeresult|`(_$_{_{0}}$)|}.
Although \scheme|a| and \scheme|b| are distinct noms, \mbox{\scheme|(== (tie a a) (tie b b))|} succeeds.  According to the rules of nominal unification, the binders \mbox{\scheme|(tie a a)|} and \mbox{\scheme|(tie b b)|} represent the same term, and therefore unify.

The reader may suspect that, as in the definition of $\alpha$-equivalence 
given above, nominal unification uses substitution to equate binders

\wspace

\centerline{\mbox{\scheme|(tie a a)|} $\equiv _{\alpha}$ \mbox{\scheme|(tie b [b/a]a)|}}

\wspace

\noindent however, this is not the case.

Unfortunately, naive substitution does not preserve
$\alpha$-equivalence of terms, as shown in the following example given
by \citet{Urban-Pitts-Gabbay/04}. Consider the $\alpha$-equivalent
terms \mbox{\scheme|(tie a b)|} and \mbox{\scheme|(tie c b)|};
replacing all free occurrences of \scheme|b| with \scheme|a| in both
terms yields \mbox{\scheme|(tie a a)|} and \mbox{\scheme|(tie c a)|},
which are no longer $\alpha$-equivalent.

Rather than using capture-avoiding substitution to address this
problem, nominal logic uses the simple and elegant notion of a
\emph{nom swap}.  Instead of performing a uni-directional substitution
of \scheme|a| for \scheme|b|, the unifier exchanges all occurrences of
\scheme|a| and \scheme|b| within a term, regardless of whether those
noms appear free, bound, or in the binding position of a
\scheme|tie|-constructed binder.  Applying the swap \mbox{\scheme|`(,a ,b)|}
 to \mbox{\scheme|(tie a b)|} and \mbox{\scheme|(tie c b)|}
yields the $\alpha$-equivalent terms \mbox{\scheme|(tie b a)|} and
\mbox{\scheme|(tie c a)|}.

When unifying \mbox{\scheme|(tie a a)|} and \mbox{\scheme|(tie b b)|}
in the \scheme|run*| expression above, the nominal unifier first
creates the swap \mbox{\scheme|`(,a ,b)|} containing the noms in the
binding positions of the two terms.  The unifier then applies this
swap to \mbox{\scheme|(tie a a)|}, yielding \mbox{\scheme|(tie b b)|}
(or equivalently, applies the swap to \mbox{\scheme|(tie b b)|},
yielding \mbox{\scheme|(tie a a)|}).  Obviously \mbox{\scheme|(tie b
    b)|} unifies with itself, according to the standard rules of
unification, and thus the nominal unification succeeds.

Of course, the terms being unified might contain unbound variables.
In the simple example

\wspace

\noindent\scheme|(run* (q) (fresh (a b) (== (tie a q) (tie b b))))| $\Rightarrow$ \begin{schemeresponsebox}(a$_{_{0}}$)\end{schemeresponsebox}

\wspace

\noindent 
the swap \mbox{\scheme|`(,a ,b)|} can be applied to
\mbox{\scheme|(tie b b)|}, yielding \mbox{\scheme|(tie a a)|}.
The terms \mbox{\scheme|(tie a a)|} and \mbox{\scheme|(tie a q)|}
are then unified, associating \scheme|q| with \scheme|a|.
However, in some cases a swap cannot be performed until a variable has become at
least partially instantiated.  For example, in the first call to \scheme|==| in 

\schemedisplayspace
\begin{schemedisplay}
(run* (q)
  (fresh (a b)
    (exist (x y)
      (== (tie a (tie a x)) (tie a (tie b y)))
      (== `(,x ,y) q))))
\end{schemedisplay}

\noindent the unifier cannot apply the swap \mbox{\scheme|`(,a ,b)|} 
to either \mbox{\scheme|x|} or \mbox{\scheme|y|}, since they are
both unbound.
(The unifier does not generate a swap for the outer binders, since they
have the same nom in their binding positions.)

Nominal unification solves this problem by introducing the notion of a \emph{suspension}, 
which is a record of \emph{delayed swaps} that may be applied later.
We represent a suspension using the \scheme|susp-tag| data structure, 
which comprises a list of suspended swaps and a variable.

\schemedisplayspace
\begin{schemedisplay}
$\hspace{1.8cm}$`(susp-tag ((,a_n ,b_n) ... (,a_1 ,b_1)) ,x)
\end{schemedisplay}

\noindent The swaps are deferred until the variable \scheme|x| is
instantiated (at least partially); at this point the swaps are applied
to the instantiated portion of the term associated with \scheme|x|.
Swaps are applied from right to left; that is, the result of applying
the swaps to a term \scheme|t| can be determined by first exchanging
all occurrences of noms \scheme|a_1| and \scheme|b_1| within
\scheme|t|, then exchanging \scheme|a_2| and \scheme|b_2| within the
resulting term, and continuing in this fashion until finally
exchanging \scheme|a_n| with \scheme|b_n|.

Now that we have the notion of a suspension, we can define equality on
binders (adapted from \citealt{Urban-Pitts-Gabbay/04}):

\begin{quotation}
\noindent \mbox{\scheme|(tie a M)|} and \mbox{\scheme|(tie b N)|} are $\alpha$-equivalent if and only if
 \scheme|a| and \scheme|b| are the same nom and \scheme|M| is $\alpha$-equivalent to \scheme|N|, or if 
 \mbox{\scheme|`(susp-tag ((,a ,b)) ,M)|} is $\alpha$-equivalent to \scheme|N| and \mbox{\scheme|(hash b M)|}.
\end{quotation}

\noindent The side condition \mbox{\scheme|(hash b M)|} is necessary,
since if \scheme|b| occurred free in \scheme|M|, then \scheme|b| would be inadvertently 
captured (and replaced with \scheme|a|) by the suspension \mbox{\scheme|`(susp-tag ((,a ,b)) ,M)|}.


Having defined equality on binders, 
we can examine the result of the previous \scheme|run*| expression.

\schemedisplayspace
\begin{schemedisplay}
(run* (q)
  (fresh (a b)
    (exist (x y)
      (== (tie a (tie a x)) (tie a (tie b y)))
      (== `(,x ,y) q)))) $\Rightarrow$ 
\end{schemedisplay}
\nspace
\begin{schemeresponse}((((susp-tag ((a$_{_{0}}$ a$_{_{1}}$)) _$_{_{0}}$) _$_{_{0}}$) : ((a$_{_{0}}$ . _$_{_{0}}$))))\end{schemeresponse}

\noindent The first call to \scheme|==| 
applies the swap \mbox{\scheme|`(,a ,b)|}
to the unbound variable \scheme|y|, and then
associates 
the resulting suspension \mbox{\scheme|`(susp-tag ((,a ,b)) ,y)|}
with \scheme|x|.  
Of course, the unifier
could have applied the swap to \scheme|x| instead of \scheme|y|,
resulting in a symmetric answer.
The freshness constraint states that the nom \scheme|a| can never
occur free within \scheme|y|, as required by the definition of 
binder equivalence.

Here is a translation of a quiz presented in \citet{Urban-Pitts-Gabbay/04}, 
demonstrating some of the finer points of nominal unification.

\schemedisplayspace
\begin{schemedisplay}
(run* (q)
  (fresh (a b)
    (exist (x y)
      (conde
        ((== (tie a (tie b `(,x ,b))) (tie b (tie a `(,a ,x)))))
        ((== (tie a (tie b `(,y ,b))) (tie b (tie a `(,a ,x)))))
        ((== (tie a (tie b `(,b ,y))) (tie b (tie a `(,a ,x)))))
        ((== (tie a (tie b `(,b ,y))) (tie a (tie a `(,a ,x))))))
      (== `(,x ,y) q)))) $\Rightarrow$
\end{schemedisplay}
\nspace
\begin{schemeresponse}
((a$_{_{0}}$ a$_{_{1}}$)
 (_$_{_{0}}$ (susp-tag ((a$_{_{0}}$ a$_{_{1}}$)) _$_{_{0}}$))
 ((_$_{_{0}}$ (susp-tag ((a$_{_{1}}$ a$_{_{0}}$)) _$_{_{0}}$)) : ((a$_{_{1}}$ . _$_{_{0}}$))))
\end{schemeresponse}
The first \scheme|conde| clause fails, since \scheme|x| cannot 
be associated with both \scheme|a| and \scheme|b|.  
The second clause succeeds, associating \scheme|x| with \scheme|a| and \scheme|y|
with \scheme|b|.  The third clause applies the swap \mbox{\scheme|`(,a ,b)|}
to \mbox{\scheme|(tie a `(,a ,x))|}, yielding
\mbox{\scheme|`(tie-tag ,b (,b (susp-tag ((,a ,b)) ,x)))|}.  This term is
then unified with \mbox{\scheme|(tie b `(,b ,y))|}, 
associating \scheme|y| with the suspension
\mbox{\scheme|`(susp-tag ((,b ,a)) ,x)|}.
The fourth clause should look familiar---it is similar 
to the previous \scheme|run*| expression.

We can interpret the successful unification of binders
\mbox{\scheme|(tie a a)|} and \mbox{\scheme|(tie b b)|} 
as showing that the $\lambda$-calculus
terms $\lambda a.a$ and $\lambda b.b$ are identical, up to
$\alpha$-equivalence. We need not restrict our interpretation to
$\lambda$ terms, however, since other scoping mechanisms have similar
properties. For example, the same successful unification also shows that $\forall a . a$
and $\forall b . b$ are equivalent in first-order logic, and
similarly, that $\exists a . a$ and $\exists b . b$ are equivalent.

We can tag terms in order to disambiguate their interpretation.  For example, this program shows that $\lambda a . \lambda b . a$ and $\lambda c . \lambda d . c$ are equivalent.

\schemedisplayspace
\begin{schemedisplay}
(run* (q)
  (exist (t u)
    (fresh (a b c d)
      (== `(lam (tie-tag ,a (lam (tie-tag ,b (vartag ,a))))) t)
      (== `(lam (tie-tag ,c (lam (tie-tag ,d (vartag ,c))))) u)))) $\Rightarrow$
\end{schemedisplay}
\nspace
\begin{schemeresponse}
(_$_{_{0}}$)
\end{schemeresponse}

Of course, not all $\lambda$-calculus terms are equivalent.

\schemedisplayspace
\begin{schemedisplay}
(run* (q)
  (exist (t u)
    (fresh (a b c d)
      (== `(lam (tie-tag ,a (lam (tie-tag ,b (vartag ,a))))) t)
      (== `(lam (tie-tag ,c (lam (tie-tag ,d (vartag ,d))))) u)))) $\Rightarrow$
\end{schemedisplay}
\nspace
\begin{schemeresponse}
()
\end{schemeresponse}

\noindent Here \mbox{\scheme|(== `(lam ,t2) `(lam ,u2))|} fails, 
showing that terms $\lambda a . \lambda b . a$ and $\lambda c . \lambda d . d$ 
are not $\alpha$-equivalent.

\section{Capture-avoiding Substitution}\label{aksubst}
We now consider a simple, but useful, nominal logic program 
adapted from \citet{CheneyU04} that performs capture-avoiding
substitution (that is, $\beta$-substitution).
\scheme|substo| implements the relation \mbox{\scheme|[new/a]e = out|}
where \scheme|e|, \scheme|new|, and \scheme|out|
are tagged lists representing $\lambda$-calculus terms, and
where \scheme|a| is a nom representing a variable name.
(We refer the interested reader to \citeauthor{CheneyU04}
for a full description of \scheme|substo|.)

\schemedisplayspace
\begin{schemedisplay}
(define substo  
  (lambda (e new a out)
    (match-e `(,e ,out)
      (`((vartag ,a) ,new))
      (`((vartag ,y) (vartag ,y))
       (hash a y))
      (`((app ,rator ,rand) (app ,ratorres ,randres))
       (substo rator new a ratorres)
       (substo rand new a randres))
      (`((lam (tie-tag ,@c ,body)) (lam (tie-tag ,@c ,bodyres)))
       (hash c a)
       (hash c new)
       (substo body new a bodyres)))))
\end{schemedisplay}

The first \scheme|substo| example shows that
$[b/a]\lambda a.ab \equiv _{\alpha} \lambda c.cb$.

\schemedisplayspace
\begin{schemedisplay}
(run* (q)
  (fresh (a b)
    (substo `(lam (tie-tag ,a (app (vartag ,a) (vartag ,b)))) `(vartag ,b) a q))) $\Rightarrow$
\end{schemedisplay}
\nspace
\begin{schemeresponse}
((lam (tie-tag a$_{_{0}}$ (app (vartag a$_{_{0}}$) (vartag a$_{_{1}}$)))))
\end{schemeresponse}

\noindent Naive substitution would have produced $\lambda b.bb$ instead.

This second example shows that $[a/b]\lambda a.b \equiv _{\alpha} \lambda c.a$.

\schemedisplayspace
\begin{schemedisplay}
(run* (x)
  (fresh (a b)
    (substo `(lam (tie-tag ,a (vartag ,b))) `(vartag ,a) b x))) $\Rightarrow$
\end{schemedisplay}
\nspace
\begin{schemeresponse}
((lam (tie-tag a$_{_{0}}$ (vartag a$_{_{1}}$))))
\end{schemeresponse}

\noindent Naive substitution would have produced $\lambda a.a$.


\section{Type Inferencer}\label{aktypeinf}

Let us consider a second non-trivial \alphakanrensp example: a type
inferencer for a subset of Scheme\footnote{This program is an extended
  and adapted version of the inferencer for the simply-type
  $\lambda$-calculus presented in \citet{CheneyU04}.}.  We begin with
the typing rule for integer constants, which are tagged with the
symbol \scheme|'intc|.

\schemedisplayspace
\begin{schemedisplay}
(define int-rel
  (lambda (g exp t)
    (exist (n)
      (== `(intc ,n) exp)
      (== 'int t))))
\end{schemedisplay}

\noindent The \scheme{!-} relation\footnote{\scheme{!-} is entered as
  {\tt !-} and is pronounced ``turnstile''.}  relates an expression
\scheme{exp} to its type \scheme{t} in the type environment
\scheme{g}.

\schemedisplayspace
\begin{schemedisplay}
(define !-
  (lambda (g exp t)
    (conde
      ((int-rel g exp t)))))
\end{schemedisplay}

\noindent We can now infer the types of integer constants: 
\mbox{\scheme|(run* (q) (!- '() '(intc 5) q))|} returns \scheme|'(int)|.

Inferring the types of integer constants is not very interesting.  We
therefore add typing rules for variables, $\lambda$ expressions, and
application.

\schemedisplayspace
\begin{schemedisplay}
(define var-rel
  (lambda (g exp t)
    (exist (x)
      (== `(vartag ,x) exp)
      (lookupo x t g))))
\end{schemedisplay}

\begin{schemedisplay}
(define lambda-rel
  (lambda (g exp t)
    (exist (body trand tbody)
      (fresh (a)
        (== `(lam ,(tie a body)) exp)
        (== `(--> ,trand ,tbody) t)
        (!- `((,a . ,trand) . ,g) body tbody)))))

(define app-rel
  (lambda (g exp t)
    (exist (rator rand trand)
      (== `(app ,rator ,rand) exp)
      (!- g rator `(--> ,trand ,t))
      (!- g rand trand))))
\end{schemedisplay}

\noindent The \scheme{lookupo} helper relation finds the type
\scheme{tx} associated with the type variable \scheme{x}
in the current type environment \scheme{g}.

\schemedisplayspace
\begin{schemedisplay}
(define lookupo
  (lambda (x tx g)
    (exist (a d)
      (== `(,a . ,d) g)
      (conde
        ((== `(,x . ,tx) a))
        ((exist (x^ tx^)
           (== `(,x^ . ,tx^) a)
           (hash x x^)
           (lookupo x tx d)))))))
\end{schemedisplay}

\noindent We redefine \scheme|!-| to include the new typing rules.

\schemedisplayspace
\begin{schemedisplay}
(define !-
  (lambda (g exp t)
    (conde
      ((var-rel g exp t))
      ((int-rel g exp t))
      ((lambda-rel g exp t))
      ((app-rel g exp t)))))
\end{schemedisplay}

We can now show that \scheme|(lambda (x) (lambda (y) x))|
has type \mbox{\scheme|($\alpha$ --> ($\beta$ --> $\alpha$))|}.

\wspace

\noindent\scheme|(run* (q) (!- '() (parse '(lambda (x) (lambda (y) x))) q))| $\Rightarrow$ \begin{schemeresponsebox}((--> _.0 (--> _.1 _.0)))\end{schemeresponsebox}

\wspace

\noindent Here we use the parser from Appendix~\ref{akinferparser} to make the code more readable.

The next example shows that self-application doesn't type check, since the nominal unifier 
uses the occurs check \cite{lloyd:lp}.

\wspace

\noindent\scheme|(run* (q) (!- '() (parse '(lambda (x) (x x))) q))| $\Rightarrow$ \begin{schemeresponsebox}()\end{schemeresponsebox}

\wspace

This example is more interesting, since it searches for expressions
that \emph{inhabit} the type \mbox{\schemeresult|(--> int int)|}.

\schemedisplayspace
\begin{schemedisplay}
(run5 (q) (!- '() q '(--> int int))) $\Rightarrow$
\end{schemedisplay}
\nspace
\begin{schemeresponse}
((lam (tie-tag a.0 (intc _.0)))
 (lam (tie-tag a.0 (vartag a.0)))
 (lam (tie-tag a.0 (app (lam (tie-tag a.1 (intc _.0))) (intc _.1))))
 (lam (tie-tag a.0 (app (lam (tie-tag a.1 (intc _.0))) (vartag a.0))))
 (app (lam (tie-tag a.0 (vartag a.0))) (lam (tie-tag a.1 (intc _.0)))))
\end{schemeresponse}

\noindent These expressions are equivalent to (in order)

\enlargethispage{1em}

\wspace

\noindent\scheme|(lambda (x) n-const)|

\noindent\scheme|(lambda (x) x)|

\noindent\scheme|(lambda (x) ((lambda (y) n-const) m-const))|

\noindent\scheme|(lambda (x) ((lambda (y) n-const) x))|

\noindent\scheme|((lambda (x) x) (lambda (y) n-const))|

\wspace

\noindent where \scheme|n-const| and \scheme|m-const| are some integer constants.
Each expression inhabits the type \mbox{\schemeresult|(int --> int)|}, although the principal type
of the expression is either
\mbox{\schemeresult|($\alpha$ --> $\alpha$)|} (for the identity function) or
\mbox{\schemeresult|($\alpha$ --> int)|} (for the remaining expressions).

We now extend the language even further, adding boolean constants,
\scheme|zero?|, \scheme|sub1|, multiplication,
\scheme|if|-expressions, and a fixed-point operator for defining
recursive functions.

\schemedisplayspace
\begin{schemedisplay}
(define bool-rel
  (lambda (g exp t)
    (exist (b)
      (== `(boolc ,b) exp)
      (== 'bool t))))

(define zero?-rel
  (lambda (g exp t)
    (exist (e)
      (== `(zero? ,e) exp)
      (== 'bool t)
      (!- g e 'int))))

(define sub1-rel
  (lambda (g exp t)
    (exist (e)
      (== `(sub1 ,e) exp)
      (== t 'int)
      (!- g e 'int))))

 (define *-rel
  (lambda (g exp t)
    (exist (e1 e2)
      (== `(* ,e1 ,e2) exp)
      (== t 'int)
      (!- g e1 'int)
      (!- g e2 'int))))

(define if-rel
  (lambda (g exp t)
    (exist (test conseq alt)
      (== `(if ,test ,conseq ,alt) exp)
      (!- g test 'bool)
      (!- g conseq t)
      (!- g alt t))))
\end{schemedisplay}

\begin{schemedisplay}
(define fix-rel
  (lambda (g exp t)
    (exist (rand)
      (== `(fix ,rand) exp)
      (!- g rand `(--> ,t ,t)))))
\end{schemedisplay}

We redefine \scheme|!-| one last time.

\enlargethispage{2em}

\schemedisplayspace
\begin{schemedisplay}
(define !-
  (lambda (g exp t)
    (conde
      ((var-rel g exp t))
      ((int-rel g exp t))
      ((bool-rel g exp t))
      ((zero?-rel g exp t))
      ((sub1-rel g exp t))
      ((fix-rel g exp t))
      ((*-rel g exp t))
      ((lambda-rel g exp t))
      ((app-rel g exp t))
      ((if-rel g exp t)))))
\end{schemedisplay}

We can now infer the type of the factorial function.

\schemedisplayspace
\begin{schemedisplay}
(run* (q)
  (!- '() (parse '((fix (lambda (!)
                          (lambda (n)
                            (if (zero? n)
                                1
                                (* (! (sub1 n)) n))))) 5))
      q)) $\Rightarrow$
\end{schemedisplay}
\nspace
\begin{schemeresponse}
(int)
\end{schemeresponse}

We can also generate pairs of expressions and their types.

\schemedisplayspace
\begin{schemedisplay}
(run13 (q)
  (exist (exp t)
    (!- '() exp t)
    (== `(,exp ,t) q))) $\Rightarrow$
\end{schemedisplay}
\nspace
\begin{schemeresponse}
(((intc _.0) int)
 ((boolc _.0) bool)
 ((zero? (intc _.0)) bool)
 ((sub1 (intc _.0)) int)
 ((zero? (sub1 (intc _.0))) bool)
 ((sub1 (sub1 (intc _.0))) int)
 ((zero? (sub1 (sub1 (intc _.0)))) bool)
 ((sub1 (sub1 (sub1 (intc _.0)))) int)
 ((zero? (sub1 (sub1 (sub1 (intc _.0))))) bool)
 ((* (intc _.0) (intc _.1)) int)
 ((lam (tie-tag a.0 (intc _.0))) (--> _.1 int))
 ((zero? (* (intc _.0) (intc _.1))) bool)
 ((lam (tie-tag a.0 (vartag a.0))) (--> _.0 _.0)))
\end{schemeresponse}

\noindent For example, the last answer shows that the identity
function has type \mbox{\schemeresult|($\alpha$ --> $\alpha$)|}.

This ends the introduction to \alphakanren.  For additional simple examples of
nominal logic programming, we suggest \citet{cheneyurban08}, \citet{CheneyThesis},
\citet{CheneyU04}, \citet{Urban-Pitts-Gabbay/04}, and
\citet{lakin2007}, which are also excellent choices for understanding
the theory of nominal logic.


\chapter{Applications II:  \alphatap}\label{alphatapchapter}

In this chapter we examine a second application of nominal logic
programming, a declarative theorem prover for first-order classical
logic. We call this prover \alphatap, since it is based on the
\leantapsp\cite{beckert95leantap} prover and written in
\alphakanren. Our prover is a relation, without mode restrictions;
given a logic variable as the theorem to be proved, \alphatapsp
\textit{generates} valid theorems.

\leantapsp is a lean tableau-based theorem prover for first-order
logic due to \citet{beckert95leantap}.  Written in
Prolog, it is extremely concise and is capable of a high rate of
inference. \leantapsp uses Prolog's cut (\texttt{!}) in three of its
five clauses in order to avoid nondeterminism, and uses
\mbox{\texttt{copy\_term/2}} to make copies of universally quantified
formulas. Although Beckert and Posegga take advantage of Prolog's
unification and backtracking features, their use of the impure cut and
\mbox{\texttt{copy\_term/2}} makes \leantapsp non-declarative.

% : reordering goals within the prover may cause divergence.

%% new definition of nondeclarative?

In this chapter we translate \leantapsp from Prolog to impure
miniKanren, using \scheme|match-a| to mimic Prolog's cut, and
\scheme|copy-termo| to mimic \mbox{\texttt{copy\_term/2}}.  We then show how
to eliminate these impure operators from our translation. To eliminate the
use of \scheme|match-a|, we introduce a tagging scheme that makes our
formulas unambiguous.  To eliminate the use of \scheme|copy-termo|, we
use substitution instead of copying terms.  Universally quantified
formulas are used as templates, rather than instantiated directly;
instead of representing universally quantified variables with logic
variables, we use the noms of nominal logic. We then use nominal
unification to write a substitution relation that replaces quantified
variables with logic variables, leaving the original template
untouched.

The resulting declarative theorem prover is interesting for two
reasons. First, because of the technique used to arrive at its
definition: we use declarative substitution rather than
\scheme|copy-termo|.  To our knowledge, there is no method for
copying arbitrary terms declaratively. Our solution is not completely
general but is useful when a term is used as a template for copying,
as in the case of \leantap.  Second, because of the flexibility of the
prover itself: \alphatapsp is capable of instantiating non-ground
theorems during the proof process, and accepts non-ground
\textit{proofs}, as well.  Whereas \leantapsp is fully automated and
either succeeds or fails to prove a given theorem, \alphatapsp can
accept guidance from the user in the form of a partially-instantiated
proof, regardless of whether the theorem is ground.

We present an implementation of \alphatapsp in
section~\ref{implementation} , demonstrating our technique for
eliminating cut and \mbox{\texttt{copy\_term/2}} from \leantap. Our
implementation demonstrates our contributions: first, it illustrates a
method for eliminating common impure operators, and demonstrates the
use of nominal logic for representing formulas in first-order logic;
second, it shows that the tableau process can be represented as a
relation between formulas and their tableaux; and third, it
demonstrates the flexibility of relational provers to mimic the full
spectrum of theorem provers, from fully automated to fully dependent
on the user.

This chapter is organized as follows. In section~\ref{tableau} we
describe the concept of tableau theorem proving. In
section~\ref{alphatap} we motivate our declarative prover by examining
its declarative properties and the proofs it returns. In
section~\ref{implementation} we present the implementation of
\alphatap, and in section~\ref{performance} we briefly examine
\alphatap's performance. Familiarity with tableau theorem proving
would be helpful; for more on this topic, see the references given in
section~\ref{tableau}.  In addition, a reading knowledge of Prolog
would be useful, but is not necessary; for readers unfamiliar with
Prolog, carefully following the miniKanren and \alphakanrensp code
should be sufficient for understanding all the ideas in this chapter.

\section{Tableau Theorem Proving}\label{tableau}

We begin with an introduction to tableau theorem proving and its
implementation in \leantap.


Tableau is a method of proving first-order theorems that works by
refuting the theorem's negation. In our description we assume basic
knowledge of first-order logic; for coverage of this subject and a
more complete description of tableau proving, see
\citet{fitting1996fol}.  For simplicity, we consider only
formulas in Skolemized \textit{negation normal form} (NNF).
Converting a formula to this form requires removing existential
quantifiers through Skolemization, reducing logical connectives so
that only $\wedge$, $\vee$, and $\neg$ remain, and pushing negations
inward until they are applied only to literals---see section~3 of
\citet{beckert95leantap} for details.

To form a tableau, a compound formula is expanded into branches
recursively until no compound formulas remain.  The leaves of this
tree structure are referred to as \textit{literals}. \leantapsp forms
and expands the tableau according to the following rules. When the
prover encounters a conjunction $x \wedge y$, it expands both $x$ and
$y$ on the same branch. When the prover encounters a disjunction $x
\vee y$, it splits the tableau and expands $x$ and $y$ on separate
branches.  Once a formula has been fully expanded into a tableau, it
can be proved unsatisfiable if on each branch of the tableau there
exist two complementary literals $a$ and $\neg a$ (each branch is
\textit{closed}).  In the case of propositional logic, syntactic
comparison is sufficient to find complementary literals; in
first-order logic, sound unification must be used. A closed tableau
represents a proof that the original formula is unsatisfiable.

 The addition of universal quantifiers makes the expansion process more
 complicated. To prove a universally quantified formula \mbox{$\forall x. M$}, 
 \leantapsp generates a logic variable $v$ and expands $M$,
 replacing all occurrences of $x$ with $v$ (i.e., it expands $M^{\prime}$ where
 $M^{\prime} = M[v/x]$).  If \leantapsp is unable to close the current branch
 after this expansion, it has the option of generating another logic
 variable and expanding the original formula again. When the prover
 expands the universally quantified formula \mbox{$\forall x.  F(x) \wedge ( \neg F({\sf a})
   \vee \neg F({\sf b}) )$}, for example, \mbox{$\forall x.  F(x)$}
 must be expanded twice, since $x$ cannot be instantiated to both
 \textsf{a} and \textsf{b}.

\section{Introducing \alphatap}\label{alphatap}

We begin by presenting some examples of \alphatap's abilities, both in
proving ground theorems and in generating theorems. We also explore
the proofs generated by \alphatap, and show how passing
partially-instantiated proofs to the prover can greatly improve its
performance.

\subsection{Running Forwards}\label{forwards}

Both \leantapsp and \alphatapsp can prove ground theorems; in
addition, \alphatap\ produces a proof.  This proof is a list
representing the steps taken to build a closed tableau for the
theorem; \citet{paulson99generic} has shown that translation to
a more standard format is possible. Since a closed tableau represents
an unsatisfiable formula, such a list of steps proves that the
negation of the formula is valid. If the list of steps is ground, the
proof search becomes deterministic, and \alphatapsp acts as a proof
checker.

\leantapsp encodes first-order formulas using Prolog terms.  For
example, the term \mbox{\texttt{(p(b),all(X,(-p(X);p(s(X)))))}}
represents \mbox{$p($\textsf{b}$) \wedge \forall x . \neg p(x) \vee
  p(s(x))$}. In our prover, we represent formulas using Scheme lists
with extra tags:

%, and in our final version we adopt a more extensive tagging
%scheme. The \schemeresult|forall| binder is represented by
%\alphakanren's \scheme|tie|, and variables are represented by noms.
%Our example formula is represented by the ground list:

\schemedisplayspace
\begin{schemeresponse}
(and-tag (pos (app p (app b))) (forall (tie anom (or-tag (neg (app p (var-tag anom))) 
                                                     (pos (app p (app s (var-tag anom))))))))

\end{schemeresponse}

% The Prolog query \mbox{\texttt{prove(Fml,[],[],[],VarLim)}} succeeds
% if the formula \texttt{Fml} is unsatisfiable.  Similarly, the
% \alphakanrensp goal \mbox{\scheme|(proveo fml '() '() '() proof)|}
% succeeds if \scheme|fml| can be shown to be unsatisfiable via the
% proof \scheme|proof|.

Consider Pelletier Problem 18~\cite{pelletier1986sfp}: \mbox{$\exists
  y.  \forall x. F(y) \Rightarrow F(x)$}. To prove this theorem in
\alphatap, we transform it into the following \textit{negation} of the
NNF:

\schemedisplayspace
\begin{schemeresponse}
(forall (tie anom (and-tag (pos (app f (var-tag anom))) (neg (app f (app g1 (var-tag anom)))))))
\end{schemeresponse}

\noindent where \schemeresult|`(app ,g1 (var-tag anom))| represents the
application of a Skolem function to the universally quantified
variable $a$. Passing this formula to the prover, we obtain the proof
\schemeresult|`(univ conj savefml savefml univ conj close)|. This proof
lists the steps the prover (presented in section~\ref{matcha}) follows to close
the tableau. Because both conjuncts of the formula contain the nom
$a$, we must expand the universally quantified formula more than once.

Partially instantiating the proof helps \alphatapsp prove theorems
with similar subparts. We can create a non-ground proof that describes
in general how to prove the subparts and have \alphatapsp fill in the
trivial differences. This can speed up the search for a proof
considerably. By inspecting the negated NNF of Pelletier Problem~21,
for example, we can see that there are at least two portions of the
theorem that will have the same proof. By specifying the structure of
the first part of the proof and constraining the identical portions by
using the same logic variable to represent both, we can give the
prover some guidance without specifying the whole proof. We pass the
following non-ground proof to \alphatap:

\schemedisplayspace
\vspace{-2pt}
\begin{centering}
\begin{schemeresponse}
(conj univ split (conj savefml savefml conj split Xvar Xvar)
      (conj savefml savefml conj split (close) (savefml split Yvar Yvar)))
\end{schemeresponse}
\end{centering}
\vspace{-2pt}

\noindent On our test machine, our prover solves the original problem
with no help in 68 milliseconds (ms); given the knowledge that the
later parts of the proof will be duplicated, the prover takes only 27
ms. This technique also yields improvement when applied to Pelletier
Problem 43: inspecting the negated NNF of the formula, we see two
parts that look nearly identical. The first part of the negated
NNF---the part representing the theorem itself---has the following
form:

\schemedisplayspace
\vspace{-2pt}
\begin{centering}
\begin{schemeresponse}
(and-tag (or-tag (and-tag (neg (app Q (app g4) (app g3)))
              (pos (app Q (app g3) (app g4))))
         (and-tag (pos (app Q (app g4) (app g3)))
              (neg (app Q (app g3) (app g4))))) ...)
\end{schemeresponse}
\end{centering}
\vspace{-2pt}

\noindent Since we suspect that the same proof might suffice for both
branches of the theorem, we give the prover the partially-instantiated
proof \mbox{\schemeresult|`(conj split Xvar Xvar)|}. Given just this
small amount of help, \alphatapsp proves the theorem in 720 ms,
compared to 1.5 seconds when the prover has no help at all.  While
situations in which large parts of a proof are identical are rare,
this technique also allows us to handle situations in which different
parts of a proof are merely similar by instantiating as much or as
little of the proof as necessary.

\subsection{Running Backwards}\label{backwards}

% \begin{figure}[H]
% \begin{centering}
% \begin{tabular}{| r | c | c | c | c |}
%   \hline 
%   Problem & \thinspace \leantap \thinspace\footnotemark[4] \thinspace &
%   Translation\footnotemark[3] & \thinspace \alphatap\footnotemark[3]
%   \thinspace & \thinspace \alphatap$\!_G$\footnotemark[4]$^,$\footnotemark[6]  \\
%   \hline
%   1 & ? & 

%   \hline
% \end{tabular}
% \caption{\alphatap's Performance on Pelletier's Problems\protect\footnotemark[2]
%   \label{fig:performance}}
% \end{centering}
% \end{figure}



%\vspace{-6pt}

%  Testing our prover on
% several of Pelletier's 75 problems~\cite{pelletier1986sfp} shows that
% \alphatapsp is about three to five times slower than our translation
% of \leantap. The translation solves problem 32, for example, in about
% one second, while \alphatapsp takes about three
% seconds; problem 26 takes our translation of \leantapsp about 13
% seconds, while \alphatapsp needs 36 seconds.

Unlike \leantap, \alphatapsp can generate valid theorems.  Some
interpretation of the results is required since the theorems generated
are negated formulas in NNF.\footnote{The full implementation of
  \alphatapsp includes a simple declarative translator from negated
  NNF to a positive form.}  In the example

\smallskip

\scheme|(run1 (q) (exist (x) (proveo q '() '() '() x)))|

\hspace{0.1cm}$\Rightarrow$
\schemeresult|`((and-tag (pos (app _.0)) (neg (app _.0))))|

\smallskip

\noindent 
the reified logic variable \schemeresult|_.0| represents any
first-order formula $p$, and the entire answer represents the formula
$p \wedge \neg p$.  Negating this formula yields the original theorem:
$\neg p \vee p$, or the law of excluded middle.  We can also generate
more complicated theorems; here we use the ``generate and test'' idiom
to find the first theorem matching the negated NNF of the inference
rule {\it modus ponens}:

\schemedisplayspace
\begin{schemedisplay}
(run1 (q)
  (exist (x)
    (proveo x '() '() '() q)
    (== `(and-tag (and-tag (or-tag (neg (app a)) (pos (app b))) (pos (app a))) (neg (app b)))
        x)))
\end{schemedisplay}
 \vspace{-.1cm}
\noindent $\Rightarrow$ \schemeresult|`((conj conj split (savefml close) (savefml savefml close)))|

\smallskip

\noindent This process takes about 5.1 seconds; {\it modus ponens} is the
173rd theorem to be generated, and the prover also generates a proof
of its validity. When this proof is given to \alphatap, {\it modus ponens}
is the sixth theorem generated, and the process takes only 20 ms.

Thus the declarative nature of \alphatapsp is useful both for
generating theorems and for producing proofs. Due to this flexibility,
\alphatapsp could become the core of a larger proof system.  Automated
theorem provers like \leantapsp are limited in the complexity of the
problems they can solve, but given the ability to accept assistance
from the user, more problems become tractable.

%can solve more difficult problems.


%\footnotetext[7]{\alphatap$\!_G$ uses the unique name and preprocessor
%  approach described in section 4.2.}

As an example, consider Pelletier Problem 47: Schubert's Steamroller.
This problem is difficult for tableau-based provers like \leantapsp
and \alphatap, and neither can solve it
automatically~\cite{beckert95leantap}.  Given some help, however,
\alphatapsp can prove the Steamroller. Our approach is to prove a
series of smaller lemmas that act as stepping stones toward the final
theorem; as each lemma is proved, it is added as an assumption in
proving the remaining ones.  The proof process is automated---the user
need only specify which lemmas to prove and in what order. Using this
strategy, \alphatapsp proves the Steamroller in about five seconds;
the proof requires twenty lemmas.


\alphatapsp thus offers an interesting compromise between large proof
assistants and smaller automated provers. It achieves some of the
capabilities of a larger system while maintaining the lean deduction
philosophy introduced by \leantap. Like an automated prover, it is
capable of proving simple theorems without user guidance. Confronted
with a more complex theorem, however, the user can provide a
partially-instantiated proof; \alphatapsp can then check the proof and
fill in the trivial parts the user has left out.  Because \alphatapsp
is declarative, the user may even leave required axioms out of the
theorem to be proved and have the system derive them. This flexibility
comes at no extra cost to the user---the prover remains both concise
and reasonably efficient.

%% New

The flexibility of \alphatapsp means that it could be made interactive
through the addition of a read-eval-print loop and a simple proof
translator between \alphatap's proofs and a more human-readable
format. Since the proof given to \alphatapsp may be partially
instantiated, such an interface would allow the user to conveniently
guide \alphatapsp in proving complex problems. With the addition of
equality and the ability to perform single beta steps, this
flexibility would become more interesting---in addition to reasoning
about programs and proving properties about them, \alphatapsp would
instantiate non-ground programs during the proof process.




\section{Implementation}\label{implementation}

We now present the implementation of \alphatap. We begin with a
translation of \leantapsp from Prolog into \alphakanren. We then show
how to eliminate the translation's impure features through a
combination of substitution and tagging.


\leantapsp implements both expansion and closing of the tableau. When
the prover encounters a conjunction, it uses its argument
\texttt{UnExp} as a stack (Figure~\ref{fig:translation}): \leantapsp
expands the first conjunct, pushing the second onto the stack for
later expansion. If the first conjunct cannot be refuted, the second
is popped off the stack and expansion begins again.  When a
disjunction is encountered, the split in the tableau is reflected by
two recursive calls. When a universal quantifier is encountered, the
quantified variable is replaced by a new logic variable, and the
formula is expanded.  The \texttt{FreeV} argument is used to avoid
replacing the free variables of the formula.  \leantapsp keeps a list
of the literals it has encountered on the current branch of the
tableau in the argument \texttt{Lits}.  When a literal is encountered,
\leantapsp attempts to unify its negation with each literal in
\texttt{Lits}; if any unification succeeds, the branch is closed.
Otherwise, the current literal is added to \texttt{Lits} and expansion
continues with a formula from \texttt{UnExp}.


\subsection{Translation to \alphakanren}\label{translation}

While \alphakanrensp is similar to Prolog with the addition of nominal
unification, \alphakanrensp uses a variant of interleaving
depth-first search~\cite{backtracking}, so the order of
\scheme|conde| or \scheme|match-e| clauses in \alphakanrensp is irrelevant. Because of
Prolog's depth-first search, \leantapsp must use \texttt{VarLim} to
limit its search depth; in \alphakanren, \texttt{VarLim} is not
necessary, and thus we omit it.


In Figure~\ref{fig:translation} we present mK\leantap, our translation
of \leantapsp into \alphakanren; we label two clauses (\onet, \twot),
since we will modify these clauses later. To express Prolog's cuts,
our definition uses \scheme|match-a|.  The final two clauses of
\leantapsp do not contain Prolog cuts; in mK\leantap, they are
combined into a single clause containing a \scheme|conde|.  In place
of \leantap\thinspace's recursive call to \texttt{prove} to check the
membership of \texttt{Lit} in \texttt{Lits}, we call \scheme|membero|,
which performs a membership check using sound unification.\footnote{We define \scheme|membero| in Figure~\ref{fig:ending}; \scheme|membero| \emph{must} use sound unification, and cannot use \scheme|==-no-check|.}  % Prolog's \texttt{copy\_term/2} is
% not built into \alphakanren; this addition is available as part of the
% mK\leantapsp source code.


%\begin{figure}[ht]
\begin{figure}[H]
%\vspace{-.3in}

\begin{tabular}{l l}

 &

\begin{minipage}{2.3in}
\begin{schemedisplay}
 (define proveo
   (lambda (fml unexp lits freev)
     (match-a fml
\end{schemedisplay}
\end{minipage} \\


\begin{minipage}{2.3in}
\begin{verbatim}
prove((E1,E2),UnExp,Lits,
      FreeV,VarLim) :- !,
  prove(E1,[E2|UnExp],Lits,
        FreeV,VarLim).
\end{verbatim}
\end{minipage}
 &
\begin{minipage}{2in}
\begin{schemedisplay}
      (`(and-tag ,e1 ,e2)
        (proveo e1 `(,e2 . ,unexp) lits freev))
\end{schemedisplay}
\end{minipage}
\\

\begin{minipage}{2in}
\begin{verbatim}
prove((E1;E2),UnExp,Lits,
      FreeV,VarLim) :- !,
  prove(E1,UnExp,Lits,FreeV,VarLim),
  prove(E2,UnExp,Lits,FreeV,Varlim).
\end{verbatim}
\end{minipage}
 &
\begin{minipage}{2in}
\vspace{1mm}
\begin{schemedisplay}
      (`(or-tag ,e1 ,e2)
        (proveo e1 unexp lits freev)
        (proveo e2 unexp lits freev))
\end{schemedisplay}
\vspace{1mm}
\end{minipage}
\\

\begin{minipage}{2in}
\begin{verbatim}
prove(all(X,Fml),UnExp,Lits,
      FreeV,VarLim) :- !,
  \+ length(FreeV,VarLim),
  copy_term((X,Fml,FreeV),
            (X1,Fml1,FreeV)),
  append(UnExp,[all(X,Fml)],UnExp1),
  prove(Fml1,UnExp1,Lits,
        [X1|FreeV],VarLim).
\end{verbatim}
\end{minipage}
 &
\begin{minipage}{2in}
\begin{schemedisplay}
     $\onet$(`(forall ,x ,body)
         (exist (x1 body1 unexp1)
           (copy-termo `(,x ,body ,freev) 
                       `(,x1 ,body1 ,freev))
           (appendo unexp `(,fml) unexp1)
           (proveo body1 unexp1 lits 
                   `(,x1 . ,freev))))
\end{schemedisplay}
\end{minipage}
\\

\begin{minipage}{2in}
\begin{verbatim}
prove(Lit,_,[L|Lits],_,_) :-
  (Lit = -Neg; -Lit = Neg) ->
   (unify(Neg,L); 
    prove(Lit,[],Lits,_,_)).
\end{verbatim}

\end{minipage}
 &
\begin{minipage}{2in}
\begin{schemedisplay}
     $\twot$(fml
         (conde
           ((match-a `(,fml ,neg)
              (`((not ,neg) ,neg))
              (`(,fml (not ,fml))))
            (membero neg lits))
           \end{schemedisplay}
           \end{minipage}
           \\

           \begin{minipage}{2in}
           \begin{verbatim}
           prove(Lit,[Next|UnExp],Lits,
                    FreeV,VarLim) :-
           prove(Next,UnExp,[Lit|Lits],
                           FreeV,VarLim).
           \end{verbatim} 
           \end{minipage}
           &
           \begin{minipage}{2in}
           \begin{schemedisplay}
        ((exist (next unexp1)
           (== `(,next . ,unexp1) unexp)
           (proveo next unexp1 `(,fml . ,lits) 
                   freev))))))))
\end{schemedisplay}
\end{minipage}
\\


\end{tabular}
\caption{\leantapsp and mK\leantap\thinspace: a translation from Prolog to \alphakanren
  \label{fig:translation}}
%\vspace{-.3in}
\end{figure}


\subsection{Eliminating \copytermo}\label{copytermo}
\enlargethispage{1\baselineskip} %

Since \scheme|copy-termo| is an impure operator, its use makes
\scheme|proveo| non-declarative: reordering the goals in the prover
can result in different behavior. For example, moving the call to
\scheme|copy-termo| after the call to \scheme|proveo| causes the
prover to diverge when given any universally quantified formula. To
make our prover declarative, we must eliminate the use of
\scheme|copy-termo|.

Tagging the logic variables that represent universally quantified
variables allows the use of a declarative technique that creates two
pristine copies of the original term: one copy may be expanded and the
other saved for later copying.  Unfortunately, this copying examines
the entire body of each quantified formula and instantiates the
original term to a potentially invalid formula.

Another approach is to represent quantified variables with symbols or
strings. When a new instantiation is needed, a new variable name can
be generated, and the new name can be substituted for the old without
affecting the original formula. This solution does not destroy the
prover's input, but it is difficult to ensure that the provided data
is in the correct form declaratively: if the formula to be proved is
non-ground, then the prover must generate unique names.  If the
formula \textit{does} contain these names, however, the prover must
\textit{not} generate new ones. This problem can be solved with a
declarative preprocessor that expects a logical formula
\textit{without} names and puts them in place. If the preprocessor is
passed a non-ground formula, it instantiates the formula to the
correct form. %We have implemented this strategy in a Prolog prover we
%call \alphatap$\!_G$; 
The requirement of a preprocessor, however,
means the prover itself is not declarative.

We use nominal logic to solve the \scheme|copy-termo| problem.
Nominal logic is a good fit for this problem, as it is designed to
handle the complexities of dealing with names and binders
declaratively.
%Using
%noms to represent universally quantified variables and the
%\scheme|tie| operator to represent the $\forall$ binder allows us to
%avoid the use of logic variables to represent quantified variables.
Since noms represent unique names, we achieve the benefits of the
symbol or string approach without the use of a preprocessor. We can
generate unique names each time we encounter a universally quantified
formula, and use nominal unification to perform the renaming of the
quantified variable. If the original formula is uninstantiated, our
newly-generated name is unique and is put in place correctly; we no
longer need a preprocessor to perform this function.

Using the tools of nominal logic, we can modify mK\leantapsp to
represent universally quantified variables using noms and to perform
substitution instead of copying.  When the prover reaches a literal,
however, it must replace each nom with a logic variable, so that
unification may successfully compare literals. To accomplish this, we
associate a logic variable with each unique nom, and replace every nom
with its associated variable before comparing literals. These
variables are generated each time the prover expands a quantified
formula.

To implement this strategy, we change our representation of formulas
slightly. Instead of representing $\forall x. F(x)$ as
\mbox{\schemeresult|`(forall Xvar (f Xvar))|}, we use a nom wrapped in
a \scheme|var-tag| tag to represent a variable reference, and the
term constructor \scheme|tie| to represent the $\forall$ binder:
\mbox{\schemeresult|`(forall (tie anom (f (var-tag anom))))|}, where $a$ is
a nom.  The \scheme|var-tag| tag allows us to distinguish noms
representing variables from other formulas. We now write a relation
\scheme|subst-lito| to perform substitution of logic variables for
tagged noms in a literal, and we modify the literal case of
\scheme|proveo| to use it. We also replace the clause handling
\schemeresult|forall| formulas and define \scheme|lookupo|. The two
clauses of \scheme|lookupo| overlap, but since each mapping in the
environment is from a unique nom to a logic variable, a particular nom
will never appear twice.

We present the changes needed to eliminate \scheme|copy-termo| from
mK\leantapsp in Figure~\ref{fig:changes}. Instead of copying the body
of each universally quantified formula, we generate a logic variable
\scheme|x| and add an association between the nom representing the
quantified variable and \scheme|x| to the current environment. When we
prepare to close a branch of the tableau, we call \scheme|subst-lito|,
replacing the noms in the current literal with their associated logic
variables.


\begin{figure}[H]

\noindent \begin{tabular}{l l}
\begin{minipage}{2.5in}
\small
\begin{schemedisplay}
$\onet$(`(forall (tie-tag ,@a ,body))
   (exist (x unexp1)
     (appendo unexp `(,fml) unexp1)
     (proveo body unexp1 lits
             `((,a . ,x) . ,env))))

$\twot$(fml
  (exist (lit)
    (subst-lito fml env lit)
    (conde
      ((match-a `(,lit ,neg)
         (`((not ,neg) ,neg))
         (`(,lit (not ,lit))))
       (membero neg lits))
      ((exist (next unexp1)
         (== `(,next . ,unexp1) unexp)
         (proveo next unexp1 `(,lit . ,lits) 
                 env))))))
\end{schemedisplay}

\vspace{.1cm}
\end{minipage}
&


\begin{minipage}{1.2in}
\small
%\schemeinput{code/lookupo}
\begin{schemedisplay}
(define lookupo
  (lambda (a env out)
    (match-e env
      (`((,a . ,out) . ,rest))
      (`(,first . ,rest)
       (lookupo a rest out)))))
\end{schemedisplay}

\begin{schemedisplay}
(define subst-lito
  (lambda (fml env out)
    (match-a `(,fml ,out)
      (`((var-tag ,a) ,out)
       (lookupo a env out))
      (`((,e1 . ,e2) (,r1 . ,r2))
       (subst-lito e1 env r1)
       (subst-lito e2 env r2))
      (`(,fml ,fml)))))
\end{schemedisplay}

\end{minipage}

\end{tabular}

\caption{Changes to mK\leantapsp to eliminate \protect\scheme|copy-termo|
  \label{fig:changes}}
%\vspace{-.2in}
\end{figure}

The original \mbox{\texttt{copy\_term/2}} approach used by \leantapsp and
mK\leantapsp avoids replacing free variables by copying the list
\scheme|`(,x ,body ,freev)|. The copied version is unified with the list
\scheme|`(x1 body1 ,freev)|, so that \textit{only} the variable
\scheme|x| will be replaced by a new logic variable---the free
variables will be copied, but those copies will be unified with the
original variables afterwards. Since our substitution strategy does
not affect free variables, the \scheme|freev| argument is no longer
needed, and so we have eliminated it.


\subsection{Eliminating \matchasymbol}\label{matcha}

Both \scheme|proveo| and \scheme|subst-lito| use \scheme|match-a|
because the clauses that recognize literals overlap with the other
clauses. To solve this problem, we have designed a tagging scheme that
ensures that the clauses of our substitution and \scheme|proveo|
relations do not overlap.  To this end, we tag both positive and
negative literals, applications, and variables. Constants are
represented by applications of zero arguments. Our prover thus accepts
formulas of the following form:


% \begin{center}
%   \begin{tabular}{lcl}
%     $<$Fml$>$ & $\rightarrow$ & $($\textsf{or} $<$Fml$>$ $<$Fml$>)$ 
% % \\ & $|$ & 
% $|$ $($\textsf{and} $<$Fml$>$ $<$Fml$>)$ 
%  \\ & $|$ & 
% $($\textsf{forall} $<$nom$>$ $<$Fml$>)$ 
% % \\ & $|$ & 
% $|$ $($\textsf{lit} $<$Lit$>)$ 
% \\
%     $<$Lit$>$ & $\rightarrow$ & $($\textsf{pos} $<$Term$>)$ 
% % \\ & $|$ & 
% $|$ $($\textsf{neg} $<$Term$>)$ 
%  \\ 
% $<$Term$>$ & $\rightarrow$ & $($\textsf{sym} $<$symbol$>)$ 
% % \\ & $|$ & 
% $|$ $($\textsf{var} $<$nom$>)$ 
% % \\ & $|$ & 
% $|$ $($\textsf{app} $<$symbol$>$ $<$Term$>$*$)$ \\
%   \end{tabular}
% \end{center}

% \begin{center}
%   \begin{tabular}{lcl}
%     Fml & $\rightarrow$ & $($\textsf{and} Fml Fml$)$ 
% % \\ & $|$ & 
% $|$ $($\textsf{or} Fml Fml$)$ 
% % \\ & $|$ & 
% $|$ $($\textsf{forall} $($\scheme|tie| nom Fml$))$ 
% % \\ & $|$ & 
% $|$ Lit 
% \\
%     Lit & $\rightarrow$ & $($\textsf{pos} Term$)$ 
% % \\ & $|$ & 
% $|$ $($\textsf{neg} Term$)$ 
%  \\ 
% Term & $\rightarrow$ & %$($\textsf{sym} symbol$)$ 
% % \\ & $|$ & 
% %$|$ 
% $($\textsf{var} nom$)$ 
% % \\ & $|$ & 
% $|$ $($\textsf{app} symbol Term*$)$ \\
%   \end{tabular}
% \end{center}

%\vspace{-.2cm}

\begin{center}
  \begin{tabular}{lcl}
    \textit{Fml} & $\rightarrow$ & $($\textsf{and}  \textit{Fml}  \textit{Fml}$)$ 
$|$ $($\textsf{or}  \textit{Fml} \textit{Fml}$)$ 
$|$ $($\textsf{forall} $($\scheme|tie| \textit{Nom} \textit{Fml}$))$ 
$|$ \textit{Lit}
\\
    \textit{Lit} & $\rightarrow$ & $($\textsf{pos} \textit{Term}$)$ 
$|$ $($\textsf{neg} \textit{Term}$)$ 
 \\ 
\textit{Term} & $\rightarrow$ &
$($\textsf{var} \textit{Nom}$)$ 
$|$ $($\textsf{app} \textit{Symbol} \textit{Term}*$)$ \\
  \end{tabular}
\end{center}

%\vspace{-.2cm}

This scheme has been chosen carefully to allow unification to compare
literals. In particular, the tags on variables \textit{must} be
discarded before literals are compared.  Consider the two non-ground
literals \mbox{\schemeresult|`(not (f Xvar))|} and
\mbox{\schemeresult|`(f (p Yvar))|}.  These literals are complementary:
the negation of one unifies with the other, associating $x$ with
\mbox{\schemeresult|`(p Yvar)|}. When we apply our tagging scheme,
however, these literals become \mbox{\schemeresult|`(neg (app f (var-tag Xvar)))|} and \mbox{\schemeresult|`(pos (app f (app p (var-tag Yvar))))|}, respectively, and are no longer complementary: their
subexpressions \mbox{\schemeresult|`(var-tag Xvar)|} and
\mbox{\schemeresult|`(app p (var-tag Yvar))|} do not unify. To avoid this
problem, our substitution relation discards the \textsf{var} tag when
it replaces noms with logic variables.

\begin{figure}[H]
%\vspace{-.2in}
\hspace{-.1in}
\begin{tabular}{l l}
\begin{minipage}{1.8in}
%\schemeinput{code/alphatapleft}
\begin{schemedisplay}
(define proveo
  (lambda (fml unexp lits env proof)
    (match-e `(,fml ,proof)
      (`((and-tag ,e1 ,e2) (conj . ,prf))
       (proveo e1 `(,e2 . ,unexp)
               lits env prf))
      (`((or-tag ,e1 ,e2) (split ,prf1 ,prf2))
       (proveo e1 unexp lits env prf1)
       (proveo e2 unexp lits env prf2))
      (`((forall (tie-tag ,@a ,body)) (univ . ,prf))
       (exist (x unexp1)
         (appendo unexp `(,fml) unexp1)
         (proveo body unexp1 lits
                 `((,a . ,x) . ,env) prf)))
      (`(,fml ,proof)
       (exist (lit)
         (subst-lito fml env lit)         
         (conde
           ((== `(close) proof)
            (match-e `(,lit ,neg)
              (`((pos ,tm) (neg ,tm)))
              (`((neg ,tm) (pos ,tm))))              
            (membero neg lits))
           ((exist (next unexp1 prf)
              (== `(,next . ,unexp1) unexp)
              (== `(savefml . ,prf) proof)
              (proveo next unexp1 `(,lit . ,lits)
                      env prf)))))))))
\end{schemedisplay}
%\vspace{1.3cm}
\end{minipage}

& 

\hspace{-0.3in}
\begin{minipage}{1.8in}
%\schemeinput{code/alphatapright}
\begin{schemedisplay}
(define appendo
  (lambda-e (ls s out)
    (`(() ,s ,s))
    (`((,a . ,d) ,s (,a . ,r))
     (appendo d s r))))

(define subst-lito
  (lambda-e (fml env out)
    (`((pos ,l) ,env (pos ,r))
     (subst-termo l env r))
    (`((neg ,l) ,env (neg ,r))
     (subst-termo l env r))))

(define subst-termo
  (lambda-e (fml env out)
    (`((var-tag ,a) ,env ,out)
     (lookupo a env out))
    (`((app ,f . ,d) ,env (app ,f . ,r))
     (subst-term* d env r))))

(define subst-term*
  (lambda-e (tm* env out)
    (`(() __ ()))
    (`((,e1 . ,e2) ,env (,r1 . ,r2))
     (subst-termo e1 env r1)
     (subst-term* e2 env r2))))

(define membero
  (lambda (x ls)
    (exist (a d)
      (== `(,a . ,d) ls)
      (conde
        ((== a x))
        ((membero x d))))))
\end{schemedisplay}
%\vspace{1.0cm}
\end{minipage}

\end{tabular}
\caption{Final definition of \alphatap
  \label{fig:ending}}
%\vspace{-.3in}
\end{figure}

Given our new tagging scheme, we can easily rewrite our substitution
relation without the use of \scheme|match-a|. We simply follow the
production rules of the grammar, defining a relation to recognize
each.

Finally, we modify \scheme|proveo| to take advantage of the same tags.
We also add a \scheme|proof| argument to \scheme|proveo|.  We call
this version of the prover \alphatap, and present its definition in
Figure~\ref{fig:ending}. It is declarative, since we have eliminated
the use of \scheme|copy-termo| and every use of \scheme|match-a|. In
addition to being a sound and complete theorem prover for first-order
logic, \alphatapsp can now generate valid first-order theorems.






\section{Performance}\label{performance}
\enlargethispage{1\baselineskip} %

Like the original \leantap, \alphatapsp can prove many theorems in
first-order logic. Because it is declarative, \alphatapsp is generally
slower at proving ground theorems than mK\leantap, which is slower
than the original \leantap. Figure~\ref{fig:performance} presents a
summary of \alphatap's performance on the first 46 of Pelletier's 75
problems~\cite{pelletier1986sfp}, showing it to be roughly twice as
slow as mK\leantap.

These performance numbers suggest that while there is a penalty to be
paid for declarativeness, it is not so severe as to cripple the
prover. The advantage mK\leantapsp enjoys over the original \leantapsp
in Problem 34 is due to \alphakanren's interleaving search strategy;
as the result for mK\leantapsp shows, the original \leantapsp is faster
than \alphatapsp for any given search strategy.

Many automated provers now use the TPTP problem
library~\cite{stucliffe1994tpl} to assess performance. Even though it
is faster than \alphatap, \leantapsp solves few of the TPTP
problems. The Pelletier Problems, on the other hand, fall into the
class of theorems \leantapsp was designed to prove, and so we feel
they provide a better set of tests for the comparison between
\leantapsp and \alphatap.

\begin{figure}[h]
%\vspace{-.2in}
\begin{centering}
\begin{tabular}{l l}

\hspace{-.1in}
\begin{minipage}{2.7in}
\begin{tabular}{| r | c | c | c | } %c |
  \hline 
  \thinspace \thinspace \# & \thinspace \leantap  \thinspace &
  mK\leantap \thinspace & \thinspace \alphatap
  \thinspace %& \thinspace \alphatap$\!_G$\footnotemark[5]$^,$\footnotemark[7]
  \\
  \hline
1 & 0.1 & 0.7 & 2.0 \\ 
2 & 0.0 & 0.1 & 0.3 \\ 
3 & 0.0 & 0.2 & 0.5 \\ 
4 & 0.0 & 1.0 & 1.7 \\ 
5 & 0.1 & 1.2 & 2.5 \\ 
6 & 0.0 & 0.1 & 0.2 \\ 
7 & 0.0 & 0.1 & 0.2 \\ 
8 & 0.0 & 0.3 & 0.8 \\ 
9 & 0.1 & 4.3 & 9.7 \\ 
10 & 0.3 & 5.5 & 10.2 \\ 
11 & 0.0 & 0.3 & 0.6 \\ 
12 & 0.6 & 17.7 & 31.9 \\ 
13 & 0.1 & 3.7 & 8.2 \\ 
14 & 0.1 & 4.2 & 9.7 \\ 
15 & 0.0 & 0.8 & 1.9 \\ 
16 & 0.0 & 0.2 & 0.6 \\ 
17 & 1.1 & 9.2 & 18.1 \\ 
18 & 0.1 & 0.5 & 1.2 \\ 
19 & 0.3 & 15.1 & 33.5 \\ 
20 & 0.5 & 8.1 & 12.7 \\ 
21 & 0.4 & 22.1 & 38.7 \\ 
22 & 0.1 & 3.4 & 6.4 \\ 
23 & 0.1 & 2.5 & 5.4 \\ 

  \hline
\end{tabular}

\end{minipage}

&

\begin{minipage}{2.5in}
\begin{tabular}{| r | c | c | c |} %c |
  \hline 
  \# & \thinspace \leantap  \thinspace &
  mK\leantap \thinspace & \thinspace \alphatap
  \thinspace %& \thinspace \alphatap$\!_G$\footnotemark[5]$^,$\footnotemark[7]  
\\
  \hline
24 & 1.7 & 31.9 & 60.3 \\ 
25 & 0.2 & 7.5 & 14.1 \\ 
26 & 0.8 & 130.9 & 187.5 \\ 
27 & 2.3 & 40.4 & 79.3 \\ 
28 & 0.3 & 19.1 & 29.6 \\ 
29 & 0.1 & 27.9 & 57.0 \\ 
30 & 0.1 & 4.2 & 9.6 \\ 
31 & 0.3 & 13.2 & 23.1 \\ 
32 & 0.2 & 23.9 & 42.4 \\ 
33 & 0.1 & 15.9 & 39.2 \\ 
34 & 199129.0  & 7272.9 & 8493.5 \\ 
35 & 0.1 & 0.5 & 1.1 \\ 
36 & 0.2 & 6.7 & 12.4 \\ 
37 & 0.8 & 123.3 & 169.2 \\ 
38 & 8.9 & 4228.8 & 8363.8 \\ 
39 & 0.0 & 1.1 & 2.8 \\ 
40 & 0.2 & 8.1 & 19.2 \\ 
41 & 0.1 & 6.9 & 17.0 \\ 
42 & 0.4 & 15.0 & 32.1 \\ 
43 & 43.2 & 668.4 & 1509.6 \\ 
44 & 0.3 & 15.1 & 35.7 \\ 
45 & 3.4 & 145.3 & 239.7 \\ 
46 & 7.7 & 505.5 & 931.2 \\ 

  \hline
\end{tabular}
\end{minipage}
\end{tabular}

\caption{Performance of \leantap, mK\leantap, and \alphatapsp on the
  first 46 Pelletier Problems. 
  All times are in milliseconds, averaged over 100 trials.
  All tests were run \mbox{under} Debian
  Linux on an IBM Thinkpad 
  X40 with a 1.1GHz Intel Pentium-M processor and 768MB RAM. 
  \leantapsp tests were run under SWI-Prolog 5.6.55;
  mK\leantapsp and \alphatapsp tests were run under Ikarus Scheme
  0.0.3+.
  \label{fig:performance}}
\end{centering}
%\vspace{-.2in}

\end{figure}


\section{Applicability of These Techniques}

To avoid the use of \scheme|copy-termo|, we have represented
universally quantified variables with noms rather than logic
variables, allowing us to perform substitution instead of copying.  To
eliminate \scheme|match-a|, we have enhanced the tagging scheme for
representing formulas.

Both of these transformations are broadly applicable. When
\scheme|match-a| is used to handle overlapping clauses, a carefully
crafted tagging scheme can often be used to eliminate
overlapping. When terms must be copied, substitution can often be used
instead of \scheme|copy-termo|---in the case of \alphatap, we use a
combination of nominal unification and substitution.


\chapter{Implementation IV:  \alphakanren}\label{akimplchapter}

\enlargethispage{1em}

In this chapter we present two implementations of \alphakanrensp based
on two implementations of nominal unification: one using idempotent
substitutions, and one using triangular substitutions.  The idempotent
implementation mirrors the mathematical description of nominal
unification given by \citet{Urban-Pitts-Gabbay/04}, while the
triangular implementation is more efficient.

This chapter is organized as follows. In section~\ref{nominalunif} we
present our implementation of nominal unification using idempotent
substitutions. In section~\ref{akgoalconstructorimplsection} we
implement \alphakanren's goal constructors, using the unifier of
section~\ref{nominalunif}, and in section~\ref{akreifysection} we
implement reification.  In section~\ref{triangularsection} we present
a second implementation of nominal unification, using triangular
substitutions.  

% In section~\ref{akperfsection} we compare the performance of the two
% \alphakanrensp implementations, using the relational theorem prover of
% \ref{alphatapchapter}.

\section{Nominal Unification with Idempotent Substitutions}\label{nominalunif}

Nominal unification occurs in two distinct phases: the first processes
equations, while the second processes constraints.  The first phase
takes a set of equations \scheme{eqns} and transforms it into a
substitution \scheme{sigma} and a set of unresolved constraints
\scheme{delta}.  The second phase combines the unresolved constraints
with the previously resolved constraints, which have both been brought
up to date using \scheme{apply-subst}.  Then, the unifier transforms
these combined constraints into a set of resolved constraints
$\nabla$, and returns the list \mbox{\scheme|`(,sigma ,nabla)|} as a
package.

Nominal unification uses several data structures.  A set of
equations \scheme{eqns} is represented as a list of pairs of terms.
A substitution \scheme{sigma} is represented as an association list of variables to terms.
A set of constraints \scheme{delta} is represented as a list of pairs associating
noms to terms; a \scheme{nabla} is a \scheme{delta} in which all terms are unbound variables.
In a substitution, a variable may have at most one association.  
In a \scheme{delta} (and therefore in a \scheme{nabla}) a nom may have multiple associations.

We represent a variable as a suspension containing an empty
list of swaps.  Several functions reconstruct suspensions that represent
variables.  However, our implementation of nominal unification assumes
that variables can be compared using \scheme{eq?}.

In order to ensure that a variable is always \scheme{eq?} to itself,
regardless of how many times it is reconstructed, we use a
\scheme{letrec} trick: a suspension representing a variable
contains a procedure of zero arguments (a \emph{thunk})
that, when invoked, returns the suspension,
thus maintaining the desired \scheme{eq?}-ness property.
(In the text we conflate variables with their associated thunks.)

\schemedisplayspace
\begin{schemedisplay}
(define ak-var
  (lambda (ignore)
    (letrec ((s (list 'susp-tag '() (lambda () s))))
      s)))
\end{schemedisplay}

\scheme{unify} attempts to solve a set of equations \scheme{eqns}
in the context of a package \mbox{\scheme|`(,sigma ,nabla)|}.
\scheme{unify} applies \scheme{sigma} to \scheme{eqns}, 
and then calls \scheme{apply-sigma-rules} on the resulting 
set of equations.
\scheme{apply-sigma-rules} either successfully completes the first
phase of nominal unification by returning a new \scheme{sigma} and
\scheme{delta}, or invokes the failure continuation \scheme{fk},
a jump-out continuation similar to Lisp's
\scheme{catch} \cite{Steele:1990:CLL}.

\schemedisplayspace
\begin{schemedisplay}
(define unify
  (lambda (eqns sigma nabla fk)
    (let ((eqns (apply-subst sigma eqns)))
      (mv-let ((sigma^ delta) (apply-sigma-rules eqns fk))
        (unifyhash delta (compose-subst sigma sigma^) nabla fk)))))
\end{schemedisplay}
\noindent \scheme{mv-let}, defined in Appendix~\ref{pmatch}, deconstructs a list of values.

In the second phase of nominal unification, \scheme{unifyhash} calls \scheme{apply-subst}
to bring \scheme{nabla} and \scheme{delta} up to date, then
passes their union to \scheme{apply-nabla-rules}.

\schemedisplayspace
\begin{schemedisplay}
(define unifyhash
  (lambda (delta sigma nabla fk)
    (let ((delta (apply-subst sigma delta))
          (nabla (apply-subst sigma nabla)))
      (let ((delta (delta-union nabla delta)))
        (list sigma (apply-nabla-rules delta fk))))))
\end{schemedisplay}

\scheme{apply-sigma-rules} is a recursive function whose only task is
to combine results returned by \scheme{sigma-rules}.
\scheme{sigma-rules} takes two arguments: a single equation and the
rest of the equations.  If \scheme{sigma-rules} fails, then
\scheme{apply-sigma-rules} invokes \scheme{fk}, and the result of
\scheme{unify} is \scheme{#f}.  Each successful call to
\scheme{sigma-rules} returns a new set of equations \scheme{eqns}, a
new \scheme{sigma}, and a set of (unresolved) constraints
\scheme{delta}.  Successive calls to \scheme{sigma-rules} resolve the
equations in \scheme{eqns} until there are no equations left.

\newpage

\schemedisplayspace
\begin{schemedisplay}
(define apply-sigma-rules
  (lambda (eqns fk)
    (cond
      ((null? eqns) `(,empty-sigma ,empty-delta))
      (else
       (let ((eqn (car eqns)) (eqns (cdr eqns)))
         (mv-let ((eqns sigma delta) (or (sigma-rules eqn eqns) (fk)))
           (mv-let ((sigma^ delta^) (apply-sigma-rules eqns fk))
             (list (compose-subst sigma sigma^) (delta-union delta^ delta)))))))))
\end{schemedisplay}

\scheme{apply-nabla-rules} is similar to \scheme{apply-sigma-rules},
but takes constraints instead of equations, and combines the results
returned by \scheme{nabla-rules}.

\schemedisplayspace
\begin{schemedisplay}
(define apply-nabla-rules
  (lambda (delta fk)
    (cond
      ((null? delta) empty-nabla)
      (else
       (let ((c (car delta)) (delta (cdr delta)))
         (mv-let ((delta nabla) (or (nabla-rules c delta) (fk)))
           (delta-union nabla (apply-nabla-rules delta fk))))))))
\end{schemedisplay}

\noindent \scheme{empty-sigma}, \scheme{empty-delta}, and
\scheme{empty-nabla} are defined in section~\ref{akgoalconstructorimplsection}.

In both \scheme{sigma-rules} and \scheme{nabla-rules}
we use \scheme{untagged?} 
to distinguish untagged pairs from specially tagged pairs that
represent binders, noms, and suspensions.

\schemedisplayspace
\begin{schemedisplay}
(define untagged?
  (lambda (x)
    (not (memv x '(tie-tag nom-tag susp-tag)))))
\end{schemedisplay}

Here are the transformation rules of the nominal unification
algorithm, derived from the rules in \citet{Urban-Pitts-Gabbay/04}.
(\scheme{sigma-rules} relies on \scheme{pmatch}, which is defined in
Appendix~\ref{pmatch}.)

\newpage

\schemedisplayspace
\begin{schemedisplay}
(define sigma-rules  
  (lambda (eqn eqns)
    (pmatch eqn
      (`(,c . ,c^)
       (guard (not (pair? c)) (equal? c c^))
       `(,eqns ,empty-sigma ,empty-delta))
      (`((tie-tag ,a ,t) . (tie-tag ,a^ ,t^))
       (guard (eq? a a^))
       `(((,t . ,t^) . ,eqns) ,empty-sigma ,empty-delta))
      (`((tie-tag ,a ,t) . (tie-tag ,a^ ,t^))
       (guard (not (eq? a a^)))
       (let ((u^ (apply-pi `((,a ,a^)) t^)))
         `(((,t . ,u^) . ,eqns) ,empty-sigma ((,a . ,t^)))))
      (`((nom-tag __) . (nom-tag __))
       (guard (eq? (car eqn) (cdr eqn)))
       `(,eqns ,empty-sigma ,empty-delta))
      (`((susp-tag ,pi ,x) . (susp-tag ,pi^ ,x^))
       (guard (eq? (x) (x^)))
       (let ((delta (map (lambda (a) (cons a (x))) 
                         (disagreement-set pi pi^))))
         `(,eqns ,empty-sigma ,delta)))
      (`((susp-tag ,pi ,x) . ,t)
       (guard (not (occurs-check (x) t)))
       (let ((x (x)) (t (apply-pi (reverse pi) t)))
         (let ((sigma `((,x . ,t))))
           (list (apply-subst sigma eqns) sigma empty-delta))))
      (`(,t . (susp-tag ,pi ,x))
       (guard (not (occurs-check (x) t)))
       (let ((x (x)) (t (apply-pi (reverse pi) t)))
         (let ((sigma `((,x . ,t))))
           (list (apply-subst sigma eqns) sigma empty-delta))))
      (`((,t1 . ,t2) . (,t1^ . ,t2^))
       (guard (untagged? t1) (untagged? t1^))
       `(((,t1 . ,t1^) (,t2 . ,t2^) . ,eqns) ,empty-sigma ,empty-delta))
      (else #f))))
\end{schemedisplay}

Clauses two and three in \scheme{sigma-rules}
implement $\alpha$-equivalence of binders, as defined
in section~\ref{akintro} of Chapter~\ref{akchapter}.
Clause five unifies two suspensions that have the same variable;
in this case, \scheme{sigma-rules} creates as many new freshness constraints
as there are noms in the \emph{disagreement set} (defined below) of the
suspensions' swaps.
Clauses six and seven are similar: each clause unifies a
suspension containing a variable \scheme{x} and a list of swaps \scheme{pi}
with a term \scheme{t}.
\scheme{sigma-rules} creates a substitution associating
\scheme{x} with the result of applying the swaps in \scheme{pi} to \scheme{t} in
\emph{reverse order}, with the newest swap in \scheme{pi} applied first.
This substitution is applied to the context \scheme{eqns}.

\newpage

\scheme{apply-pi}, below, applies a list of swaps \scheme{pi} to a term \scheme{v}.

\schemedisplayspace
\begin{schemedisplay}
(define apply-pi
  (lambda (pi v)
    (pmatch v
      (`,c (guard (not (pair? c))) c)
      (`(tie-tag ,a ,t)
       (let ((a (apply-pi pi a))
             (t (apply-pi pi t)))
         `(tie-tag ,a ,t)))
      (`(nom-tag __)
       (let loop ((v v) (pi pi))
         (if (null? pi)
             v
             (apply-swap (car pi) (loop v (cdr pi))))))
      (`(susp-tag ,pi^ ,x)
       (let ((pi (append pi pi^)))
         (if (null? pi)
             (x)
             `(susp-tag ,pi ,x))))
      (`(,a . ,d) (cons (apply-pi pi a) (apply-pi pi d))))))
\end{schemedisplay}

\noindent If \scheme{v} is a nom, then  \scheme{pi}'s swaps are applied,
with the oldest swap applied first.  If \scheme{v} is a suspension
with a list of swaps \scheme{pi^} and variable \scheme{x}, then
the swaps in \scheme{pi} are added to the swaps in \scheme{pi^}.
If this list is empty, then \scheme{x}'s suspension is
returned; otherwise, a new suspension is created with those swaps.

\schemedisplayspace
\begin{schemedisplay}
(define apply-swap
  (lambda (swap a)
    (pmatch swap
      (`(,a1 ,a2)
       (cond
         ((eq? a a2) a1)
         ((eq? a a1) a2)
         (else a))))))
\end{schemedisplay}

The \scheme{nabla-rules} are much simpler than the \scheme{sigma-rules}.  
In the second clause, the nom \scheme{a^} in the binding position of the binder 
is the same as \scheme{a}, so \scheme{a} can never appear free in \scheme{t}.
In the fifth clause, the list of swaps \scheme{pi} in the suspension are
applied, in reverse order, to the nom \scheme{a}, yielding another nom.
\scheme{nabla-rules} then adds a new constraint associating this nom with
the suspension's variable.

\newpage

\schemedisplayspace
\begin{schemedisplay}
(define nabla-rules
  (lambda (d delta)
    (pmatch d
      (`(,a . ,c)
       (guard (not (pair? c)))
       `(,delta ,empty-nabla))
      (`(,a . (tie-tag ,a^ ,t))
       (guard (eq? a^ a))
       `(,delta ,empty-nabla))
      (`(,a . (tie-tag ,a^ ,t))
       (guard (not (eq? a^ a)))
       `(((,a . ,t) . ,delta) ,empty-nabla))
      (`(,a . (nom-tag __))
       (guard (not (eq? a (cdr d))))
       `(,delta ,empty-nabla))
      (`(,a . (susp-tag ,pi ,x))
       (let ((a (apply-pi (reverse pi) a)) (x (x)))
         `(,delta ((,a . ,x)))))
      (`(,a . (,t1 . ,t2))
       (guard (untagged? t1))
       `(((,a . ,t1) (,a . ,t2) . ,delta) ,empty-nabla))
      (else #f))))
\end{schemedisplay}

Finding the disagreement set of two lists of swaps
\scheme{pi} and \scheme{pi^} requires
forming a set of all the noms in those lists, then applying both \scheme{pi}
and \scheme{pi^} to each nom \scheme{a} in this set.
If \mbox{\scheme|(apply-pi pi a)|} and \mbox{\scheme|(apply-pi pi^ a)|}
produce different noms, then \scheme{a} is in the \emph{dis}agreement set.
(\scheme{filter} and \scheme{remove-duplicates} are defined 
in Appendix~\ref{helpers}.)

\schemedisplayspace
\begin{schemedisplay}
(define disagreement-set
  (lambda (pi pi^)
    (filter
      (lambda (a) (not (eq? (apply-pi pi a) (apply-pi pi^ a))))
      (remove-duplicates
        (append (apply append pi) (apply append pi^))))))
\end{schemedisplay}

The \scheme{occurs-check} is what one might expect.

\schemedisplayspace
\begin{schemedisplay}
(define occurs-check
  (lambda (x v)
    (pmatch v
      (`,c (guard (not (pair? c))) #f)
      (`(tie-tag __ ,t) (occurs-check x t))      
      (`(nom-tag __) #f)
      (`(susp-tag __ x^) (eq? (x^) x))
      (`(,x^ . ,y^) (or (occurs-check x x^) (occurs-check x y^)))
      (else #f))))
\end{schemedisplay}

\subsection{Idempotent Substitutions}\label{applysubst}

\scheme{compose-subst}'s definition is taken from \citet{lloyd:lp}.  It
takes two substitutions \scheme{sigma} and \scheme{tau}, and constructs a new
substitution \scheme{sigma^} in which each association \mbox{\scheme|`(,x . ,v)|}
in \scheme{sigma} is replaced by \mbox{\scheme|`(,x . v^)|}, where \scheme{v^} is the result
of applying \scheme{tau} to \scheme{v}.
Any association in \scheme{tau} whose
variable has an association in \scheme{sigma^} is then filtered from \scheme{tau}.
Also, any association of the form \mbox{\scheme|`(,x . ,x)|} is filtered from \scheme{sigma^}.
These filtered substitutions are then appended.

\schemedisplayspace
\begin{schemedisplay}
(define compose-subst
  (lambda (sigma tau)
    (let ((sigma^ (map
                    (lambda (a) (cons (car a) (apply-subst tau (cdr a))))
                    sigma)))
      (append
        (filter (lambda (a) (not (assq (car a) sigma^))) tau)
        (filter (lambda (a) (not (eq? (car a) (cdr a)))) sigma^)))))
\end{schemedisplay}

Next we define \scheme{apply-subst}.  In the suspension case,
\scheme{apply-subst} applies the list of swaps \scheme{pi} to a variable,
or to its binding.

\schemedisplayspace
\begin{schemedisplay}
(define apply-subst
  (lambda (sigma v)
    (pmatch v
      (`,c (guard (not (pair? c))) c)
      (`(tie-tag ,a ,t)
       (let ((t (apply-subst sigma t)))
         `(tie-tag ,a ,t)))      
      (`(nom-tag __) v)
      (`(susp-tag ,pi ,x) (apply-pi pi (get (x) sigma)))
      (`(,x . ,y) (cons (apply-subst sigma x) (apply-subst sigma y))))))
\end{schemedisplay}

\noindent \scheme{get}, which is defined in Appendix~\ref{helpers},
finds the binding of a variable in a substitution or returns the
variable if no binding exists.

\subsection{{\it \deltaunionsymbol}}

Finally we define \scheme{delta-union}, which forms the union of
two \scheme{delta}'s.

\schemedisplayspace
\begin{schemedisplay}
(define delta-union
  (lambda (delta delta^)
    (pmatch delta
      (`() delta^)
      (`(,d . ,delta)
       (if (term-member? d delta^)
           (delta-union delta delta^)
           (cons d (delta-union delta delta^)))))))
\end{schemedisplay}
\newpage
\begin{schemedisplay}
(define term-member?
  (lambda (v v*)
    (pmatch v*
      (`() #f)
      (`(,v^ . ,v*) 
       (or (term-equal? v^ v) (term-member? v v*))))))

(define term-equal?
  (lambda (u v)
    (pmatch `(,u ,v)
      (`(,c ,c^) (guard (not (pair? c)) (not (pair? c^)))
       (equal? c c^))
      (`((tie-tag ,a ,t) (tie-tag ,a^ ,t^))
       (and (eq? a a^) (term-equal? t t^)))
      (`((nom-tag __) (nom-tag __)) (eq? u v))
      (`((susp-tag ,pi ,x) (susp-tag ,pi^ ,x^))
       (and (eq? (x) (x^)) (null? (disagreement-set pi pi^))))
      (`((,x . ,y) (,x^ . ,y^))
       (and (term-equal? x x^) (term-equal? y y^)))
      (else #f))))
\end{schemedisplay}

Recall that \scheme{delta} denotes a set of unresolved
constraints, where a constraint is a pair of a nom \scheme{a} and a
term \scheme{t}.  \scheme{delta-union} uses \scheme{term-member?},
which uses \scheme{term-equal?} when comparing two constraints.  The
definition of \scheme{term-equal?} is straightforward except when
comparing two suspensions, in which case their variables must be the
same, and the disagreement set of their lists of swaps must be empty.

%\section{Goals and Packages}\label{akpackagerepsection}

% Our \alphakanren\ implementation comprises three kinds of operators: the interface
% operator \scheme{run}; goal constructors \scheme{==}, \scheme{hash},
% \scheme{conde}, \scheme{exist}, and \scheme{fresh}, which take a package
% \emph{implicitly}; and functions such as \scheme{reify},
% and the already defined \scheme{unify} and \scheme{unify#}, which take a
% package \emph{explicitly}.  


% (As in Chapter~\ref{mkimplchapter}, we notate \scheme{lambda} as
% \scheme{lambdag@} when creating such a function~\scheme{g}.)


% \schemedisplayspace
% \begin{schemedisplay}
% (define-syntax lambdag@ 
%   (syntax-rules () ((__ (p) e) (lambda (p) e))))
% \end{schemedisplay}

% Because a sequence of packages may be infinite, we represent it
% not as a list but as a \mbox{\scheme|p-inf|}, a special kind of
% stream that can contain either zero, one, or more packages
% \cite{hinze2000,Wadler85}.
% We use \mbox{\scheme|#f|} to represent the empty stream of
% packages. If \scheme{p} is a package, then \scheme{p} itself
% represents the stream containing
% just~\scheme{p}.  To represent a stream containing multiple
% packages, we use \mbox{\scheme|(choice p f)|}, where \scheme{p}
% is the first package in the stream, and where \scheme{f} is a
% thunk that, when invoked, produces the remainder of the stream. 
% (For clarity, we notate \scheme{lambda}
% as \scheme{lambdaf@} when creating such a function~\scheme{f}.)
% To represent an incomplete stream, we use \mbox{\scheme|(inc e)|}, 
% where \scheme{e} is an \emph{expression} that evaluates to
% a \mbox{\scheme|p-inf|}---thus \scheme{inc} creates an \scheme{f}.

% \schemedisplayspace
% \begin{schemedisplay}
% (define-syntax lambdaf@ 
%   (syntax-rules () ((__ () e) (lambda () e))))

% (define-syntax choice
%   (syntax-rules () ((__ a f) (cons a f))))

% (define-syntax inc 
%   (syntax-rules () ((__ e) (lambdaf@ () e))))
% \end{schemedisplay}

% \noindent A singleton stream \scheme{p} is the same as
% \begin{schemebox}(choice p (lambdaf@ () #f))\end{schemebox}.  However, 
% for the goals that return only a single package, using this special
% representation of a singleton stream
% avoids the cost of unnecessarily building and taking apart pairs, 
% and creating and invoking thunks.

% To ensure that the values produced by these four kinds of
% \mbox{\scheme|p-inf|}'s can be distinguished, we assume that a package
% is never \schemeresult{#f}, a function, or a pair whose \scheme{cdr}
% is a function.  To discriminate among these four cases, we define
% \mbox{\scheme|case-inf|}.

% \newpage
% \schemedisplayspace
% \begin{schemedisplay}
% (define-syntax case-inf
%   (syntax-rules ()
%     ((_ a-inf (() e0) ((f^) e1) ((a^) e2) ((a f) e3))
%      (pmatch a-inf
%        (#f e0)
%        (`,f^ (guard (procedure? f^)) e1)
%        (`,a^ (guard (not 
%                      (and (pair? a^) 
%                           (procedure? (cdr a^)))))
%             e2)
%        (`(,a . ,f) e3)))))
% \end{schemedisplay}

% The interface operator \scheme{run} 
% uses \scheme{take} (defined below) to convert an \scheme{f} to an
% \emph{even} stream \cite{Lazysml98}.  
% The definition of \scheme{run} places an artificial goal at the tail
% of \mbox{\scheme|g0 g ...|}; this artificial goal reifies the variable
%  \scheme{x} using the final package \scheme{p}
% produced by running the goals \mbox{\scheme|g0 g ...|}
% in the empty package \mbox{\scheme|`(,empty-sigma ,empty-nabla)|}.

% \schemedisplayspace
% \begin{schemedisplay}
% (define-syntax run
%   (syntax-rules ()
%     ((_ n (x) g0 g ...)
%      (take n (lambdaf@ ()
%                ((exist (x) g0 g ... 
%                   (lambdag@ (p) (cons (reify x p) '())))
%                 `(,empty-sigma ,empty-nabla)))))))
% \end{schemedisplay}

% \noindent Wrapping the reified value 
% in a list allows \scheme{#f} to appear as a value.

% If the first argument to \scheme{take} is \scheme{#f}, 
% then \scheme{take} returns the entire stream of reified values as a list,
% thereby providing the behavior of \scheme{run*}.
% The \scheme{and} expressions within \scheme{take} 
% detect this \scheme{#f} case.

% \newpage
% \schemedisplayspace
% \begin{schemedisplay}
% (define take
%   (lambda (n f)
%     (if (and n (zero? n)) 
%       '()
%       (case-inf (f)
%         (() '())
%         ((f) (take n f))
%         ((a) a)
%         ((a f) (cons (car a)
%                  (take (and n (- n 1)) f)))))))
% \end{schemedisplay}

% \scheme{run*} is trivially defined in terms of \scheme{run}.

% \schemedisplayspace
% \begin{schemedisplay}
% (define-syntax run*
%   (syntax-rules ()
%     ((_ (x) g0 g ...)
%      (run #f (x) g0 g ...))))
% \end{schemedisplay}

\section{Goal Constructors}\label{akgoalconstructorimplsection}

In the core miniKanren implementation of Chapter~\ref{mkimplchapter},
a goal is a function that maps a substitution \scheme{s} to an ordered
sequence of zero or more substitutions (see
section~\ref{goalconstructors}).  In \alphakanren, a goal \scheme{g}
is a function that maps a package \scheme{p} to an ordered
sequence \scheme|p-inf| of zero or more packages.  

We represent the empty substitution, 
along with the empty unresolved and resolved
constraint sets, as the empty list.

\wspace

\noindent \scheme|(define empty-sigma '()) $~~~~~$ (define empty-delta '()) $~~~~~$ (define empty-nabla '())|

\wspace

\scheme{==} and \scheme{hash} construct goals that return either a
singleton stream or an empty stream.

\schemedisplayspace
\begin{schemedisplay}
(define-syntax ==
  (syntax-rules ()
    ((_ u v)
     (unifier unify `((,u . ,v))))))
\end{schemedisplay}
\newpage
\begin{schemedisplay}
(define-syntax hash
  (syntax-rules ()
    ((_ a t)
     (unifier unifyhash `((,a . ,t))))))

(define unifier
  (lambda (fn set)
    (lambdag@ (p)
      (mv-let ((sigma nabla) p)
        (call/cc (lambda (fk) (fn set sigma nabla (lambda () (fk #f)))))))))
\end{schemedisplay}

% To take the conjunction of goals, we define \scheme{exist}, a goal
% constructor that first lexically binds variables built by
% \scheme{var}, and then combines successive goals using \scheme{bind*}.
The goal constructor \scheme{fresh} is identical to \scheme{exist},
except that it lexically binds noms instead of variables.

% (define-syntax exist
%   (syntax-rules ()
%     ((_ (x ...) g0 g ...)
%      (lambdag@ (p)
%        (inc
%          (let ((x (var)) ...)
%            (bind* (g0 p) g ...)))))))


\schemedisplayspace
\begin{schemedisplay}
(define-syntax fresh
  (syntax-rules ()
    ((_ (a ...) g0 g ...)
     (lambdag@ (p)
       (inc
         (let ((a (nom 'a)) ...)
           (bind* (g0 p) g ...)))))))

(define nom
  (lambda (a)
    (list 'nom-tag (symbol->string a))))
\end{schemedisplay}

% \scheme{bind*} is short-circuiting: since the empty stream is
% represented by \mbox{\scheme|#f|}, any failed goal causes
% \scheme{bind*} to immediately return \scheme{#f}.  \scheme{bind*}
% relies on \scheme{bind} \cite{moggi91notions,Wadler92}, which applies the
% goal \scheme{g} to each element in the stream \scheme{p-inf}.  The
% resulting \scheme{p-inf}'s are then merged using \scheme{mplus}, which
% combines a \scheme{p-inf} and an \scheme{f} to yield a single
% \scheme{p-inf}.  (\scheme{bind} is similar to Lisp's \scheme{mapcan}
% but uses \scheme{mplus} (not \scheme{append}) to interleave the values
% of streams.)

% \schemedisplayspace
% \begin{schemedisplay}
% (define-syntax bind*
%   (syntax-rules ()
%     ((_ e) e)
%     ((_ e g0 g ...)
%      (let ((a-inf e))
%        (and a-inf (bind* (bind a-inf g0) g ...))))))

% (define bind
%   (lambda (a-inf g)
%     (case-inf a-inf
%       (() #f)
%       ((f) (inc (bind (f) g)))
%       ((a) (g a))
%       ((a f) (mplus (g a) (lambdaf@ () (bind (f) g)))))))

% (define mplus
%   (lambda (a-inf f)
%     (case-inf a-inf
%       (() (f))
%       ((f^) (inc (mplus (f) f^)))
%       ((a) (choice a f))
%       ((a f^) (choice a (lambdaf@ () (mplus (f) f^)))))))
% \end{schemedisplay}

% To take the disjunction of goals we define \scheme{conde}, a goal
% constructor that combines successive \scheme{conde}-lines using
% \scheme{mplus*}, which in turn relies on \scheme{mplus}.  
% We use the same implicit package \scheme{p} for each \scheme{conde}-line.
% To avoid unwanted divergence, we treat the
% \scheme{conde}-lines as a single \scheme{inc} stream.  

% \schemedisplayspace
% \begin{schemedisplay}
% (define-syntax conde
%   (syntax-rules ()
%     ((_ (g0 g ...) (g1 g^ ...) ...)
%      (lambdag@ (p)
%        (inc (mplus* (bind* (g0 p) g ...)
%                     (bind* (g1 p) g^ ...)
%                     ...))))))
% \end{schemedisplay}

% \begin{schemedisplay}
% (define-syntax mplus*
%   (syntax-rules ()
%     ((_ e) e)
%     ((_ e0 e ...) (mplus e0 (lambdaf@ () (mplus* e ...))))))
% \end{schemedisplay}

\section{Reification}\label{akreifysection}

\enlargethispage{2em}

As described in section~\ref{mkreification}, \emph{reification} is the
process of turning a miniKanren (or \alphakanren) value into a Scheme
value.

\alphakanren's version of \scheme{reify} takes a variable \scheme{x}
and a package \scheme{p}, and returns the value associated with
\scheme{x} in \scheme{p} (along with any relevant constraints), first
replacing all variables and noms with symbols representing those
entities.  A constraint \scheme{`(,a . ,y)} is \emph{relevant} if both
\scheme{a} and \scheme{y} appear in the value associated with
\scheme{x}.

The first \scheme{cond} clause in the definition of \scheme{reify}
below returns only the reified value associated with \scheme{x}, when
there are no relevant constraints.  The \scheme{else} clause returns
both the reified value of \scheme{x} and the reified set of relevant
constraints; we have arbitrarily chosen the colon `:' to separate the
reified value from the list of reified constraints.

\schemedisplayspace
\begin{schemedisplay}
(define reify
  (lambda (x p)
    (mv-let ((sigma nabla) p)
      (let* ((v (get x sigma)) (s (reify-s v)) (v (walk* v s)))
        (let ((nabla (filter (lambda (a) (and (symbol? (car a)) (symbol? (cdr a))))
                       (walk* nabla s))))
          (cond
            ((null? nabla) v)
            (else `(,v : ,nabla))))))))
\end{schemedisplay}

\scheme{reify-s} is the heart of the reifier.  \scheme{reify-s} takes
an arbitrary value \scheme{v}, and returns a substitution that maps
every distinct nom and variable in \scheme{v} to a unique symbol.  The
trick to maintaining left-to-right ordering of the subscripts on these
symbols is to process \scheme{v} from left to right, as can be seen in
the last \scheme{pmatch} clause.  When \scheme{reify-s} encounters a
nom or variable, it determines if we already have a mapping for that
entity.  If not, \scheme{reify-s} extends the substitution with an
association between the nom or variable and a new, appropriately
subscripted symbol.

\schemedisplayspace
\begin{schemedisplay}
(define reify-s
  (letrec
    ((r-s (lambda (v s)
            (pmatch v
              (`,c (guard (not (pair? c))) s)
              (`(tie-tag ,a ,t) (r-s t (r-s a s)))
              (`(nom-tag ,n)
               (cond
                 ((assq v s) s)
                 ((assp nom? s)
                  => (lambda (p)
                       (let ((n (reify-n (cdr p))))
                         (cons `(,v . ,n) s))))
                 (else (cons `(,v . a.0) s))))
              (`(susp-tag () __)
               (cond
                 ((assq v s) s)
                 ((assp ak-var? s)
                  => (lambda (p)
                       (let ((n (reify-n (cdr p))))
                         (cons `(,v . ,n) s))))
                 (else (cons `(,v . __.0) s))))
              (`(susp-tag ,pi ,x)
               (r-s (x) (r-s pi s)))
              (`(,a . ,d) (r-s d (r-s a s)))))))
      (lambda (v)
        (r-s v '()))))
\end{schemedisplay}

\scheme{walk*} applies a special substitution \scheme{s}, which
maps noms and variables to symbols, to an arbitrary value \scheme{v}.

\schemedisplayspace
\begin{schemedisplay}
(define walk*
  (lambda (v s)
    (pmatch v
      (,c (guard (not (pair? c))) c)
      (`(tie-tag ,a ,t) (list 'tie-tag (get a s) (walk* t s)))
      (`(nom-tag __) (get v s))
      (`(susp-tag () __) (get v s))
      (`(susp-tag ,pi ,x) (list 'susp-tag (walk* pi s) (get (x) s)))
      (`(,a . ,d) (cons (walk* a s) (walk* d s))))))
\end{schemedisplay}

\begin{schemedisplay}
(define ak-var?
  (lambda (x)
    (pmatch x
      (`(susp-tag () __) #t)
      (else #f))))

(define nom?
  (lambda (x)
    (pmatch x
      (`(nom-tag __) #t)
      (else #f))))
\end{schemedisplay}

\scheme{reify-n} returns a symbol representing an individual
variable or nom; this symbol always ends with a period 
followed by a non-negative integer. 

\schemedisplayspace
\begin{schemedisplay}
(define reify-n
  (lambda (a)
    (let ((str* (string->list (symbol->string a))))
      (let ((c* (memv #\. str*)))
        (let ((rn (string->number (list->string (cdr c*)))))
          (let ((n-str (number->string (+ rn 1))))
            (string->symbol 
              (string-append
                (string (car str*)) "." n-str))))))))
\end{schemedisplay}


% \subsection{Impure Control Operators:}

% For completeness, we define three additional \alphakanren\ goal constructors 
% not used in this paper: \scheme{project}, which can be used to
% access the values of variables, and 
% \scheme{conda} and \scheme{condu}, which can be used to prune 
% the search tree of a program.
% The examples from chapter 10 of \emph{The Reasoned Schemer}~\cite{reasoned} 
% demonstrate how \scheme{conda} and
% \scheme{condu} can be useful, and the pitfalls that await the
% unsuspecting reader.
% %\newpage

% \schemedisplayspace
% \begin{schemedisplay}
% (define-syntax project 
%   (syntax-rules ()                       
%     ((_ (x ...) g0 g ...)  
%      (lambdag@ (p)
%        (mv-let ((sigma nabla) p)
%          (let ((x (get x sigma)) ...)
%            (bind* (g0 p) g ...)))))))
% \end{schemedisplay}

% \begin{schemedisplay}
% (define-syntax conda
%   (syntax-rules ()
%     ((_ (g0 g ...) (g1 g^ ...) ...)
%      (lambdag@ (p)
%        (inc (ifa ((g0 p) g ...) ((g1 p) g^ ...) ...))))))

% (define-syntax ifa
%   (syntax-rules ()
%     ((_) #f)
%     ((_ (e g ...) b ...)
%      (let loop ((a-inf e))
%        (case-inf a-inf
%          (() (ifa b ...))
%          ((f) (inc (loop (f))))
%          ((a) (bind* a-inf g ...))
%          ((a f) (bind* a-inf g ...)))))))
% \end{schemedisplay}

% \begin{schemedisplay}
% (define-syntax condu
%   (syntax-rules ()
%     ((_ (g0 g ...) (g1 g^ ...) ...)
%      (lambdag@ (p)
%        (inc (ifu ((g0 p) g ...) ((g1 p) g^ ...) ...))))))
% \end{schemedisplay}
% \newpage
% \begin{schemedisplay}
% (define-syntax ifu
%   (syntax-rules ()
%     ((_) #f)
%     ((_ (e g ...) b ...)
%      (let loop ((a-inf e))
%        (case-inf a-inf
%          (() (ifu b ...))
%          ((f) (inc (loop (f))))
%          ((a) (bind* a-inf g ...))
%          ((a f) (bind* a g ...)))))))
% \end{schemedisplay}

\section{Nominal Unification with Triangular Substitutions}\label{triangularsection}

\enlargethispage{1em}

In this section we modify the idempotent nominal unification
implementation to work with triangular substitutions, significantly
improving the performance of \alphakanren\footnote{This implementation
  of triangular nominal unification is due to Joseph Near.  Ramana
  Kumar has implemented a somewhat faster triangular unifier; however,
  the resulting code bears little resemblance to the idempotent
  algorithm of~\cite{Urban-Pitts-Gabbay/04}.}.  We present only the
definitions that differ from those already presented.

Like the core miniKanren implementation of
Chapter~\ref{mkimplchapter}, our triangular unifier relies on a
\scheme|walk| function for looking up values in a triangular
substitution.  The nominal \scheme|walk| function is complicated by
the need to handle suspensions and permutations.

\schemedisplayspace
\begin{schemedisplay}
(define walk
  (lambda (x s)
    (let loop ((x x) (pi '()))
      (pmatch x
        (`(susp-tag ,pi^ ,v)
         (let ((v (assq (v) s)))
           (cond
             (v (loop (cdr v) (append pi^ pi)))
             (else (apply-pi pi x)))))
        (else (apply-pi pi x))))))
\end{schemedisplay}

We can now redefine \scheme|walk*| in terms of \scheme|walk|.

\schemedisplayspace
\begin{schemedisplay}
(define walk*
  (lambda (v s)
    (let ([v (walk v s)])
      (pmatch v
        (`(tie-tag ,a ,t) (list 'tie-tag a (walk* t s)))
        (`(,a . ,d) (guard (untagged? a))
         (cons (walk* a s) (walk* d s)))
        (else v)))))
\end{schemedisplay}

\scheme|unify| no longer uses \scheme|compose-subst| or
\scheme|apply-subst|.

\schemedisplayspace
\begin{schemedisplay}
(define unify
  (lambda (eqns sigma nabla fk)
    (mv-let ((sigma^ delta) (apply-sigma-rules eqns sigma fk))
      (unifyhash delta sigma^ nabla fk))))
\end{schemedisplay}

Similarly, \scheme|unifyhash| no longer uses \scheme|apply-subst|.

\schemedisplayspace
\begin{schemedisplay}
(define unifyhash
  (lambda (delta sigma nabla fk)
    (let ((delta (delta-union nabla delta)))
      (list sigma (apply-nabla-rules delta sigma fk)))))
\end{schemedisplay}

\scheme|apply-sigma-rules| now takes \scheme|sigma| as an additional
argument, which it passes to \scheme|sigma-rules|; also,
\scheme|apply-sigma-rules| no longer uses \scheme|compose-subst|.

\schemedisplayspace
\begin{schemedisplay}
(define apply-sigma-rules
  (lambda (eqns sigma fk)
    (cond
      ((null? eqns) `(,sigma ,empty-delta))
      (else
       (let ((eqn (car eqns)) (eqns (cdr eqns)))
         (mv-let ((eqns sigma delta) (or (sigma-rules eqn sigma eqns) (fk)))
           (mv-let ((sigma^ delta^) (apply-sigma-rules eqns sigma fk))
             (list sigma^ (delta-union delta^ delta)))))))))
\end{schemedisplay}

\scheme|apply-nabla-rules| also takes \scheme|sigma| as an additional
argument, which it passes to \scheme|nabla-rules|.

\schemedisplayspace
\begin{schemedisplay}
(define apply-nabla-rules
  (lambda (delta sigma fk)
    (cond
      ((null? delta) empty-nabla)
      (else
       (let ((c (car delta)) (delta (cdr delta)))
         (mv-let ((delta nabla) (or (nabla-rules c sigma delta) (fk)))
           (delta-union nabla (apply-nabla-rules delta sigma fk))))))))
\end{schemedisplay}

\scheme|sigma-rules| no longer uses \scheme|apply-subst|, but now
walks \scheme|eqns| in \scheme|sigma|, which is passed in as an
additional argument.

\schemedisplayspace
\begin{schemedisplay}
(define sigma-rules  
  (lambda (eqn sigma eqns)
    (let ((eqn (cons (walk (car eqn) sigma) (walk (cdr eqn) sigma))))
      (pmatch eqn
        (`(,c . ,c^)
         (guard (not (pair? c)) (equal? c c^))
         `(,eqns ,sigma ,empty-delta))
        (`((tie-tag ,a ,t) . (tie-tag ,a^ ,t^))
         (guard (eq? a a^))
         `(((,t . ,t^) . ,eqns) ,sigma ,empty-delta))
        (`((tie-tag ,a ,t) . (tie-tag ,a^ ,t^))
         (guard (not (eq? a a^)))
         (let ((u^ (apply-pi `((,a ,a^)) t^)))
           `(((,t . ,u^) . ,eqns) ,sigma ((,a . ,t^)))))
        (`((nom-tag __) . (nom-tag __))
         (guard (eq? (car eqn) (cdr eqn)))
         `(,eqns ,sigma ,empty-delta))
        (`((susp-tag ,pi ,x) . (susp-tag ,pi^ ,x^))
         (guard (eq? (x) (x^)))
         (let ((delta (map (lambda (a) (cons a (x))) 
                           (disagreement-set pi pi^))))
           `(,eqns ,sigma ,delta)))
        (`((susp-tag ,pi ,x) . ,t)
         (guard (not (occurs-check (x) t)))
         (let ((sigma (ext-s (x) (apply-pi (reverse pi) t) sigma)))
           `(,eqns ,sigma ,empty-delta)))
        (`(,t . (susp-tag ,pi ,x))
         (guard (not (occurs-check (x) t)))
         (let ((sigma (ext-s (x) (apply-pi (reverse pi) t) sigma)))
           `(,eqns ,sigma ,empty-delta)))
        (`((,t1 . ,t2) . (,t1^ . ,t2^))
         (guard (untagged? t1) (untagged? t1^))
         `(((,t1 . ,t1^) (,t2 . ,t2^) . ,eqns) ,sigma ,empty-delta))
        (else #f)))))
\end{schemedisplay}

\newpage

\scheme|nabla-rules| also takes \scheme|sigma| as an additional
argument, which it uses to walk \scheme|d|.

\schemedisplayspace
\begin{schemedisplay}
(define nabla-rules
  (lambda (d sigma delta)
    (let ((d (cons (walk (car d) sigma) (walk (cdr d) sigma))))
      (pmatch d
        (`(,a . ,c)
         (guard (not (pair? c)))
         `(,delta ,empty-nabla))
        (`(,a . (tie-tag ,a^ ,t))
         (guard (eq? a^ a))
         `(,delta ,empty-nabla))
        (`(,a . (tie-tag ,a^ ,t))
         (guard (not (eq? a^ a)))
         `(((,a . ,t) . ,delta) ,empty-nabla))
        (`(,a . (nom-tag __))
         (guard (not (eq? a (cdr d))))
         `(,delta ,empty-nabla))
        (`(,a . (susp-tag ,pi ,x))
         (let ((a (apply-pi (reverse pi) a)) (x (x)))
           `(,delta ((,a . ,x)))))
        (`(,a . (,t1 . ,t2))
         (guard (untagged? t1))
         `(((,a . ,t1) (,a . ,t2) . ,delta) ,empty-nabla))
        (else #f)))))
\end{schemedisplay}

The redefinition of \scheme|reify| uses the new \scheme|apply-reify-s|
function in place of some uses of \scheme|walk*|.

\schemedisplayspace
\begin{schemedisplay}
(define reify
  (lambda (x p)
    (mv-let ((sigma nabla) p)
      (let* ((v (walk* x sigma)) (s (reify-s v)) (v (apply-reify-s v s)))
        (let ((nabla (filter (lambda (a) (and (symbol? (car a)) (symbol? (cdr a))))
                       (apply-reify-s nabla s))))
          (cond
            ((null? nabla) v)
            (else `(,v : ,nabla))))))))
\end{schemedisplay}

\newpage

\scheme|apply-reify-s| is new, but is almost identical to the old
definition of \scheme|walk*| in section~\ref{akreifysection}.

\schemedisplayspace
\begin{schemedisplay}
(define apply-reify-s
  (lambda (v s)
    (pmatch v
      (`,c (guard (not (pair? c))) c)
      (`(tie-tag ,a ,t) (list 'tie-tag (get a s) (apply-reify-s t s)))
      (`(nom-tag __) (get v s))
      (`(susp-tag () __) (get v s))
      (`(susp-tag ,pi ,x) 
       (list 'susp-tag
         (map (lambda (swap)
                (pmatch swap
                  (`(,a ,b) (list (get a s) (get b s)))))
              pi)
         (get (x) s)))
      (`(,a . ,d) (cons (apply-reify-s a s) (apply-reify-s d s))))))
\end{schemedisplay}

By using triangular rather than idempotent substitutions, unification
is as much as ten times faster and is more memory efficient.

An important limitation of both the triangular and idempotent
implementations is that neither currently supports disequality
constraints.

% \section{Performance Comparison: Idempotent vs. Triangular Substitutions}\label{akperfsection}

% I'm not using anything fancy. The occurs-check is easy to turn off, but I've left it on here. I tried to keep this file as close to alphamk.scm as possible. I wrote a simple test that should take time linear in the length of the list. Even without any other optimizations, it's a huge improvement:

% Idempotent:
% > (time (run* (q) (exist (x) (testo (make-list 400) x))))
% running stats for (run* (q) (exist (x) (testo (make-list 400) x))):
%     1373 collections
%     36043 ms elapsed cpu time, including 4156 ms collecting
%     38922 ms elapsed real time, including 4609 ms collecting
%     5745917976 bytes allocated
% (_.0)

% Triangular:
% > (time (run* (q) (exist (x) (testo (make-list 400) x))))
% running stats for (run* (q) (exist (x) (testo (make-list 400) x))):
%     8 collections
%     396 ms elapsed cpu time, including 12 ms collecting
%     411 ms elapsed real time, including 14 ms collecting
%     35858608 bytes allocated
% (_.0)

% The difference in allocation was something of a surprise to me, but it makes sense. Of course, Ramana's version with unsound unification is still faster, but it's not nearly as big an improvement over the "naive" triangular version as that version is over the original:
% > (time (run #f (q) (exist (x) (testo (make-list 400) x))))
% running stats for (run #f (q) (exist (x) (testo (make-list 400) x))):
%     1 collection
%     176 ms elapsed cpu time, including 52 ms collecting
%     187 ms elapsed real time, including 55 ms collecting
%     3028096 bytes allocated
% (*0)

% This version runs most of the Pelletier problems using alphaTAP. There are a few that it can't solve without the proof (like 34). I don't *think* this is due to errors in the implementation, but I could be wrong. The aim is to make it fast while staying as close to the paper version as possible, right?

% Joe


\part{Tabling}\label{tablingpart}

% Part~\ref{tablingpart} adds tabling to our implementation of core
% miniKanren.  Tabling is a form of memoization: the answers produced by
% a tabled relation are ``remembered'' (that is, stored in a table), so
% that subsequent calls to the relation can avoid recomputing the
% answers.  Tabling allows our programs to run more efficiently in many
% cases; more importantly, many programs that would otherwise diverge
% terminate when using tabling.  Chapter~\ref{tablingchapter} introduces
% the notion of tabling, and explains which programs benefit from
% tabling.  
% Chapter~\ref{tablingimplchapter} presents our streams-based
% implementation of tabling, which demonstrates the advantage of
% embedding miniKanren in a language with higher-order functions.

\chapter{Techniques III: Tabling}\label{tablingchapter}

This chapter introduces \emph{tabling}, an extension of memoization to
logic programming.  We present a full implementation of tabling for
miniKanren in Chapter~\ref{tablingimplchapter}.

This chapter is organized as follows.  In section~\ref{tablingmemo} we
review memoization as used in functional programming.
Section~\ref{tablingintro} introduces tabling, explains how tabling
differs from memoization, and describes a few of the many applications
of tabling.  In section~\ref{tablingform} we present the
\scheme|tabled| form, used to create tabled relations.  In
section~\ref{tablingexamples} we examine several examples of tabled
relations, and in section~\ref{tablinglimitations} we discuss the
limitations of tabling.

\section{Memoization}\label{tablingmemo}

Consider the naive Scheme implementation of the Fibonacci function.

\schemedisplayspace
\begin{schemedisplay}
(define fib
  (lambda (n)
    (cond
      ((= 0 n) 0)
      ((= 1 n) 1)
      (else (+ (fib (- n 1)) (fib (- n 2)))))))
\end{schemedisplay}

\noindent 
The call \mbox{\scheme|(fib 5)|} results in calls to
\mbox{\scheme|(fib 4)|} and \mbox{\scheme|(fib 3)|}; the resulting
call to \mbox{\scheme|(fib 4)|} also calls \mbox{\scheme|(fib 3)|}.
The call \mbox{\scheme|(fib 5)|} therefore results in two calls to
\mbox{\scheme|(fib 3)|}, the second of which performs duplicate work.
Similarly, \mbox{\scheme|(fib 5)|} results in three calls to
\mbox{\scheme|(fib 2)|}, five calls to \mbox{\scheme|(fib 1)|}, and
three calls to \mbox{\scheme|(fib 0)|}.  Due to these redundant calls,
the time complexity of \scheme|fib| is exponential in $n$.

To avoid this duplicate work, we could record each distinct call to
\scheme|fib| in a table, along with the answer returned by that call.
Whenever a duplicate call to \scheme|fib| is made, \scheme|fib| would
return the answer stored in the table instead of recomputing the
result.  This optimization technique, known as \emph{memoization}
\cite{Michie68}, can result in a lower complexity class for the
running time of the memoized function.  Indeed, the memoized version
of \scheme|fib| runs in linear rather than exponential time.

Memoization is a common technique in functional programming, since it
often improves performance of recursive functions.  In this chapter we
consider the related technique of \scheme|tabling|, which generalizes
memoization to logic programming.

\section{Tabling}\label{tablingintro}

Tabling is a generalization of memoization; tabling allows a relation
to store and reuse its previously computed results.  Tabling a
relation is more complicated than memoizing a function, since a
relation returns a potentially infinite stream of substitutions rather
than a single value.  Also, the arguments to a tabled relation can
contain unassociated logic variables or partially instantiated terms,
which complicates determining whether a call is a variant of a
previously seen call.

Tabling, like memoization, can result in dramatic performance gains
for some programs.  For example, combining tabling with Prolog's Definite
Clause Grammars~\cite{dcg86} makes it trivial to write efficient
recursive descent parsers that handle left-recursion\footnote{One
  important use of tabling by Prolog systems is to handle
  left-recursive definitions of goals; due to Prolog's incomplete
  depth-first search, calls to left-recursive goals often diverge.
  Since miniKanren uses a complete search strategy, handling
  left-recursion is not a problem.  However, we will see in
  section~\ref{tablingexamples} that there are other programs we want
  to write that terminate under tabling but diverge
  otherwise.}~\cite{DBLP:conf/padl/BecketS08}---these parsers are
equivalent to ``packrat'' parsing~\cite{packrat02}.  Tabling is also
useful for writing programs that must calculate fixed points, such as
abstract interpreters and model checkers~\cite{memoingforlp,dra09}.
However, the real reason we are interested in tabling is that many
relations that would otherwise diverge terminate under tabling, as we
will see in section~\ref{tablingexamples}.

An excellent introduction to tabling and its uses is Warren's survey~\cite{memoingforlp}.

\section{The {\bf tabled} Form}\label{tablingform}

Tabled relations are constructed using the \scheme|tabled| form:

\wspace

\mbox{\scheme|(tabled (x ...) g g* ...)|}

\wspace

\noindent For example, 

\wspace

\mbox{\scheme|(define fo (tabled (z) (== z 5)))|}

\wspace

\noindent defines a top-level tabled goal constructor named
\scheme|fo|.  Each tabled goal constructor has its own local table,
which can be garbage collected once there are no live references to
the goal constructor.  Keep in mind that the table is associated with
the goal constructor, not the goal returned by the goal constructor.

Calls to a tabled relation come in two flavors: master calls and slave
calls.  A \emph{master call} is a call to a tabled relation whose
arguments are not (yet) stored in the table.  A \emph{slave call} is a
call whose arguments are found in the table; each slave call is a
\emph{variant} of some master call.

Two calls to the same tabled relation are variants of each other if
their arguments are the same, up to consistent renaming of
unassociated logic variables\footnote{In other words, the two lists of arguments
  to the relation, when reified with respect to their ``current'' substitutions, 
  must be \scheme|equal?|.}. For example, consider the calls
\mbox{\scheme|(mulo y z 5)|} and \mbox{\scheme|(mulo w w x)|} in the
substitutions \mbox{\scheme|`((y . z))|} and \mbox{\scheme|`((x . 5))|},
respectively.  Taking the substitutions into account, these calls are equivalent to
\mbox{\scheme|(mulo z z 5)|} and \mbox{\scheme|(mulo w w 5)|}, which are variants of 
each other.  However, the calls \mbox{\scheme|(mulo w w x)|} and \mbox{\scheme|(mulo y z z)|}
are variants only if \scheme|w| is associated with \scheme|x|, and \scheme|y| is associated with \scheme|z|,
respectively.  For the same reason, 
\mbox{\scheme|(mulo w 5 6)|} and \mbox{\scheme|(mulo y z 6)|} are variants only if 
\scheme|z| is associated with 5 in the substitution in place for the second call.

\section{Tabling Examples}\label{tablingexamples}

We are now ready to examine examples of tabled relations.  The
canonical example relation, \scheme|patho|\footnote{The path examples
  in this section are taken from~\citet{memoingforlp}.}, finds all
paths between two nodes in a directed graph.  The goal
\mbox{\scheme|(patho x y)|} succeeds if there is a directed edge from
\scheme|x| to \scheme|y|, or if there is an edge from \scheme|x| to
some node \scheme|z| and there is a path from \scheme|z| to
\scheme|y|.

\schemedisplayspace
\begin{schemedisplay}
(define patho
  (lambda (x y)
    (conde
      ((arco x y))
      ((exist (z)
         (arco x z)
         (patho z y))))))
\end{schemedisplay}

\noindent The goal \mbox{\scheme|(arco x y)|} succeeds if there is a
directed edge from node \scheme|x| to node \scheme|y|.

\schemedisplayspace
\begin{schemedisplay}
(define arco
  (lambda (x y)
    (conde
      ((== 'a x) (== 'b y))
      ((== 'c x) (== 'b y))
      ((== 'b x) (== 'd y)))))
\end{schemedisplay}

\noindent This definition of \scheme|arco| represents edges from
\scheme|'a| to \scheme|'b|, \scheme|'c| to \scheme|'b|, and
\scheme|'b| to \scheme|'d|.

\noindent The expression \mbox{\scheme|(run* (q) (patho 'a q))|}
returns \mbox{\scheme|'(b d)|}, indicating that only the nodes
\scheme|'b| and \scheme|'d| are reachable from \scheme|'a|.

Now let us redefine \scheme|arco| to represent a different set of
directed edges, this time with a circularity between nodes \scheme|'a|
and \scheme|'b|.

\newpage

\schemedisplayspace \begin{schemedisplay}
(define arco
  (lambda (x y)
    (conde
      ((== 'a x) (== 'b y))
      ((== 'b x) (== 'a y))
      ((== 'b x) (== 'd y)))))
\end{schemedisplay}

\noindent Using the new definition of \scheme|arco|, the expression
\mbox{\scheme|(run* (q) (patho 'a q))|} now diverges.  We can
understand the cause of this divergence if we replace \scheme|run*|
with \scheme|run10|.

\wspace

\noindent\scheme|(run10 (q) (patho 'a q))|  $\Rightarrow$ \scheme|'(b a d b a d b a d b)|

\wspace

\noindent Because of the circular path between \scheme|'a| and
\scheme|'b|, \mbox{\scheme|(patho 'a q)|} keeps finding longer and
longer paths between \scheme|'a| and the nodes \scheme|'b|, \scheme|'a|,
and \scheme|'d|.  To avoid this problem, we can table
\scheme|patho|.

\schemedisplayspace
\begin{schemedisplay}
(define patho
  (tabled (x y)
    (conde
      ((arco x y))
      ((exist (z)
         (arco x z)
         (patho z y))))))
\end{schemedisplay}

\noindent \mbox{\scheme|(run* (q) (patho 'a q))|} then converges,
returning \mbox{\scheme|'(b a d)|}.

Now let us consider a mutually recursive program.

\schemedisplayspace
\begin{schemedisplay}
(letrec ((fo (lambda (x)
               (conde
                 ((== 0 x))
                 ((go x)))))
         (go (lambda (x)
               (conde
                 ((== 1 x))
                 ((fo x))))))
  (run* (q) (fo q)))
\end{schemedisplay}

\noindent This expression diverges.  If we replace
\scheme|run*| with \scheme|run10| the program converges with the value
\mbox{\scheme|'(0 1 0 1 0 1 0 1 0 1)|}.  If we table either
\scheme|fo|, \scheme|go|, or both, \mbox{\scheme|(run* (q) (fo q))|}
converges with the value \mbox{\scheme|'(0 1)|}.


\section{Limitations of Tabling}\label{tablinglimitations}

Tabling is a remarkably useful addition to miniKanren, and can be
used to improve efficiency of relations and
(sometimes) avoid divergence.  Unfortunately, tabling is not a
panacea.  In fact, tabling can be trivially defeated by changing one
or more arguments in each call to a tabled relation.  For example,
consider the ternary multiplication relation \scheme|mulo| from
Chapter~\ref{arithchapter}.  The arguments in the call

\wspace

\noindent
\mbox{\scheme|(mulo `(1 1 . ,x) x `(0 0 0 1 . ,x))|}\footnote{This example is due to Oleg
  Kiselyov (personal communication).}

\wspace

\noindent
all share the variable \scheme|x|.
The resulting goal succeeds only if there exists a non-negative
integer $x$ that satisfies $(3 + 4x) \cdot x = 8 + 16x$.  \scheme|mulo| 
enumerates all non-negative integer values for \scheme|x| until it
finds one that satisfies this equation.  However, if no such
\scheme|x| exists the call to \scheme|mulo| will diverge.  Tabling
will not help, since the value of \scheme|x| keeps changing.

Another disadvantage of tabling is that it can greatly increase the
memory consumption of a program.  This is a problem with memoization
in general.  For example, consider the tail-recursive
accumulator-passing-style Scheme definition of factorial\footnote{The
  call \mbox{\scheme|(!-aps n 1)|} calculates the factorial of
  \scheme|n|.}.

\schemedisplayspace
\begin{schemedisplay}
(define !-aps
  (lambda (n a)
    (cond
      ((zero? n) a)
      (else (!-aps (sub1 n) (* n a))))))
\end{schemedisplay}

\noindent Other than the space used to represent numbers, this
function uses a bounded amount of memory\footnote{Scheme
  implementations are required to handle tail calls properly---thus
  \scheme|!-aps| uses a constant amount of stack space}.  However, the
memoized version of \scheme|!-aps| uses an unbounded amount of memory
if \scheme|n| is negative, and otherwise uses an amount of memory
linear in \scheme|n|.

Chapter~\ref{tablingimplchapter} presents a complete implementation of
tabling for miniKanren; this implementation has several limitations.
The first limitation is that tabled relations must be closed; a tabled
goal constructor cannot contain free logic variables, since
associations for those variables would be thrown away.  This is a
consequence of not storing entire substitutions in a relation's table,
as described in section~\ref{tablingrep}.

Another limitation is that arguments passed to tabled relations must
be ``printable'' (or ``reifiable'') values.  For example, tabled
relations should never be passed functions, including goals, since all
functions reify to the same value\footnote{Pure relations should never
take functions as arguments anyway, since miniKanren does not support
higher-order unification, and cannot meaningfully construct functions
when running backwards.}.

The most significant limitation of our tabling implementation is that
it does not currently support disequality constraints, nominal
unification, or freshness constraints.  How to best combine tabling
and constraints is an open research problem~\cite{TCHR08}.


\chapter{Implementation V: Tabling}\label{tablingimplchapter}

In this chapter we implement the tabling scheme described in
Chapter~\ref{tablingchapter}.  Our tabling implementation extends the
streams-based implementation of miniKanren from
Chapter~\ref{mkimplchapter}, preserving the original implementation's
interleaving search behavior.

This chapter is organized as follows.  In section~\ref{tablingrep} we
describe the core data structures used in the implementation.
Section~\ref{tablingalgorithm} gives a high-level description of the
tabling algorithm.  In section~\ref{tablingcaseinf} we introduce a new
type of \emph{waiting} stream, which requires extending both
\scheme|case-inf| and the operators that use it: \scheme|take|,
\scheme|bind|, and \scheme|mplus|.  Section~\ref{tablingreify} extends
the reifier from Chapter~\ref{mkimplchapter} with a new function
\scheme|reify-var|.  Finally in section~\ref{tablingmainimpl} we
present the heart of the tabling implementation: the user-level
\scheme|tabled| form, and the \scheme|master| and \scheme|reuse|
functions to handle master and slave calls, respectively.

% [TODO May want to describe data structures before giving a high-level
% description of the algorithm---I'm not sure which is better.]

\section{Answer Terms, Caches, and Suspended Streams}\label{tablingrep}

Like any goal, a goal returned by a tabled goal constructor is a
function mapping a substitution to a stream of substitutions.  The
goal constructor's table does not store entire substitutions; rather,
the table stores \emph{answer terms}.  An answer term is a list of the
arguments from a master call, perhaps partially or fully instantiated
as a result of running the goal's body.  A \scheme|cache| 
associates each master call with a set of answer terms.  A subsequent
slave call reuses the master call's tabled answers by unifying each
answer term in the cache with the slave call's actual parameters,
producing a stream of answer substitutions.

There may be multiple slave calls associated with each master call;
each slave call ``consumes'' \emph{all} the tabled answer terms in the
cache.  Evaluation of the master call and its slave calls are
interleaved---slave calls may start consuming answer terms before the
master call has finished producing them.  When a master call produces
new answer terms, the consumption of these answers by associated slave
calls can result in new master or slave calls.  The algorithm reaches
a fixed point when all master calls have finished producing answers,
and each slave call has consumed every answer term produced by its
associated master call.

To understand why we table answer terms rather than full
substitutions, consider this \scheme|run*| expression.

\schemedisplayspace
\begin{schemedisplay}
(let ((f (tabled (z) (== z 6))))
  (run* (q)
    (exist (x y)
      (conde
        ((== x 5) (f y))
        ((f y)))
      (== `(,x ,y) q))))
\end{schemedisplay}

\noindent 
Imagine that the first \scheme|conde| clause is evaluated completely
before the second clause.  When the master call \mbox{\scheme|(f y)|} 
in the first clause succeeds, the substitution will be \mbox{\scheme|`((,y . 6) (,x . 5))|}.
If we were to table the full substitution, including the association for \scheme|x|, 
the slave call in the second clause would incorrectly associate \scheme|x| with 5.
The \scheme|run*| expression would therefore return
\mbox{\scheme|'((5 6) (5 6))|} instead of the correct answer
\mbox{\scheme|'((5 6) (_.0 6))|}.

Since the table records answer terms rather than entire substitutions,
a tabled goal constructor must be closed with respect to logic
variables; values associated with free logic variables would be
forgotten.  For example, the \scheme|run*| expression

\schemedisplayspace
\begin{schemedisplay}
(run* (q)
  (exist (x y)
    (let ((f (lambda (z) (exist () (== x 5) (== z 6)))))
      (conde
        ((f y) (== `(,x ,y) q))
        ((f y) (== `(,x ,y) q))))))
\end{schemedisplay}

\noindent returns \mbox{\scheme|`((5 6) (5 6))|}, as expected.  However, 
if we were to table \scheme|f| by replacing 

\noindent \mbox{\scheme|(lambda (z) (exist () (== x 5) (== z 6)))|} with 
\mbox{\scheme|(tabled (z) (exist () (== x 5) (== z 6)))|}, 
the \scheme|run*| expression would instead return \mbox{\scheme|`((5 6) (_.0 6))|}.

Each tabled goal constructor has its own local table represented as a
list of \mbox{\scheme|`(,key . ,cache)|} pairs, where \scheme|key| is
a list of reified arguments from a master call, and where
\scheme|cache| contains the set of answer terms for that master call.

A cache is represented as a tagged vector, and contains a list
of tabled answer terms.  Each master call is associated with a single
cache.

\schemedisplayspace
\begin{schemedisplay} 
(define make-cache (lambda (ansv*) (vector 'cache ansv*)))
(define cache-ansv* (lambda (c) (vector-ref c 1)))
(define cache-ansv*-set! (lambda (c val) (vector-set! c 1 val)))
\end{schemedisplay}

Each slave call is associated with a single \emph{suspended stream},
or \scheme|ss|.  \noindent Each suspended stream is represented as a
tagged vector containing a cache, a list of tabled answer terms
\scheme|ansv*|, and a thunk that produces the remainder of the stream
(an \scheme|f|, as described in section~\ref{goalconstructors}).

\schemedisplayspace
\begin{schemedisplay}
(define make-ss (lambda (cache ansv* f) (vector 'ss cache ansv* f)))
(define ss? (lambda (x) (and (vector? x) (eq? (vector-ref x 0) 'ss))))
(define ss-cache (lambda (ss) (vector-ref ss 1)))
(define ss-ansv* (lambda (ss) (vector-ref ss 2)))
(define ss-f (lambda (ss) (vector-ref ss 3)))
\end{schemedisplay}

\noindent The \scheme|ansv*| list indicates which of the master call's
answer terms the suspended stream has already
processed---\scheme|ansv*| is always a suffix of the list in
\scheme|cache|.  There may be many suspended streams associated with a
single \scheme|cache|---each of these \scheme|ss|'s may contain
a different \scheme|ansv*| list, representing a different ``already
seen'' suffix of answer terms from the cache.

The \scheme|ss-ready?| predicate indicates whether a suspended
stream's cache contains new answer terms not yet consumed by the
stream.

\schemedisplayspace
\begin{schemedisplay}
(define ss-ready? (lambda (ss) (not (eq? (cache-ansv* (ss-cache ss)) (ss-ansv* ss)))))
\end{schemedisplay}

% \scheme|ansv*| is only used for comparison with the cached answer
% terms in \scheme|cache|, to determine if there are new answer terms
% for a slave call.  If so, \scheme|ansv*| will not be \scheme|eq?| to
% the list in \scheme|cache|---the prefix of \scheme|cache| up to the
% beginning of \scheme|ansv*| are the answer terms that have not yet
% been handled by the slave call.


\section{The Tabling Algorithm}\label{tablingalgorithm}

Now that we are familiar with the fundamental data structures, we can
examine in detail the steps performed when a tabled goal constructor
is called:

%\mbox{\scheme|(appendo '(a b) '(c) z)|}:

\begin{enumerate}
\item The goal constructor creates a list of the arguments
  passed to the call, \scheme|argv|, then returns a goal.
\item When passed a substitution \scheme|s|, the goal reifies
  \scheme|argv| in \scheme|s|, producing a list \scheme|key| of
  reified arguments.

% [This is just WRONG!  Standard unification is used in this step.]
%   The reification procedure used for this step differs from the
%   standard reification algorithm (Chapter~\ref{mkimplchapter}) in that
%   unassociated variables are consistently replaced with newly created
%   logic variables rather than with symbols.  This \scheme|reify-var|
%   function is reminiscent of Prolog's {\tt copy\_term/2}, but is a
%   Scheme procedure rather than a goal.  This step is necessary to
%   avoid prematurely instantiating the original logic variables later
%   in the algorithm.

\item The goal uses the reified list of arguments as the lookup key in
  the goal constructor's local table, which is an association list of
  \mbox{\scheme|`(,key . ,cache)|} pairs.
\item If the key is not in the table's association list we are making
  a new master call.  The goal constructs a new cache
  containing the empty list.  The goal then side-effects the local
  table, extending it with a pair containing the new key and cache.
  Next, a ``fake'' subgoal is added to the body of the goal.
  When passed a substitution, this ``fake'' goal checks if the answer
  term about to be cached is equivalent to an existing answer term in the
  cache; if so, the fake goal fails, keeping the master call from
  producing a duplicate answer.  Otherwise, the fake goal extends the
  cache with the new answer term, then returns the answer substitution
  as a singleton stream\footnote{This singleton stream is actually a
    \emph{waiting stream}, described in
    section~\ref{tablingcaseinf}.}.
\item If, on the other hand, the key is found in the table's
  association list, we are making a slave call.  Instead of re-running
  the body of the goal, we reuse the tabled answers from the
  corresponding master call.  The slave call produces a stream of
  answer substitutions by unifying, in the current substitution,
  \scheme|ansv*| with each cached answer term.  Due to miniKanren's
  interleaving search, a master call may not produce all of its
  answers immediately.  Therefore, the answer stream produced by a
  slave call may need to suspend periodically, ``awakening'' when the
  master call produces new answer terms for the slave to consume.
\end{enumerate}

Recall that the algorithm reaches a fixed point when all the master
calls have finished producing answers, and each slave call has
consumed every answer term produced by its corresponding master call.
In the process of consuming a cached answer term, a slave call might
make a new master or slave call.

\section{Waiting Streams}\label{tablingcaseinf}

We extend the \mbox{\scheme|a-inf|} stream datatype described in
section~\ref{goalconstructors} with a new variant: a \scheme{waiting} stream \scheme|w|
is a non-empty proper list \mbox{\scheme|`(,ss ,ss* ...)|} of suspended streams.
The waiting stream datatype allows us to express a disjunction of suspended streams; just
as importantly, the datatype makes it easier to recognize when a fixed point has been reached,
as described below.

\schemedisplayspace
\begin{schemedisplay}
(define w? (lambda (x) (and (pair? x) (ss? (car x)))))
\end{schemedisplay}

\noindent New singleton waiting streams are created in the
\scheme|reuse| function described in section~\ref{tablingmainimpl}.
The only way to create a waiting stream containing multiple suspended
streams is through disjunction (see the definition of \scheme|mplus|
below).

The addition of the waiting stream type requires us to extend the
definition of \scheme|case-inf| from section~\ref{goalconstructors}
with a new \scheme|w| clause.

\schemedisplayspace
\begin{schemedisplay} 
(define-syntax case-inf
  (syntax-rules ()
    ((_ e (() e0) ((f^) e1) ((w) ew) ((a^) e2) ((a f) e3))
     (let ((a-inf e))
       (cond
         ((not a-inf) e0)
         ((procedure? a-inf) (let ((f^ a-inf)) e1))
         ((and (pair? a-inf) (procedure? (cdr a-inf)))
          (let ((a (car a-inf)) (f (cdr a-inf))) e3))
         ((w? a-inf) (w-check a-inf
                              (lambda (f^) e1)
                              (lambda () (let ((w a-inf)) ew))))
         (else (let ((a^ a-inf)) e2)))))))
\end{schemedisplay}

The new clause of \scheme|case-inf| expands into a call to
\scheme|w-check|, which takes a waiting stream \scheme|w|, a success
continuation \scheme|sk|, and a failure continuation \scheme|fk|.
\scheme|w-check| plays a critical role in finding the fixed point of a
program.

\scheme|w-check| looks in \scheme|w| for the first suspended stream
\scheme|ss| whose cache contains new answer terms.  If none of the
suspended streams contain unseen answer terms, \scheme|w-check|
invokes the failure continuation.  Otherwise, \scheme|sk| is passed an
\scheme|f|-type stream containing the new answers produced by
\scheme|ss|, interleaved with answers from a new waiting stream
containing the remaining suspended streams in \scheme|w|.

% \scheme|w-check| determines if one of the suspended streams in
% \scheme|w| has not yet handled all of the answer terms associated with
% its cache.  If so, \scheme|sk| is passed a new stream that contains
% the new answers, interleaved with a new \scheme|w| containing all the
% other suspended streams.  If none of the suspended streams contain
% unseen answer terms, the failure continuation \scheme|fk| is invoked.

\schemedisplayspace
\begin{schemedisplay}
(define w-check
  (lambda (w sk fk)
    (let loop ((w w) (a '()))
      (cond
        ((null? w) (fk))
        ((ss-ready? (car w))
         (sk (lambdaf@ ()
               (let ((f (ss-f (car w)))
                     (w (append (reverse a) (cdr w))))
                 (if (null? w) (f) (mplus (f) (lambdaf@ () w)))))))
        (else (loop (cdr w) (cons (car w) a)))))))
\end{schemedisplay}

The \scheme|w| case of \scheme|case-inf| actually represents
\emph{two} cases: in the first case, the waiting stream can produce
new answers; in the second case, the stream cannot produce new
answers, although it may be able to in the future.

In the first case, \scheme|w| contains a suspended stream \scheme|ss|
ready to produce new answers.  \scheme|w-check| creates a new
\scheme|f|-type stream encapsulating the answers from \scheme|ss|,
along with the remainder of the \scheme|w| stream (if non-empty).
\scheme|case-inf| then processes this stream as it would any other
\scheme|f|: by evaluating the expression \scheme|e1| in an extended
environment in which \scheme|f^| is bound to the new stream. To see
this more clearly, think of the \scheme|sk| passed to
\scheme|w-check|, \mbox{\scheme|(lambda (f^) e1)|}, as the equivalent
\mbox{\scheme|(lambda (a-inf) (let ((f^ a-inf)) e1))|}, which exactly
mirrors the code produced in the \scheme|f^| case of
\scheme|case-inf|.

In the second case, none of the suspended streams in \scheme|w| can
produce new answers.  We have therefore reached a fixed point, at
least temporarily; this case is analogous to the \scheme|()| case of
\scheme|case-inf|.  Unlike in the \scheme|()| case, however,
\scheme|w| might produce answers later.  \scheme|ew| is evaluated in
an extended environment in which \scheme|w| is bound to the waiting
stream---this is made most clear in the \scheme|w| case of
\scheme|mplus|, below.

% In the second case, the waiting stream cannot (yet) produce more
% answers---therefore, this case is most similar to the () case of
% case-inf.  However, unlike in the () case, there is still the possibility
% that the waiting stream will produce more answers later on.

Since we have added a clause to \scheme|case-inf| we must redefine
\scheme|take|, \scheme|bind|, and \scheme|mplus|.  These functions
differ from their definitions in section~\ref{goalconstructors} only
in the addition of the \scheme|w| case.  However, the \scheme|w| case
implicitly uses the expression specified for the \scheme|f| case as
well\footnote{Or the \scheme|f^| case, in the definition of
  \scheme|mplus|.}, if \scheme|w| contains a suspended stream ready to
produce new answers.  In this event, a new \scheme|f|-type stream is
constructed that contains not only these new answers but also the
remaining suspended streams in \scheme|w|; this new stream is then
handled by the \scheme|f| case of \scheme|case-inf|.

\newpage

Here is the updated definition of \scheme|take|.

\schemedisplayspace
\begin{schemedisplay} 
(define take
  (lambda (n f)
    (if (and n (zero? n)) 
      '()
      (case-inf (f)
        (() '())
        ((f) (take n f))
        ((w) '())
        ((a) a)
        ((a f) (cons (car a) (take (and n (- n 1)) f)))))))
\end{schemedisplay}

\noindent 
If \scheme|w| contains a suspended stream ready to produce a new
stream of answers, this new stream is handled by the \scheme|f| case
of \scheme|case-inf|.  Otherwise, we have reached a fixed
point---therefore, \scheme|take| returns the empty list.

Here is the updated definition of \scheme|bind|.

\schemedisplayspace
\begin{schemedisplay} 
(define bind
  (lambda (a-inf g)
    (case-inf a-inf
      (() (mzero))
      ((f) (inc (bind (f) g)))
      ((w) (map (lambda (ss)
                  (make-ss (ss-cache ss) (ss-ansv* ss)
                           (lambdaf@ () (bind ((ss-f ss)) g))))
                w))
      ((a) (g a))
      ((a f) (mplus (g a) (lambdaf@ () (bind (f) g)))))))
\end{schemedisplay}

\noindent If \scheme|w| contains a suspended stream ready to produce a new
stream of answers, this new stream is handled by the \scheme|f| case
of \scheme|case-inf|.  Otherwise, the binding of answer substitutions
to \scheme|g| must be delayed, because the streams in \scheme|w| are
all suspended.  \scheme|bind| reconstructs the list \scheme|w|,
pushing the bind operation into each rebuilt suspended stream.  If a
stream is awakened later, it will then bind its new answers to
\scheme|g|.

Here is the updated definition of \scheme|mplus|.

\schemedisplayspace
\begin{schemedisplay} 
(define mplus
  (lambda (a-inf f)
    (case-inf a-inf
      (() (f))
      ((f^) (inc (mplus (f) f^)))
      ((w) (lambdaf@ () (let ((a-inf (f)))
                          (if (w? a-inf)
                              (append a-inf w)
                              (mplus a-inf (lambdaf@ () w))))))
      ((a) (choice a f))
      ((a f^) (choice a (lambdaf@ () (mplus (f) f^)))))))
\end{schemedisplay}

\noindent If \scheme|w| contains a suspended stream ready to produce a
new stream of answers, this new stream is handled by the \scheme|f|
case of \scheme|case-inf|.  Otherwise, \scheme|mplus| returns a new
\scheme|f|-type stream.  If the second argument to \scheme|mplus|
produces a waiting stream \scheme|w^|, then \scheme|mplus| appends the
lists \scheme|w^| and \scheme|w|, creating a single combined waiting
stream.  If \scheme|mplus|'s second argument produces an
\scheme|a-inf| that is \emph{not} a waiting stream, then \scheme|w| is
``pushed'' to the back of the new stream.  Accumulating all suspended
streams in a single waiting stream at the end of an \scheme|f|-type
stream allows \scheme|w-check| to easily determine if a fixed point
has been reached.

\section{Extending and Abstracting Reification}\label{tablingreify}

To avoid prematurely instantiating logic variables, the
\scheme|master| and \scheme|reuse| procedures in
section~\ref{tablingmainimpl} copy the list \scheme|argv| of arguments
passed to the tabled goal constructor.  This operation is performed by the
\scheme|reify-var| function, which is similar in spirit to Prolog's
\mbox{{\tt copy\_term/2}}, but is implemented by a function rather than a
user-level goal constructor.

Our implementation of \scheme|reify-var| is identical to that of the
\scheme|reify| function from Chapter~\ref{mkimplchapter}, except that
unassociated variables are consistently replaced with newly created
logic variables rather than with symbols.  We therefore abstract the
reification operators, defining them in terms of the
\scheme|make-reify| helper, which in turn uses an abstracted version
of \scheme|reify-s|.

\schemedisplayspace
\begin{schemedisplay} 
(define make-reify
  (lambda (rep)
    (lambda (v s)
      (let ((v (walk* v s)))
        (walk* v (reify-s rep v empty-s))))))

(define reify (make-reify reify-name))

(define reify-var (make-reify reify-v))

(define reify-v
  (lambda (n)
    (var n)))

(define reify-s
  (lambda (rep v s)
    (let ((v (walk v s)))
      (cond
        ((var? v) (ext-s-no-check v (rep (length s)) s))
        ((pair? v) (reify-s rep (cdr v) (reify-s rep (car v) s)))
        (else s)))))
\end{schemedisplay}


\section{Core Tabling Operators}\label{tablingmainimpl}

We are now ready to define the core tabling operators.  The \scheme|tabled|
user-level form creates a tabled goal constructor, complete with an empty local
association list \scheme|table| that will contain \mbox{\scheme|`(,key . ,cache)|} pairs.
Section~\ref{tablingalgorithm} describes the behavior of tabled goal constructors
at a high level; most of the interesting work is performed in the
\scheme|master| and \scheme|reuse| helpers, defined below.

\schemedisplayspace
\begin{schemedisplay}
(define-syntax tabled
  (syntax-rules ()
    ((_ (x ...) g g* ...)
     (let ((table '()))
       (lambda (x ...)
         (let ((argv (list x ...)))
           (lambdag@ (s)
             (let ((key (reify argv s)))
               (cond
                 ((assoc key table)
                  => (lambda (key.cache) (reuse argv (cdr key.cache) s)))
                 (else (let ((cache (make-cache '())))
                         (set! table (cons `(,key . ,cache) table))
                         ((exist () g g* ... (master argv cache)) s))))))))))))
\end{schemedisplay}

The \scheme|master| function is invoked during a master call, and
returns a ``fake'' goal run at the end of the body of the tabled
goal. This fake goal checks if the answer term about to be cached is
equivalent to an answer term already in the cache.  If so, the call to
the fake goal fails, to avoid producing a duplicate answer.
Otherwise, the goal succeeds, caching the new answer term
before returning the answer substitution.

\schemedisplayspace
\begin{schemedisplay}
(define master
  (lambda (argv cache)
    (lambdag@ (s)
      (and
        (for-all
          (lambda (ansv) (not (alpha-equiv? argv ansv s)))
          (cache-ansv* cache))
        (begin
          (cache-ansv*-set! cache (cons (reify-var argv s) (cache-ansv* cache)))
          s)))))
\end{schemedisplay}

\scheme|alpha-equiv?| returns true if \scheme|x| and \scheme|y|
represent the same term, modulo consistent replacement of unassociated
logic variables.

\schemedisplayspace
\begin{schemedisplay}
(define alpha-equiv?
  (lambda (x y s)
    (equal? (reify x s) (reify y s))))
\end{schemedisplay}

\scheme|reuse| constructs a stream of answer substitutions for a slave
call, using the cached answer terms from the corresponding master
call.  Like \scheme|w-check|, \scheme|reuse| plays a critical role in
calculating the fixed point of a program.  Each call to \scheme|loop|
returns an \mbox{\scheme|`(,a . ,f)|}-type stream until all the answer
terms in the cache have been consumed.  \scheme|reuse| then returns a
waiting stream\footnote{This is the only code that introduces a new
  waiting stream, as opposed to rebuilding or appending existing
  waiting streams.} encapsulating a single suspended stream whose
\scheme|f| calls the outer \scheme|fix| loop, consuming any answer
terms produced by the master call while the stream was suspended.
Invoking \scheme|f| restarts the search for a fixed point; to avoid
divergence, \scheme|w-check| does not invoke the \scheme|f| of any
suspended stream that does not contain unseen answer terms.

\schemedisplayspace
\begin{schemedisplay}
(define reuse
  (lambda (argv cache s)
    (let fix ((start (cache-ansv* cache)) (end '()))
      (let loop ((ansv* start))
        (if (eq? ansv* end)
            (list (make-ss cache start (lambdaf@ () (fix (cache-ansv* cache) start))))
            (choice (subunify argv (reify-var (car ansv*) s) s)
                    (lambdaf@ () (loop (cdr ansv*)))))))))
\end{schemedisplay}

\scheme|reuse| depends on \scheme|subunify| to unify the list of
unreified arguments in the slave call with a copy of each cached
answer term.  Since we know that the unification will succeed, the
definition of \scheme|subunify| is shorter and more efficient than the
definition of \scheme|unify| from Chapter~\ref{mkimplchapter}.

\schemedisplayspace
\begin{schemedisplay}
(define subunify
  (lambda (arg ans s)
    (let ((arg (walk arg s)))
      (cond
        ((eq? arg ans) s)
        ((var? arg) (ext-s-no-check arg ans s))
        ((pair? arg) (subunify (cdr arg) (cdr ans)
                       (subunify (car arg) (car ans) s)))
        (else s)))))
\end{schemedisplay}

The code in this chapter is short but extremely subtle.  This subtlety
is due to the use of side effects, interaction of multiple functions
to calculate fixed points, and introduction of the suspended stream
and waiting stream datatypes.  Understanding how the manipulation of
waiting streams by \scheme|mplus| makes the definition of
\scheme|w-check| possible is especially subtle.  To fully appreciate
this last point, the reader is encouraged to modify the implementation
by replacing all uses of waiting streams with suspended streams, and
then ascertain why many tabled programs diverge as a result.


\part{Ferns}\label{fernspart}

% Part~\ref{fernspart} presents a bottom-avoiding data structure called
% \emph{ferns}, and shows how ferns can be used to avoid
% expression-level divergence.  Chapter~\ref{fernschapter} introduces
% the ferns data structure and implements a simple, miniKanren-like
% language using ferns.  Chapter~\ref{fernsimpl} presents our embedding
% of ferns in Scheme.

%\chapter{Advanced Techniques V:  Ferns-based miniKanren}\label{fernschapter}
\chapter{Techniques IV:  Ferns}\label{fernschapter}

In this chapter we provide a bottom-avoiding generalization of core
miniKanren using \emph{ferns} \cite{ferns81}, a shareable data
structure designed to avoid divergence.

The chapter is organized as follows. Section~\ref{fernsintro}
introduces the ferns data structure and shows examples of familiar
recursive functions using ferns.  Section~\ref{Sharing} describes the
promotion algorithm \cite{Friedman79b} that characterizes the
necessary sharing properties of ferns.  Section~\ref{lp-system}
defines bottom-avoiding logic programming goal constructors,
corresponding to core miniKanren with non-interleaving search.
Chapter~\ref{fernsimpl} presents a complete \emph{shallow
  embedding} \cite{Boulton92tassel.experience} of the ferns data
structure and related operators.

\section{Introduction to Ferns}\label{fernsintro}

Ferns are constructed with \scheme|cons| and \scheme|frons|,
originally called \textbf{frons} \cite{DFried80}, and accessed by
\scheme|fcar| and \scheme|fcdr|, generalizations of \scheme|car| and
\scheme|cdr|, respectively.  Ferns built with \scheme|frons| are like
streams in that the \emph{evaluation} of elements is delayed,
permitting unbounded data structures.  In contrast to streams, the
\emph{ordering} of elements is also delayed: convergent values form
the prefix in some unspecified order, while divergent values form the
suffix.

We begin with several examples that illustrate the
properties of ferns, showing their similarities to
and differences from traditional lists and streams.  Later, we include
examples that show that a natural recursive style can be used when
programming with ferns and point out the advantages ferns afford the
user.

\subsection{Two Simple Programs}

Convergent elements of a fern form its prefix in some unspecified
order. For example, evaluating the expression

\schemedisplayspace
\schemeinput{fernscode/intro-1}

\noindent 
prints either \schemeresult|010| or \schemeresult|101|,
demonstrating that the order of values within a fern is not specified in
advance but remains consistent once determined, while

\schemedisplayspace
\schemeinput{fernscode/intro-2}

\noindent returns \mbox{\schemeresult|(720 . 120)|}, demonstrating
that accessing a fern avoids divergence as much as possible.
(\scheme|bottom| is any expression whose evaluation diverges.)  In the
latter example, each fern contains only one convergent value; taking
the \scheme|fcdr| of \scheme|s1| or the \scheme|fcadr| of \scheme|s2|
results in divergence.

Ferns are \emph{shareable} data structures; sharing, combined with
delayed ordering of values, can result in surprising behavior.  For
example, consider these expressions:

\schemedisplayspace
\schemeinput{fernscode/not-so-weird-sharing}

\noindent and


\schemedisplayspace
\schemeinput{fernscode/weird-sharing-1}

\noindent
The first expression must evaluate to \mbox{\schemeresult|(1 2
  2)|}.  The second expression may also return this value---as
expected, the car of \scheme|b| would then be equal to the cadr of
\scheme|a|.  The second expression might instead return
\mbox{\schemeresult|(2 1 2)|} however; in this case, the car of \scheme|b|
would be equal to the car of \scheme|a| rather than to its cadr.
Section~\ref{Sharing} discusses sharing in detail.

\subsection{Recursion}\label{fernsRecursiveExamples}

We now present examples of the use of ferns in simple recursive
functions. Consider the definition of \scheme|ints-bottom|\footnote{\scheme|timed-lambda| is identical to
  \scheme|lambda|, except it creates preemptible procedures. 
(See Appendix~\ref{nestable-engines}.)}.

\schemedisplayspace
\schemeinput{fernscode/ints}

\noindent Then \mbox{\scheme|(fcaddr (ints-bottom 0))|} 
could return any non-negative integer, whereas a
stream version would return \schemeresult|2|.

%But wait---there's more!

% \schemedisplayspace
%\schemeinput{fernscode/code/ints-bottom}

There is a tight relationship between ferns and lists, since every
cons pair is a fern.  The empty
fern is also represented by \scheme|`()|, and
\mbox{\scheme|(pair? (frons e1 e2))|} returns \schemeresult|#t| for
all \scheme|e1| and \scheme|e2|.  After replacing the list constructor
\scheme|cons| with the fern constructor \scheme|frons|, many recursive
functions operating on lists avoid divergence.  For example,
\scheme|map-bottom| is defined
by replacing \scheme|cons| with \scheme|frons|, \scheme|car| with
\scheme|fcar|, and \scheme|cdr| with \scheme|fcdr| in the definition of
\scheme|map|, and can map a function over an unbounded fern:
% inlined this from code/map-example.ss
the value of \mbox{\scheme|(fcaddr (map-bottom add1 (ints-bottom 0)))|} can be any positive integer.


%The similarities between ferns and the more traditional list and
%stream data structures make it possible to write bottom-avoiding
%functions in the natural recursive style---indeed, this ability to
%define functions in a familiar style is one of the advantages of using
%ferns.

Ferns work especially well with \emph{annihilators}. True values are
annihilators for \scheme|or-bottom|

\schemedisplayspace
\schemeinput{fernscode/or-fn}

\noindent which searches in a fern for a true convergent value
and avoids divergence if it finds one:
\mbox{\scheme|(or-bottom (fern bottom (odd? 1) (! 5) bottom (odd? 0)))|}
returns some true value, where \scheme|fern| is defined as follows.

\schemedisplayspace
\schemeinput{fernscode/fern}

% We get additional benefit when annihilators are common. Scheme's
% \scheme|or|, for example, uses any true value as an annihilator. We
% can define \scheme|or-bottom|, similar to \scheme|or|, that takes any
% number of expressions and returns a true value nondeterministically.
% Thus, \scheme|(or-bottom (odd? 0) bottom (! 5) (odd? 1) bottom)|
% returns either \scheme|#t| or \scheme|120|, both of which
% are true values in Scheme. \scheme|or-bottom| is defined as a macro
% that wraps its arguments in \scheme|fern| before passing them to a
% recursive auxiliary for two reasons: first, to make the auxiliary
% lazy, and second, to allow the auxiliary to choose convergent
% arguments before divergent ones.  This strategy will be used again in
% Section~\ref{mplus-bottom-and-bind-bottom} to define \scheme|mplus-bottom|.

%\schemedisplayspace
%\schemeinput{fernscode/code/or-fn}

\noindent
Let us define \scheme|append-bottom| for ferns. 

\schemedisplayspace
\schemeinput{fernscode/append}

\noindent
To observe the behavior of \scheme|append-bottom|, we define \scheme|take-bottom| whose
first argument is either \scheme|#f| (all results) or $n > 0$ (no more than $n$ results).

\schemedisplayspace
\schemeinput{fernscode/take}

\noindent 
When determining the $n$th value, it is necessary to avoid taking the
\scheme|fcdr| after the $n$th value is determined, since it is that \scheme|fcdr|
that might not terminate and we already have $n$ results.

The definition of \scheme|append-bottom| appears to work as expected:

\wspace
\noindent
\scheme|(take-bottom 2 (append-bottom (fern 1) (fern bottom 2)))| \schemeresult|=> (1 2)|.
\wspace

\noindent Moving \scheme|bottom| from the second argument to the
first, however, reveals a problem:

\wspace
\noindent
\scheme|(take-bottom 2 (append-bottom (fern bottom 1) (fern 2)))| \schemeresult|=> bottom|.
\wspace

\noindent
Even though the result of the call to
\scheme|append-bottom| should contain two convergent elements, taking the first
two elements of that result diverges. This is because the
definition of \scheme|append-bottom| requires that \scheme|s1| be completely
exhausted before any elements from \scheme|s2| can appear in the result.
If one of the elements of \scheme|s1| is \scheme|bottom|, then no
element from \scheme|s2| will ever appear.  The same is true if
\scheme|s1| contains an unbounded number of convergent elements: since
\scheme|s1| is never null, the result will never contain elements from
\scheme|s2|.  With the definition of \scheme|mplus-bottom| in
Section~\ref{mplus-bottom-and-bind-bottom}, it becomes clear that the solution to these problems is to
interleave the elements from \scheme|s1| and \scheme|s2| in the
resulting fern as in the next example.

Functional programs often share rather than copy data, and ferns are
designed to encourage this programming style. Consider a procedure to
compute the Cartesian product of two ferns:

\schemedisplayspace
\schemeinput{fernscode/cartesian}

\schemeinput{fernscode/cartesian-example}\schemeresult|~> `((a . x) (a . y) (b . x) (a . z) (b . y) (b . z))|
\belowcodeskip \medskipamount
\medskip

\noindent
where \schemeresult|~>| indicates \emph{one} of the possible values.
This definition ensures that the resulting fern shares
elements with the ferns passed as arguments. Many references to a
particular element may be made without repeating computations, hence the
expression

\wspace

\noindent
\scheme|(take-bottom 2 (Cartesian-product-bottom (fern (begin (display #t) 5)) (fern 'a bottom 'b)))|

\tspace

\noindent
\schemeresult|~> `((5 . a) (5 . b))|
\belowcodeskip \medskipamount
\medskip

\noindent prints \schemeresult|#t| \emph{exactly once}.
(There are more examples of the use of ferns in~\citet{Johnson-83}, \citet{Filman-Friedman-84}, and~\citet{Jeschke-PHD-95}.)

In the next section we look at how the sharing properties of ferns are
maintained alongside bottom-avoidance.

\section{Sharing and Promotion}\label{Sharing}

In this section, we provide examples and a high-level description of
the \emph{promotion algorithm} of Friedman and Wise~\cite{Friedman79b}. 
The values in a fern are computed and
\emph{promoted} across the fern while ensuring that the correct values
are available from each subfern, \scheme|bottom|'s are avoided, and
non-\scheme|bottom| values are computed only once.
% A proper fern is either the empty list or a pair whose cdr is another fern. 
Ferns have structure, and there may be references to more than one
subfern of a particular fern. Consider the example expression

\schemedisplayspace
\schemeinput{fernscode/sharing-1}

\begin{schemeresponse}~> `((6 120 720) (6 120 720) (6 720) (720))
\end{schemeresponse}

\noindent assuming \scheme|list| evaluates its arguments
left-to-right.  Importantly,
accessing \scheme|delta| cannot retrieve values in the prefix of the
enclosing fern \scheme|alpha|. We now describe in detail how the
result of \mbox{\scheme|(take-bottom 3 alpha)|} is determined along with
the necessary changes to the fern data structure during this
process. Whenever we encounter a choice, we shall assume a choice
consistent with the value returned in the example.

During the first access of \scheme|alpha| the cdrs are evaluated, as
indicated by the arrows in Figure~\ref{fig:solid}a.
Figure~\ref{fig:solid}b depicts the data structure after
\mbox{\scheme|(fcar alpha)|} is evaluated. We assume that, of the
possible values for \mbox{\scheme|(fcar alpha)|}, namely
\scheme|bottom| (which is never chosen), \mbox{\scheme|(! 5)|},
\mbox{\scheme|(! 3)|}, and \mbox{\scheme|(! 6)|}, the value of
\mbox{\scheme|(! 3)|} is chosen and promoted. Since the value
of \mbox{\scheme|(! 3)|} might be a value for \mbox{\scheme|(fcar beta)|}
 and \mbox{\scheme|(fcar gamma)|}, we replace the cars of all
three pairs with the value of \mbox{\scheme|(! 3)|}, which
is \schemeresult|6|. We replace the cdrs of \scheme|alpha| and
\scheme|beta| with new frons pairs containing \scheme|bottom| and
\mbox{\scheme|(! 5)|}, which were not chosen. The new frons pairs are
linked together, and linked at the end to the old cdr of
\scheme|gamma|. Thus \scheme|alpha|, \scheme|beta|, and \scheme|gamma|
become a fern with \schemeresult|6| in the car and a fern of the
rest of their original possible values in their cdrs.  As a result of
the promotion, $\alpha$, $\beta$, and $\gamma$ become cons pairs,
represented in the figures by rectangles.

Figure~\ref{fig:solid}c depicts the data structure after
\mbox{\scheme|(fcadr alpha)|} is evaluated. This time,
\mbox{\scheme|(! 5)|} is chosen from \scheme|bottom|, \mbox{\scheme|(! 5)|}, 
and \mbox{\scheme|(! 6)|}.  Since the value of \mbox{\scheme|(! 5)|} is
also a possible value for \mbox{\scheme|(fcadr beta)|}, we replace
the cadrs of both \scheme|alpha| and \scheme|beta| with the value of
\mbox{\scheme|(! 5)|}, which is \schemeresult|120|, and replace the cddr of
\scheme|alpha| with a frons pair containing the \scheme|bottom| that
was not chosen and a pointer to \scheme|delta|. The cddr of \scheme|beta| points to \scheme|delta|; no
new fern with remaining possible values is needed because the value
chosen for \mbox{\scheme|(fcadr beta)|} was the first value
available. As before, the pairs containing values become cons pairs.

% \renewcommand{\figurename}{Program}

% \begin{figure}[H]
% \begin{schemeregion}
% \schemeinput{fernscode/code/sharing-1}
% \end{schemeregion}
% \caption{Program that constructs and accesses fern $\alpha$.\label{fig:code/sharing-1}}
% \end{figure}

\renewcommand{\figurename}{Figure}
\addtocounter{figure}{0}

% \newcommand{\ahwd}{4}
% \newcommand{\ahht}{3}
% \newcommand{\dahwd}{5}
% \newcommand{\hahht}{2}
% \newcommand{\thahht}{1}
% \newcommand{\dahht}{6}
% \newcommand{\namegap}{8}
% \newcommand{\ftxht}{12}
% \newcommand{\ctxht}{10}
% \newcommand{\ebxwd}{3}
% \newcommand{\qbxwd}{10}
% \newcommand{\hbxwd}{20}
% \newcommand{\sbxwd}{6}
% \newcommand{\fbxwd}{40}

\newcommand{\ahwd}{4}
\newcommand{\ahht}{3}
\newcommand{\dahwd}{5}
\newcommand{\hahht}{2}
\newcommand{\thahht}{1}
\newcommand{\dahht}{6}
\newcommand{\namegap}{8}
\newcommand{\ftxht}{10}
\newcommand{\ctxht}{8}
\newcommand{\ebxwd}{2}
\newcommand{\qbxwd}{7}
\newcommand{\hbxwd}{14}
\newcommand{\sbxwd}{4}
\newcommand{\fbxwd}{28}


\Define\namebox(1) {
  \Move(0,\hbxwd) 
  \Move(0,\namegap) 
  \Text(--#1--) 
  \Move(0,-\namegap) 
  \Move(0,-\hbxwd)
}

\Define\namefig(3) { % name, half width, height
  \Move(#2,-#3)
  \Move(0,-\namegap)
  \Text(--$\mbox{(#1)}$--)
  \Move(0,\namegap)
  \Move(-#2,#3)
}

\newcommand{\boxtext}{Z}

\Define\fronsbox {
  \Line(0,\hbxwd) 
  \MarkLoc(p1)
  \Move(\qbxwd,\qbxwd)
  \MarkLoc(q1)
  \Move(\hbxwd,-\hbxwd)
  \MarkLoc(q2)
  \Move(\qbxwd,\qbxwd)
  \MarkLoc(p2) 
  \Curve(p1,q1,q2,p2) 
  \Line(0,-\hbxwd)
  \MarkLoc(p1)
  \Move(-\qbxwd,-\qbxwd)
  \MarkLoc(q1)
  \Move(-\hbxwd,\hbxwd)
  \MarkLoc(q2)
  \Move(-\qbxwd,-\qbxwd)
  \MarkLoc(p2)
  \Curve(p1,q1,q2,p2)
  \Move(\qbxwd,\ftxht)
  \Text(--\boxtext--)
  \Move(\qbxwd,-\ftxht)
  \Line(0,\hbxwd)
  \Move(\qbxwd,-\qbxwd)
}

\Define\arrowhead {
  \Move(-\ahwd,-\ahht) \Line(\ahwd,\ahht)
  \Move(-\ahwd,\ahht) \Line(\ahwd,-\ahht)
  \Move(1,0)
}

\Define\rightdotted {
  \Do(0,\qbxwd){\Line(1,0)\Move(2,0)}
  \arrowhead
  \Move(0,-\qbxwd)
}

\Define\rightsolid {
  \Do(0,\qbxwd){\Line(3,0)}
  \arrowhead
  \Move(0,-\qbxwd)
}

\Define\boxnil {
  \Move(-\qbxwd,-\qbxwd)
  \Line(\hbxwd,\hbxwd)
  \Move(0,-\qbxwd)
}

\Define\dottednil {
  \Move(-\qbxwd,-\qbxwd)
  \Do(0,\sbxwd){\Line(1,1)\Move(2,2)}
  \Move(0,-\qbxwd)
}

\Define\blankbox {
  \Move(\hbxwd,0)
  \Move(\qbxwd,0)
  \Do(0,\qbxwd){\Move(3,0)}
  \Move(1,0)
}

\Define\consbox {
  \Line(0,\hbxwd) 
  \Line(\fbxwd,0)
  \Line(0,-\hbxwd)
  \Line(-\fbxwd,0)
  \Move(\qbxwd,\ctxht)
  \Text(--\boxtext--)
  \Move(\qbxwd,-\ctxht)
  \Line(0,\hbxwd)
  \Move(\qbxwd,-\qbxwd)
}

\Define\downhead {
  \Move(-\ahht,\ahwd) \Line(\ahht,-\ahwd)
  \Move(\ahht,\ahwd) \Line(-\ahht,-\ahwd)
  \Move(0,-1)
}

\Define\downsolid {
  \Line(0,-\hbxwd)
  \Line(0,-\qbxwd)
  \downhead
  \Move(\qbxwd,0)
  \Move(-\fbxwd,-\hbxwd)
}

\Define\longdownsolid {
  \Line(0,-\hbxwd)
  \Line(0,-\qbxwd)
  \Line(0,-\ebxwd)
  \downhead
  \Move(\qbxwd,\ebxwd)
  \Move(-\fbxwd,-\hbxwd)
}

\Define\blankleft {
  \Move(-\hbxwd,0)
  \Move(-\qbxwd,0)
  \Do(0,\qbxwd){\Move(-3,0)}
  \Move(-1,0)
}

\Define\blankup {
  \Move(0,\fbxwd)
  \Move(0,1)
}

\Define\rightuphead {
  \Move(-\dahwd,-\hahht) \Line(\dahwd,\hahht)
  \Move(0,-\dahht) \Line(0,\dahht)
}

\Define\tiltedrightuphead {
  \Move(-\dahwd,-\thahht) \Line(\dahwd,\thahht)
  \Move(-1,-\dahht) \Line(1,\dahht)
}

\Define\rightupdoublesolid {
  \MarkLoc(p1)
  \Move(\fbxwd,0)
  \MarkLoc(q1)
  \Move(\fbxwd,0)
  \MarkLoc(q2)
  \blankup
  \Move(\hbxwd,-\qbxwd)
  \Move(-1,-1)
  \MarkLoc(p2)
  \Curve(p1,q1,q2,p2)
  \rightuphead
  \Move(1,1)
}

\Define\rightupdoublelonger {
  \MarkLoc(p1)
  \Move(\fbxwd,0)
  \MarkLoc(q1)
  \Move(\fbxwd,0)
  \MarkLoc(q2)
  \blankup
  \Move(\hbxwd,-\qbxwd)
  \Move(4,4)
  \MarkLoc(p2)
  \Curve(p1,q1,q2,p2)
  \rightuphead
  \Move(-4,-4)
}

\Define\blankdown {
  \Move(0,-\fbxwd)
}

\Define\rightuptriplesolid {
  \MarkLoc(p1)
  \blankbox
  \blankbox
  \MarkLoc(q1)
  \MarkLoc(q2)
  \Do(0,\qbxwd){\Move(3,0)}
  \Move(1,0)
  \blankup
  \Move(0,\hbxwd)
  \Move(0,\qbxwd)
  \Move(-3,0)
  \MarkLoc(p2)
  \Curve(p1,q1,q2,p2)
% \rightuphead
  \Move(3,0)
  \blankleft \blankleft \blankdown
}


\vspace{20pt}

\Define\edown {
% \Line(0,-\hbxwd)
% \Line(0,-\qbxwd)
  \Do(0,\qbxwd){\Line(0,-2)}
  \downhead
  \Move(0,-1)
}

\Define\unedown {
  \Move(0,1)
  \Move(0,1)
  \Do(0,\qbxwd){\Move(0,2)}
% \Move(0,\qbxwd)
% \Move(0,\hbxwd)
}

\newcommand\egap{3}

\Define\lengine {
  \edown
  \Move(\egap,0)
  \Line(-\hbxwd,0)
  \Line(0,-\hbxwd)
  \Line(\qbxwd,0)
  \Move(0,\qbxwd)
  \Text(--\boxtext--)
  \Move(0,-\qbxwd)
  \Line(\qbxwd,0)
  \Line(0,\hbxwd)
  \Move(-\egap,0)
  \unedown
}

\Define\rengine {
  \edown
  \Move(-\egap,0)
  \Line(\hbxwd,0)
  \Line(0,-\hbxwd)
  \Line(-\qbxwd,0)
  \Move(0,\qbxwd)
  \Text(--\boxtext--)
  \Move(0,-\qbxwd)
  \Line(-\qbxwd,0)
  \Line(0,\hbxwd)
  \Move(\egap,0)
  \unedown
}

\Define\nodownrengine {
  \Move(0,-\hbxwd)
  \Move(0,-\qbxwd)
  \Move(0,-2)
  \Move(-\egap,0)
  \Line(\hbxwd,0)
  \Line(0,-\hbxwd)
  \Line(-\qbxwd,0)
  \Move(0,\qbxwd)
  \Text(--\boxtext--)
  \Move(0,-\qbxwd)
  \Line(-\qbxwd,0)
  \Line(0,\hbxwd)
  \Move(\egap,0)
  \Move(0,2)
  \Move(0,\qbxwd)
  \Move(0,\hbxwd)
}

\Define\efronsbox {
  \Line(0,\hbxwd) 
  \MarkLoc(p1)
  \Move(\qbxwd,\qbxwd)
  \MarkLoc(q1)
  \Move(\hbxwd,-\hbxwd)
  \MarkLoc(q2)
  \Move(\qbxwd,\qbxwd)
  \MarkLoc(p2) 
  \Curve(p1,q1,q2,p2) 
  \Line(0,-\hbxwd)
  \MarkLoc(p1)
  \Move(-\qbxwd,-\qbxwd)
  \MarkLoc(q1)
  \Move(-\hbxwd,\hbxwd)
  \MarkLoc(q2)
  \Move(-\qbxwd,-\qbxwd)
  \MarkLoc(p2)
  \Curve(p1,q1,q2,p2)
  \Move(\qbxwd, \qbxwd)
  \lengine
  \Move(\qbxwd, -\qbxwd)
  \Line(0,\hbxwd)
  \Move(\qbxwd,-\qbxwd)
}

\Define\rightunright{
  \Do(0,\qbxwd){\Line(3,0)}
  \arrowhead
  \Move(-1,0)
  \Do(0,\qbxwd){\Move(-3,0)}
}

%%\enlargethispage{20pt}

\renewcommand\egap{7}

\begin{figure}[h]

  \wspace

  \wspace  

  \wspace
  
\begin{schemeregion}
\begin{picture}(160,60)(0,-60)
\Draw\PenSize(1pt)
\namefig(a,80,45)
\namebox(\scheme|alpha|)
\renewcommand{\boxtext}{\scheme|bottom|}
\efronsbox\rightsolid
\namebox(\scheme|beta|)
\renewcommand{\boxtext}{\scheme|! 5|}
\efronsbox\rightsolid
\namebox(\scheme|gamma|)
\renewcommand{\boxtext}{\scheme|! 3|}
\efronsbox\rightsolid
\namebox(\scheme|delta|)
\renewcommand{\boxtext}{\scheme|! 6|}
\efronsbox\boxnil
\EndDraw
\end{picture}
\end{schemeregion}

\begin{schemeregion}
\begin{picture}(160,0)(-180,-70)
\Draw\PenSize(1pt)
\namefig(b,80,45)
\namebox(\scheme|alpha|)
\renewcommand{\boxtext}{\schemeresult|six|}
\consbox\longdownsolid
\renewcommand{\boxtext}{\scheme|bottom|}
\efronsbox\rightsolid
\renewcommand{\boxtext}{\scheme|! 5|}
\efronsbox\rightupdoublesolid
\blankleft \blankleft \Move(-1,0)
\namebox(\scheme|beta|)
\renewcommand{\boxtext}{\schemeresult|six|}
\consbox\longdownsolid
\blankup \blankbox
\namebox(\scheme|gamma|)
\renewcommand{\boxtext}{\schemeresult|six|}
\consbox\rightsolid
\namebox(\scheme|delta|)
\renewcommand{\boxtext}{\scheme|! 6|}
\efronsbox\boxnil
\EndDraw
\end{picture}
\end{schemeregion}

\begin{schemeregion}
\begin{picture}(160,110)(0,-90)
\Draw\PenSize(1pt)
\namefig(c,80,95)
\namebox(\scheme|alpha|)
\renewcommand{\boxtext}{\schemeresult|six|}
\consbox\downsolid
\renewcommand{\boxtext}{\schemeresult|onetwenty|}
\consbox\longdownsolid
\renewcommand{\boxtext}{\scheme|bottom|}
\efronsbox\rightuptriplesolid
\renewcommand{\boxtext}{\schemeresult|onetwenty|}
\consbox\rightupdoublesolid
\blankleft \blankleft
\namebox(\scheme|beta|)
\renewcommand{\boxtext}{\schemeresult|six|}
\consbox\downsolid
\blankup \blankbox
\namebox(\scheme|gamma|)
\renewcommand{\boxtext}{\schemeresult|six|}
\consbox\rightsolid
\namebox(\scheme|delta|)
\renewcommand{\boxtext}{\scheme|! 6|}
\efronsbox\boxnil
\EndDraw
\end{picture}
\end{schemeregion}

\begin{schemeregion}
\begin{picture}(160,0)(-180,-100)
\Draw\PenSize(1pt)
\namefig(d,80,95)
\namebox(\scheme|alpha|)
\renewcommand{\boxtext}{\schemeresult|six|}
\consbox\downsolid
\renewcommand{\boxtext}{\schemeresult|onetwenty|}
\consbox\downsolid
%\renewcommand{\boxtext}{\schemeresult|720|}
\renewcommand{\boxtext}{\schemeresult|seventwenty|}
\consbox\longdownsolid
\renewcommand{\boxtext}{\scheme|bottom|}
\efronsbox\boxnil
\blankup
\Move(\hbxwd,\hbxwd)
\Move(\qbxwd,\qbxwd)
\Move(-3,1)
\renewcommand{\boxtext}{\schemeresult|onetwenty|}
\consbox\rightupdoublesolid
\blankleft \blankleft
\namebox(\scheme|beta|)
\renewcommand{\boxtext}{\schemeresult|six|}
\consbox\downsolid
\blankup \blankbox
\namebox(\scheme|gamma|)
\renewcommand{\boxtext}{\schemeresult|six|}
\consbox\rightsolid
\namebox(\scheme|delta|)
\renewcommand{\boxtext}{\schemeresult|seventwenty|}
\consbox\boxnil
\EndDraw
\end{picture}
\end{schemeregion}

\vspace{20pt}

\caption{Fern $\alpha$ immediately after evaluation of cdrs, but before any cars have finished evaluation (a) and after the values, $6$ (b), $120$ (c), and $720$ (d) have been promoted.\label{fig:solid}}
\end{figure}

\renewcommand\egap{3}

Figure~\ref{fig:solid}d depicts the data structure after
\mbox{\scheme|(fcaddr alpha)|} is evaluated. Of \scheme|bottom| and
\mbox{\scheme|(! 6)|}, it comes as no surprise that \mbox{\scheme|(!
  6)|} is chosen. Since the value of \mbox{\scheme|(! 6)|}, which is
\schemeresult|720|, is also a possible value for \mbox{\scheme|(fcar
  delta)|} (and in fact the only one), we update the car of
\scheme|delta| and the car of the cddr of \mbox{\scheme|alpha|} with
\schemeresult|720|.  The cdr of \scheme|delta| remains as the empty
list, and the cdr of the cddr of \mbox{\scheme|alpha|} becomes a new
frons pair containing \scheme|bottom|. The cdr of the new frons pair
is the empty list copied from the cdr of \scheme|delta|.  The
remaining values are obvious given the final state of the data
structure. No further manipulation of the data structure is necessary
to evaluate the three remaining calls to \scheme|take-bottom|.

In Figure~\ref{fig:solid}d  each of the ferns
\scheme|alpha|, \scheme|beta|, \scheme|gamma|, and \scheme|delta|
contains some permutation of its original possible values, and
\scheme|bottom| has been pushed to the end of
\scheme|alpha|. Furthermore, if there are no shared references to
\scheme|beta|, \scheme|gamma|, and \scheme|delta|, the number of
accessible pairs is linear in the length of the fern.  If there are
references to subferns, for a fern of size $n$, the worst case is
$(n^2+n)/2$.  But, as these shared references vanish, so do the
additional cons pairs.

If \scheme|list| evaluated from right-to-left instead of evaluating
from left-to-right, the example expression would return
\mbox{\scheme|`((720 6 120) (720 6 120) (720 6) (720))|}.  Each list
would be independent of the others and the last pair of
\scheme|alpha| would be a frons pair with \scheme|bottom| in the car
and the empty list in the cdr.  This demonstrates that if there is
sharing of these lists, the lists contain four pairs, three pairs, two
pairs, and one pair, respectively.  If the example expression just
returned \scheme|alpha|, then only four pairs would be accessible.

%\caption{Final state of fern $\alpha$ after promotion of $120$ (top) and $720$ (bottom). \label{fig:p3}}

%\begin{figure}[H]
%\end{figure}




%\begin{figure}[H]
%\end{figure}
%%%%%

%\vspace{10pt}

% In our implementation, it is the requesting of a value from a fern that
% provokes evaluation of the potential values. Therefore changing the
% order of requests affects how the data structure is manipulated.
% Program~\ref{fig:code/sharing-2} is a variation of
% Program~\ref{fig:code/sharing-1}, with the same initial data structure
% but with the \scheme|display| lines in reverse order; this code might,
% on a particular run, print:
% \begin{schemeresponse}
% (720)
% (720 6)
% (720 6 120)
% (720 6 120)
% \end{schemeresponse}

% Figures~\ref{fig:dotted}, \ref{fig:singles} and \ref{fig:bottom}
% depict some milestones in the evolution of the fern $\alpha$ during
% the evaluation of Program~\ref{fig:code/sharing-2}.
% Figure~\ref{fig:dotted} is still applicable because
% Programs~\ref{fig:code/sharing-1} and \ref{fig:code/sharing-2} differ
% only in the bodies of their innermost \scheme|let| expressions.
% The bottom part of Figure~\ref{fig:bottom} shows that if there are references to
% \scheme|beta|, \scheme|gamma|, and \scheme|delta|, then there are
% respectively three additional cons pairs if $\beta$ is referenced;
% two additional cons pairs if $\gamma$ is referenced; and one additional 
% cons pair if $\delta$ is referenced.  Thus, for a fern
% of size $n$, the worst case is $(n^2+n)/2$.  But, as these sharings
% vanish, the number of additional cons pairs approaches zero.

% is not applicable,
% since Program~\ref{fig:code/sharing-2} evaluates \mbox{\scheme|(fcar
%   delta)|} first, and the links between previous pairs remain
% dotted. Figure~\ref{fig:singles} depicts the changes to $\alpha$ as
% \mbox{\scheme|(take-bottom 1 delta)|}, \mbox{\scheme|(take-bottom 2 gamma)|}, and
% \mbox{\scheme|(take-bottom 3 beta)|} are evaluated: in each case the first
% frons pair of the respective subfern contains the only potential
% value, so that frons pair is simply turned into a cons pair.

% \addtocounter{figure}{-4}
% \renewcommand{\figurename}{Program}

% \begin{figure}[H]
% \begin{schemeregion}
% \schemeinput{fernscode/code/sharing-2}
% \end{schemeregion}
% \caption{Variant of Program~\ref{fig:code/sharing-1} with \textit{displays} reversed.\label{fig:code/sharing-2}}
% \end{figure}

% \renewcommand{\figurename}{Figure}
% \addtocounter{figure}{3}

% \begin{figure}[H]
% \begin{schemeregion}
% \begin{picture}(160,40)(0,-8)
% \Draw\PenSize(1pt)
% \namebox(\scheme|alpha|)
% \renewcommand{\boxtext}{\scheme|bottom|}
% \fronsbox\rightdotted
% \namebox(\scheme|beta|)
% \renewcommand{\boxtext}{\scheme|(!! 5)|}
% \fronsbox\rightdotted
% \namebox(\scheme|gamma|)
% \renewcommand{\boxtext}{\scheme|(!! 3)|}
% \fronsbox\rightdotted
% \namebox(\scheme|delta|)
% \renewcommand{\boxtext}{\schemeresult|720|}
% \consbox\boxnil
% \EndDraw
% \end{picture}
% \end{schemeregion}
% \caption{Fern $\alpha$ after promotion of the first value, 720. (See Program~\ref{fig:code/sharing-2}.)\label{fig:singles}}
% \end{figure}


% \begin{schemeregion}
% \begin{picture}(160,40)(0,-10)
% \Draw\PenSize(1pt)
% \namebox(\scheme|alpha|)
% \renewcommand{\boxtext}{\scheme|bottom|}
% \fronsbox\rightdotted
% \namebox(\scheme|beta|)
% \renewcommand{\boxtext}{\scheme|(!! 5)|}
% \fronsbox\rightdotted
% \namebox(\scheme|gamma|)
% \renewcommand{\boxtext}{\scheme|6|}
% \consbox\rightsolid
% \namebox(\scheme|delta|)
% \renewcommand{\boxtext}{\schemeresult|720|}
% \consbox\boxnil
% \EndDraw
% \end{picture}
% \end{schemeregion}

% \begin{schemeregion}
% \begin{picture}(160,40)(0,-10)
% \Draw\PenSize(1pt)
% \namebox(\scheme|alpha|)
% \renewcommand{\boxtext}{\scheme|bottom|}
% \fronsbox\rightdotted
% \namebox(\scheme|beta|)
% \renewcommand{\boxtext}{\scheme|120|}
% \consbox\rightsolid
% \namebox(\scheme|gamma|)
% \renewcommand{\boxtext}{\scheme|6|}
% \consbox\rightsolid
% \namebox(\scheme|delta|)
% \renewcommand{\boxtext}{\schemeresult|720|}
% \consbox\boxnil
% \EndDraw
% \end{picture}
% \end{schemeregion}

% Figure~\ref{fig:singles} shows that \mbox{\scheme|(take-bottom 1 delta)|},
% which is the same as \mbox{\scheme|(fcar delta)|}, evaluates only
% \mbox{\scheme|(! 7)|}, accounting for \scheme|720| in
% $\delta$. Once we have a cons pair in a fern, its car will generally
% dominate any frons pairs that refer to it directly or indirectly, so
% we end up in turn with \mbox{\scheme|(fcar gamma)|},
% \mbox{\scheme|(fcar beta)|}, and \mbox{\scheme|(fcar alpha)|} also
% being \scheme|720|.  The values in the other pairs dominate for
% similar reasons, but then it is for the \scheme|fcadr| and finally for
% the \scheme|fcaddr|. These lead to the before and after figures
% for the last promotion that corresponds to taking \mbox{\scheme|(fcaddr alpha)|}
% as shown in the top diagram of Figure~\ref{fig:bottom}.

 \Define\rightdoubleupsolid {
   \MarkLoc(p1)
   \blankbox
   \MarkLoc(q1)
   \MarkLoc(q2)
   \Do(0,\qbxwd){\Move(3,0)}
   \Move(1,-\qbxwd)
   \blankup
   \MarkLoc(p2)
   \Curve(p1,q1,q2,p2)
   \tiltedrightuphead
   \blankleft
 }

 \Define\righttripleupsolid {
   \MarkLoc(p1)
   \blankbox
   \blankbox
   \blankup
   \MarkLoc(q1)
   \MarkLoc(q2)
   \Do(0,\qbxwd){\Move(3,0)}
   \Move(1,-\qbxwd)
   \blankup
   \MarkLoc(p2)
   \Curve(p1,q1,q2,p2)
   \tiltedrightuphead
   \blankleft
   \blankleft
}

%\begin{figure}[H]
% \begin{schemeregion}
% \begin{picture}(160,80)(0,-50)
% \Draw\PenSize(1pt)
% \namebox(\scheme|alpha|)
% \renewcommand{\boxtext}{\schemeresult|120|}
% \consbox\longdownsolid
% \renewcommand{\boxtext}{\scheme|bottom|}
% \fronsbox\rightdoubleupsolid
% \namebox(\scheme|beta|)
% \renewcommand{\boxtext}{\scheme|120|}
% \consbox\rightsolid
% \namebox(\scheme|gamma|)
% \renewcommand{\boxtext}{\scheme|6|}
% \consbox\rightsolid
% \namebox(\scheme|delta|)
% \renewcommand{\boxtext}{\schemeresult|720|}
% \consbox\boxnil
% \EndDraw
% \end{picture}
% \end{schemeregion}

% \begin{schemeregion}
% \begin{picture}(160,120)(0,-90)
% \Draw\PenSize(1pt)
% \namebox(\scheme|alpha|)
% \renewcommand{\boxtext}{\schemeresult|120|}
% \consbox\downsolid
% \renewcommand{\boxtext}{\scheme|6|}
% \consbox\longdownsolid
% \renewcommand{\boxtext}{\scheme|bottom|}
% \fronsbox\righttripleupsolid
% \namebox(\scheme|beta|)
% \renewcommand{\boxtext}{\scheme|120|}
% \consbox\rightsolid
% \namebox(\scheme|gamma|)
% \renewcommand{\boxtext}{\scheme|6|}
% \consbox\rightsolid
% \namebox(\scheme|delta|)
% \renewcommand{\boxtext}{\schemeresult|720|}
% \consbox\boxnil
% \EndDraw
% \end{picture}
% \end{schemeregion}

%%%% penultimate promotion.
% \begin{figure}[H]
% \begin{schemeregion}
% \begin{picture}(160,160)(0,-130)
% \Draw\PenSize(1pt)
% \namebox(\scheme|alpha|)
% \renewcommand{\boxtext}{\schemeresult|720|}
% \consbox\downsolid
% \renewcommand{\boxtext}{\scheme|6|}
% %\consbox\downsolid
% %\renewcommand{\boxtext}{\schemeresult|120|}
% \consbox\longdownsolid
% \renewcommand{\boxtext}{\scheme|bottom|}
% %\fronsbox\boxnil\blankup\blankup
% \fronsbox\rightsolid\blankup\blankup
% %\Move(\hbxwd,-\qbxwd)
% \Move(3,0)
% \namebox(\scheme|beta|)
% \renewcommand{\boxtext}{\scheme|720|}
% %%%
% \consbox\downsolid
% \renewcommand{\boxtext}{\scheme|6|}
% \consbox\downsolid
% \renewcommand{\boxtext}{\schemeresult|120|}
% \consbox\boxnil\blankup\blankup
% \Move(\hbxwd,-\qbxwd)\Move(3,0)
% %\consbox\downsolid
% %%%
% \namebox(\scheme|gamma|)
% \renewcommand{\boxtext}{\scheme|720|}
% \consbox\downsolid
% \renewcommand{\boxtext}{\scheme|6|}
% \consbox\boxnil\blankup
% \Move(\hbxwd,-\qbxwd)\Move(3,0)
% \namebox(\scheme|delta|)
% \renewcommand{\boxtext}{\schemeresult|720|}
% \consbox\boxnil
% \EndDraw
% \end{picture}
% \end{schemeregion}

% \begin{schemeregion}
% \begin{picture}(160,160)(0,-130)
% \Draw\PenSize(1pt)
% \namebox(\scheme|alpha|)
% \renewcommand{\boxtext}{\schemeresult|720|}
% \consbox\downsolid
% \renewcommand{\boxtext}{\scheme|6|}
% \consbox\downsolid
% \renewcommand{\boxtext}{\schemeresult|120|}
% \consbox\longdownsolid
% \renewcommand{\boxtext}{\scheme|bottom|}
% \fronsbox\boxnil\blankup\blankup\blankup
% \Move(\hbxwd,-\qbxwd)\Move(3,0)
% \namebox(\scheme|beta|)
% \renewcommand{\boxtext}{\scheme|720|}
%%%
% \consbox\downsolid
% \renewcommand{\boxtext}{\scheme|6|}
% \consbox\downsolid
% \renewcommand{\boxtext}{\schemeresult|120|}
% \consbox\boxnil\blankup\blankup
% \Move(\hbxwd,-\qbxwd)\Move(3,0)
% %\consbox\downsolid
% %%%
% \namebox(\scheme|gamma|)
% \renewcommand{\boxtext}{\scheme|720|}
% \consbox\downsolid
% \renewcommand{\boxtext}{\scheme|6|}
% \consbox\boxnil\blankup
% \Move(\hbxwd,-\qbxwd)\Move(3,0)
% \namebox(\scheme|delta|)
% \renewcommand{\boxtext}{\schemeresult|720|}
% \consbox\boxnil
% \EndDraw
% \end{picture}
% \end{schemeregion}
% \caption{Final promotion within fern $\alpha$ from Program~\ref{fig:code/sharing-2}.\label{fig:bottom}}
% \end{figure}

The example presented in this section provides a direct view of
promotion. When a fern is accessed by multiple computations, the
promotion algorithm must be able to handle various issues such as
multiple values becoming available for promotion at once. The code
presented in Chapter~\ref{fernsimpl} handles these details.

We are now ready to consider a ferns-based implementation of
miniKanren.

\section{Ferns-based miniKanren}\label{lp-system}

In this section we describe a simple bottom-avoiding logic programming
language, which corresponds to core miniKanren with non-interleaving
search.
% In this section, we use the task of logic programming as an extended
% example of the use of ferns in avoiding bottom while maintaining a
% natural recursive style. 
%We compare two sets of goal constructors, one
%using streams and the other using ferns.  
We begin by describing and
implementing operators \scheme|mplus-bottom| and \scheme|bind-bottom| over ferns,
and go on to implement goal constructors in terms of these
operators.  The fern-based goal constructors are shown to be more
general than the standard stream-based ones presented in Chapter~\ref{mkimplchapter}\footnote{See~\citet{Wand04relatingmodels} for a historical account of logic combinators.}.

\subsection{\protect\scheme|mplus-bottom| and \protect\scheme|bind-bottom|}\label{mplus-bottom-and-bind-bottom}

Since we are developing goal constructors in Scheme, a call-by-value language,
we make \scheme|mplus-bottom| itself lazy to avoid diverging when one or
more of its arguments diverge. This is accomplished by defining \scheme|mplus-bottom| as a
macro that wraps its two arguments in \scheme|fern| before passing
them to \scheme|mplus-fn-bottom|. In addition, \scheme|mplus-bottom| must interleave
elements from both of its arguments so that a fern of unbounded length
in the first argument will not cause the second argument to be
ignored.

%% TODO: bind should be lazy!
%% Bind should not need to be lazy, since its arguments are streams
%% and functions and we know all the places it is called.

%% TODO: examples of why we need frons?

% The \scheme|mplus-func-bottom| procedure does exactly that, returning a fern
% that is guaranteed to contain all the convergent elements of both
% \scheme|s1| and \scheme|s2|, even if both contain divergent elements
% or are infinite in length.

% \noindent Because \scheme|mplus-func-bottom| immediately returns a fern,
% \mbox{\scheme|(mplus-func-bottom s1 s2) converges|} for all ferns \scheme|s1| and
% \scheme|s2|. In addition, since the results are interleaved, divergent
% values are pushed to the end of the resulting fern. Thus,

% \belowcodeskip 0pt
% \schemeinput{fernscode/code/mplus-example-1}\schemeresult|=> `(5 6)|
% \belowcodeskip \medskipamount

% \medskip

% \noindent \scheme|mplus-func-bottom| even works on infinite ferns, interleaving
% them:

% \belowcodeskip 0pt
% \schemeinput{fernscode/code/mplus-example-2}\schemeresult|=> `(1 200 201 2)|
% \belowcodeskip \medskipamount

% \medskip

% Unfortunately, this definition of \scheme|mplus-func-bottom| can handle only one
% type of divergence: while \mbox{\scheme|(mplus-func-bottom s1 s2) converges|} when
% \mbox{\scheme|s1| \scheme|converges|} and \mbox{\scheme|s2| \scheme|converges|}, 
% \mbox{\scheme|(mplus-func-bottom s1 bottom) => bottom|}. 
% Since \scheme|s1| may have non-\scheme|bottom|
% values, we would prefer that \scheme|mplus-func-bottom| avoid this type of
% divergence by waiting to evaluate its arguments until necessary. We can
% accomplish this by writing \scheme|mplus-func-bottom| as a macro that creates a
% two-value fern of its arguments before passing that fern to the
% \scheme|mplus-fn-bottom| procedure.  $

\schemedisplayspace
\schemeinput{fernscode/mplus-fn}

\schemeinput{fernscode/bind}

% We use a fern constructor to make \scheme|mplus-bottom| lazy: if one of the ferns in the 
% argument to \scheme|mplus-fn-bottom| is divergent, it can select the other
% one. For example, consider \mbox{\scheme|(fcar (mplus-bottom bottom (fern 5)))|},
% which evaluates to \schemeresult|5|. 
\noindent
\scheme|bind-bottom| avoids the same types of
divergence as \scheme|map-bottom| described in Section~\ref{fernsRecursiveExamples} but
uses \scheme|mplus-bottom| to merge the results of the calls to \scheme|f|.
Thus, \mbox{\scheme|(bind-bottom (ints-bottom 0) ints-bottom)|} is an unbounded fern of
integers; for every (nonnegative) integer $n$, it contains the
integers starting from $n$ and therefore every nonnegative integer $n$
is contained $n+1$ times.  The interleaving leads to
duplicates in the following example:

\wspace
\noindent\scheme|(take-bottom 13 (bind-bottom (ints-bottom 0) ints-bottom))| \schemeresult|~> `(0 1 2 1 3 4 5 6 7 8 9 2 10)|.
\wspace

\noindent
The addition of \scheme|unit-bottom| and \scheme|mzero-bottom| rounds out the set of
operators we need to implement a minimal miniKanren-like language.

\schemedisplayspace
\schemeinput{fernscode/mzerounit}

\noindent Using these definitions, we can run programs that require
multiple unbounded ferns, such as this program inspired by Seres and
Spivey~\cite{CombinatorsforLP} that searches for a pair $a$ and $b$ of
divisors of $9$ by enumerating the integers from $2$ in a fern of
possible values for $a$ and similarly for $b$:

\schemedisplayspace
\belowcodeskip 0pt
\schemeinput{fernscode/spivey}\schemeresult|=> `(3 3)|.
\belowcodeskip \medskipamount
\medskip

\noindent Using streams instead of ferns in this example, which would be
like nesting ``for'' loops, would result in divergence since $2$ does
not evenly divide $9$.

\subsection{Goal Constructors}

%\enlargethispage{30pt}
We are now ready to define three goal constructors:
\scheme|==-bottom|, which unifies terms; \scheme|disj-bottom|, which
performs disjunction over goals; and \scheme|conj-bottom|, which
performs conjunction over goals\footnote{\scheme|disj-bottom| is just a simplified version of \scheme|conde|, while \scheme|conj-bottom| is just a simplified version of \scheme|exist|.}. These goal constructors are required
to terminate, and they always return a goal.  A \emph{goal} is a
procedure that takes a substitution and returns a fern of
substitutions (rather than a stream of substitutions, as in
Chapter~\ref{mkimplchapter}).

\schemedisplayspace
\schemeinput{fernscode/mk-without-run}

A logic program evaluates to a goal; to obtain answers, this goal is
applied to the empty substitution. The result is a fern of
substitutions representing answers.  We define \scheme|run-bottom| in terms
of \scheme|take-bottom|, described in Section \ref{fernsRecursiveExamples}, to
obtain a list of answers from the fern of substitutions

\schemedisplayspace
\schemeinput{fernscode/mk-run}
\noindent where \scheme|n| is a non-negative integer (or \scheme|#f|) and 
\scheme|g| is a goal.  
% (See Section~\ref{lp-helpers} for the rest of the definitions used in this section.)

Given two logic variables \scheme|x| and \scheme|y|, here are
some simple logic programs that produce the same answers using both
fern-based and stream-based goal constructors.

\medskip

\indent \scheme|(run-bottom #f (==-bottom 1 x))| \schemeresult|=> `($\{x/1\}$)| \\
\indent \scheme|(run-bottom 1 (conj-bottom (==-bottom y 3) (==-bottom x y)))| %$
\schemeresult|=> `($\{x/3, y/3\}$)| \\
\indent \scheme|(run-bottom 1 (disj-bottom (==-bottom x y) (==-bottom y 3)))| %$
\schemeresult|=> `($\{x/y\}$)| \\
 \indent \scheme|(run-bottom 5 (disj-bottom (==-bottom x y) (==-bottom y 3)))| %$
\schemeresult|=> `($\{x/y\}$ $\{y/3\}$)| \\
 \indent \scheme|(run-bottom 1 (conj-bottom (==-bottom x 5) (conj-bottom (==-bottom x y) (==-bottom y 4))))|
\schemeresult|=> `()| \\
\indent \scheme|(run-bottom #f (conj-bottom (==-bottom x 5) (disj-bottom (==-bottom x 5) (==-bottom x 6))))|
\schemeresult|=> `($\{x/5\}$)|

\medskip

\noindent It is not difficult, however, to find examples of logic
programs that diverge when using stream-based goal constructors but
converge using fern-based constructors:

\medskip

\scheme|(run-bottom 1 (disj-bottom bottom (==-bottom x 3)))| \schemeresult|=> `($\{x/3\}$)| \\
\indent \scheme|(run-bottom 1 (disj-bottom (==-bottom bottom x) (==-bottom x 5)))| \schemeresult|=> `($\{x/5\}$)|

\medskip

\noindent and given idempotent substitutions \cite{lloyd:lp}, the
fern-based operators can even avoid some circularity-based divergence
without the occurs-check, while stream-based operators cannot:

\wspace

\scheme|(run-bottom 1 (disj-bottom (==-bottom (list x) x) (==-bottom x 6)))| \schemeresult|=> `($\{x/6\}$)|

\wspace

There are functions that represent relations. The relation
\scheme|always-five-bottom| associates 5 with its argument an unbounded number of times:

\schemedisplayspace
\schemeinput{fernscode/always-five}

\noindent Because both stream and fern constructors
do not evaluate their arguments,
we may safely evaluate the goal \mbox{\scheme|(always-five-bottom x)|}, %$
obtaining an unbounded collection of answers.  Using
\scheme|run-bottom|, we can ask for a finite number of these answers. Because
the ordering of streams is determined at construction time, however, the
stream-based operators cannot even determine the first answer in that
collection. This is because the definition of \scheme|always-five-bottom| is left recursive.  The fern-based operators, however, compute as many answers 
as desired:

\medskip

\mbox{\scheme|(run-bottom 4 (always-five-bottom x))| \schemeresult|=> `($\{x/5\}$ $\{x/5\}$ $\{x/5\}$ $\{x/5\}$)|}.

\medskip


% \scheme|(take-bottom #f ((==-bottom 1 x) empty-s)) =>| \schemeresult|`($\{x/1\}$)|

% \scheme|(take-bottom #f ((conj-bottom (==-bottom 1 x) (==-bottom 2 x)) empty-s) =>|
% \schemeresult|`()|

% \scheme|(take-bottom #f ((disj-bottom (==-bottom 2 x) (==-bottom 3 x)) empty-s)) =>| \schemeresult|`($\{x/2\}$ $\{x/3\}$)|

% \noindent With \scheme|cons| instead of \scheme|frons|, the following examples diverge:

% \scheme|(take-bottom 1 ((disj-bottom bottom (==-bottom 4 x)) empty-s)) =>| \schemeresult|`($\{x/4\}$)|

% \scheme|(take-bottom 1 ((disj-bottom (==-bottom bottom x) (==-bottom 5 x)) empty-s)) =>| \schemeresult|`($\{x/5\}$)|

% \noindent Without the occurs check, the following example also diverges
% with \scheme|cons| instead of \scheme|frons|:

% \scheme|(take-bottom 1 ((disj-bottom (==-bottom `(,x) x) (==-bottom x 6)) empty-s)) =>|
% \schemeresult|`($\{x/6\}$)|

% \section{Helper Functions}\label{lp-helpers}

% To complete the implementation of the bottom-avoiding logic
% programming operators presented in Section~\ref{lp-system}, we provide
% a logic variable constructor \scheme|make-var|, a unification
% algorithm \scheme|unify|, and substitution helpers \scheme|empty-s|,
% \scheme|ext-s|, and \scheme|walk|. We represent logic variables by
% R$^6$RS~\cite{r6rs} records (syntactic layer); defining the record
% type \scheme|var| creates the constructor \scheme|make-var|
% automatically. We represent substitutions as association lists, and
% use the triangular substitution model~\cite{FBaade01}.

% \schemedisplayspace
% \schemeinput{fernscode/subst}


\chapter{Implementation VI: Ferns}\label{fernsimpl}

In this chapter we present a complete, portable, \RsixRSsp
compliant~\cite{r6rs} implementation of ferns\footnote{The ferns
  library is available at
  \url{http://www.cs.indiana.edu/~webyrd/ferns.html}}. We begin with a
description of \emph{engines}~\cite{CHayne87}, which we use to handle
suspended, preemptible computations. We then describe and implement
frons pairs, the building blocks of ferns. Next we present
\scheme|fcar| and \scheme|fcdr|, which work on both frons pairs and
cons pairs. Taking the \scheme|fcar| of a frons pair involves choosing
one of the possible values in the fern and promoting the chosen value.
Taking the \scheme|fcdr| of a frons pair ensures the first value in
the pair is determined and returns the rest of the fern.  Taking the
\scheme|fcar| (\scheme|fcdr|) of a cons pair is the same as taking its
\scheme|car| (\scheme|cdr|).

\section{Engines}\label{fernsenginessection}

An engine is a procedure that computes a delayed value in steps\footnote{See Appendix~\ref{nestable-engines} for our implementation of nestable engines.}.
%\cite{CHayne87}. 
To demonstrate the use of engines, consider the procedure

\schemedisplayspace
\schemeinput{fernscode/wait}

\noindent To create an engine \scheme|e| to delay a call to
\scheme|(wait 20)|, we write

\wspace
\scheme|(define e (engine (wait 20)))|
\wspace

\noindent 
To partially compute \mbox{\scheme|(wait 20)|}, we call \scheme|e|
with a number of \scheme|ticks|: \mbox{\scheme|(e 5)|}, which returns
either a pair with false in the car and a new advanced engine (one
advanced 5 ticks) in the cdr or a pair with unused ticks (always true)
in the car and the value of the computation (here \schemeresult|done|)
in the cdr.  In our embedding of engines, a tick is spent on each
call to a procedure defined with \scheme|timed-lambda|. Consider

\schemedisplayspace

\belowcodeskip 0pt

\schemeinput{fernscode/coaxexample-2}
\nspace
\begin{schemeresponse}
$\Rightarrow$ `(#f #f #f #f 4 done).
\end{schemeresponse}

\wspace

\noindent In this example, \mbox{\scheme|(wait 20)|} calls
\scheme|wait| a total of $21$ times (including the initial call), so
on the fifth engine invocation, it terminates with 4 unused ticks.

The delayed computation in an engine may involve creating and calling
more engines. When a \emph{nested engine}
\cite{hieb94subcontinuations} consumes a tick, every
frons-enclosing engine also consumes a tick. To see this, we define
\scheme|choose-bottom| using engines:

\schemedisplayspace
\schemeinput{fernscode/coaxchoose}

\wspace

\noindent Nested calls to \scheme|choose-bottom|, for example 
\mbox{\scheme|(choose-bottom v1 (choose-bottom v2 v3))|}, rely on nestable engines.  
This implementation of \scheme|choose-bottom| is fair because our embedding
of nested engines is fair: every tick given to the second engine in
the outer call to \scheme|choose-aux-bottom| is passed on to exactly one of
the engines, alternating between the engines for \scheme|v2| and
\scheme|v3|, in the inner call to \scheme|choose-aux-bottom|.

\section{The Ferns Data Type}\label{fernsdatatype}

We represent a frons pair by a cons pair that contains at least one
tagged engine (\scheme|te|). Engines are tagged with either \scheme|L|
when locked (being advanced by another computation) or \scheme|U| when
unlocked (runnable).  We distinguish between locked and unlocked
engines because the \scheme|fcar| of a fern may be requested more than
once simultaneously. Thus, to manage effects, the locks prevent the same
engine from being advanced in more than one computation\footnote{A lock creates a localized critical region
that corresponds to the intended use of \scheme|sting-unless|~\cite{FriedmanWise78}.}.

We define simple predicates \scheme|engine-tag-L-car|,
\scheme|engine-tag-U-car|, \scheme|engine-tag-L-cdr|, and
\scheme|engine-tag-U-cdr| for testing whether one side of a frons pair
contains a locked or unlocked engine. 

\schemedisplayspace
\schemeinput{fernscode/lockedorunhuh}

\wspace

The procedure \scheme|coax-d| (\scheme|coax-a|) takes a frons pair
with an unlocked tagged engine in the cdr (car) and locks and advances
the tagged engine by \scheme|nsteps| ticks. If
\scheme|coax|ing~\cite{Friedman79b} the engine does not finish, the
tagged engine is unlocked and updated with the advanced engine. If
\scheme|coax|ing the engine finishes with value \scheme|v|, then
\scheme|v| becomes the frons pair's cdr (car). In addition, the tagged
engine will be updated with an unlocked dummy engine that returns
\scheme|v|. We do this because the cdrs of multiple frons pairs may
share a single engine, as will be explained at the end of this
section.  Although the cars of frons pairs never share engines, we do
the same for the cars.

\schemedisplayspace
\schemeinput{fernscode/coaxskcdr}
\schemeinput{fernscode/replacebang}

\wspace

\noindent
Now we present the implementation of the fern operators.

%\subsection{\protect\scheme|frons|}\label{frons}

\section{\fronssymbol, \fcarsymbol, and \fcdrsymbol}\label{frons}\label{car}\label{cdr}\label{fronsconstructorsection}

\scheme|frons| constructs a frons pair by placing unlocked engines of
its unevaluated operands in a cons pair.

\schemedisplayspace
\schemeinput{fernscode/frons}

%\section{\protect\scheme|fcar|}\label{car}

\wspace

\noindent
When the \scheme|fcar| (definition below), which is defined only for
ferns, is requested, parallel evaluation of the possible values is
accomplished by a round-robin race of the engines in the
fern.  During its turn, each engine is advanced a fixed,
arbitrary number of ticks until a value is produced.  The race is
accomplished by two mutually recursive functions: \scheme|race-car|,
which works on the possible values of the fern, and \scheme|race-cdr|,
which moves onto the next frons pair by either following the cdr of
the current frons pair or starting again at the beginning.

\schemedisplayspace
\schemeinput{fernscode/car}

\wspace

\noindent
\scheme|race-car| dispatches on the current pair or value \scheme|q|.
When the car of \scheme|q| is a locked engine, \scheme|race-car| waits
for it to become unlocked by waiting \scheme|nsteps| ticks and then
calling \scheme|race-cdr|. The call to \scheme|wait| is required to
allow \scheme|race-car| to be preempted at this point, so the owner of
the lock does not starve. When the car is an unlocked engine,
\scheme|race-car| advances the unlocked engine \scheme|nsteps| ticks,
then continues the race by calling \scheme|race-cdr|.  When \scheme|q|
is not a pair, \scheme|race-car| simply starts the race again from the
beginning.  This happens when racing over a finite fern and emerges
from the \scheme|else| clause of \scheme|race-cdr|. When the car
contains a value, we call \scheme|promote| which ensures a value is promoted to
the car of \scheme|p|, then return that value.

One subtlety of the definition of \scheme|race-car| is that after
coaxing an engine it does not check if the coaxing has led to
completion.  If it has, the value will be picked up the next time the
race comes around, if necessary.  Calling \scheme|promote!|
immediately would be incorrect because an engine may be preempted
while advancing, at which point promotion from \scheme|p| may be
performed by another computation with a different value for the car of
\scheme|p|.  

%\enlargethispage{40pt}

\scheme|race-cdr| also dispatches on \scheme|q|, this time examining its
cdr. When the cdr of \scheme|q| is a locked engine,
\scheme|race-cdr|, being unable to proceed further down the fern,
restarts the race by calling \scheme|race-car| on \scheme|p|. When the
cdr of \scheme|q| contains an unlocked engine, \scheme|race-cdr|
advances the engine \scheme|nsteps| ticks as in \scheme|race-car|, and
then restarts the race. If that engine finishes with a new frons pair,
the new pair will then be competing in the race and will be examined
next time around. When the cdr of \scheme|q| is a value, usually a fern,
\scheme|race-cdr| continues the race by passing it to
\scheme|race-car|; if a non-pair value is at the end of a fern, it will
be picked up by the third clause in \scheme|race-car|.

\scheme|fcar| avoids starvation by running each engine in a subfern for the
same number of ticks. During a race, a subfern of the fern in question
is in a fair state: for some (potentially empty) prefix of the subfern there are no engines in the cdrs, so
each potential value in a fair subfern is considered equally. When this fair
subfern is not the entire fern, the race devotes the same number of
ticks to lengthening the fair subfern as it does to each element of that
subfern. Since cdr engines often evaluate to pairs quickly, the
entire fern usually becomes fair in a number of races equal to the
length of the fern. When cdr engines do not finish quickly, however, the
process of making the entire fern fair can take much longer, especially
for long ferns. The cost of finding the value of an element occurring near
the end of such a fern can be much greater than the cost for an element
near the beginning.

%\enlargethispage{30pt}

Starting from \scheme|p|, \scheme|promote!| (definition below) finds
the first pair \scheme|r| whose car contains a convergent value, and
propagates that value back to \scheme|p|.  Each frons pair in this
chain (excluding \scheme|r|) is transformed into a cons pair whose car
is the convergent value. These new frons pairs are connected as a fern
and the last one shares \scheme|r|'s cdr.  When \scheme|promote!| is
called from \scheme|race-car|, we know that \scheme|q|'s car is a
value but we don't know for certain that there is no other pair, say
\scheme|r|, in the chain from \scheme|p| to \scheme|q|.  Thus, we must
search from \scheme|p| without preemption to find the closest value to
\scheme|p|.  This situation can arise when there are two calls to
\scheme|fcar| on the same fern competing:

\schemedisplayspace
\schemeinput{fernscode/whypromotion.ss}

\wspace

\noindent
If a call to \scheme|race-car| finds the value \schemeresult|720| and
tries to promote it, but the value \schemeresult|120| has already been
promoted, we don't want to change the car of \scheme|alpha|. Instead,
the call to \scheme|promote| when \schemeresult|720| is found will
find the \schemeresult|120| first and stop.

% Each finds a different value; since each call is nested, the two calls are
% effectively running in parallel. Then the one with the value closer to
% \scheme|p| will promote first even if \scheme|promote!| is non-preemptible.

\schemedisplayspace
\schemeinput{fernscode/promote}

%\section{\protect\scheme|fcdr|}\label{cdr}

\wspace

The \scheme|fcdr| of a fern (definition below) cannot be determined until the
fern's \scheme|fcar| has been determined.  Once the car has been determined,
there is no longer parallel competition between potential cdrs.  Thus,
we can use \scheme|cdrdollar|, which takes the cdr of a stream. Then,
since \scheme|p|'s car has been determined, \scheme|p| has therefore
become a cons pair, so \scheme|fcdr| returns the value in \scheme|p|'s
cdr.  (\scheme|cardollar|'s definition follows by replacing all
\scheme|d|s by \scheme|a|s.  \scheme|consdollar| is the same as
\scheme|frons|, and the definitions of the other stream operators
follow the definitions with operators $f_{\perp}$ replaced by $f_s$.)

\schemedisplayspace
\schemeinput{fernscode/cdr}
\schemeinput{fernscode/cdrdollar}

\wspace

If the engine being advanced by \scheme|cdrdollar| completes,
\scheme|cdrdollar| indicates that \scheme|coax-d| should replace the
tagged engine in \scheme|p| by the computed value.  However,
\scheme|race-cdr| and \scheme|fcdr| are required not only to update
the frons pair with the calculated value, but also to update
the tagged engine because there might be a fern other than \scheme|p|
sharing this engine. Consider the following expression where we assume
\scheme|list| evaluates its arguments from left to right.

\schemedisplayspace
\begin{schemedisplay}
(let ((beta (frons 1 (ints-bottom 2))))
  (let ((alpha (frons bottom beta)))
    (list (fcar alpha) (fcadr beta) (fcadr alpha))))
\end{schemedisplay}
\nspace
\begin{schemeresponse}
~> `(1 2 2)
\end{schemeresponse}

\wspace

\noindent Figure~\ref{fig:sk} shows the data structures involved in
evaluating the expression. 

\schemedisplayspace

\vspace{-5pt}

\begin{figure}[h]
\begin{schemeregion}
\begin{picture}(40,50)(-40,-30)
\Draw\PenSize(1pt)
\namefig(a,15,35)
\namebox(\scheme|alpha|)
\renewcommand{\boxtext}{\scheme|bottom|}
\efronsbox
\renewcommand{\boxtext}{\scheme|beta|}
\rengine
\EndDraw
\end{picture}
\end{schemeregion}

\begin{schemeregion}
\begin{picture}(80,0)(-120,-40)   %%% 
\Draw\PenSize(1pt)
\namefig(b,35,35)
\namebox(\scheme|alpha|)
\renewcommand{\egap}{\qbxwd}
\renewcommand{\boxtext}{\scheme|bottom|}
\efronsbox
\rightsolid
\namebox(\scheme|beta|)
\renewcommand{\boxtext}{\schemeresult|one|}
\fronsbox
\renewcommand{\boxtext}{$\iota_2$}
\rengine
\EndDraw
\end{picture}
\end{schemeregion}

\begin{schemeregion}
\begin{picture}(80,0)(-240,-50)
\Draw\PenSize(1pt)
\namefig(c,35,35)
\namebox(\scheme|alpha|)
\renewcommand{\boxtext}{\schemeresult|one|}
\consbox\downsolid
\renewcommand{\boxtext}{\scheme|bottom|}
\renewcommand{\egap}{6}
\efronsbox\rightunright\blankup\Move(\hbxwd,-\qbxwd)
\namebox(\scheme|beta|)
\renewcommand{\boxtext}{\schemeresult|one|}
\fronsbox
\renewcommand{\boxtext}{$\iota_2$}
\renewcommand{\egap}{8}
\rengine
\EndDraw
\end{picture}
\end{schemeregion}

\begin{schemeregion}
\begin{picture}(80,80)(-40,-50)
\Draw\PenSize(1pt)
\namefig(d,40,60)
\namebox(\scheme|alpha|)
\renewcommand{\boxtext}{\schemeresult|one|}
\consbox\downsolid
\renewcommand{\boxtext}{\scheme|bottom|}
\renewcommand{\egap}{3}
\efronsbox
\renewcommand{\boxtext}{\scheme|gamma|}
\rengine
\blankup\Move(\hbxwd,-\qbxwd)
\namebox(\scheme|beta|)
\renewcommand{\boxtext}{\schemeresult|one|}
\consbox
\rightsolid
\namebox(\scheme|gamma|)
\renewcommand{\boxtext}{\schemeresult|two|}
\fronsbox
\renewcommand{\egap}{6}
\renewcommand{\boxtext}{$\iota_3$}
\rengine
\EndDraw
\end{picture}
\end{schemeregion}

\begin{schemeregion}
\begin{picture}(80,0)(-180,-60)
\Draw\PenSize(1pt)
\namefig(e,40,60)
\namebox(\scheme|alpha|)
\renewcommand{\boxtext}{\schemeresult|one|}
\consbox\downsolid
\renewcommand{\boxtext}{\schemeresult|two|}
\consbox\downsolid
\renewcommand{\boxtext}{\scheme|bottom|}
\renewcommand{\egap}{6}
\efronsbox \rightupdoublelonger
\blankup \blankleft
\Move(-\qbxwd,0)
\namebox(\scheme|beta|)
\renewcommand{\boxtext}{\schemeresult|one|}
\consbox \rightsolid
\namebox(\scheme|gamma|)
\renewcommand{\boxtext}{\schemeresult|two|}
\fronsbox
\renewcommand{\boxtext}{$\iota_3$}
\lengine
\EndDraw
\end{picture}
\end{schemeregion}

\vspace{25pt}
\caption{Fern $\alpha$ after construction (a); after $\beta$ in the cdr of $\alpha$ has been evaluated (b); after $1$ from the car of $\beta$ has been promoted to the car of $\alpha$, resulting in a shared tagged engine (c); after the shared engine is run, while evaluating ({\it{cadr$_\perp$} $\beta$}), to produce a fern $\gamma$ (d); after $2$ from the car of $\gamma$ has been promoted to the cadr of \mbox{$\alpha$ (e)}.\label{fig:sk}}
\end{figure}

\noindent Figure~\ref{fig:sk}a shows \scheme|alpha| immediately after
it has been constructed, with engines delaying evaluation of
\scheme|bottom| and \scheme|beta|. In evaluating \mbox{\scheme|(fcar alpha)|}, 
the engine for \scheme|beta| finishes, resulting in
Figure~\ref{fig:sk}b. \scheme|beta| can now participate in the race
for \scheme|(fcar alpha)|. Suppose the value \scheme|1| found in the
car of \scheme|beta| is chosen and promoted. The result is Figure~\ref{fig:sk}c, in which the engine
delaying \mbox{\scheme|(ints-bottom 2)|} is shared by both \scheme|beta| and
the cdr of \scheme|alpha|. \scheme|(fcadr beta)| forces calculation of
\mbox{\scheme|(ints-bottom 2)|}, which results in a fern, \scheme|gamma|, whose
first value (in this example) is \scheme|2|. Figure~\ref{fig:sk}d now
shows why \scheme|coax-d| updates the current pair (\scheme|beta|)
and creates a new dummy engine with the calculated value (\scheme|gamma|):
the cddr of \scheme|alpha| needs the new engine to avoid recalculation
of \mbox{\scheme|(ints-bottom 2)|}. In Figure~\ref{fig:sk}e when \mbox{\scheme|(fcadr alpha)|}
is evaluated, the value \scheme|2|, calculated already by
\mbox{\scheme|(fcadr beta)|}, is promoted and the engine delaying
\mbox{\scheme|(ints-bottom 3)|} is shared by both \scheme|alpha| and \scheme|beta|.


\part{Context and Conclusions}\label{contextpart}

% Part~\ref{contextpart} provides context and conclusions for the work
% in this dissertation.  Chapter~\ref{relatedworkchapter} describes
% related work, while Chapter~\ref{futureworkchapter} proposes future
% research.  We offer our conclusions in Chapter~\ref{conclusionchapter}

\chapter{Related Work}\label{relatedworkchapter}

This chapter describes some of the work by other researchers that is
related to the research presented in this dissertation.  

% Logic programming is a huge field that has existed for almost forty
% years; it would be impossible to describe all of the relevant related
% work.

\citeauthor{lloyd:lp} \citeyearpar{lloyd:lp} is the standard work on
the theoretical foundations of logic programming; \citeauthor{doets}
\citeyearpar{doets} has written a more recent introduction to the
theory of logic programming.

The most popular logic programming language is Prolog
\cite{ISO:1995:IIIe,ISO:2000:IIIf}.  \citeauthor{ClocksinMellish}
\citeyearpar{ClocksinMellish} have written one of the most popular
introductions to the language.  Prolog was designed by Colmerauer
\cite{prologtenfigs,prologthree}; \citeauthor{birthofprolog}
\citeyearpar{birthofprolog} describe the early history of Prolog.

Most modern implementations of Prolog are based on the Warren Abstract
Machine (WAM) \cite{AICPub641:1983}; \citeauthor{wamtutorial}
\citeyearpar{wamtutorial} presents a tutorial reconstruction of the
WAM.  \citeauthor{DBLP:journals/jlp/Roy94}
\citeyearpar{DBLP:journals/jlp/Roy94} describes in detail the first
decade of sequential Prolog implementation techniques after the
invention of the WAM.

\citeauthor{declproginprolog} has advocated using Prolog for
declarative programming \citeyearpar{declproginprolog}; unfortunately,
Prolog's design and implementation encourages the use of cut and other
non-logical features.  For example, \citeauthor{Naish:1995pr}
\citeyearpar{Naish:1995pr} argues that Prolog programming without cut
is impractical.

% Horn logic

%\section{Combining Functional and Logic Programming}\label{functionalrelated}


There is a long tradition of embedding logic programming operators in
Scheme
\cite{logscheme,schelog,translitprolog,
% nondetinlogscheme, % interpreter, not an embedding
% SriOxlSri85, % not sure if this is an embedding
sicp,metacircularlogicalscheme,logiccontinuationsjlp}.
Most of this work was done during the mid-1980's to early-1990's, and
most of these embeddings can be seen as attempts to combine
Prolog's unification and backtracking search with Scheme's lexical
scope and first-class functions.  Similarly, there have been attempts to
embed logic programming in other functional languages, such as Lisp
\cite{LogLisp,commonlispextendedprolog,ConcurrentLogLISP,qlog,DBLP:books/eh/campbell84/KahnC84} and
Haskell
\cite{embeddingprologinhaskell,Seres:jucsfunctionalreadingoflogic,CombinatorsforLP,kc+pl:haskell00-typedlp,continuationshaskelllp}.
However, the extent to which these languages truly integrate
functional programming and logic programming is debatable; as with
miniKanren, these embeddings are not functional logic programming
languages in the modern sense; they do not provide higher-order
unification or higher-order pattern matching, as in
\lambdaprolog~\cite{nadathur88,lambdaprologdescr}, nor do they use
narrowing or residuation.

Two modern languages that combine logic programming with functional
programming are Mercury \cite{Somogyi95mercury} and Curry
\cite{Hanus95curry:a}. The syntax and type systems of both languages
are inspired by Haskell.

The Mercury compiler uses programmer-supplied type, mode, and
determinism annotations to compile each goal into multiple functions.
this results in very efficient code, which is essential to the Mercury
team's objective of facilitating declarative programming
``in-the-large''.  Unfortunately, this emphasis on the efficiency
comes at the expense of relational programming---forcing, or even
permitting, a programmer to explicitly specify an argument's mode as
``input'' or ``output'' is the antithesis of relational programming.

The Curry language takes a different approach, integrating functional
and logic programming through the single implementation strategy of
narrowing \cite{AntoyHanusMasseySteiner01PPDP}; that is, lazy term
rewriting, with the ability to instantiate logic variables.  Curry
also supports residuation, which allows a goal to suspend if its
arguments are not sufficiently instantiated. For example, a goal that
performs addition might suspend if its first two arguments are not
ground. While residuation is a useful language feature, it inhibits
relational programming since the program will diverge if the arguments
never become instantiated.

miniKanren is the descendant of Kanren \cite{kanrensite}, another
embedding of logic programming in Scheme.  Kanren is closer in spirit
to Prolog than is miniKanren.  Philosophically, Kanren was designed
for efficiency rather than for relational programming. Kanren 
supports neither nominal logic, disequality constraints, nor tabling. Kanren
allows programmers to easily extend existing relations\footnote{This
  can be done in miniKanren as well, through the technique of function
  extension. However, Kanren provides an explicit form for extending a
  relation.}.

Sokuza Kanren is a minimal embedding of logic programming in Scheme;
it is essentially a stripped down version of the core miniKanren
implementation from Chapter~\ref{mkimplchapter}\footnote{For example,
  Sokuza Kanren does not include a reifier.}.






% Maybe in history of miniKanren:
% Daniel P. Friedman and Oleg Kiselyov. ``A Logic System with First-Class
% Relations''. May 2004. Available online: ps pdf.

% Maybe in history of miniKanren:
% R. Kent Dybvig, Daniel P. Friedman, and Michael
% Y. Levin. ``Implementation strategies for Scheme-based Prolog
% systems''. 1998. Available online: ps pdf.

% Mitchell Wand. ``A Semantic Algebra for Logic Programming''. Indiana
% University. TR-148. August 1983.

% Michael C. Rubenstein and Richard M. Salter. ``Computationally Extended
% Logic Programming''. Journal of Computer
% Languages. Vol. 12. Num. 1. 1987.
% \cite{compextendedlp}

% J. Michael Ashley and Richard M. Salter. "A Revised State Space Model
% for a Logic Programming Embedding in Scheme". BIGRE Bulletin. 65. July
% 1989.



%(perhaps footnote on Planner)


%Silvija Seres, Michael J. Spivey, C.A.R. Hoare. Algebra of Logic Programming. ICLP'99. November 1999.
%\cite{algebraoflogicprogramming}


% Implementing functional logic languages using multiple threads and stores
% Tolmach, Andrew and Antoy, Sergio and Nita, Marius
% \cite{multiplethreadsstores}

% Deriving backtracking monad transformers
% Ralf Hinze
% \cite{hinze2000}
% Abstract: In a paper about pretty printing J. Hughes introduced two fundamental techniques for deriving programs from their specification, where a specification consists of a signature and properties that the operations of the signature are required to satisfy. Briefly, the first technique, the term implementation, represents the operations by terms and works by defining a mapping from operations to observations --- this mapping can be seen as defining a simple interpreter. The second, the context-passing implementation, represents operations as functions from their calling context to observations. We apply both techniques to derive a backtracking monad transformer that adds backtracking to an arbitrary monad. In addition to the usual backtracking operations --- failure and nondeterministic choice --- the prolog cut and an operation for delimiting the effect of a cut are supported.


%\section{Core miniKanren}\label{coremkrelated}

% Reification--Prolog operator

% unification  Kevin Knight
% \cite{knightunifsurvey}

% M and M algorithm
% \cite{MartelliMontanari}

% linear time unification

%idempotent vs. triangular substitutions

%disunification: a survey
%\cite{Comon91disunification:a}

%ommiting occurs check (Apt)
%\cite{occurcheck}

%CPS
%\cite{genericaccountcps}
%A-Normal Form, for unnesting
%\cite{essencecompiling}

%BinProlog (for unnesting/CPS-style intermediate language)
%\cite{binprolog}


% Backtracking, interleaving, and terminating monad transformers
% Amr, Dan, Ken, and Oleg
% \cite{backtracking}

% Streams (see references from TRS:  pages 132, 145, 148, 159)
% Philip L. Wadler
% How to replace failure by a list of successes: a method for exception handling, backtracking, and pattern matching in lazy functional languages.
% \cite{failurelistsuccesses}

% Wand and Val
% \cite{Wand04relatingmodels}

\section{Purely Relational Arithmetic}\label{arithrelated}

Chapter~\ref{arithchapter} presents a purely relational binary
arithmetic system.

We first presented arithmetic predicates over binary natural numbers
(including division and logarithm) in a book \cite{trs}.  That
presentation had no detailed explanations, proofs, or formal analysis;
this was the focus of a later paper \cite{conf/flops/KiselyovBFS08}
that presented the arithmetic relations in Prolog rather than
miniKanren.  A lengthier, unpublished version of this
paper\footnote{\url{http://okmij.org/ftp/Prolog/Arithm/arithm.pdf}}
includes appendices containing additional proofs.

Bra{\ss}el, Fischer, and Huch's paper \citeyearpar{numbers-Curry}
appears to be the only previous description of declarative arithmetic.
It is a practical paper, based on the functional logic language Curry.
It argues for declaring numbers and their operations in the language
itself, rather than using external numeric data types and operations.
It also uses a little-endian binary encoding of positive integers
(later extended to signed integers).

Whereas our implementation of arithmetic uses a pure
logic programming language, Bra{\ss}el, Fischer, and Huch
use a non-strict functional-logic programming language.  Therefore,
our implementations use wildly different strategies and
are not directly comparable.  Also, we implement the logarithm relation.

Bra{\ss}el, Fischer, and Huch leave it to future work to prove
termination of their predicates.  In contrast, we have formulated and
proved decidability of our predicates under interleaving search (as
used in miniKanren) and depth-first search (used in Prolog).

Our approach is minimalist and pure; therefore, its methodology can be
used in other logic systems---specifically, Haskell's type classes.
% The logic programming of Haskell typeclasses is
% assuredly pure, with no cut, |var/1|, or negation.
Hallgren \citeyearpar{hallgren01fun} first implemented (unary)
arithmetic in such a system, but with restricted modes. Kiselyov
\citeyearpar[\S6]{npt} treats decimal addition more relationally.
Kiselyov and Shan \citeyearpar{lightweight-resources} first
demonstrated all-mode arithmetic relations for arbitrary binary
numerals, to represent numerical equality and inequality constraints
in the type system.  Their type-level declarative arithmetic library
enables resource-aware programming in Haskell with expressive static
guarantees.

\section{\alphakanren}\label{alphakanrenrelated}

\alphakanren, presented in Chapters~\ref{akchapter} and
\ref{akimplchapter}, is a nominal logic programming language; it was
based on both miniKanren and
\alphaprolog~\cite{CheneyThesis,CheneyU04}.

Early versions of \alphaprologsp implemented equivariant
unification~\cite{DBLP:conf/rta/Cheney05}, which allows the
permutations associated with suspensions to contain logic variables.
The expense of equivariant unification~\cite{DBLP:conf/icalp/Cheney04}
led \citeauthor{DBLP:conf/tlca/UrbanC05} to replace full equivariant
unification with nominal unification \cite{DBLP:conf/tlca/UrbanC05}.
\citeauthor{CheneyThesis}'s dissertation presents numerous examples of
nominal logic programming in \alphaprologsp \cite{CheneyThesis}.
% Unfortunately, it appears that \alphaprologsp is no longer under development.

MLSOS \cite{lakin2007} is another nominal logic language, designed for
easily expressing the rules and side-conditions of Structured
Operational Semantics \cite{Plotkin:2004:SAO}.  MLSOS uses nominal
unification, and introduces \emph{name constraints}, which are
essentially disequality constraints restricted to noms (or to
suspensions that will become noms).

Nominal logic was introduced by \citeauthor{Pitts03}
\citeyearpar{Pitts03}.  Nominal functional languages include FreshML
\cite{ShinwellPG03}, Fresh O'Caml \cite{journals/entcs/Shinwell06},
and C$\alpha$ml \cite{pottier06}.

The first nominal unification algorithm was presented and proved
correct by \citeauthor{Urban-Pitts-Gabbay/04}
\citeyearpar{Urban-Pitts-Gabbay/04}; the algorithm was described using
idempotent substitutions.  

A naive implementation of the \citeauthor{Urban-Pitts-Gabbay/04}
algorithm has exponential time complexity; however, by representing
nominal terms as graphs, and by lazily pushing in swaps, it is
possible to implement a polynomial-time version of nominal unification
\cite{DBLP:journals/tcs/CalvesF08,implnomunif}.

More recently, \citet{DowekEtAl} presented a variant
of nominal unification using ``permissive'' nominal terms, which do
not require explicit freshness constraints.  To our knowledge, there
are no programming languages that currently support permissive nominal
terms.

\section{\alphatap}\label{alphataprelated}

The \alphatapsp relational theorem prover presented in
Chapter~\ref{alphatapchapter} is based on \leantap, a lean
tableau-based prover for first-order logic due
to~\citet{beckert95leantap}.

Through his integration of \leantapsp with the Isabelle theorem
prover, \citet{paulson99generic} shows that it is possible to
modify \leantapsp to produce a list of Isabelle tactics representing a
proof.  This approach could be reversed to produce a proof translator
from Isabelle proofs to \alphatapsp proofs, allowing \alphatapsp to
become interactive as discussed in section~\ref{backwards}.

The \leantapsp Frequently Asked Questions
\cite{beckert-leantsupsuppfaq} states that \leantapsp might be made
declarative through the elimination of Prolog's cuts but does not
address the problem of \mbox{\texttt{copy\_term/2}} or specify how the cuts
might be eliminated.  Other provers written in Prolog include those of
\citet{manthey1988stp} and \citet{stickel1988ptt}, but each uses some
impure feature and is thus not declarative.

\citet{christiansen1998arc} uses constraint logic programming and
metavariables (similar to nominal logic's names) to build a
declarative interpreter based on Kowalski's non-declarative
\texttt{demonstrate} predicate~\cite{kowalski79}.  This approach is
similar to ours, but the Prolog-like language is not complicated by
the presence of binders.

Higher-order abstract syntax (HOAS), presented in
\citet{pfenning1988hoa}, can be used instead of nominal logic to
perform substitution on quantified formulas. \citet{felty1988stp} were
among the first to develop a theorem prover using HOAS to represent
formulas; \citet{pfenning1999sdt} also use a HOAS encoding for
formulas.

Kiselyov uses a HOAS encoding for universally quantified formulas in
his original translation of \leantapsp into
miniKanren~\cite{kanrensite}. Since miniKanren does not implement
higher-order unification, the prover cannot generate theorems.

\citeauthor{lisitsalambdald}'s $\lambda$\leantapsp
\citeyearpar{lisitsalambdald} is a prover written in \lambdaprologsp
that addresses the problem of \mbox{\texttt{copy\_term/2}} using HOAS, and is
perhaps closest to our own work.  Like \alphatap, $\lambda$\leantapsp
replaces universally quantified variables with logic variables using
substitution. However, $\lambda$\leantapsp is not declarative, since
it contains cuts.  Even if we use our techniques to remove the cuts
from $\lambda$\leantap, the prover does not generate theorems, since
\lambdaprologsp uses a depth-first search strategy.  Generating
theorems requires the addition of a tagging scheme and iterative
deepening on \emph{every clause} of the program.  Even with these
additions, however, $\lambda$\leantapsp often generates theorems that
do not have the proper HOAS encoding, since that encoding is not
specified in the prover.

% \section{Type Inferencer}\label{inferencerrelated}

% simply-typed lambda calculus
% \cite{lamcalcwithtypes}

% Hindley-Milner

% A Theory of Type Polymorphism in Programming
% Robin Milner
% \cite{Milner78}


% Algorithm W

% Principal type-schemes for functional programs
% Damas, Luis and Milner, Robin
% \cite{hindleymilner}


% Pierce
% \cite{tapl}

% intuitionistic logic



% Curry-Howard Isomorphism
% \cite{Howard80}

% A formulae-as-type notion of control
% Griffin, Timothy G.
% \cite{Griffin90}

% Oleg's type inhabitation tool (Haskell and Scheme)

% Djinn (other, competing type inhabitation tool)

% Formula Tree Lab (type inhabitation)

% Coq

% Twelf

% type inhabitation,
% type habitation,
% term reconstruction



% \section{Term Reducer}\label{reducerrelated}

% The relational term reducer in Chapter~\ref{reducerchapter} was
% inspired by PLT Redex, a domain-specific language for specifying
% operational semantics \cite{DBLP:conf/rta/MatthewsFFF04}.  Like
% \alphakanren, PLT Redex is embedded in Scheme.  PLT Redex is more
% sophisticated than the reducer of Chapter~\ref{reducerchapter}; given
% a grammar and a set of rewrite rules, PLT Redex will automatically
% generate a stepper.  However, PLT Redex cannot ``run backwards'', and
% does not support nominal terms.

% Two other popular term rewriting systems are Maude \cite{Maude2:03}
% and Stratego \cite{stratego}. \citeauthor{termrewritingandallthat}
% have written an excellent introduction to term rewriting
% \cite{termrewritingandallthat}.  How to best combine term rewriting
% with nominal logic is an active area of research
% \cite{nominalrewriting,nomrewritingsystems,nomrewritingwithnamegen,hierarchicalnomterms}.

\section{Tabling}\label{tablingrelated}

Tabling is essentially an efficient way to find fixed points.  Tabling
can be used to implement model checkers, abstract interpreters,
deductive databases, and other useful programs that must calculate
fixed points~\cite{dra09,memoingforlp}.

Many Prolog implementations support some form of tabling.  XSB
Prolog~\cite{xsb}, which uses SLG Resolution~\cite{SLGresolution} and
the SLG-WAM abstract machine~\cite{SLGwam}, remains the standard
testbed for advanced tabling implementation.  Our implementation was
originally inspired by the Dynamic Reordering of Alternatives (DRA)
approach to
tabling~\cite{dra09,simpleimplementingtabling}.


% OLD resolution with tabulation (OLDT)~\cite{oldt}, SLG
% Resolution~\cite{SLGresolution}, the SLG-WAM abstract
% machine~\cite{SLGwam}.  

% CAT: The Copying Approach to Tabling
% Demoen, Bart and Sagonas, Konstantinos F.
% \cite{meow}

% CHAT: the copy-hybrid approach to tabling
% Demoen, Bart and Sagonas, Konstantinos
% \cite{chat}

% adding closures to the WAM

% streams-based and continuations-based tabling


\section{Ferns}\label{fernsrelated}

Chapter~\ref{fernschapter} describes ferns, a shareable,
bottom-avoiding data structure invented by \citet{ferns81}.
Chapter~\ref{fernsimpl} presents our shallow embedding of ferns in
Scheme.

Previous implementations of ferns have been for a call-by-need
language.  The work of \citet{Friedman79b,DFried80,ferns81} presumes a
deep embedding whereas our approach is a shallow embedding.  The
function \scheme|coax| is taken from their conceptualization~\cite{Friedman79b}:

\begin{quote}
  {COAX is a function which takes a suspension as an argument and
    returns a field as a value; that field may have its \emph{exists}
    bit \emph{true} and its pointer referring to its \emph{exist}ent
    value, or it may have its \emph{exists} bit false and its pointer
    referring to another suspension.}
\end{quote}

\noindent
Thus, engines are a user-level, first-class manifestation of
suspensions where \emph{true} above corresponds to the unused ticks.  
\citeauthor{Johnson-77}'s master's thesis \citeyearpar{Johnson-77} 
under Friedman's direction presents
a deep embedding in Pascal for a lazy ferns language.  Subsequently,
Johnson and his doctoral student Jeschke implemented a series of
native C symbolic multiprocessing systems based on the Friedman and
Wise model.  This series culminated with the parallel implementation
Jeschke describes in his dissertation \cite{Jeschke-PHD-95}.  In their
\emph{Daisy} language, ferns are the means of expressing explicit
concurrency \cite{Johnson-83}.



% [TODO Integrate the related work sections of the various papers.
% Also, the footnotes throughout the dissertation describe related
% work.]

% \noindent [look at related work sections of each paper and of TRS]

% \noindent [One of Siskind's PhD students said that there is a researcher working
% on making functions ``run backwards''.  Dan thinks his name is
% C. Hennie.  Dan doesn't think his work is directly relevant, but I
% need to check anyway.  Maybe he has some useful techniques.]

% \noindent [Apt's papers: 'Declarative Programming in Prolog' and 'Declarative
% Interpretation Reconsidered']

% %%%\noindent [Work on design patterns for logic programs.]

% \noindent [Mercury, Curry]

% \noindent [Dan thinks I need to discuss modes, perhaps in the context of
% Mercury]

% \noindent [Disjkstra guard for non-overlapping property]

% \noindent [CPS, and esp. A-Normal Form, for unnesting]

% \noindent [BinProlog (for unnesting/CPS-style intermediate language)]

% \noindent [Combinators for Logic Programming.  Work of Hanus.]

% \noindent [Schelog/Prolog on a Page.  Other implementations of logic languages
% in Scheme or functional languages]

% \noindent [Kanren/Sokouza Kanren]

% \noindent [alphaProlog, MLSOS, CaML, nominal logic/unification/Urban, Pitts and
% Gabbay]

% \noindent [Competing relational arithmetic system from German researchers]

% \noindent [ICFP paper by Amr, Dan, Ken, and Oleg on backtracking monads]

% \noindent [HOAS as an alternative to nominal logic]

% \noindent [CLP]


\chapter{Future Work}\label{futureworkchapter}

In this chapter we propose future work related to miniKanren, and to
relational programming in general.

This chapter is organized as follows.  In
section~\ref{futureformalization} we discuss how our work on
miniKanren might be formalized.  Section~\ref{futureimplementation}
presents possible improvements to the existing miniKanren
implementation, while section~\ref{futureextensions} suggests how the
miniKanren language might be extended.  Section~\ref{futureidioms}
considers future work on relational idioms, while
section~\ref{futureapplications} proposes future applications of
miniKanren. Finally, in section~\ref{futuretools} we propose tools
that might ease the burden on relational programmers.

\section{Formalization}\label{futureformalization}

From a formalization standpoint, the most important future work is to
create a formal semantics for miniKanren.  Perhaps the simplest
approach would be to start from the operational semantics of the
nominal logic programming language MLSOS, as described
in~\citet{lakin2007}.  Of course, miniKanren's semantics would become
more complex if the language extensions proposed in
section~\ref{futureextensions} were added.  Indeed, it is the
interaction between different language features (nominal unification
and constraint logic programming, for example) that will make
extending miniKanren challenging.

The core miniKanren implementation presented in
Chapter~\ref{mkimplchapter} uses a stream-based interleaving search
strategy.  The use of \scheme|inc|s (thunks) to force interleaving
makes it difficult to exactly characterize the search behavior, and
therefore the order in which miniKanren produces answers.  It would be
both interesting and useful to mathematically describe this
interleaving behavior (see section~\ref{futureimplementation}).

% Prove properties of \scheme|walk| (recent work by Dave, Andy, and Dan)

In Chapter~\ref{alphatapchapter} we replaced \leantap's use of
Prolog's \mbox{{\tt copy\_term/2}} with a purely declarative combination of
tagging and nominal unification; this technique was key to making
\alphatapsp purely relational.  Unfortunately, this approach can only
be used when the programmer knows the structure of the terms to be
copied.  It would be useful to formalize this technique, to better
understand its applicability and limitations.

%formalize technique in 
%revist unhappy family
%encasulation of search in Curry (cloning argument)

The relational arithmetic system presented in
Chapter~\ref{arithchapter} uses bounds on term sizes to provide strong
termination guarantees for arithmetic relations\footnote{At least, for
  single arithmetic relations whose arguments do not share
  unassociated logic variables.}.  A systematic approach to deriving
such bounds on term sizes would be very helpful for relational
programmers.  Of course, G\"{o}del and Turing showed that it is
impossible to guarantee termination for all goals we might wish to
write, so in general we will not be able to achieve finite failure
through bounds, or any other technique\footnote{For example, the
  strong termination guarantees for our arithmetic system do not hold
  for conjunction of addition and multiplication goals.}.  However,
even when such bounds exist, it may be difficult to express them in
miniKanren.  Indeed, poorly expressed bounds may themselves cause
divergence---for example, by attempting to eagerly determine the
length of an uninstantiated (and therefore unbounded)
list\footnote{See Chapter~\ref{divergencechapter} for more on the
  difficulty of expressing bounds on term sizes.}.  A systematic
approach to expressing bounds already derived by the programmer would
be most useful.

Section~\ref{triangularsection} presents a Scheme implementation of a
nominal unifier that uses triangular substitutions.  This algorithm
should be formalized and proved correct, similar to the presentation
of (idempotent) nominal unification in~\citet{Urban-Pitts-Gabbay/04}.
% [perhaps Joe's would be easier to prove]

\citet{HermanWand08:Hygiene} use nominal logic to describe an
idealized version of Scheme's \scheme|syntax-rules| hygienic macro
system.  It would be interesting to extend this work to the full
\scheme|syntax-rules| system, perhaps by implementing the macro system
as an \alphakanrensp relation.

A more speculative area of future work is the connection between the
various causes of divergence described in
Chapter~\ref{divergencechapter}.  As discussed in the conclusion of
this dissertation, there may be a deep connection between these causes
of divergence, and between the techniques for avoiding them.  Since
divergence is an effect, monads~\cite{moggi91notions} or
arrows~\cite{hughes98generalisingmonads} may provide the best
framework for exploring these ideas.

%monads + mk
%monads/arrows + divergence


\section{Implementation}\label{futureimplementation}

The core miniKanren implementation presented in
Chapter~\ref{mkimplchapter} uses streams to implement backtracking
search\footnote{Although one could argue that the stream-based
  implementation performs backtracking search without actually
  backtracking.}.  As described in \citet{Wand04relatingmodels}, our
use of streams could be modelled using explicit success and failure
continuations.  When extending the miniKanren language, it is
sometimes more convenient to use this two-continuation model of
backtracking---for example, the first implementation of tabling for
miniKanren used continuations rather than streams.

The streams implementation of miniKanren makes liberal use of
\scheme|inc|s (thunks) to force interleaving in the search.
Unfortunately, it is difficult to exactly replicate this interleaving
search behavior in the two-continuation model.  As a result,
continuation-based implementations of miniKanren may produce answers
in a different order than stream-based implementations, which makes it
difficult to test, benchmark, or otherwise compare different
implementations.  It therefore would be extremely convenient to have a
continuation-based implementation of miniKanren that exactly mirrors
the search behavior of the streams-based implementation from
Chapter~\ref{mkimplchapter}.  This may require a formal
characterization of the stream-based search strategy, as discussed in
section~\ref{futureformalization}.

We currently use association lists to represent substitutions; we may
wish to consider other purely functional representations of
substitutions that would make variable lookup less expensive.  For
example, Abdulaziz Ghuloum previously implemented a trie-based
representation of substitutions that performs at least as well as the
fastest walk-based algorithm presented in
Chapter~\ref{walkimpl}. Using a trie-based representation of
substitutions may mean giving up on the clever method of implementing
disequality constraints described in Chapter~\ref{diseqimplchapter}.

Relational programming is inherently parallelizable.  In fact, we have
already implemented two parallel versions of miniKanren: one written
in Scheme and one in Erlang~\cite{armstrong03}. However, neither
parallel implementation runs as quickly as the sequential
implementation of miniKanren presented in Chapter~\ref{mkimplchapter}.
One difficulty in making a parallel implementation run efficiently is
that miniKanren suffers from an ``embarrassment of parallelism''.  For
example, a recursive goal might contain a \scheme|conde| whose first
clause contains a single unification. The overhead of sending this
single unification to a new core or processor may be more expensive
than just performing the unification. Ciao Prolog solves this problem
by performing a ``granularity analysis'' to determine which parts of a
program perform enough computation to offset the overhead of
parallelization~\cite{granularity90,lopez96methodology}.

Our purely functional implementation of miniKanren also implies a
different set of design choices than would be made when parallelizing
a Prolog implementation based on the Warren Abstract Machine. In
particular, our stream-based search implementation, combined with our
functional representation of
substitutions\footnote{\citet{gupta93analysis} have explored the
  tradeoffs of different environment representations in the context of
  parallel logic programming.}, means that disjunction is truly
parallel: failure of one disjunct does not require communication with
other disjuncts.

Reification of nominal terms is another area for future work.  The
core-miniKanren reifier presented in Chapter~\ref{mkimplchapter}
enforces several important invariants: swapping adjacent calls to
\scheme|==|, swapping arguments within a single call to \scheme|==|,
or reordering nested \scheme|exist| clauses\footnote{Assuming this is
  done without inadvertently shadowing variables, or leaving
  previously bound variables unbound.} cannot affect reified answers.
We would like \alphakanrensp to ensure similar invariants; however,
reification in \alphakanrensp is more complicated, since each term
containing a \scheme|tie| now represents an infinite equivalence class
of $\alpha$-equivalent terms. Additionally, we do not have a canonical
representation for permutations associated with suspensions. Finally,
reification must also handle freshness constraints.

miniKanren uses a complete interleaving search strategy, which ensures
disjunction (\scheme|conde|) is commutative---swapping the order of
\scheme|conde| clauses can affect the order in which answers are
returned, but cannot affect whether a goal diverges. In contrast,
miniKanren's conjunction operators (\scheme|exist| and \scheme|fresh|)
are not commutative---swapping conjuncts can cause a goal that
previously failed finitely to now diverge.  It is easy to see that
commutative conjunction can be implemented: just run in parallel every
possible ordering of conjuncts. Unfortunately, this simplistic
approach is far too expensive to be used in practice. However, it may
be possible to more efficiently implement commutative conjunction by
interleaving the evaluation of conjuncts, and allowing each conjunct
to partially extend the substitution. This would allow conjuncts to
communicate with each other by extending the substitution, thereby
allowing the conjunction to ``fail fast'', and avoiding the
duplication of work inherent in the naive approach described above.
It is not clear whether this approach is efficient enough to be used
throughout an entire program; the programmer may need to restrict use
of commutative conjunction to conjunctions containing multiple
recursive goals.

Alternatively, it may be possible to \emph{simulate} commutative
conjunction using a combination of continuations, interleaving search,
and tabling.  This approach would only be a simulation of true
commutative conjunction because tabling is defeated if an argument
changes with each recursive call.

The core miniKanren implementation presented in
Chapter~\ref{mkimplchapter} is an embedding in Scheme, using a
combination of procedures and hygienic macros.  Although this
embedding allows us to easily benefit from the optimizations provided
by a host Scheme implementation, we lose the ability to analyze or
transform entire miniKanren programs. A miniKanren compiler would
allow us to perform more sophisticated program analyses. Finally, a
miniKanren interpreter\footnote{In the long tradition of writing
  meta-circular Scheme interpreters, a meta-circular miniKanren
  interpreter would be especially satisfying.} or abstract machine
would be useful from both an implementation and formalization
standpoint.


\section{Language Extensions}\label{futureextensions}

\alphakanren's support for nominal logic programming could be extended
in several ways.  Perhaps the simplest extension would be to add
MLSOS's name inequality constraint~\cite{lakin2007}, which is
essentially a disequality constraint limited to noms (and to
suspensions that will become noms).  A more ambitious extension would
be to add full disequality constraints to \alphakanren.  One might
also implement equivariant unification~\cite{DBLP:conf/rta/Cheney05},
which extends nominal unification with the ability to include logic
variables in permutations; however, the expense of equivariant
unification~\cite{DBLP:conf/icalp/Cheney04} limits its
appeal\footnote{Although~\citet{DBLP:conf/tlca/UrbanC05} show that it
  is often possible to avoid full equivariant unification in real
  programs.}.  \citet{DowekEtAl} recently presented
a variant of nominal unification using ``permissive'' nominal terms,
which do not require explicit freshness constraints; permissive
nominal terms might simplify reification of \alphakanrensp answers.

Our tabling implementation does not currently work with disequality
constraints or freshness constraints.  It would be very useful to
extend tabling to work with these constraints.  Alternatively, it may
be possible to add tabling to \alphakanrensp by using permissive
nominal terms, which do not require freshness constraints.

\citet{DBLP:conf/iclp/GuptaBMSM07} have implemented a coinductive
logic programming language that can express infinite streams using
coinductive definitions of goals.  The heart of their system is an
implementation of tabling, in which unification rather than
reification is used to determine whether a call is a variant of an
already tabled call.  It should be straightforward to add coinductive
logic programming to miniKanren, since we have already implemented
tabling.  Also, it would be interesting to investigate if other
notions of variant calls make sense---for example, what if we used
subsumption instead of reification or unification? Would we get a
different type of logic programming?  Finally, the streams that can be
created using the system of \citeauthor{DBLP:conf/iclp/GuptaBMSM07}
must have a regular structure---for example, their system cannot
represent a stream of all the prime numbers.  How might more
sophisticated streams be expressed?

One alternative to requiring the occurs check for sound unification is
to allow infinite terms, as in Prolog II. This would require changing
the reifier to print circular terms. We would also want our core
language forms, such as disequality constraints, to handle infinite
terms\footnote{SWI Prolog~\cite{Wielemaker:03b} includes many
  predicates that work on infinite terms, and might serve as an
  inspiration.}.

An extremely useful extension to miniKanren would be the addition of
constraint logic programming, or
CLP~\cite{DBLP:journals/jlp/JaffarM94}\footnote{Actually, miniKanren
  and \alphakanrensp already support several types of constraints:
  unification (\scheme|==|) and dis-unification (\scheme|=/=|)
  constraints, and the freshness constraints of nominal logic.
  However, there are many other types of constraints we might want to
  add.}.  The notation `CLP(X)' refers to constraint logic programming
over some domain `X'; common domains include the integers (CLP(Z)),
rational numbers (CLP(Q)), real numbers (CLP(R)), and finite domains
(CLP(FD)).  Most useful for existing applications of miniKanren would
be CLP(FD) and CLP(Z), which would allow us to declaratively express
arithmetic in a more efficient manner than the arithmetic system of
Chapter~\ref{arithchapter}\footnote{The declarative arithmetic system
  of Chapter~\ref{arithchapter} has several advantages over the
  constraint approach, however.  As opposed to CLP(FD), our system
  works on numbers of arbitrary size.  Our system is also implemented
  entirely at the user-level language, without any constraints other
  than unification, while adding CLP(FD) or CLP(Z) requires
  significant changes to the underlying implementation, and may
  interact in undesirable ways with other language features.}.

miniKanren, like Scheme, is dynamically-typed. \citet{siek06:gradual}
show how \emph{gradual typing} can be used to add a sophisticated type
system to a dynamically typed language, without giving up the
flexibility of dynamic typing\footnote{There has also been recent work
  on adding something like gradual typing to Prolog (see
  ~\cite{towardstypedprolog}, although it is unclear whether these
  researchers are aware of the Scheme community's work on gradual
  typing and soft typing~\cite{softtyping}.}.  It would be interesting
to apply this typing scheme to miniKanren, since supporting logic
variables and constraints may require extending the notions of gradual
typing.

Relational goals often append two lists; if the first list is an
uninstantiated logic variable, this results in infinitely many
answers, which can easily lead to divergence. It may be possible to
create an \scheme|append| constraint that represents the delayed
appending of two lists, and avoids enumerating infinitely many
appended lists.

Another line of future work would be to implement non-standard logics
for relational programming, such as temporal logic, linear logic, and
modal logic.  Of course, supporting any of these logics would require
significant changes to miniKanren, and would require careful
consideration of how various language extensions would interact with
the new logic.

Modern functional logic programming languages like Curry are based on
narrowing~\cite{AntoyHanusMasseySteiner01PPDP}, which combines term
rewriting with the ability to instantiate logic variables. It would be
interesting to implement a language based on \emph{nominal}
narrowing---that is, narrowing based on nominal
rewriting~\cite{nominalrewriting}. This would allow a single
implementation to express nominal functional programming (as in
FreshML~\cite{ShinwellPG03} or C$\alpha$ml~\cite{pottier06}), nominal
logic programming (as in \alphaprolog~\cite{CheneyU04},
MLSOS~\cite{lakin2007}, or \alphakanren), hygienic macros (as in
Scheme\footnote{\citet{HermanWand08:Hygiene} describe a simplified
  version of Scheme's \scheme|syntax-rules| macro system using nominal
  logic.}), and nominal term rewriting (as in Maude~\cite{Maude2:03},
Stratego~\cite{stratego}, or PLT
Redex~\cite{DBLP:conf/rta/MatthewsFFF04}, but with the addition of
nominal logic).

Like MLSOS and \alphaprolog, \alphakanrensp is well suited for
expressing the rules and side-conditions of Structural Operational
Semantics (SOS)~\cite{Plotkin:2004:SAO}.  It would be informative to
explore which SOS rules or side-conditions \emph{cannot} be easily
expressed in \alphakanren; such an exercise would likely result in new
constraints and other language extensions.  Similarly, it would be
informative to investigate which Scheme, Prolog, and Curry programs we
cannot satisfactorily express in a purely relational manner.

Perhaps the greatest challenge in extending miniKanren is to combine
all of these language features in a meaningful way. Ciao Prolog
attempts to control interactions between language features through a
module system~\cite{DBLP:journals/tcs/GrasH99}. The addition of
libraries to the \RsixRSsp Scheme standard~\cite{r6rs} should allow us
to do the same. However, a more sophisticated approach based on monads
and monad transformers may better control the interaction between
language features.



\section{Idioms}\label{futureidioms}

\citet{Okasaki:1999} has investigated the use of purely functional
data structures, many of which are comparable in efficiency to
imperative data structures\footnote{Indeed, uses of purely functional
  data structures can be even more efficient than uses of imperative
  data structures, due to sharing.}.  Even more so than in functional
programming, data representation is essential to relational
programming.  Therefore, it would be interesting and useful to
investigate the use of \emph{purely relational} data structures---that
is, data structures and data representations that are especially
well-suited for relational programming.  Some of these data structures
might take advantage of relational language features such as nominal
unification or constraints.

Also, as mentioned in section~\ref{futureformalization}, it would be
useful to formalize our combination of tagging and nominal unification
to emulate Prolog's \mbox{{\tt copy\_term/2}} in a purely declarative manner.



\section{Applications}\label{futureapplications}

It should be relatively easy to extend the arithmetic system of
Chapter~\ref{arithchapter} to handle rational numbers.  Probably the
most difficult part of this exercise would be maintaining fractions in
simplified form.

An interesting extension to the type inferencer in
section~\ref{aktypeinf} would be to support
polymorphic-\scheme|let|~\cite{tapl}.  At a minimum, this would
require a declarative way to perform a combination of substitution and
term copying.  Of course, the implementation of \alphatapsp in
Chapter~\ref{alphatapchapter} also uses these techniques.  However,
there may be enough differences between \alphatapsp and the type
inferencer to make applying these techniques difficult or impossible.
If so, a new type of constraint may be called for.

As described in Chapter~\ref{alphatapchapter}, the \alphatapsp theorem
prover allows a user to guide the proof search by partially
instantiating the prover's proof-tree argument.  It should be possible
to extend \alphatap, making it act as a rudimentary interactive proof
assistant.  This would further demonstrate the flexibility of
relational programming; more importantly, creating such a tool might
require new techniques that would be useful for writing relational
programs in general.



\section{Tools}\label{futuretools}

As mentioned in section~\ref{futureformalization}, integrating bounds
on term size into an existing relation can be difficult.  A tool that
could take a relation, along with a specification of bounds on the
argument sizes, and synthesize a new relation that incorporates those
bounds would be extremely helpful.

A tool to automatically translate Scheme programs to miniKanren would
also be handy.  Ideally, this tool would generate purely relational
miniKanren code adhering to the non-overlapping principle (see
section~\ref{reconsiderrember}).  This may be possible, at least for
many simple Scheme functions, if the programmer were to help the tool
by specifying how to represent terms, along with an appropriate
tagging scheme.  However, deriving miniKanren relations from Scheme
functions is not the real difficultly---rather, ensuring finite
failure for a wide variety of arguments is what makes relational
programming so difficult.

A Prolog-to-Scheme translator would also be useful.  Translating pure
Prolog programs into miniKanren should be very easy, especially since
the \scheme|lambda-e| pattern-matching macro is similar to Prolog's
pattern matching syntax.


% Design patterns for relational programming

%Contracts and relational programming

% Proof techiques/recursion


\chapter{Conclusions}\label{conclusionchapter}

This dissertation presents the following high-level contributions:

\begin{enumerate}
\item A collection of idioms, techniques, and language constructs for
  relational programming, including examples of their use, and a
  discussion of each technique and when it should or should not be
  used.

\item Various implementations of core miniKanren and its variants,
which utilize the full power of Scheme, are concise and easily
extensible, allow sharing of substitutions, and provide backtracking
``for free''.

\item A variety of programs demonstrating the power of relational programming.

\item A clear philosophical framework for the practicing relational
programmer.
\end{enumerate}

\noindent More specifically, this dissertation presents:

\begin{enumerate}
\item A novel constraint-free binary arithmetic system with strong termination
guarantees.

\item A novel technique for eliminating uses of \mbox{{\tt copy\_term/2}}, using
nominal logic and tagging.

%\item A collection of idioms, techniques, and language constructs for
%relational programming, including examples of their use, a
%discussion of each technique and when it should or should not be
%used.

\item A novel and extremely flexible lean tableau theorem prover that
acts as a proof generator, theorem generator, and even a simple
proof assistant.

% \item A flexible type inferencer that performs type checking and type
%  inhabitation.

% \item A relational term reducer.

\item The first implementation of nominal unification using triangular
substitutions, which is much faster than a naive implementation that
follows the formal specification by using idempotent substitutions.

\item An elegant, streams-based implementation of tabling,
demonstrating the advantage of embedding miniKanren in a language with
higher-order functions.

\item A novel \emph{walk}-based algorithm for variable lookup in
triangular substitutions, which is amenable to a variety of
optimizations.

%\item Various implementations of core miniKanren and its variants,
%which utilize the full power of Scheme, are concise and easily
%extensible, allow sharing of substitutions, and provide backtracking
%``for free''.

%\item A variety of simple examples demonstrating the power of relational
%programming in miniKanren.

%\item A clear philosophical framework for the practicing relational
%programmer.

\item A novel approach to expression-level divergence avoidance using
ferns, including the first shallow embedding of ferns.
\end{enumerate}

\noindent The result of these contributions is a set of tools and
techniques for relational programming, and example applications
informing the use of these techniques.

As stated in the introduction, the thesis of this dissertation is that
miniKanren supports a variety of relational idioms and techniques,
making it feasible and useful to write interesting programs as
relations.  The technique and implementation chapters should establish
that miniKanren supports a variety of relational idioms

\noindent 
and techniques.  The application chapters should establish that it is
feasible and useful to write interesting programs as relations in
miniKanren, using these idioms and techniques.

A common theme throughout this dissertation is divergence, and how to
avoid it.  Indeed, an alternative title for this dissertation could
be, ``Relational Programming in miniKanren: Taming
$\bot$.''\footnote{With apologies to Olin Shivers.}  As we saw in
Chapter~\ref{divergencechapter}, there are many causes of divergent
behavior, and different techniques are required to tame each type of
divergence.  Some of these techniques merely require programmer
ingenuity, such as the data representation and bounds on term size
used in the arithmetic system of Chapter~\ref{arithchapter}.  Other
techniques, such as disequality constraints and tabling, require
implementation-level support.

G\"{o}del and Turing showed that it is impossible to guarantee
termination for every goal we might wish to write.  However, this does
not mean that we should give up the fight.  Rather, it means that we
must be willing to thoughtfully employ a variety of techniques when
writing our relations---as a result, we can write surprisingly
sophisticated programs that exhibit finite failure, such as our
declarative arithmetic system.  It also means we must be creative, and
willing to invent new declarative techniques when necessary---perhaps
a new type of constraint or a clever use of nominal logic, for
example\footnote{We can draw inspiration and encouragement from work
  that has been done on NP-complete and NP-hard problems.  Knowing
  that a problem is NP hard is not the end of the story, but rather
  the beginning.  Special cases of the general problem may be
  computationally tractable, while probabilistic or approximation
  algorithms may prove useful in the general case.  (A good example is
  probabilistic primality testing, used in cryptography for decades.
  Although \citet{Agrawal02primesis} recently showed that primality
  testing can be performed deterministically in polynomial time, the
  potentially fallible probabilistic approach is still used is
  practice, since it is more efficient.)  A researcher in this area
  must be willing to master and apply a variety of techniques to
  construct tractable variants of these problems. Similarly, a
  relational programmer must be willing to master and apply a variety
  of techniques in order to construct a relation that fails finitely.
  This often involves trying to find approximations of logical
  negation (such as various types of constraints).}.

Of course, no one is forcing us to program relationally.  After trying
to wrangle a few recalcitrant relations into termination, we may be
tempted to abandon the relational paradigm, and use miniKanren's
impure features like \scheme|conda| and \scheme|project|.  We might
then view miniKanren as merely a ``cleaner'', lexically scoped version
of Prolog, with S-expression syntax and higher-order functions.
However tempting this may be, we lose more than the flexibility of
programs once we abandon the relational approach: we lose the need to
construct creative solutions to difficult yet easily describable
problems, such as the \scheme|rembero| problem in
Chapter~\ref{diseqchapter}.  

The difficulties of relational programming should be embraced, not
avoided.  The history of Haskell has demonstrated that a commitment to
purity, and the severe design constraints this commitment implies,
leads to a fertile and exciting design space.  From this perspective,
the relationship between miniKanren and Prolog is analogous to the
relationship between Haskell and Scheme.  Prolog and Scheme allow, and
even encourage, a pure style of programming, but do not require it; in
a pinch, the programmer can always use the ``escape hatch'' of an
impure operator, be it cut, \scheme|set!|, or a host of other
convenient abominations, to leave the land of purity.  miniKanren and
Haskell explore what is possible when the escape hatch is welded shut.
Haskell programmers have learned, and are still learning, to avoid
explicit effects by using an ever-expanding collection of monads;
miniKanren programmers are learning to avoid divergence by using an
ever-expanding collection of declarative techniques, many of which
express limited forms of negation in a bottom-avoiding manner.
Haskell and miniKanren show that, sometimes, painting yourself into a
corner can be liberating\footnote{President John F. Kennedy expressed this idea best, in his remarks at the dedication of the Aerospace Medical Health Center, the day before he was assassinated.
\begin{quotation}
We have a long way to go. Many weeks and months and years of long, tedious work lie ahead. There will be setbacks and frustrations and disappointments. There will be, as there always are,$\ldots$temptations to do something else that is perhaps easier. But this research here must go on. This space effort must go on. $\ldots$ That much we can say with confidence and conviction.

    Frank O'Connor, the Irish writer, tells in one of his books how, as a boy, he and his friends would make their way across the countryside, and when they came to an orchard wall that seemed too high and too doubtful to try and too difficult to permit their voyage to continue, they took off their hats and tossed them over the wall---and then they had no choice but to follow them.

    This Nation has tossed its cap over the wall of space, and we have no choice but to follow it. Whatever the difficulties, they will be overcome. Whatever the hazards, they must be guarded against. With the$\ldots$help and support of all Americans, we will climb this wall with safety and with speed---and we shall then explore the wonders on the other side.
\end{quotation}
Remarks at the Dedication of the Aerospace Medical Health Center\\
President John F. Kennedy\\
San Antonio, Texas\\
November 21, 1963
}.

% Kennedy quote
%
% Remarks at the Dedication of the Aerospace Medical Health Center
% President John F. Kennedy
% San Antonio, Texas
% November 21, 1963
%
% http://www.jfklibrary.org/Historical+Resources/Archives/Reference+Desk/Speeches/JFK/003POF03AerospaceMedicalCenter11211963.htm
%
%     I think the United States should be a leader. A country as rich and powerful as this which bears so many burdens and responsibilities, which has so many opportunities, should be second to none. And in December, while I do not regard our mastery of space as anywhere near complete, while I recognize that there are still areas where we are behind--at least in one area, the size of the booster--this year I hope the United States will be ahead. And I am for it. We have a long way to go. Many weeks and months and years of long, tedious work lie ahead. There will be setbacks and frustrations and disappointments. There will be, as there always are, pressures in this country to do less in this area as in so many others, and temptations to do something else that is perhaps easier. But this research here must go on. This space effort must go on. The conquest of space must and will go ahead. That much we know. That much we can say with confidence and conviction.
%     Frank O'Connor, the Irish writer, tells in one of his books how, as a boy, he and his friends would make their way across the countryside, and when they came to an orchard wall that seemed too high and too doubtful to try and too difficult to permit their voyage to continue, they took off their hats and tossed them over the wall--and then they had no choice but to follow them.
%     This Nation has tossed its cap over the wall of space, and we have no choice but to follow it. Whatever the difficulties, they will be overcome. Whatever the hazards, they must be guarded against. With the vital help of this Aerospace Medical Center, with the help of all those who labor in the space endeavor, with the help and support of all Americans, we will climb this wall with safety and with speed-and we shall then explore the wonders on the other side.

A final, very speculative observation: it may be possible to push the
analogy between monads and techniques for bottom avoidance further.
Before Moggi's work on monads~\cite{moggi91notions}, the relationship
between different types of effects was not understood---signaling an
error, printing a message, and changing a variable's value in memory
seemed like very different operations.  Moggi showed how these
apparently unrelated effects could be encapsulated using monads,
providing a common framework for a wide variety of effects.  Could it
be that the various types of divergence described in
Chapter~\ref{divergencechapter} are also related, in a deep and
fundamental way?  Indeed, divergence itself is an effect.  From the
monadic viewpoint, divergence is equivalent to an error, while from
the relational programming viewpoint, divergence is equivalent to
failure; is there a deeper connection?


% Our work makes the following contributions:

% \begin{enumerate}
% \item A novel constraint-free binary arithmetic system with strong termination
% guarantees.

% \item The first implementation of nominal unification using triangular
% substitutions, which is much faster than a naive implementation that
% ``follows the math'' and uses idempotent substitutions.

% \item An elegant, continuation-based implementation of tabling,
% demonstrating the advantage of embedding miniKanren in a language with
% higher-order functions.

% \item A novel, \emph{walk}-based algorithm for variable lookup in
% triangular substitutions, which is amenable to a variety of
% optimizations.

% \item Various implementations of core miniKanren and its variants,
% which utilize the full power of Scheme, are concise and easily
% extensible, allow sharing of substitutions, and provide backtracking
% ``for free''.

% \item A variety of simple examples demonstrating the power of relational
% programming in miniKanren.

% \item A novel and extremely flexible lean tableau theorem prover that
% acts as a proof generator, theorem generator, and even a simple
% proof assistant.

% \item A collection of idioms, techniques, and language constructs for
% relational programming, including examples of their use, a
% discussion of each technique and when it should or should not be
% used.

% \item A clear philosophical framework for the practicing relational
% programmer.

% \item A novel approach to expression-level divergence avoidance using
% ferns, including the first shallow embedding of ferns.

% \item A novel technique for eliminating uses of {\tt copy\_term/2}, using
% nominal logic and tagging.
% \end{enumerate}

% [what about our ability to implement disequality constraints using
% unification?  We weren't the first to invent the idea, but I doubt
% there are many implementations that use this approach, since most
% impls don't use triangular subst.  Our impl is very similar to the
% math.]

% [re-read conclusions and abstracts of our papers to see if there are
% more contributions, and to better characterize our contributions]

% [what do these contrbutions add up to?]

% My thesis is that new techniques make it increasingly feasible and useful to write
% programs as relations. [which contributions establish the claims in my thesis statement?]

% [secondary thesis?]

% [closing thought?]

% \cite{DBLP:conf/iclp/NearBF08}


\newpage

\appendix

\chapter{Familiar Helpers}\label{helpers}
The auxiliaries below are used in the implementation of \alphakanrensp in Chapter~\ref{akimplchapter}.

\schemedisplayspace
\begin{schemedisplay}
(define get
  (lambda (x s)
    (cond
      ((assq x s) => cdr)
      (else x))))

(define assp
  (lambda (p s)
    (cond
      ((null? s) #f)
      ((p (car (car s))) (car s))
      (else (assp p (cdr s))))))
\end{schemedisplay}

\begin{schemedisplay}
(define filter
  (lambda (p s)
    (cond
      ((null? s) '())
      ((p (car s)) (cons (car s) (filter p (cdr s))))
      (else (filter p (cdr s))))))

(define remove-duplicates
  (lambda (s)
    (cond
      ((null? s) '())
      ((memq (car s) (cdr s)) (remove-duplicates (cdr s)))
      (else (cons (car s) (remove-duplicates (cdr s)))))))
\end{schemedisplay}


\chapter{{\bf pmatch}}\label{pmatch}

In this appendix we describe \scheme|pmatch|, a simple pattern
matcher written by Oleg Kiselyov.
Let us first consider a simple example of \scheme|pmatch|.  

\schemedisplayspace
\begin{schemedisplay}
(define h
  (lambda (x y)
    (pmatch `(,x . ,y)
      (`(,a . ,b) (guard (number? a) (number? b)) (+ a b))
      (`(__ . ,c) (guard (number? c)) (* c c))
      (else (* x x)))))
\end{schemedisplay}

\noindent\scheme{(list (h 1 2) (h 'w 5) (h 6 'w))} $\Rightarrow$ \begin{schemeresponsebox}(3 25 36)\end{schemeresponsebox}

\wspace

\noindent
In this example, a dotted pair is matched against three
different kinds of patterns.

In the first pattern, the value of \scheme{x} is
lexically bound to \scheme{a} and the value of \scheme{y} is lexically bound to
\scheme{b}.  Before the pattern match succeeds, however, an
optional guard is run within the scope of \scheme{a} and \scheme{b}.  
The guard succeeds only if \scheme{x} and \scheme{y} are numbers; 
if so, then the sum of \scheme{x} and \scheme{y} is returned.  

The second pattern matches against a pair, 
provided that the optional guard succeeds.
If so, the value of \mbox{\scheme|y|} is lexically bound
to \scheme{c}, and the square of \scheme{y} is returned.

If the second pattern fails to match against \mbox{\scheme|`(,x . ,y)|},
because \scheme{y} is not a number, then the third and final clause is tried.
An \scheme{else} pattern matches against \emph{any} value, and never includes a guard.
In this case, the clause returns the square of \scheme{x}, which must be
a number in order to avoid an error at runtime.

Below is the definition of \scheme{pmatch}, which is implemented using
continuation-passing-style macros \cite{oai:CiteSeerPSU:392916}.

\newpage

%\schemedisplayspace
\begin{schemedisplay}
(define-syntax pmatch
  (syntax-rules (else guard)
    ((_ (op arg ...) cs ...)
     (let ((v (op arg ...)))
       (pmatch v cs ...)))
    ((_ v) (if #f #f))
    ((_ v (else e0 e ...)) (begin e0 e ...))
    ((_ v (pat (guard g ...) e0 e ...) cs ...)
     (let ((fk (lambda () (pmatch v cs ...))))
       (ppat v pat 
         (if (and g ...) (begin e0 e ...) (fk))
         (fk))))
    ((_ v (pat e0 e ...) cs ...)
     (let ((fk (lambda () (pmatch v cs ...))))
       (ppat v pat (begin e0 e ...) (fk))))))

(define-syntax ppat
  (syntax-rules (__ quote unquote)
    ((_ v __ kt kf) kt)
    ((_ v () kt kf) (if (null? v) kt kf))
    ((_ v quotelit kt kf)
     (if (equal? v quotelit) kt kf))
    ((_ v (unquote var) kt kf) (let ((var v)) kt))
    ((_ v (x . y) kt kf)
     (if (pair? v)
       (let ((vx (car v)) (vy (cdr v)))
         (ppat vx x (ppat vy y kt kf) kf))
       kf))
    ((_ v lit kt kf) (if (equal? v quotelit) kt kf))))
\end{schemedisplay}

The first clause ensures that the expression whose value is to be
\scheme{pmatch}ed against is evaluated only once.  The second clause
returns an unspecified value if no other clause matches.

The remaining clauses represent the three types of patterns supported
by \scheme{pmatch}.  The first is the trivial \scheme{else} clause,
which matches against any datum, and which behaves identically to an
\scheme{else} clause in a \scheme{cond} expression.  The other two
clauses are identical, except that the first one includes a guard
containing one or more expressions---if the datum matches against the
pattern, the guard expressions are evaluated in left-to-right order.
If a guard expression evaluates to \scheme{#f}, then it is as if the
datum had failed to match against the pattern: the remaining guard
expressions are ignored, and the next clause is tried.
The expression \scheme{(fk)} is evaluated if the pattern it is
associated with fails to match, or if the pattern matches but the
guard fails.

\scheme{ppat} does the actual pattern matching over constants and
pairs. The wild-card pattern \scheme{__} matches against \emph{any} value%
\footnote{The \scheme|pmatch| presented in~\cite{alphamk} uses a single underscore ({\tt \_}) as the wild-card pattern. Here we use a double underscore ({\tt \_\_}) for compatibility with \RsixRS.};
the second pattern matches against the empty list;
the third pattern matches against a quoted value; and
the fourth pattern matches against \scheme{any} value, 
and binds that value to a lexical variable with the specified identifier name.  
The fifth pattern matches against a pair, tears it apart, 
and recursively matches the \scheme{car} of the value against
the \scheme{car} of the pattern.  If that succeeds, the \scheme{cdr} 
of the value is recursively matched against the \scheme{cdr} of the pattern.  
(We use \scheme{let} to avoid building long \scheme{car/cdr} chains.)
The last pattern matches against constants, including symbols.

Here is the definition of \scheme{h} after expansion.  
\schemedisplayspace
\begin{schemedisplay}
(define h
  (lambda (x y)
    (let ((v `(,x . ,y)))
      (let ((fk (lambda ()
                  (let ((fk (lambda () (* x x))))
                    (if (pair? v)
                      (let ((vx (car v)) (vy (cdr v)))
                        (let ((c vy))
                          (if (number? c) (* c c) (fk))))
                      (fk))))))
        (if (pair? v)
          (let ((vx (car v)) (vy (cdr v)))
            (let ((a vx))
              (let ((b vy))
                (if (and (number? a) (number? b))
                  (+ a b)
                  (fk)))))
          (fk))))))
\end{schemedisplay}

There are four kinds of improvements that should be resolved by the
compiler.  First, \scheme{vx} is not used in the top definition of
\scheme{fk}, so it should not get a binding. Second, the binding to
\scheme{a} and \scheme{b} should be parallel \scheme{let} bindings.
Third, where \scheme{c} is bound, could have been where \scheme{vy} is
bound, and where \scheme{a} and \scheme{b} are bound, could have been
where \scheme{vx} and \scheme{vy} are bound, respectively.  Fourth,
thunk creation is unnecessary where no guard is present.

The \scheme{mv-let} macro used in Chapter~\ref{akimplchapter} can be defined using \scheme{pmatch}.
\schemedisplayspace
\begin{schemedisplay}
(define-syntax mv-let
  (syntax-rules () 
    ((_ ((x ...) e) b0 b ...) (pmatch e ((,x ...) b0 b ...)))))
\end{schemedisplay}

\noindent\scheme|(mv-let ((x y z) (list 1 2 3)) (+ x y z))| $\Rightarrow$ \begin{schemeresponsebox}6\end{schemeresponsebox}


\chapter{{\bf \matchesymbol} and {\bf \lambdaesymbol}}\label{matche}

In this appendix we describe \scheme|matche| and \scheme|lambdae|,
pattern-matching macros for writing concise miniKanren programs.
These macros were designed by Will Byrd and implemented by Ramana
Kumar with the help of Dan Friedman.

To illustrate the use of \scheme|matche| and \scheme|lambdae|
we will rewrite the explicit definition of \scheme|appendo|, which uses
the core miniKanren operators \scheme|==|, \scheme|conde|, and \scheme|exist|.

\schemedisplayspace
\begin{schemedisplay}
(define appendo
  (lambda (l s out)
    (conde
      ((== '() l) (== s out))
      ((exist (a d res)
         (== `(,a . ,d) l)
         (== `(,a . ,res) out)
         (appendo d s res))))))
\end{schemedisplay}

We can shorten the \scheme|appendo| definition using \scheme|matche|.
\scheme|matche| resembles \scheme|pmatch| (Appendix~\ref{pmatch})
syntactically, but uses unification rather than uni-directional
pattern matching.  \scheme|matche| expands into a \scheme|conde|;
each \scheme|matche| clause becomes a \scheme|conde|
clause\footnote{The \scheme|matcha| and \scheme|matchu| forms are
  identical to \scheme|matche|, except they expand into uses of
  \scheme|conda| and \scheme|condu|, respectively.}.  As with
\scheme|pmatch| the first expression in each clause is an implicitly
quasiquoted pattern.  Unquoted identifiers in a pattern are introduced
as unassociated logic variables whose scope is limited to the pattern
and goals in that clause.

Here is \scheme|appendo| defined with \scheme|matche|.

\schemedisplayspace
\begin{schemedisplay}
(define appendo
  (lambda (l s out)
    (match-e `(,l ,s ,out)
      (`(() ,s ,s))
      (`((,a . ,d) ,s (,a . ,res)) (appendo d s res)))))
\end{schemedisplay}

\noindent The pattern in the first clause attempts to unify the first
argument of \scheme|appendo| with the empty list, while also unifying
\scheme|appendo|'s second and third arguments.  The same unquoted
identifier can appear more than once in a \scheme|matche| pattern;
this is not allowed in \scheme|pmatch|.

We can make \scheme|appendo| even shorter by using \scheme|lambdae|.
\scheme|lambdae| just expands into a \scheme|lambda| wrapped around a
\scheme|matche|---the \scheme|matche| matches against the
\scheme|lambda|'s argument list\footnote{The \scheme|lambdaa| and \scheme|lambdau| forms are
  identical to \scheme|lambdae|, except they expand into uses of
  \scheme|matcha| and \scheme|matchu|, respectively.}.

\schemedisplayspace
\begin{schemedisplay}
(define appendo
  (lambda-e (l s out)
    (`(() ,s ,s))
    (`((,a . ,d) ,s (,a . ,res)) (appendo d s res))))
\end{schemedisplay}

The double-underscore symbol \scheme|__| represents a pattern wildcard
that matches any value without binding it to a variable.  For example,
the pattern in \scheme|pairo|

\schemedisplayspace
\begin{schemedisplay}
(define pairo
  (lambda-e (x)
    (`((__ . __)))))
\end{schemedisplay}

\noindent matches any pair, regardless of the values of its car and
cdr.

\scheme|lambdae| and \scheme|matche| also support nominal logic (see
Chapter~\ref{akchapter}).  Just as unquoted identifiers in a pattern
are introduced as unassociated logic variables, using unquote splicing
in a pattern introduces a fresh nom whose scope is limited to the
pattern and goals in that clause.  For example, the goal constructor

\schemedisplayspace
\begin{schemedisplay}
(define foo
  (lambda (t)
    (fresh (a b)
      (exist (x y)
        (conde
          ((== (tie a (tie b `(,x ,b))) t))
          ((== (tie a (tie b `(,y ,b))) t))
          ((== (tie a (tie b `(,b ,y))) t))
          ((== (tie a (tie b `(,b ,y))) t)))))))
\end{schemedisplay}

\noindent can be re-written as

\schemedisplayspace
\begin{schemedisplay}
(define foo
  (lambda-e (t)
    (`(tie-tag ,@a (tie-tag ,@b (,x ,@b))))
    (`(tie-tag ,@a (tie-tag ,@b (,y ,@b))))
    (`(tie-tag ,@a (tie-tag ,@b (,@b ,y))))
    (`(tie-tag ,@a (tie-tag ,@b (,@b ,y))))))
\end{schemedisplay}

\noindent where \scheme|'tie-tag| is the tag returned by the
\scheme|tie| constructor\footnote{Unfortunately, this explicit pattern
matching breaks the abstraction of the \scheme|tie| constructor.}.

Here is the definition of \scheme|lambdae|,
and its impure variants \scheme|lambdaa| and \scheme|lambdau|.

\schemedisplayspace
\begin{schemedisplay}
(define-syntax lambdae
  (syntax-rules ()
    ((_ (x ...) c c* ...)
     (lambda (x ...) (matche (quasiquote (unquote x) ...) (c c* ...) ())))))

(define-syntax lambdaa
  (syntax-rules ()
    ((_ (x ...) c c* ...)
     (lambda (x ...) (matcha (quasiquote (unquote x) ...) (c c* ...) ())))))

(define-syntax lambdau
  (syntax-rules ()
    ((_ (x ...) c c* ...)
     (lambda (x ...) (matchu (quasiquote (unquote x) ...) (c c* ...) ())))))
\end{schemedisplay}

Here is the definition of \scheme|matche|, and its impure variants \scheme|matcha| and \scheme|matchu|.

\schemedisplayspace
\begin{schemedisplay}
(define-syntax exist*
  (syntax-rules ()
    ((_ (x ...) g0 g ...)
     (lambdag@ (a)
       (inc
         (let* ((x (var 'x)) ...)
           (bind* (g0 a) g ...)))))))

(define-syntax fresh*
  (syntax-rules ()
    ((_ (x ...) g0 g ...)
     (lambdag@ (a)
       (inc
         (let* ((x (nom 'x)) ...)
           (bind* (g0 a) g ...)))))))

(define-syntax matche
  (syntax-rules ()
    ((_ (f x ...) g* . cs)
     (let ((v (f x ...))) (matche v g* . cs)))
    ((_ v g* . cs) (mpat conde v (g* . cs) ()))))

(define-syntax matcha
  (syntax-rules ()
    ((_ (f x ...) g* . cs)
     (let ((v (f x ...))) (matcha v g* . cs)))
    ((_ v g* . cs) (mpat conda v (g* . cs) ()))))
\end{schemedisplay}

\newpage

\begin{schemedisplay}
(define-syntax matchu
  (syntax-rules ()
    ((_ (f x ...) g* . cs)
     (let ((v (f x ...))) (matchu v g* . cs)))
    ((_ v g* . cs) (mpat condu v (g* . cs) ()))))

(define-syntax mpat
  (syntax-rules (__ quote unquote unquote-splicing expand cons)
    ((_ co v () (l ...)) (co l ...))
    ((_ co v (pat) xs as ((g ...) . cs) (l ...))
     (mpat co v cs (l ... ((fresh* as (exist* xs (== `pat v) g ...))))))
    ((_ co v ((__ g0 g ...) . cs) (l ...))
     (mpat co v cs (l ... ((exist () g0 g ...)))))
    ((_ co v (((unquote y) g0 g ...) . cs) (l ...))
     (mpat co v cs (l ... ((exist (y) (== y v) g0 g ...)))))
    ((_ co v (((unquote-splicing b) g0 g ...) . cs) (l ...))
     (mpat co v cs (l ... ((fresh (b) g0 g ...)))))
    ((_ co v ((pat g ...) . cs) ls)
     (mpat co v (pat expand) () () ((g ...) . cs) ls))
    ((_ co v (__ expand . k) (x ...) as cs ls)
     (mpat co v ((unquote y) . k) (y x ...) as cs ls))
    ((_ co v ((unquote y) expand . k) (x ...) as cs ls)
     (mpat co v ((unquote y) . k) (y x ...) as cs ls))
    ((_ co v ((unquote-splicing b) expand . k) xs (a ...) cs ls)
     (mpat co v ((unquote b) . k) xs (b a ...) cs ls))
    ((_ co v ((quote c) expand . k) xs as cs ls)
     (mpat co v (c . k) xs as cs ls))
    ((_ co v ((a . d) expand . k) xs as cs ls)
     (mpat co v (d expand a expand cons . k) xs as cs ls))
    ((_ co v (d a expand cons . k) xs as cs ls)
     (mpat co v (a expand d cons . k) xs as cs ls))
    ((_ co v (a d cons . k) xs as cs ls)
     (mpat co v ((a . d) . k) xs as cs ls))
    ((_ co v (c expand . k) xs as cs ls)
     (mpat co v (c . k) xs as cs ls))))
\end{schemedisplay}


\chapter{Nestable Engines}\label{nestable-engines}

Our implementation of ferns in Chapter~\ref{fernsimpl} requires
nestable engines \cite{RDybvi89,hieb94subcontinuations}, which we
present here with minimal comment.  The implementation uses a global
variable, \scheme|state|, which holds two values: the number of ticks
available to the currently running engine or \scheme|#f| representing
infinity; and a continuation. \scheme|make-engine| makes an engine out
of a thunk.  \scheme|engine| is a macro that makes an engine from an
expression.  \scheme|timed-lambda| is like \scheme|lambda| except that
it passes its body as a thunk to \scheme|expend-tick-to-call|, which
ensures a tick is spent before the body is evaluated and passes the
suspended body to the continuation if no ticks are available. Programs
that use this embedding of nestable engines (and by extension our
embedding of \scheme|frons|) should not use \scheme|call/cc|, because
the uses of \scheme|call/cc| in the nestable engines implementation
may interact with other uses in ways that are difficult for the
programmer to predict.

%\newpage
%\enlargethispage{30pt}
\schemedisplayspace
\schemeinput{fernscode/coaxappendix.ss}

\schemeinput{fernscode/appendix-extra.ss}


\chapter{Parser for Nominal Type Inferencer}\label{akinferparser}

This parser is used by the nominal type inferencer is
section~\ref{aktypeinf}.

\schemedisplayspace
\begin{schemedisplay}
(define parse (lambda (exp) (parse-aux exp '())))

(define parse-aux
  (lambda (exp env)
    (pmatch exp
      (`,x (guard (symbol? x))
       (let ((v (cdr (assq x env))))
         `(vartag ,v)))
      (`,n (guard (number? n)) `(intc ,n))
      (`,b (guard (boolean? b)) `(boolc ,b))
      (`(zero? ,e) (let ((e (parse-aux e env))) `(zero? ,e)))
      (`(sub1 ,e) (let ((e (parse-aux e env))) `(sub1 ,e)))
      (`(fix ,e) (let ((e (parse-aux e env))) `(fix ,e)))
      (`(* ,e1 ,e2) (let ((e1 (parse-aux e1 env)) (e2 (parse-aux e2 env))) `(* ,e1 ,e2)))
      (`(if ,e1 ,e2 ,e3)
       (let ((e1 (parse-aux e1 env)) (e2 (parse-aux e2 env)) (e3 (parse-aux e3 env)))
         `(if ,e1 ,e2 ,e3)))
      (`(lambda (,x) ,e)
       (let* ((a (nom x)) (e (tie a (parse-aux e (cons (cons x a) env)))))
         `(lam ,e)))
      (`(,e1 ,e2)
       (let ((e1 (parse-aux e1 env)) (e2 (parse-aux e2 env)))
         `(app ,e1 ,e2))))))
\end{schemedisplay}


\backmatter

% \phantomsection
\addcontentsline{toc}{chapter}{\bibname}

\bibliography{thesis}

\end{schemeregion}

%\begin{tabular}{l l}
\hspace{-0.92cm}
\begin{minipage}{2.5in}
\thispagestyle{empty}
William E. Byrd

Dept. of Computer Science

Lindley Hall 215

Indiana University

Bloomington, IN 47405

{\tt webyrd@cs.indiana.edu}

(812) 855-4885

\end{minipage}
\hspace{2cm}
\begin{minipage}{2.5in}
\vspace{-1.35cm}
home:

3488 E. Covenanter Dr.

Bloomington, IN 47401

(812) 320-8505
\end{minipage}
\end{tabular}

\bigskip

\noindent{\bf Degrees}

\medskip

\noindent \ \ B.S. in Computer Science, 1999, University of Maryland Baltimore County,  

\noindent \hspace{1cm} Baltimore, Maryland, {\it cum laude}

\medskip

\noindent \ \ B.S. in Special Education, 1994, College of Charleston, Charleston, 

\noindent \hspace{1cm} South Carolina, {\it magna cum laude}

\bigskip

\noindent{\bf Current Position}

\medskip

\noindent \ \ Assistant Instructor under the direction of Dan Friedman.

\bigskip

\noindent{\bf Honors}

\medskip

\noindent \ \ Benefitfocus.com ``Medal of Honor'', Benefitfocus.com, 2001.

\medskip

\noindent \ \ AppNet Excellence Award, AppNet, 2000.

\medskip

\noindent \ \ Outstanding Senior in Computer Science and Electrical Engineering, 

\noindent \hspace{1cm} University of Maryland, Baltimore County, 1999.

\bigskip

\noindent{\bf Books}

\medskip

\noindent\ \ Friedman, D. P., Byrd, W. E., and Kiselyov, O. {\it The Reasoned Schemer},

\noindent \hspace{1cm} The MIT Press, 2005.

% \bigskip

% \noindent{\bf Publications}

\bigskip

\noindent{\bf Conferences}

\medskip

\noindent\ \ Near, J., Byrd, W. E., and Friedman, D. P. ``\alphatap: A Declarative 

\noindent \hspace{1cm} Theorem Prover for First-Order Classical Logic'', In Proceedings of the

\noindent \hspace{1cm}  24th International Conference on Logic Programming, volume 5366 of

\noindent \hspace{1cm}  Lecture Notes in Computer Science, pp. 238--252, 2008.

\medskip

\noindent\ \ Kiselyov, O., Byrd, W. E., Friedman, D. P., and Shan, C. ``Pure, Declarative, 

\noindent \hspace{1cm} and Constructive Arithmetic Relations (Declarative Pearl)'', In Proceedings 

\noindent \hspace{1cm}  of the 9th International Symposium on Functional and Logic Programming, 

\noindent \hspace{1cm}  volume 4989 of Lecture Notes in Computer Science, pp. 64--80, 2008.

\bigskip

\noindent{\bf Workshops}

\medskip

\noindent\ \ Byrd, W. E. and Friedman, D. P. ``\alphakanren: A Fresh Name in Nominal

\noindent \hspace{1cm} Logic Programming'', In Proceedings of the 2007 Workshop on Scheme 

\noindent \hspace{1cm} and Functional Programming, Universit\'{e} Laval Technical Report 

\noindent \hspace{1cm} DIUL-RT-0701, pp. 79--90, 2007.

\medskip

\noindent\ \ Byrd, W. E. and Friedman, D. P. ``From Variadic Functions to Variadic

\noindent \hspace{1cm} Relations'', In Proceedings of the 2006 Scheme and Functional

\noindent \hspace{1cm} Programming Workshop, University of Chicago Technical Report 

\noindent \hspace{1cm} TR-2006-06, pp. 105--117, 2006.


%\bigskip

%\bigskip

%*** home info should be right flush, at same height as work info ***

%*** Are right-hand margins correct?  They don't look wide enough ***

%*** Remember to fill out publications section if ICLP or POPL papers are accepted ***


\end{document}
