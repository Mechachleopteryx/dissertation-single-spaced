\title{Relational Programming: Techniques, Applications, and Obstacles}

[Dan thinks 'Limitations' may be better than obstacles]

\chapter{Dedication} For my parents.

\chapter{Acknowledgements}

start by thanking committee members individually; Olin uses a bulleted
list, one bullet per member

thank Olin for writing an exemplary dissertation whose structure I
shamelessly ripped off

end by thanking family

\chapter{Abstract}
[check abstracts of my papers]

describe disparity between promise and reality of logic programming

brief explanation of and motivation for relational programming

describe the work this dissertation presents

\chapter{Preface}
[re-read TRS preface]

Describe history of our research program on relational programming,
including the evolution of miniKanren, the development and adoption of
new idioms and techniques, interesting applications of relational
programming, and the evolution of our philosophy and outlook.

\chapter{Introduction}

  \section{My Thesis}

[thesis sentence is first sentence of diss., and forms a complete paragraph:]

  New techniques make it increasingly feasible and useful to write
  programs as relations.

  several paragraphs describing problem

  one paragraph for each part of my thesis, explaining how I support
  that assertion in my dissertation
  
  \section{Structure of this Dissertation}

This dissertation has the following structure:

[bullet list roadmap with chunking of chapters, illustrating how the
chapters fit together]

  \section{Logic Programming and Relations}

    \subsection{Logic Programming}

    [Dan says I should do something like this: for convenience, in the
    sequel logic programming is just referred to as ``programming'']

    \subsection{Relations}

    \subsection{The Problem}

    perhaps give a simple example or two showing difficulty of writing
    relations: perhaps rembero, demonstrating the need for constraints
    [or something based on lambda calculus, demonstrating need for
    nominal logic], and an example showing divergence, demonstrating
    the need for bounds or constraints

\chapter{Introduction to Core miniKanren}

\chapter{Basic Techniques}

\chapter{Simple Examples}

\chapter{Implementation I:  Core miniKanren}

\chapter{Advanced Techniques I:  Deriving Bounds for Divergence Avoidance}

\chapter{Applications I:  Pure Binary Arithmetic}

[can we write a relational quicksort program, like in Apt's paper?
See page 3 of Apt's paper 'Declarative Programming in Prolog']

[also, can we write the trans relation w/out negation?  see page 15 of
Apt's paper]

\chapter{Advanced Techniques II:  Nominal Logic}

\chapter{Applications II:  Relational Type Inferencer}

\chapter{Applications III:  alphaleanTAP}

\chapter{Implementation II:  alphaKanren}

\chapter{Advanced Techniques III:  Advanced Divergence Avoidance}

\chapter{Applications IV:  PLT-Redex-Style Reducer} 
(alternatively, could do interpeter, although this might be too
similar to type inferencer)

\chapter{Implementation III:  Tabling}

[can we write a PLT-Redex style reducer or a small-step interpreter
that runs to completion?]

\chapter{Obstacles}

\chapter{Future Work}
[look at future work listed in each paper]

parallel/multicore

formal semantics of mk

identify current limitations of current relational techniques, and
develop new relational constructs and techniques to overcome these
limitations

identify fundamental limitations of relational programming--what can't
we do, ever?

determine how to better combine relational features in a single
implementation (the mkit approach)

tools and techniques for automatic derivation and/or ``purification''
transformations

additional constraints/CLP/CHR

prove correctness of walk agorithm and rest of core implementation

\chapter{Related Work}
[look at related work sections of each paper and of TRS]

[One of Siskind's PhD students said that there is a researcher working
on making functions ``run backwards''.  Dan thinks his name is
C. Hennie.  Dan doesn't think his work is directly relevant, but I
need to check anyway.  Maybe he has some useful techniques.]

Apt's papers: 'Declarative Programming in Prolog' and 'Declarative
Interpretation Reconsidered'

Work on design patterns for logic programs.

Mercury, Curry

Dan thinks I need to discuss modes, perhaps in the context of Mercury

BinProlog (for unnesting/CPS-style intermediate language)

Combinators for Logic Programming.  Work of Hanus.

Schelog/Prolog on a Page.  Other implementations of logic languages in
Scheme or functional languages

Kanren/Sokouza Kanren

alphaProlog, MLSOS, CaML, nominal logic/unification/Urban, Pitts and
Gabbay

Competing relational arithmetic system from German researchers

ICFP paper by Amr, Dan, Ken, and Oleg on backtracking monads

HOAS as an alternative to nominal logic

CLP

\chapter{Conclusion} 
[read the conclusions (and abstracts) of papers for more
contributions, and for more details on the contributions listed below]

Our research makes the following contributions:

--alternative wording--

Contributions of our research:

* a novel relational binary arithmetic system, which does not use
constraints (other than unification), and has stong termination
guarantees

* the first implementation of nominal unification using triangular
substitutions, greatly speeding up our implementation of nominal logic
programming

* an elegant, continuation-based implementation of tabling,
demonstrating the advantage of embedding miniKanren in a language with
higher-order functions

* A novel, \scheme{walk}-based algorithm for variable lookup in
triangular substitutions, amenable to a variety of optimizations,
allowing sharing of substitutions, and making backing ``free''

* various implementations of core miniKanren and its variants,
utilizing the full power of Scheme, and which are concise and easily
extensible

* a novel relational theorem prover that acts as a proof generator,
theorem generator, etc.

* descriptions of core miniKanren and its extensions

* a variety of examples demonstrating the power of relational
programming, including some non-trivial examples

* a collection of idioms, techniques, and langauge constructs that can
be used for relational programming, including examples of their use, a
discussion of each technique and when it should or should not be used

* a clear, if controversial, philosophical framework for the
practicing relational programmer

* a novel approach to expression-level divergence avoidance using
ferns, including the first shallow embedding of ferns

\chapter{Appendix A: Catalog of Relational Techniques}

\chapter{CV}


--------------------
Try approach of super outline, mapped to the paragraph level, complete
with topic sentences
--------------------

chapter-level content from previous dissertation attempt:

miniKanren
miniKanren implementation
pure binary arithmetic
Scheme->miniKanren translation
alphaKanren
alphaleanTAP
tabling
ferns

content and organization of dissertation must support thesis

need to motivate the desire to write programs as relations

need to talk about the obstacles to relational programming

examples from TRS
palo example
arithmetic system (based on hardware metaphors)
eigen
rembero and disequality constraints
tabling
nominal logic
Ramana's implementation of alphaKanren with triangular substitutions
Aziz's trie representation of substitution
walk and related optimizations
our philosophy and point of view (perhaps Alan Kay quote)
HOAS vs. nominal logic

Quotes for each chapter.  See Olin's diss for proper typesetting.
intermezzos discussing interesting examples or techniques?
Artwork:  Duane Bibby?  XKCD?  Scott McCloud, Mad Magazine illustrator
Full color?
Graphic design?
Flow chart for determining which techniques to use?
For each technique, may wish to list all the known relevant tools
Concept map
Illustration, a la Dragon Book, showing how all the techniques and
concepts are related
Green cover

Dan says I should imagine I'm giving a 6 hour lecture on relational
programming.  Perhaps I should actually create and give these
lectures.

May need to tighten up my thesis, to concentrate on our contributions.
Too many researchers have investigated declarative techniques and
approaches.  Perhaps title should be something like: New Techniques
for Declarative Logic Programming.
