\chapter{Techniques I:  Disequality Constraints}\label{diseqchapter}

In this chapter we naively translate a Scheme program to miniKanren,
and observe that the miniKanren relation exhibits undesirable
behavior.  This behavior is due to our inability to express negation
in core miniKanren.  We improve our miniKanren relation through the use of
disequality constraints, which can express a limited form of negation.

This chapter is organized as follows.  In
section~\ref{translaterember} we translate the Scheme function 
\mbox{\scheme|rember|} into the miniKanren relation \mbox{\scheme|rembero|}.
In section~\ref{remberotrouble} we observe that
\mbox{\scheme|rembero|} produces unexpected answers that do not
correspond to answers produced by \mbox{\scheme|rember|}.  In
section~\ref{reconsiderrember} we show that the unexpected answers are
due to our failure to translate implicit tests in the
\mbox{\scheme|rember|} function.  Section~\ref{diseqsection}
introduces disequality constraints, which allow us to express a
limited form of negation.  In section~\ref{fixingrembero} we fix our
definition of \mbox{\scheme|rembero|} by adding a disequality
constraint, thereby eliminating the unexpected answers.  Finally in
section~\ref{diseqlimits} we point out several disadvantages of
disequality constraints, and discuss when these constraints should be
used.

\section{Translating {\rembersymbol} into miniKanren}\label{translaterember}

We begin by naively translating the \scheme|rember| function into
miniKanren.  \mbox{\scheme|rember|} takes two arguments: a symbol \scheme|x| and a
list of symbols \scheme|ls|, and removes the first occurrence of
\scheme|x| from \scheme|ls|.

\wspace

\noindent\mbox{\scheme|(rember 'b '(a b c b d)) => |}\mbox{\schemeresult|`(a c b d)|}

\wspace

\noindent\mbox{\scheme|(rember 'd '(a b c)) => |}\mbox{\schemeresult|`(a b c)|}

%\wspace

\newpage

Here is \mbox{\scheme|rember|}

\schemedisplayspace
\begin{schemedisplay}
(define rember
  (lambda (x ls)
    (cond
      ((null? ls) '())
      ((eq? (car ls) x) (cdr ls))
      (else (cons (car ls) (rember x (cdr ls)))))))
\end{schemedisplay}

To translate \mbox{\scheme|rember|} into the miniKanren relation
\mbox{\scheme|rembero|} we add a third argument \mbox{\scheme|out|},
change \mbox{\scheme|cond|} to \mbox{\scheme|conde|}, and replace uses
of \mbox{\scheme|null?|}, \mbox{\scheme|eq?|}, \mbox{\scheme|cons|},
\mbox{\scheme|car|}, and \mbox{\scheme|cdr|} with calls to
\mbox{\scheme|==|}.  We also unnest the recursive call, using a
temporary variable \mbox{\scheme|res|} to hold the ``output'' value of
the recursive call.

\schemedisplayspace
\begin{schemedisplay}
(define rembero
  (lambda (x ls out)
    (conde
      ((== '() ls) (== '() out))
      ((exist (a d)
         (== `(,a . ,d) ls)
         (== a x)
         (== d out)))
      ((exist (a d res)
         (== `(,a . ,d) ls)
         (== `(,a . ,res) out)
         (rembero x d res))))))
\end{schemedisplay}

\section{The Trouble with {\remberosymbol}}\label{remberotrouble}

For simple tests, it may seem that \mbox{\scheme|rembero|} works as
expected, mimicking the behavior of \mbox{\scheme|rember|}.

\wspace

\noindent\mbox{\scheme|(run1 (q) (rembero 'b '(a b c b d) q)) => |} \mbox{\schemeresult|`((a c b d))|}

\wspace

\noindent\mbox{\scheme|(run1 (q) (rembero 'd '(a b c) q)) => |} \mbox{\schemeresult|`((a b c))|}

\wspace

\noindent However, we notice a problem if we replace the \mbox{\scheme|run1|}
with \mbox{\scheme|run*|}.

\wspace

\noindent\mbox{\scheme|(run* (q) (rembero 'b '(a b c b d) q)) => |} \mbox{\schemeresult|`((a c b d) (a b c d) (a b c b d))|}

\wspace

\noindent Now there are multiple answers.  The first answer is expected, but in
the second answer \mbox{\scheme|rembero|} removes the second
occurrence of \mbox{\scheme|`b|} rather than the first occurrence.  The
last answer is even worse---\mbox{\scheme|rembero|} does not remove
either \mbox{\scheme|`b|}, as is evidenced by the \mbox{\scheme|run|}
expression

\wspace

\noindent\mbox{\scheme|(run* (q) (rembero 'b '(b) '(b))) => |} \mbox{\schemeresult|`(_.0)|}

\section{Reconsidering {\rembersymbol}}\label{reconsiderrember}

Where did we go wrong?  Is our miniKanren translation not faithful to
the original Scheme program?

Not quite. The problem is that \mbox{\scheme|cond|} tries its clauses
in order, stopping at the first clause whose test evaluates to a true
value, while \mbox{\scheme|conde|} tries every possible clause.  But
isn't there only one \mbox{\scheme|cond|} clause that matches any
given values of \mbox{\scheme|x|} and \mbox{\scheme|ls|}?  Actually,
no.

Let us examine the definition of \mbox{\scheme|rember|} once again.

\schemedisplayspace
\begin{schemedisplay}
(define rember
  (lambda (x ls)
    (cond
      ((null? ls) '())
      ((eq? (car ls) x) (cdr ls))
      (else (cons (car ls) (rember x (cdr ls)))))))
\end{schemedisplay}

\noindent Consider the call \mbox{\scheme|(rember 'a '(a b c))|}.
Clearly the \mbox{\scheme|null?|} test keeps the first clause from
returning an answer, while the \mbox{\scheme|eq?|} test allows the
second clause to produce an answer.  But the test of the final clause,
the ``always-true'' \mbox{\scheme|else|} keyword, 
is equivalent to the trivial \mbox{\scheme|#t|} test.

\schemedisplayspace
\begin{schemedisplay}
(define rember
  (lambda (x ls)
    (cond
      ((null? ls) '())
      ((eq? (car ls) x) (cdr ls))
      (#t (cons (car ls) (rember x (cdr ls)))))))
\end{schemedisplay}

\noindent If it were not for the second clause, the third clause would produce
an answer for the call \mbox{\scheme|(rember 'a '(a b c))|}.  In fact,
if we swap the last two clauses

\schemedisplayspace
\begin{schemedisplay}
(define rember
  (lambda (x ls)
    (cond
      ((null? ls) '())
      (#t (cons (car ls) (rember x (cdr ls))))
      ((eq? (car ls) x) (cdr ls)))))
\end{schemedisplay}

\noindent the call \mbox{\scheme|(rember 'a '(a b c))|} returns
\mbox{\scheme|`(a b c)|} rather than \mbox{\scheme|`(b c)|}.

What does the \mbox{\scheme|else|} test really mean in the original
definition of \scheme|rember|?  It means that the tests in all the above
clauses must evaluate to \mbox{\scheme|#f|}.  Similar reasoning holds
for the \mbox{\scheme|eq?|} test of the second clause---the test
implies that the \mbox{\scheme|null?|} test in the first clause
returned \mbox{\scheme|#f|}.  We can therefore redefine
\mbox{\scheme|rember|} to make the implicit tests explicit.

\newpage

%\schemedisplayspace
\begin{schemedisplay}
(define rember
  (lambda (x ls)
    (cond
      ((null? ls) '())
      ((and (not (null? ls)) (eq? (car ls) x))
       (cdr ls))
      ((and (not (null? ls)) (not (eq? (car ls) x)))
       (cons (car ls) (rember x (cdr ls)))))))
\end{schemedisplay}

\mbox{\scheme|rember|} now produces the same answers no matter how we
reorder the clauses; the clauses are now \emph{non-overlapping},
since only a single clause can produce an answer for any specific call
to \mbox{\scheme|rember|}\footnote{Throughout this dissertation we strive to write programs that adhere to the \emph{non-overlapping principle}, to avoid duplicate or misleading answers.  Such programs are similar to the guarded command programs described in \citet{guardedcommands,disciplineprog}.}.

\schemedisplayspace
\begin{schemedisplay}
(define rember
  (lambda (x ls)
    (cond
      ((and (not (null? ls)) (not (eq? (car ls) x)))
       (cons (car ls) (rember x (cdr ls))))
      ((and (not (null? ls)) (eq? (car ls) x))
       (cdr ls))
      ((null? ls) '()))))
\end{schemedisplay}

\noindent Even though we have reordered the \scheme|cond| clauses,
\scheme|rember| works as expected.

\wspace

\noindent\mbox{\scheme|(rember 'a '(a b c)) => |}\mbox{\schemeresult|`(b c)|}

\section{Disequality Constraints}\label{diseqsection}

Now we can reconsider our definition of \mbox{\scheme|rembero|},
adding the equivalent of the explicit tests to make our
\mbox{\scheme|conde|} clauses non-overlapping\footnote{More than one
  \mbox{\scheme|conde|} clause may succeed if \mbox{\scheme|rembero|}
  is passed fresh variables.  However, only one clause will succeed if
  the first two arguments to \mbox{\scheme|rembero|} are fully
  ground.}.

Unfortunately, we do not have a way to express negation in core
miniKanren\footnote{The impure operators \scheme|conda| and
  \scheme|condu| from section~\ref{impureoperatorssection} can be used to express
  ``negation as failure'', as is commonly done in Prolog programs, but
  we eschew this non-declarative approach.}.  However, we do not need
full negation to express the test \mbox{\scheme|(not (null? ls))|},
since if \mbox{\scheme|ls|} is not null it must be a
pair\footnote{This assumes, of course, that the second argument to
  \mbox{\scheme|rembero|} can be unified with a proper list.  Passing
  in \mbox{\scheme|5|} as the \mbox{\scheme|ls|} argument makes no
  more sense for \mbox{\scheme|rembero|} than it does for
  \mbox{\scheme|rember|}.}.  In fact, we are already expressing the
\mbox{\scheme|(not (null? ls))|} test implicitly, through the
unification \mbox{\scheme|(== `(,a . ,d) ls)|} that appears in the
last two \mbox{\scheme|conde|} clauses.

The only remaining test is \mbox{\scheme|(not (eq? (car ls) x))|} in
the last clause.  How might we express that the car of
\mbox{\scheme|ls|} is not \mbox{\scheme|x|}?  We could attempt to
unify the car of \mbox{\scheme|ls|} with every symbol other than
\mbox{\scheme|x|}.  Even if \mbox{\scheme|x|} were instantiated, to the
symbol \mbox{\scheme|`a|} for example, we would have to unify
\mbox{\scheme|x|} with every symbol \emph{other} than
\mbox{\scheme|`a|}, of which there are infinitely many.  Clearly this
is problematic: enumerating an infinite domain can easily lead to
divergent behavior\footnote{It is possible to enumerate some infinite
  domains using a finite number of cases, through the use of clever
  data representation.  For example, using the binary list notation
  from Chapter~\ref{arithchapter} we can express that a natural number
  \mbox{\scheme|x|} is not \mbox{\scheme|5|} by unifying
  \mbox{\scheme|x|} with the patterns \mbox{\scheme|`()|},
  \mbox{\scheme|`(1)|}, \mbox{\scheme|`(,a 1)|}, \mbox{\scheme|`(0 ,a 1)|}, 
  \mbox{\scheme|`(1 1 1)|}, and \mbox{\scheme|`(,a ,b ,c ,d . ,rest)|}.  Although this
  approach avoids divergence, it requires us to know the domain and
  representation of \mbox{\scheme|x|}.  Furthermore, this approach may result in
  duplicate answers even for programs that adhere to the
  non-overlapping principle, which can be a problem even when
  enumerating finite domains.}.

Compare the tests in the second and third \mbox{\scheme|rember|} clauses: \mbox{\scheme|(eq? (car ls) x)|} and \mbox{\scheme|(not (eq? (car ls) x))|}.  We use \mbox{\scheme|(== a x)|} to express that
the car of \mbox{\scheme|ls|} (which is \mbox{\scheme|a|}) is equal to \mbox{\scheme|x|}.  What we need is the ability
to express the \emph{disequality constraint}\footnote{As opposed to an \emph{equality constraint}, such as \mbox{\scheme|(== a x)|}.  Disequality is also known as \emph{disunification}.} \mbox{\scheme|(=/= a x)|}\footnote{We may also wish to introduce an operator \mbox{\scheme|=/=-no-check|} that performs \emph{unsound} disunification, to avoid the cost of the occurs check.}, which
asserts that \mbox{\scheme|a|} and \mbox{\scheme|x|} are not equal, and can never be made equal
through unification.

Before we add a disequality constraint to \mbox{\scheme|rembero|}, let us examine some
simple uses of \mbox{\scheme|=/=|}.
In the first example, we unify \mbox{\scheme|q|} with
\mbox{\scheme|5|}, then specify that \mbox{\scheme|q|} can never be
\mbox{\scheme|5|}.  As expected, the call to \mbox{\scheme|=/=|}
fails.

\wspace

\noindent\mbox{\scheme|(run* (q) (== 5 q) (=/= 5 q)) => |} \mbox{\schemeresult|`()|}

\wspace

If we swap the goals, the program behaves the same.

\wspace

\noindent\mbox{\scheme|(run* (q) (=/= 5 q) (== 5 q)) => |} \mbox{\schemeresult|`()|}

\wspace

\mbox{\scheme|=/=|} can take arbitrary expressions, as shown in the next two examples.

\wspace

\noindent\mbox{\scheme|(run* (q) (=/= (+ 2 3) 5)) => |} \mbox{\schemeresult|`()|}

\wspace

\noindent\mbox{\scheme|(run* (q) (=/= (* 2 3) 5)) => |} \mbox{\schemeresult|`(_.0)|}

\wspace

In this \mbox{\scheme|run*|} expression we assert that
\mbox{\scheme|q|} can never be \mbox{\scheme|5|} or \mbox{\scheme|6|}.
We express the latter constraint indirectly, by constraining
\mbox{\scheme|x|}.

\schemedisplayspace
\begin{schemedisplay}
(run* (q)
  (exist (x)
    (=/= 5 q)
    (== x q)
    (=/= 6 x))) =>
\end{schemedisplay}
\nspace
\begin{schemeresponse}
`((_.0 : (never-equal ((_.0 . 5)) ((_.0 . 6)))))
\end{schemeresponse}

\noindent The answer includes two reified constraints indicating that the output
variable (\mbox{\scheme|q|}) can never be \mbox{\scheme|5|} or \mbox{\scheme|6|}.

\newpage

Consider this \mbox{\scheme|run*|} expression.

\schemedisplayspace
\begin{schemedisplay}
(run* (q)
  (exist (y z)
    (=/= `(,y . ,z) q))) =>
\end{schemedisplay}
\nspace
\begin{schemeresponse}
`(_.0)
\end{schemeresponse}

\noindent It may seem that the constraint on \mbox{\scheme|q|} should
be reified.  However, this constraint can only be violated if
\mbox{\scheme|q|} is unified with \mbox{\scheme|`(,y . ,z)|}.  Since
\mbox{\scheme|y|} and \mbox{\scheme|z|} are not reified, the
constraint is not \emph{relevant} and is therefore not reified.

To reify a constraint, we must reify all of the variables involved in the constraint.

\schemedisplayspace
\begin{schemedisplay}
(run* (q)
  (exist (x y z)
    (=/= `(,y . ,z) x)
    (== `(,x ,y ,z) q))) =>
\end{schemedisplay}
\nspace
\begin{schemeresponse}
`(((_.0 _.1 _.2) : (never-equal ((_.0 _.1 . _.2)))))
\end{schemeresponse}

\noindent The constraint is easier to interpret if we remember that 
\mbox{\scheme|`(never-equal ((_.0 _.1 . _.2)))|} is equivalent to 
\mbox{\scheme|`(never-equal ((_.0 . (_.1 . _.2))))|}.

Here is a slightly more complicated example of \mbox{\scheme|=/=|}.

\schemedisplayspace
\begin{schemedisplay}
(run* (q)
  (exist (x y z)
    (== `(,y . ,z) x)
    (=/= '(5 . 6) x)
    (== 5 y)
    (== `(,x ,y ,z) q))) =>
\end{schemedisplay}
\nspace
\begin{schemeresponse}
`((((5 . _.0) 5 _.0) : (never-equal ((_.0 . 6)))))
\end{schemeresponse}

Here is the same program, but with \mbox{\scheme|(== 6 y)|} instead of \mbox{\scheme|(== 5 y)|}.

\schemedisplayspace
\begin{schemedisplay}
(run* (q)
  (exist (x y z)
    (== `(,y . ,z) x)
    (=/= '(5 . 6) x)
    (== 6 y)
    (== `(,x ,y ,z) q))) =>
\end{schemedisplay}
\nspace
\begin{schemeresponse}
`(((6 . _.0) 6 _.0))
\end{schemeresponse}

\noindent Since \mbox{\scheme|y|} cannot be \mbox{\scheme|5|}, \mbox{\scheme|(=/= '(5 . 6) x)|} cannot be violated and is
therefore discarded.

We end this section with a final example, to demonstrate how to interpret more complicated reified constraints.

\newpage

%\schemedisplayspace
\begin{schemedisplay}
(run* (q)
  (exist (x y z)
    (=/= 5 x)
    (=/= 6 x)
    (=/= `(,y 1) `(2 ,z))
    (== `(,x ,y ,z) q))) =>
\end{schemedisplay}
\nspace
\begin{schemeresponse}
`(((_.0 _.1 _.2) : (never-equal ((_.1 . 2) (_.2 . 1)) ((_.0 . 6)) ((_.0 . 5)))))
\end{schemeresponse}

\noindent The constraints \mbox{\scheme|`((_.0 . 6))|} and
\mbox{\scheme|`((_.0 . 5))|} are independent of each other, and
indicate that \mbox{\scheme|x|} can never be \mbox{\scheme|5|} or
\mbox{\scheme|6|}. However, \mbox{\scheme|`((_.1 . 2) (_.2 . 1))|}
represents a \emph{single} constraint, indicating that
\mbox{\scheme|y|} cannot be \mbox{\scheme|2|} if \mbox{\scheme|z|} is
\mbox{\scheme|1|}\footnote{Reifying constraints in a friendly manner
  is non-trivial, as we will see in Chapter~\ref{diseqimplchapter}.}.

\section{Fixing {\remberosymbol}}\label{fixingrembero}
\enlargethispage{2em}


Now that we understand \mbox{\scheme|=/=|}, and how to interpret reified constraints,
we are ready to add the disequality constraint \mbox{\scheme|(=/= a x)|} to the last
clause of \mbox{\scheme|rembero|}.

\schemedisplayspace
\begin{schemedisplay}
(define rembero
  (lambda (x ls out)
    (conde
      ((== '() ls) (== '() out))
      ((exist (a d)
         (== `(,a . ,d) ls)
         (== a x)
         (== d out)))
      ((exist (a d res)
         (== `(,a . ,d) ls)
         (=/= a x)
         (== `(,a . ,res) out)
         (rembero x d res))))))
\end{schemedisplay}

If we re-run the programs from section~\ref{remberotrouble} we see
that \mbox{\scheme|rembero|}'s behavior is consistent with that of \mbox{\scheme|rember|}.

\wspace

\noindent\mbox{\scheme|(run* (q) (rembero 'b '(a b c b d) q)) => |} \mbox{\schemeresult|`((a c b d))|}

\wspace

\noindent\mbox{\scheme|(run* (q) (rembero 'b '(b) '(b))) => |} \mbox{\schemeresult|`()|}

\wspace

Of course, \mbox{\scheme|rembero|} is more flexible than \mbox{\scheme|rember|}.

\schemedisplayspace
\begin{schemedisplay}
(run* (q)
  (exist (x out)
    (rembero x '(a b c) out)
    (== `(,x ,out) q))) =>
\end{schemedisplay}
\nspace
\begin{schemeresponse}
`((a (b c))
  (b (a c))
  (c (a b))
  ((_.0 (a b c)) : (never-equal ((_.0 . c)) ((_.0 . b)) ((_.0 . a)))))
\end{schemeresponse}

The final answer indicates that removing a symbol \mbox{\scheme|x|}
from the list \mbox{\scheme|`(a b c)|} results in the original list,
provided that \mbox{\scheme|x|} is not \mbox{\scheme|`a|},
\mbox{\scheme|`b|}, or \mbox{\scheme|`c|}.


\section{Limitations of Disequality Constraints}\label{diseqlimits}

Disequality constraints add expressive power to core
miniKanren\footnote{It seems that disequality constraints were present
  in a very early version of Prolog~\cite{birthofprolog}, although
  they were apparently removed after several years.  Prolog
  II~\cite{prologtenfigs} reintroduced disequality constraints, which
  are now standard in most Prolog systems.}, allowing us to express a
limited form of negation.  However, disequality constraints have
several limitations and disadvantages.

First, the \mbox{\scheme|=/=|} operator can only express that two
terms are never the same.  This is much more limited than the ability
to express full negation.  For example, consider the test
\mbox{\scheme|(and (not (null? ls)) (not (eq? (car ls) x)))|} from the
version of \mbox{\scheme|rember|} in section~\ref{reconsiderrember}.
By de Morgan's law, this test is logically equivalent to
\mbox{\scheme|(not (or (null? ls) (eq? (car ls) x)))|}.  We can use
disequality constraints to express the first version of the test, but
not the second.

Answers containing reified disequality constraints can be more
difficult to interpret than answers without constraints.  Also, it is
not always obvious why a constraint was \emph{not} reified (whether it
was not relevant or could not be violated).

Disequality constraints also complicate the implementation of the
unifier, and especially the reifier.  Disequality constraints can also
be expensive, since every constraint must be checked after each
successful unification.

Because of these disadvantages, it is preferable to use
\mbox{\scheme|==|} rather than \mbox{\scheme|=/=|} whenever practical.
For example, it is better to express the test \mbox{\scheme|(not (null? ls))|} 
as \mbox{\scheme|(== `(,a . ,d) ls)|} rather than as
\mbox{\scheme|(=/= '() ls)|}.

Still, disequality constraints add expressive power to core miniKanren, and are
generally preferable to enumerating infinite (or even finite) domains.






% [membero example]

% [rembero  (Scheme -> mk translation)
% duplicate answers

% simplest example of CLP

% implicit negation in cond lines

% Dijkstra guard

% must be able to swap cond clauses in Scheme

% tagging (for application expressions in lambda calculus, for example)]

% [surpriseo]

% [an arithmetic example---some number is not 5]

% can represent infinitely many terms using a single constraint---useful
% in avoiding divergence

% only relevant constraints are printed

% pairs

% constraint simplification

% [look at test programs from =/= test cases]

% [all-different]
