\chapter{Future Work}\label{futureworkchapter}

In this chapter we propose future work related to miniKanren, and to
relational programming in general.

This chapter is organized as follows.  In
section~\ref{futureformalization} we discuss how our work on
miniKanren might be formalized.  Section~\ref{futureimplementation}
presents possible improvements to the existing miniKanren
implementation, while section~\ref{futureextensions} suggests how the
miniKanren language might be extended.  Section~\ref{futureidioms}
considers future work on relational idioms, while
section~\ref{futureapplications} proposes future applications of
miniKanren. Finally, in section~\ref{futuretools} we propose tools
that might ease the burden on relational programmers.

\section{Formalization}\label{futureformalization}

From a formalization standpoint, the most important future work is to
create a formal semantics for miniKanren.  Perhaps the simplest
approach would be to start from the operational semantics of the
nominal logic programming language MLSOS, as described
in~\citet{lakin2007}.  Of course, miniKanren's semantics would become
more complex if the language extensions proposed in
section~\ref{futureextensions} were added.  Indeed, it is the
interaction between different language features (nominal unification
and constraint logic programming, for example) that will make
extending miniKanren challenging.

The core miniKanren implementation presented in
Chapter~\ref{mkimplchapter} uses a stream-based interleaving search
strategy.  The use of \scheme|inc|s (thunks) to force interleaving
makes it difficult to exactly characterize the search behavior, and
therefore the order in which miniKanren produces answers.  It would be
both interesting and useful to mathematically describe this
interleaving behavior (see section~\ref{futureimplementation}).

% Prove properties of \scheme|walk| (recent work by Dave, Andy, and Dan)

In Chapter~\ref{alphatapchapter} we replaced \leantap's use of
Prolog's \mbox{{\tt copy\_term/2}} with a purely declarative combination of
tagging and nominal unification; this technique was key to making
\alphatapsp purely relational.  Unfortunately, this approach can only
be used when the programmer knows the structure of the terms to be
copied.  It would be useful to formalize this technique, to better
understand its applicability and limitations.

%formalize technique in 
%revist unhappy family
%encasulation of search in Curry (cloning argument)

The relational arithmetic system presented in
Chapter~\ref{arithchapter} uses bounds on term sizes to provide strong
termination guarantees for arithmetic relations\footnote{At least, for
  single arithmetic relations whose arguments do not share
  unassociated logic variables.}.  A systematic approach to deriving
such bounds on term sizes would be very helpful for relational
programmers.  Of course, G\"{o}del and Turing showed that it is
impossible to guarantee termination for all goals we might wish to
write, so in general we will not be able to achieve finite failure
through bounds, or any other technique\footnote{For example, the
  strong termination guarantees for our arithmetic system do not hold
  for conjunction of addition and multiplication goals.}.  However,
even when such bounds exist, it may be difficult to express them in
miniKanren.  Indeed, poorly expressed bounds may themselves cause
divergence---for example, by attempting to eagerly determine the
length of an uninstantiated (and therefore unbounded)
list\footnote{See Chapter~\ref{divergencechapter} for more on the
  difficulty of expressing bounds on term sizes.}.  A systematic
approach to expressing bounds already derived by the programmer would
be most useful.

Section~\ref{triangularsection} presents a Scheme implementation of a
nominal unifier that uses triangular substitutions.  This algorithm
should be formalized and proved correct, similar to the presentation
of (idempotent) nominal unification in~\citet{Urban-Pitts-Gabbay/04}.
% [perhaps Joe's would be easier to prove]

\citet{HermanWand08:Hygiene} use nominal logic to describe an
idealized version of Scheme's \scheme|syntax-rules| hygienic macro
system.  It would be interesting to extend this work to the full
\scheme|syntax-rules| system, perhaps by implementing the macro system
as an \alphakanrensp relation.

A more speculative area of future work is the connection between the
various causes of divergence described in
Chapter~\ref{divergencechapter}.  As discussed in the conclusion of
this dissertation, there may be a deep connection between these causes
of divergence, and between the techniques for avoiding them.  Since
divergence is an effect, monads~\cite{moggi91notions} or
arrows~\cite{hughes98generalisingmonads} may provide the best
framework for exploring these ideas.

%monads + mk
%monads/arrows + divergence


\section{Implementation}\label{futureimplementation}

The core miniKanren implementation presented in
Chapter~\ref{mkimplchapter} uses streams to implement backtracking
search\footnote{Although one could argue that the stream-based
  implementation performs backtracking search without actually
  backtracking.}.  As described in \citet{Wand04relatingmodels}, our
use of streams could be modelled using explicit success and failure
continuations.  When extending the miniKanren language, it is
sometimes more convenient to use this two-continuation model of
backtracking---for example, the first implementation of tabling for
miniKanren used continuations rather than streams.

The streams implementation of miniKanren makes liberal use of
\scheme|inc|s (thunks) to force interleaving in the search.
Unfortunately, it is difficult to exactly replicate this interleaving
search behavior in the two-continuation model.  As a result,
continuation-based implementations of miniKanren may produce answers
in a different order than stream-based implementations, which makes it
difficult to test, benchmark, or otherwise compare different
implementations.  It therefore would be extremely convenient to have a
continuation-based implementation of miniKanren that exactly mirrors
the search behavior of the streams-based implementation from
Chapter~\ref{mkimplchapter}.  This may require a formal
characterization of the stream-based search strategy, as discussed in
section~\ref{futureformalization}.

We currently use association lists to represent substitutions; we may
wish to consider other purely functional representations of
substitutions that would make variable lookup less expensive.  For
example, Abdulaziz Ghuloum previously implemented a trie-based
representation of substitutions that performs at least as well as the
fastest walk-based algorithm presented in
Chapter~\ref{walkimpl}. Using a trie-based representation of
substitutions may mean giving up on the clever method of implementing
disequality constraints described in Chapter~\ref{diseqimplchapter}.

Relational programming is inherently parallelizable.  In fact, we have
already implemented two parallel versions of miniKanren: one written
in Scheme and one in Erlang~\cite{armstrong03}. However, neither
parallel implementation runs as quickly as the sequential
implementation of miniKanren presented in Chapter~\ref{mkimplchapter}.
One difficulty in making a parallel implementation run efficiently is
that miniKanren suffers from an ``embarrassment of parallelism''.  For
example, a recursive goal might contain a \scheme|conde| whose first
clause contains a single unification. The overhead of sending this
single unification to a new core or processor may be more expensive
than just performing the unification. Ciao Prolog solves this problem
by performing a ``granularity analysis'' to determine which parts of a
program perform enough computation to offset the overhead of
parallelization~\cite{granularity90,lopez96methodology}.

Our purely functional implementation of miniKanren also implies a
different set of design choices than would be made when parallelizing
a Prolog implementation based on the Warren Abstract Machine. In
particular, our stream-based search implementation, combined with our
functional representation of
substitutions\footnote{\citet{gupta93analysis} have explored the
  tradeoffs of different environment representations in the context of
  parallel logic programming.}, means that disjunction is truly
parallel: failure of one disjunct does not require communication with
other disjuncts.

Reification of nominal terms is another area for future work.  The
core-miniKanren reifier presented in Chapter~\ref{mkimplchapter}
enforces several important invariants: swapping adjacent calls to
\scheme|==|, swapping arguments within a single call to \scheme|==|,
or reordering nested \scheme|exist| clauses\footnote{Assuming this is
  done without inadvertently shadowing variables, or leaving
  previously bound variables unbound.} cannot affect reified answers.
We would like \alphakanrensp to ensure similar invariants; however,
reification in \alphakanrensp is more complicated, since each term
containing a \scheme|tie| now represents an infinite equivalence class
of $\alpha$-equivalent terms. Additionally, we do not have a canonical
representation for permutations associated with suspensions. Finally,
reification must also handle freshness constraints.

miniKanren uses a complete interleaving search strategy, which ensures
disjunction (\scheme|conde|) is commutative---swapping the order of
\scheme|conde| clauses can affect the order in which answers are
returned, but cannot affect whether a goal diverges. In contrast,
miniKanren's conjunction operators (\scheme|exist| and \scheme|fresh|)
are not commutative---swapping conjuncts can cause a goal that
previously failed finitely to now diverge.  It is easy to see that
commutative conjunction can be implemented: just run in parallel every
possible ordering of conjuncts. Unfortunately, this simplistic
approach is far too expensive to be used in practice. However, it may
be possible to more efficiently implement commutative conjunction by
interleaving the evaluation of conjuncts, and allowing each conjunct
to partially extend the substitution. This would allow conjuncts to
communicate with each other by extending the substitution, thereby
allowing the conjunction to ``fail fast'', and avoiding the
duplication of work inherent in the naive approach described above.
It is not clear whether this approach is efficient enough to be used
throughout an entire program; the programmer may need to restrict use
of commutative conjunction to conjunctions containing multiple
recursive goals.

Alternatively, it may be possible to \emph{simulate} commutative
conjunction using a combination of continuations, interleaving search,
and tabling.  This approach would only be a simulation of true
commutative conjunction because tabling is defeated if an argument
changes with each recursive call.

The core miniKanren implementation presented in
Chapter~\ref{mkimplchapter} is an embedding in Scheme, using a
combination of procedures and hygienic macros.  Although this
embedding allows us to easily benefit from the optimizations provided
by a host Scheme implementation, we lose the ability to analyze or
transform entire miniKanren programs. A miniKanren compiler would
allow us to perform more sophisticated program analyses. Finally, a
miniKanren interpreter\footnote{In the long tradition of writing
  meta-circular Scheme interpreters, a meta-circular miniKanren
  interpreter would be especially satisfying.} or abstract machine
would be useful from both an implementation and formalization
standpoint.


\section{Language Extensions}\label{futureextensions}

\alphakanren's support for nominal logic programming could be extended
in several ways.  Perhaps the simplest extension would be to add
MLSOS's name inequality constraint~\cite{lakin2007}, which is
essentially a disequality constraint limited to noms (and to
suspensions that will become noms).  A more ambitious extension would
be to add full disequality constraints to \alphakanren.  One might
also implement equivariant unification~\cite{DBLP:conf/rta/Cheney05},
which extends nominal unification with the ability to include logic
variables in permutations; however, the expense of equivariant
unification~\cite{DBLP:conf/icalp/Cheney04} limits its
appeal\footnote{Although~\citet{DBLP:conf/tlca/UrbanC05} show that it
  is often possible to avoid full equivariant unification in real
  programs.}.  \citet{DowekEtAl} recently presented
a variant of nominal unification using ``permissive'' nominal terms,
which do not require explicit freshness constraints; permissive
nominal terms might simplify reification of \alphakanrensp answers.

Our tabling implementation does not currently work with disequality
constraints or freshness constraints.  It would be very useful to
extend tabling to work with these constraints.  Alternatively, it may
be possible to add tabling to \alphakanrensp by using permissive
nominal terms, which do not require freshness constraints.

\citet{DBLP:conf/iclp/GuptaBMSM07} have implemented a coinductive
logic programming language that can express infinite streams using
coinductive definitions of goals.  The heart of their system is an
implementation of tabling, in which unification rather than
reification is used to determine whether a call is a variant of an
already tabled call.  It should be straightforward to add coinductive
logic programming to miniKanren, since we have already implemented
tabling.  Also, it would be interesting to investigate if other
notions of variant calls make sense---for example, what if we used
subsumption instead of reification or unification? Would we get a
different type of logic programming?  Finally, the streams that can be
created using the system of \citeauthor{DBLP:conf/iclp/GuptaBMSM07}
must have a regular structure---for example, their system cannot
represent a stream of all the prime numbers.  How might more
sophisticated streams be expressed?

One alternative to requiring the occurs check for sound unification is
to allow infinite terms, as in Prolog II. This would require changing
the reifier to print circular terms. We would also want our core
language forms, such as disequality constraints, to handle infinite
terms\footnote{SWI Prolog~\cite{Wielemaker:03b} includes many
  predicates that work on infinite terms, and might serve as an
  inspiration.}.

An extremely useful extension to miniKanren would be the addition of
constraint logic programming, or
CLP~\cite{DBLP:journals/jlp/JaffarM94}\footnote{Actually, miniKanren
  and \alphakanrensp already support several types of constraints:
  unification (\scheme|==|) and dis-unification (\scheme|=/=|)
  constraints, and the freshness constraints of nominal logic.
  However, there are many other types of constraints we might want to
  add.}.  The notation `CLP(X)' refers to constraint logic programming
over some domain `X'; common domains include the integers (CLP(Z)),
rational numbers (CLP(Q)), real numbers (CLP(R)), and finite domains
(CLP(FD)).  Most useful for existing applications of miniKanren would
be CLP(FD) and CLP(Z), which would allow us to declaratively express
arithmetic in a more efficient manner than the arithmetic system of
Chapter~\ref{arithchapter}\footnote{The declarative arithmetic system
  of Chapter~\ref{arithchapter} has several advantages over the
  constraint approach, however.  As opposed to CLP(FD), our system
  works on numbers of arbitrary size.  Our system is also implemented
  entirely at the user-level language, without any constraints other
  than unification, while adding CLP(FD) or CLP(Z) requires
  significant changes to the underlying implementation, and may
  interact in undesirable ways with other language features.}.

miniKanren, like Scheme, is dynamically-typed. \citet{siek06:gradual}
show how \emph{gradual typing} can be used to add a sophisticated type
system to a dynamically typed language, without giving up the
flexibility of dynamic typing\footnote{There has also been recent work
  on adding something like gradual typing to Prolog (see
  ~\cite{towardstypedprolog}, although it is unclear whether these
  researchers are aware of the Scheme community's work on gradual
  typing and soft typing~\cite{softtyping}.}.  It would be interesting
to apply this typing scheme to miniKanren, since supporting logic
variables and constraints may require extending the notions of gradual
typing.

Relational goals often append two lists; if the first list is an
uninstantiated logic variable, this results in infinitely many
answers, which can easily lead to divergence. It may be possible to
create an \scheme|append| constraint that represents the delayed
appending of two lists, and avoids enumerating infinitely many
appended lists.

Another line of future work would be to implement non-standard logics
for relational programming, such as temporal logic, linear logic, and
modal logic.  Of course, supporting any of these logics would require
significant changes to miniKanren, and would require careful
consideration of how various language extensions would interact with
the new logic.

Modern functional logic programming languages like Curry are based on
narrowing~\cite{AntoyHanusMasseySteiner01PPDP}, which combines term
rewriting with the ability to instantiate logic variables. It would be
interesting to implement a language based on \emph{nominal}
narrowing---that is, narrowing based on nominal
rewriting~\cite{nominalrewriting}. This would allow a single
implementation to express nominal functional programming (as in
FreshML~\cite{ShinwellPG03} or C$\alpha$ml~\cite{pottier06}), nominal
logic programming (as in \alphaprolog~\cite{CheneyU04},
MLSOS~\cite{lakin2007}, or \alphakanren), hygienic macros (as in
Scheme\footnote{\citet{HermanWand08:Hygiene} describe a simplified
  version of Scheme's \scheme|syntax-rules| macro system using nominal
  logic.}), and nominal term rewriting (as in Maude~\cite{Maude2:03},
Stratego~\cite{stratego}, or PLT
Redex~\cite{DBLP:conf/rta/MatthewsFFF04}, but with the addition of
nominal logic).

Like MLSOS and \alphaprolog, \alphakanrensp is well suited for
expressing the rules and side-conditions of Structural Operational
Semantics (SOS)~\cite{Plotkin:2004:SAO}.  It would be informative to
explore which SOS rules or side-conditions \emph{cannot} be easily
expressed in \alphakanren; such an exercise would likely result in new
constraints and other language extensions.  Similarly, it would be
informative to investigate which Scheme, Prolog, and Curry programs we
cannot satisfactorily express in a purely relational manner.

Perhaps the greatest challenge in extending miniKanren is to combine
all of these language features in a meaningful way. Ciao Prolog
attempts to control interactions between language features through a
module system~\cite{DBLP:journals/tcs/GrasH99}. The addition of
libraries to the \RsixRSsp Scheme standard~\cite{r6rs} should allow us
to do the same. However, a more sophisticated approach based on monads
and monad transformers may better control the interaction between
language features.



\section{Idioms}\label{futureidioms}

\citet{Okasaki:1999} has investigated the use of purely functional
data structures, many of which are comparable in efficiency to
imperative data structures\footnote{Indeed, uses of purely functional
  data structures can be even more efficient than uses of imperative
  data structures, due to sharing.}.  Even more so than in functional
programming, data representation is essential to relational
programming.  Therefore, it would be interesting and useful to
investigate the use of \emph{purely relational} data structures---that
is, data structures and data representations that are especially
well-suited for relational programming.  Some of these data structures
might take advantage of relational language features such as nominal
unification or constraints.

Also, as mentioned in section~\ref{futureformalization}, it would be
useful to formalize our combination of tagging and nominal unification
to emulate Prolog's \mbox{{\tt copy\_term/2}} in a purely declarative manner.



\section{Applications}\label{futureapplications}

It should be relatively easy to extend the arithmetic system of
Chapter~\ref{arithchapter} to handle rational numbers.  Probably the
most difficult part of this exercise would be maintaining fractions in
simplified form.

An interesting extension to the type inferencer in
section~\ref{aktypeinf} would be to support
polymorphic-\scheme|let|~\cite{tapl}.  At a minimum, this would
require a declarative way to perform a combination of substitution and
term copying.  Of course, the implementation of \alphatapsp in
Chapter~\ref{alphatapchapter} also uses these techniques.  However,
there may be enough differences between \alphatapsp and the type
inferencer to make applying these techniques difficult or impossible.
If so, a new type of constraint may be called for.

As described in Chapter~\ref{alphatapchapter}, the \alphatapsp theorem
prover allows a user to guide the proof search by partially
instantiating the prover's proof-tree argument.  It should be possible
to extend \alphatap, making it act as a rudimentary interactive proof
assistant.  This would further demonstrate the flexibility of
relational programming; more importantly, creating such a tool might
require new techniques that would be useful for writing relational
programs in general.



\section{Tools}\label{futuretools}

As mentioned in section~\ref{futureformalization}, integrating bounds
on term size into an existing relation can be difficult.  A tool that
could take a relation, along with a specification of bounds on the
argument sizes, and synthesize a new relation that incorporates those
bounds would be extremely helpful.

A tool to automatically translate Scheme programs to miniKanren would
also be handy.  Ideally, this tool would generate purely relational
miniKanren code adhering to the non-overlapping principle (see
section~\ref{reconsiderrember}).  This may be possible, at least for
many simple Scheme functions, if the programmer were to help the tool
by specifying how to represent terms, along with an appropriate
tagging scheme.  However, deriving miniKanren relations from Scheme
functions is not the real difficultly---rather, ensuring finite
failure for a wide variety of arguments is what makes relational
programming so difficult.

A Prolog-to-Scheme translator would also be useful.  Translating pure
Prolog programs into miniKanren should be very easy, especially since
the \scheme|lambda-e| pattern-matching macro is similar to Prolog's
pattern matching syntax.


% Design patterns for relational programming

%Contracts and relational programming

% Proof techiques/recursion
