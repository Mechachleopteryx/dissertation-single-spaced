\chapter{Introduction}\label{introchapter}

\section{My Thesis}
\begin{quotation}
\begin{flushright}
\textit{A beginning is a very delicate time.}\\
---Princess Irulan
\end{flushright}
\end{quotation}


\noindent miniKanren supports a variety of relational idioms and
techniques, making it feasible and useful to write interesting
programs as relations.

The promise of logic programming is that programs can be written {\em
  relationally}, without distinguishing between input and output
arguments.  Each relation produces meaningful answers, even when all
of its arguments are unbound logic variables.  Relational programs are
remarkably flexible---for example, a relational type inferencer can
also perform type checking and type inhabitation.  Similarly, a
relational theorem prover can also be used as a proof checker, proof
generator, theorem generator, and even as a primitive proof assistant.

% Unfortunately, writing remarkably flexible relational programs is
% remarkably difficult.  Relational programming requires a variety of
% unusual and advanced tools and techniques: for example, a relational
% interpreter for a subset of Scheme might use nominal unification to
% support variable binding and scope, Constraint Logic Programming over
% Finite Domains (CLP(FD)) to implement relational arithmetic, and
% tabling to improve termination behavior.  In addition, relational
% programmers must give up many standard but non-declarative programming
% constructs used in traditional logic programming, such as Prolog's
% infamous cut operator\footnote{Prolog's cut operator ({\tt !}) prunes
%   the program's search tree, which can greatly improve efficiency and
%   even avoid divergence for infinite search trees.  Unfortunately,
%   careless use of cut can prune nodes that contain answers, resulting
%   in an unsound program.}.  As a result of these difficulties,
% non-trivial Prolog programs are rarely written as relations.  Newer
% languages such as Curry and Mercury attempt to solve this problem by
% using only declative language features.  While these languages offer
% many interesting and useful declarative constructs, 

% potentially diverging when used with logic variables rather than
% ground terms.

% Thus, the promise of
% logic programming remains largely unfulfilled.  What is needed is a
% language for {\em relational programming}, as opposed to logic
% programming or even declarative programming.

Unfortunately, writing remarkably flexible relational programs is
remarkably difficult, and requires a variety of unusual and advanced
tools and techniques.  For example, a relational interpreter for a
subset of Scheme might use nominal unification to support variable
binding and scope, Constraint Logic Programming over Finite Domains
(CLP(FD)) to implement relational arithmetic, and tabling to improve
termination behavior.

This dissertation presents {\em miniKanren}, a family of languages
specifically designed for relational programming, and which supports a
variety of relational idioms and techniques.  We show how miniKanren
can be used to write interesting relational programs, including an
extremely flexible lean tableau theorem prover and a novel
constraint-free binary arithmetic system with strong termination
guarantees.  We also present interesting and practical techniques used
to implement miniKanren, including a nominal unifier that uses
triangular rather than idempotent substitutions and a novel
``walk''-based algorithm for variable lookup in triangular
substitutions.

Chapter~\ref{mkintrochapter} presents the core miniKanren language,
which we then extend with disequality constraints
(Chapter~\ref{diseqchapter}), nominal logic (Chapter~\ref{akchapter}),
tabling (Chapter~\ref{tablingchapter}), and expression-level
divergence avoidance using ferns (Chapter~\ref{fernschapter}).  We
provide implementations of all of these language extensions in
Chapters~\ref{mkimplchapter}, \ref{walkimpl}, \ref{diseqimplchapter},
\ref{akimplchapter}, \ref{tablingimplchapter}, and \ref{fernsimpl}.
Together, these chapters establish the first half of my thesis:
miniKanren supports a variety of relational idioms and techniques.

To illustrate the use of these techniques, we present two
non-trivial miniKanren applications.  The constraint-free relational
arithmetic system of Chapter~\ref{arithchapter} and the theorem prover
of Chapter~\ref{alphatapchapter}
% and the term reducer of Chapter~\ref{reducerchapter} 
establish the second half of my thesis:
it is feasible and useful to write interesting programs as relations
in miniKanren, using these idioms and techniques.

\section{Structure of this Dissertation}

With the exception of two early chapters
(Chapters~\ref{mkintrochapter} and~\ref{divergencechapter}), each
technical chapter in this dissertation is divided into one of three
categories: techniques, applications, or
implementations\footnote{Hence the title of this dissertation:
  \emph{Relational Programming in miniKanren: Techniques,
    Applications, and Implementations}.}.  \emph{Technique chapters}
describe language features and idioms for writing relations, such as
disequality constraints (Chapter~\ref{diseqchapter}) and nominal logic
(Chapter~\ref{akchapter}).  \emph{Application chapters} demonstrate
how to write interesting, non-trivial relations in miniKanren; these
applications demonstrate the use of many of the language forms and
idioms presented in the technique chapters.  \emph{Implementation
  chapters} show how to implement the language extensions presented in
the technique chapters.

At a higher level, the dissertation is divided into six parts, which
are organized by theme:

\begin{itemize}

\item Part~\ref{coremkpart} presents the core miniKanren language,
  which we will extend in the latter parts of the dissertation.
  Chapter~\ref{mkintrochapter} introduces the core language, along
  with a few simple examples, while Chapter~\ref{mkimplchapter}
  presents the implementation of the core language.  These two
  chapters are especially important, since they form the foundation
  for the advanced techniques and implementations that follow.  In
  Chapter~\ref{walkimpl} we optimize the \scheme|walk| algorithm
  presented in Chapter~\ref{mkimplchapter}, which is the heart of
  miniKanren's unifier.  Chapter~\ref{divergencechapter} attempts to
  categorize the many ways miniKanren programs can diverge, and
  describes techniques that can be used to avoid each type of
  divergence.  Avoiding divergence while maintaining declarativeness is
  what makes relational programming so fascinating, yet so
  challenging.  Chapter~\ref{arithchapter} presents a non-trivial
  application of core miniKanren: a constraint-free arithmetic system
  with strong termination guarantees.
  
\item Part~\ref{diseqpart} extends core miniKanren with disequality
  constraints, which allow us to express that two terms are different,
  and can never be unified.  Disequality constraints express a very
  limited form of negation, and can be seen as a very simple kind of
  constraint logic programming.  Chapter~\ref{diseqchapter} describes
  disequality constraints from the perspective of the user, while
  Chapter~\ref{diseqimplchapter} shows how we can use unification in a
  clever way to simply and efficiently implement the constraints.  We
  give special attention to \emph{constraint reification}---the
  process of displaying constraints in a human-friendly manner.

\item Part~\ref{nominallogicpart} extends core miniKanren with
  operators for expressing nominal logic; we call the resulting
  language \alphakanren.  Nominal logic allows us to easily express
  notions of scope and binding, which is useful when writing
  declarative interpreters, type inferencers, and many other relations
  that deal with variables.  Chapter~\ref{akchapter} introduces
  nominal logic, explains \alphakanren's new language constructs, and
  provides a few simple example programs.
  Chapter
  \ref{alphatapchapter} presents a non-trivial application of
  \alphakanren: a relational theorem
  prover.  In Chapter~\ref{akimplchapter} we present our
  implementation of \alphakanren, including two different
  implementations of nominal unification.

\item Part~\ref{tablingpart} adds tabling to our implementation of
  core miniKanren.  Tabling is a form of memoization: the answers
  produced by a tabled relation are ``remembered'' (that is, stored in
  a table), so that subsequent calls to the relation can avoid
  recomputing the answers.  Tabling allows our programs to run more
  efficiently in many cases; more importantly, many programs that
  would otherwise diverge terminate when using tabling.
  Chapter~\ref{tablingchapter} introduces the notion of tabling, and
  explains which programs benefit from tabling.
%   Chapter~\ref{reducerchapter} presents a non-trivial miniKanren
%   application, a relational term reducer, and shows how tabling
%   improves the termination behavior of the reducer.
  Chapter~\ref{tablingimplchapter} presents our streams-based
  implementation of tabling, which demonstrates the advantage of
  embedding miniKanren in a language with higher-order functions.

\item Part~\ref{fernspart} presents a bottom-avoiding data structure
  called a \emph{fern}, and shows how ferns can be used to avoid
  expression-level divergence.  Chapter~\ref{fernschapter} introduces
  the fern data structure and implements a simple, miniKanren-like
  language using ferns.  Chapter~\ref{fernsimpl} presents our
  embedding of ferns in Scheme.

\item Part~\ref{contextpart} provides context and conclusions for the
  work in this dissertation.  Chapter~\ref{relatedworkchapter}
  describes related work, while Chapter~\ref{futureworkchapter}
  proposes future research.  We offer our final conclusions in
  Chapter~\ref{conclusionchapter}.

\end{itemize}

The dissertation also includes four appendices.
Appendix~\ref{helpers} contains several generic helper functions that
could be part of any standard Scheme library.  Appendix~\ref{pmatch}
describes and defines \scheme|pmatch|, a simple pattern matching macro
for Scheme programs.  Appendix~\ref{matche} describes and defines
\scheme|matche| and \scheme|lambdae|, pattern matching macros for
writing concise miniKanren relations.  Appendix~\ref{nestable-engines}
contains our implementation of nestable engines, which are used in our
embedding of ferns.



% For example, the theme of
% Part~\ref{nominallogicpart} is nominal logic;
% Part~\ref{nominallogicpart} contains one technique chapter
% (Chapter~\ref{akchapter}), two application chapters
% (Chapters~\ref{typeinferencerchapter} and
% ~\ref{alphatapchapter}), and one implementation chapter
% (Chapter~\ref{akimplchapter}).


%This dissertation has the following structure:

%\noindent [bullet list roadmap with chunking of chapters, illustrating how the
%chapters fit together]

\section{Relational Programming}

Relational programming is a discipline of logic programming in which
every goal is written as a ``pure'' relation.  Each relation produces
meaningful answers, even when all of its arguments are unbound logic
variables.  For example, Chapter~\ref{arithchapter} presents
\scheme|pluso|, which performs addition over natural numbers.
\mbox{\scheme|(pluso bn1 bn2 bn3)|}\footnote{\scheme|bn1|, 
\scheme|bn2|, and \scheme|bn3| are shorthand for the little-endian binary 
lists representing the numbers 1, 2, and 3---see Chapter~\ref{arithchapter} for details.}
 succeeds, since $1+2=3$---that is,
the triple $(1,2,3)$ is in the ternary addition relation.  We can use
\scheme|pluso| to add two numbers: \mbox{\scheme|(pluso bn1 bn2 z)|}
associates the logic variable \scheme|z| with \scheme|3|.  We can also
subtract numbers using \scheme|pluso|: \mbox{\scheme|(pluso bn1 y bn3)|} 
associates \scheme|y| with 2, since $3-1=2$.  We can even
call \scheme|pluso| with only logic variables: \mbox{\scheme|(pluso x y z)|} 
produces an infinite number of answers in which the natural
numbers associated with \scheme|x|, \scheme|y|, and \scheme|z|
satisfy \scheme|x+y=z|.  For example, one such answer associates
\scheme|x| with 3, \scheme|y| with 4, and \scheme|z| with 7.

To write relational goals, programmers must avoid a variety of
powerful logic programming constructs, such as Prolog's
cut ({\tt !}), {\tt var/1}, and \mbox{{\tt copy\_term/2}} operators.  These
operators inhibit relational programming, since their proper use is
dependent upon the groundness or non-groundness of terms\footnote{A
  term is \emph{ground} if it does not contain unassociated logic
  variables.}.
%, and therefore upon the mode of the relation in which
%they appear.  
Programmers who wish to write relations must avoid these
constructs, and instead use language features compatible with the
relational paradigm.

A critical aspect of relational programming is the desire for
relations to terminate whenever possible.  Writing a goal without mode
restrictions is not very interesting if the goal diverges when passed
one or more fresh variables.  In particular, we desire the
\emph{finite failure} property for our goals---if a goal is asked to
produce an answer, yet no answer exists, that goal should fail in a
finite amount of time.  Although G\"{o}del and Turing showed that it
is impossible to guarantee termination for all goals we might wish to
write, the use of clever data encoding, nominal unification, tabling,
and the derivation of bounds on the maximum size of terms allows a
careful miniKanren programmer to write surprisingly sophisticated
programs that exhibit finite failure.

Our emphasis on both pure relations and finite failure leads to
different design choices than those of more established logic
programming languages such as
Prolog~\cite{ISO:1995:IIIe,ISO:2000:IIIf},
Mercury~\cite{Somogyi95mercury}, and
Curry~\cite{Hanus95curry:a,Hanus06}.  For example, unlike Prolog,
miniKanren uses a complete (interleaving) search strategy by default.
Unlike Mercury, miniKanren uses full unification, required to
implement goals that take only fresh logic variables as their
arguments\footnote{Mercury is statically typed, and requires
  programmers to specify ``mode
  annotations''~\cite{Apt94reasoningabout} indicating whether each
  argument to a goal is an ``input'' (that is, fully ground) or an
  ``output'' (that is, an unassociated logic variable).  Programmers
  also specify whether each goal can produce one, finitely many, or
  infinitely many answers.  Given all this information, the Mercury
  compiler can generate multiple specialized functions that perform
  the work of a single goal.  For example, a ternary goal that
  expresses addition (similar to the \scheme|pluso| function described
  above) might be compiled into separate functions that perform
  addition or subtraction; at runtime, the appropriate function will
  be called depending on which arguments are ground.  In fact,
  compiled Mercury programs do not use logic variables or unification,
  and are therefore extremely efficient.  Unfortunately, this lack of
  unification means it is not possible to write Mercury goals that
  take only ``output'' variables.}.  And our desire for termination
prevents us from adapting Curry's
residuation\footnote{\emph{Residuation}~\cite{hanus:jlp95} suspends
  certain operations on non-ground terms, until those terms become
  ground.  For example, we could use residuation to express addition
  using Scheme's built-in \scheme|+| procedure.  If we try to add
  \scheme|x| and 5, and \scheme|x| is an unassociated logic variable,
  we suspend the addition, and instead try running another goal.
  Hopefully this goal will associate \scheme|x| with a number; when
  that happens, we can perform the addition.  However, if \scheme|x|
  never becomes ground, we will be unable to perform the addition, and
  we will never produce an answer.}.

\section{miniKanren}

This dissertation presents miniKanren, a language designed for
relational programming, along with various language extensions that
add expressive power without sacrificing the ability to write
relations.

% The premise of this work is that:
% \begin{enumerate}
% \item Logic programmers should strive to write their programs as
%   relations, without mode restrictions,
% \end{enumerate}
% and
% \begin{enumerate}
% \item[2.] miniKanren's combination of relational language constructs,
%   along with its easily modifiable, purely functional implementation,
%   makes it uniquely suited for relational programming.
% \end{enumerate}

miniKanren is implemented as an embedding in Scheme, using only a
handful of special forms and functions.  The concise and purely
functional implementation of the core operators makes the language
easy to extend.  miniKanren programmers have access to all of Scheme,
including higher-order functions, first-class continuations, and
Scheme's unique and powerful hygienic macro system.  Having access to
Scheme's features makes it easy for implementers to extend miniKanren;
for example, from a single figure explaining XSB-style OLDT resolution
we were able to design and implement a tabling system for miniKanren
in under a week.

This thesis presents complete Scheme implementations of core
miniKanren and its extensions, including two versions of nominal
unification, a simple constraint system, a streams-based tabling
system, and a minimal implementation of a miniKanren-like language
using the bottom-avoiding fern data-structure.  Our implementation of
core miniKanren is purely functional, and is designed to be easily
modifiable, encouraging readers to experiment with and extend
miniKanren.

\section{Typographical Conventions}

% [META This section is likely to grow, and perhaps turn into a table.]

The code in this dissertation uses the following typographic conventions.  Lexical variables are in \scheme{italic}, forms are in {\bf boldface}, and quoted symbols are in {\sf sans serif}.
Quotes, quasiquotes, and unquotes are suppressed,
and quoted or quasiquoted lists appear with bold parentheses---for example \scheme{`()} and \mbox{\scheme|`(x . ,x)|} are entered as
\qqmagicoff 
\texttt{'()} and \mbox{\texttt{`(x . \!,x)}}, respectively.
\qqmagicon 
By our convention, names of relations end with a superscript $o$---for example \scheme{substo}, which is entered as {\tt substo}.  Relational operators do not follow this convention:  \scheme{==} (entered as {\tt ==}), \scheme{conde} (entered as {\tt conde}), and \scheme{exist}.  Chapter~\ref{diseqchapter} introduces the relational operator \scheme|=/=| (entered as {\tt =/=}), while Chapter~\ref{akchapter} introduces \scheme{fresh}, \scheme|hash| (entered as {\tt hash}), and the term constructor \scheme|tie| (entered as {\tt tie}).  Similarly, \mbox{\scheme|(run5 (q) body)|} and \mbox{\scheme|(run* (q) body)|} are entered as \mbox{{\tt (run 5 (q) body)}} and \mbox{{\tt (run* (q) body)}}, respectively.

\scheme|lambda| is entered as {\tt lambda}.
\scheme|lambdae| from Appendix~\ref{matche} is entered as {\tt lambdae}.
The arithmetic relations \scheme|<=lo| and \scheme|<=o| from
Chapter~\ref{arithchapter} are entered as {\tt <=lo} and {\tt <=o},
respectively. \scheme|occurs-check| from Chapter~\ref{mkimplchapter} 
is entered as {\tt occurs-check}.

% [TODO point out how to enter \scheme|>1o|, \scheme|<o|, 
% =lo, <lo, <=lo, <=o,
% and other
% arithmetic names.  Say something about R5RS compatibility.
% lambda, lambdae, matche,
% ]


%   \section{Where to Find the Source Code}

%   [various implementations, arithmetic system, type inferencer,
%   alphaleanTAP, etc.]

%   \section{Logic Programming and Relations}

%     \subsection{Logic Programming}

%     \noindent [Dan says I should do something like this: for
%     convenience, in the sequel logic programming is just referred to
%     as ``programming'']

%     \subsection{Relations}

%     \subsection{The Problem}

%     \noindent [perhaps give a simple example or two showing difficulty
%     of writing relations: perhaps rembero, demonstrating the need for
%     constraints [or something based on lambda calculus, demonstrating
%     need for nominal logic], and an example showing divergence,
%     demonstrating the need for bounds or constraints]
