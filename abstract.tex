\chapter*{Abstract}

%WILLIAM E. BYRD  \underline{Relational Programming in miniKanren: Techniques, Applications, and Implementations}

\noindent The promise of logic programming is that programs can be
written {\em relationally}, without distinguishing between input and
output arguments.  Relational programs are remarkably flexible---for
example, a relational type-inferencer also performs type checking and
type inhabitation, while a relational theorem prover generates
theorems as well as proofs and can even be used as a simple proof
assistant.

Unfortunately, writing relational programs is difficult, and requires
many interesting and unusual tools and techniques.  For example, a
relational interpreter for a subset of Scheme might use nominal
unification to support variable binding and scope, Constraint Logic
Programming over Finite Domains (CLP(FD)) to implement relational
arithmetic, and tabling to improve termination behavior.

In this dissertation I present {\em miniKanren}, a family of languages
specifically designed for relational programming, and which supports a
variety of relational idioms and techniques.  I show how miniKanren
can be used to write interesting relational programs, including an
extremely flexible lean tableau theorem prover and a novel
constraint-free binary arithmetic system with strong termination
guarantees.  I also present interesting and practical techniques used
to implement miniKanren, including a nominal unifier that uses
triangular rather than idempotent substitutions and a novel
``walk''-based algorithm for variable lookup in triangular
substitutions.

The result of this research is a family of languages that supports a
variety of relational idioms and techniques, making it feasible and
useful to write interesting programs as relations.

% \noindent [check abstracts of my papers]

% \noindent [describe disparity between promise and reality of logic programming]

% \noindent [brief explanation of and motivation for relational programming]

% \noindent [describe the work this dissertation presents]
