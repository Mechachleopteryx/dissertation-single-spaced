\chapter{Implementation VI: Ferns}\label{fernsimpl}

In this chapter we present a complete, portable, \RsixRSsp
compliant~\cite{r6rs} implementation of ferns\footnote{The ferns
  library is available at
  \url{http://www.cs.indiana.edu/~webyrd/ferns.html}}. We begin with a
description of \emph{engines}~\cite{CHayne87}, which we use to handle
suspended, preemptible computations. We then describe and implement
frons pairs, the building blocks of ferns. Next we present
\scheme|fcar| and \scheme|fcdr|, which work on both frons pairs and
cons pairs. Taking the \scheme|fcar| of a frons pair involves choosing
one of the possible values in the fern and promoting the chosen value.
Taking the \scheme|fcdr| of a frons pair ensures the first value in
the pair is determined and returns the rest of the fern.  Taking the
\scheme|fcar| (\scheme|fcdr|) of a cons pair is the same as taking its
\scheme|car| (\scheme|cdr|).

\section{Engines}\label{fernsenginessection}

An engine is a procedure that computes a delayed value in steps\footnote{See Appendix~\ref{nestable-engines} for our implementation of nestable engines.}.
%\cite{CHayne87}. 
To demonstrate the use of engines, consider the procedure

\schemedisplayspace
\schemeinput{fernscode/wait}

\noindent To create an engine \scheme|e| to delay a call to
\scheme|(wait 20)|, we write

\wspace
\scheme|(define e (engine (wait 20)))|
\wspace

\noindent 
To partially compute \mbox{\scheme|(wait 20)|}, we call \scheme|e|
with a number of \scheme|ticks|: \mbox{\scheme|(e 5)|}, which returns
either a pair with false in the car and a new advanced engine (one
advanced 5 ticks) in the cdr or a pair with unused ticks (always true)
in the car and the value of the computation (here \schemeresult|done|)
in the cdr.  In our embedding of engines, a tick is spent on each
call to a procedure defined with \scheme|timed-lambda|. Consider

\schemedisplayspace

\belowcodeskip 0pt

\schemeinput{fernscode/coaxexample-2}
\nspace
\begin{schemeresponse}
$\Rightarrow$ `(#f #f #f #f 4 done).
\end{schemeresponse}

\wspace

\noindent In this example, \mbox{\scheme|(wait 20)|} calls
\scheme|wait| a total of $21$ times (including the initial call), so
on the fifth engine invocation, it terminates with 4 unused ticks.

The delayed computation in an engine may involve creating and calling
more engines. When a \emph{nested engine}
\cite{hieb94subcontinuations} consumes a tick, every
frons-enclosing engine also consumes a tick. To see this, we define
\scheme|choose-bottom| using engines:

\schemedisplayspace
\schemeinput{fernscode/coaxchoose}

\wspace

\noindent Nested calls to \scheme|choose-bottom|, for example 
\mbox{\scheme|(choose-bottom v1 (choose-bottom v2 v3))|}, rely on nestable engines.  
This implementation of \scheme|choose-bottom| is fair because our embedding
of nested engines is fair: every tick given to the second engine in
the outer call to \scheme|choose-aux-bottom| is passed on to exactly one of
the engines, alternating between the engines for \scheme|v2| and
\scheme|v3|, in the inner call to \scheme|choose-aux-bottom|.

\section{The Ferns Data Type}\label{fernsdatatype}

We represent a frons pair by a cons pair that contains at least one
tagged engine (\scheme|te|). Engines are tagged with either \scheme|L|
when locked (being advanced by another computation) or \scheme|U| when
unlocked (runnable).  We distinguish between locked and unlocked
engines because the \scheme|fcar| of a fern may be requested more than
once simultaneously. Thus, to manage effects, the locks prevent the same
engine from being advanced in more than one computation\footnote{A lock creates a localized critical region
that corresponds to the intended use of \scheme|sting-unless|~\cite{FriedmanWise78}.}.

We define simple predicates \scheme|engine-tag-L-car|,
\scheme|engine-tag-U-car|, \scheme|engine-tag-L-cdr|, and
\scheme|engine-tag-U-cdr| for testing whether one side of a frons pair
contains a locked or unlocked engine. 

\schemedisplayspace
\schemeinput{fernscode/lockedorunhuh}

\wspace

The procedure \scheme|coax-d| (\scheme|coax-a|) takes a frons pair
with an unlocked tagged engine in the cdr (car) and locks and advances
the tagged engine by \scheme|nsteps| ticks. If
\scheme|coax|ing~\cite{Friedman79b} the engine does not finish, the
tagged engine is unlocked and updated with the advanced engine. If
\scheme|coax|ing the engine finishes with value \scheme|v|, then
\scheme|v| becomes the frons pair's cdr (car). In addition, the tagged
engine will be updated with an unlocked dummy engine that returns
\scheme|v|. We do this because the cdrs of multiple frons pairs may
share a single engine, as will be explained at the end of this
section.  Although the cars of frons pairs never share engines, we do
the same for the cars.

\schemedisplayspace
\schemeinput{fernscode/coaxskcdr}
\schemeinput{fernscode/replacebang}

\wspace

\noindent
Now we present the implementation of the fern operators.

%\subsection{\protect\scheme|frons|}\label{frons}

\section{\fronssymbol, \fcarsymbol, and \fcdrsymbol}\label{frons}\label{car}\label{cdr}\label{fronsconstructorsection}

\scheme|frons| constructs a frons pair by placing unlocked engines of
its unevaluated operands in a cons pair.

\schemedisplayspace
\schemeinput{fernscode/frons}

%\section{\protect\scheme|fcar|}\label{car}

\wspace

\noindent
When the \scheme|fcar| (definition below), which is defined only for
ferns, is requested, parallel evaluation of the possible values is
accomplished by a round-robin race of the engines in the
fern.  During its turn, each engine is advanced a fixed,
arbitrary number of ticks until a value is produced.  The race is
accomplished by two mutually recursive functions: \scheme|race-car|,
which works on the possible values of the fern, and \scheme|race-cdr|,
which moves onto the next frons pair by either following the cdr of
the current frons pair or starting again at the beginning.

\schemedisplayspace
\schemeinput{fernscode/car}

\wspace

\noindent
\scheme|race-car| dispatches on the current pair or value \scheme|q|.
When the car of \scheme|q| is a locked engine, \scheme|race-car| waits
for it to become unlocked by waiting \scheme|nsteps| ticks and then
calling \scheme|race-cdr|. The call to \scheme|wait| is required to
allow \scheme|race-car| to be preempted at this point, so the owner of
the lock does not starve. When the car is an unlocked engine,
\scheme|race-car| advances the unlocked engine \scheme|nsteps| ticks,
then continues the race by calling \scheme|race-cdr|.  When \scheme|q|
is not a pair, \scheme|race-car| simply starts the race again from the
beginning.  This happens when racing over a finite fern and emerges
from the \scheme|else| clause of \scheme|race-cdr|. When the car
contains a value, we call \scheme|promote| which ensures a value is promoted to
the car of \scheme|p|, then return that value.

One subtlety of the definition of \scheme|race-car| is that after
coaxing an engine it does not check if the coaxing has led to
completion.  If it has, the value will be picked up the next time the
race comes around, if necessary.  Calling \scheme|promote!|
immediately would be incorrect because an engine may be preempted
while advancing, at which point promotion from \scheme|p| may be
performed by another computation with a different value for the car of
\scheme|p|.  

%\enlargethispage{40pt}

\scheme|race-cdr| also dispatches on \scheme|q|, this time examining its
cdr. When the cdr of \scheme|q| is a locked engine,
\scheme|race-cdr|, being unable to proceed further down the fern,
restarts the race by calling \scheme|race-car| on \scheme|p|. When the
cdr of \scheme|q| contains an unlocked engine, \scheme|race-cdr|
advances the engine \scheme|nsteps| ticks as in \scheme|race-car|, and
then restarts the race. If that engine finishes with a new frons pair,
the new pair will then be competing in the race and will be examined
next time around. When the cdr of \scheme|q| is a value, usually a fern,
\scheme|race-cdr| continues the race by passing it to
\scheme|race-car|; if a non-pair value is at the end of a fern, it will
be picked up by the third clause in \scheme|race-car|.

\scheme|fcar| avoids starvation by running each engine in a subfern for the
same number of ticks. During a race, a subfern of the fern in question
is in a fair state: for some (potentially empty) prefix of the subfern there are no engines in the cdrs, so
each potential value in a fair subfern is considered equally. When this fair
subfern is not the entire fern, the race devotes the same number of
ticks to lengthening the fair subfern as it does to each element of that
subfern. Since cdr engines often evaluate to pairs quickly, the
entire fern usually becomes fair in a number of races equal to the
length of the fern. When cdr engines do not finish quickly, however, the
process of making the entire fern fair can take much longer, especially
for long ferns. The cost of finding the value of an element occurring near
the end of such a fern can be much greater than the cost for an element
near the beginning.

%\enlargethispage{30pt}

Starting from \scheme|p|, \scheme|promote!| (definition below) finds
the first pair \scheme|r| whose car contains a convergent value, and
propagates that value back to \scheme|p|.  Each frons pair in this
chain (excluding \scheme|r|) is transformed into a cons pair whose car
is the convergent value. These new frons pairs are connected as a fern
and the last one shares \scheme|r|'s cdr.  When \scheme|promote!| is
called from \scheme|race-car|, we know that \scheme|q|'s car is a
value but we don't know for certain that there is no other pair, say
\scheme|r|, in the chain from \scheme|p| to \scheme|q|.  Thus, we must
search from \scheme|p| without preemption to find the closest value to
\scheme|p|.  This situation can arise when there are two calls to
\scheme|fcar| on the same fern competing:

\schemedisplayspace
\schemeinput{fernscode/whypromotion.ss}

\wspace

\noindent
If a call to \scheme|race-car| finds the value \schemeresult|720| and
tries to promote it, but the value \schemeresult|120| has already been
promoted, we don't want to change the car of \scheme|alpha|. Instead,
the call to \scheme|promote| when \schemeresult|720| is found will
find the \schemeresult|120| first and stop.

% Each finds a different value; since each call is nested, the two calls are
% effectively running in parallel. Then the one with the value closer to
% \scheme|p| will promote first even if \scheme|promote!| is non-preemptible.

\schemedisplayspace
\schemeinput{fernscode/promote}

%\section{\protect\scheme|fcdr|}\label{cdr}

\wspace

The \scheme|fcdr| of a fern (definition below) cannot be determined until the
fern's \scheme|fcar| has been determined.  Once the car has been determined,
there is no longer parallel competition between potential cdrs.  Thus,
we can use \scheme|cdrdollar|, which takes the cdr of a stream. Then,
since \scheme|p|'s car has been determined, \scheme|p| has therefore
become a cons pair, so \scheme|fcdr| returns the value in \scheme|p|'s
cdr.  (\scheme|cardollar|'s definition follows by replacing all
\scheme|d|s by \scheme|a|s.  \scheme|consdollar| is the same as
\scheme|frons|, and the definitions of the other stream operators
follow the definitions with operators $f_{\perp}$ replaced by $f_s$.)

\schemedisplayspace
\schemeinput{fernscode/cdr}
\schemeinput{fernscode/cdrdollar}

\wspace

If the engine being advanced by \scheme|cdrdollar| completes,
\scheme|cdrdollar| indicates that \scheme|coax-d| should replace the
tagged engine in \scheme|p| by the computed value.  However,
\scheme|race-cdr| and \scheme|fcdr| are required not only to update
the frons pair with the calculated value, but also to update
the tagged engine because there might be a fern other than \scheme|p|
sharing this engine. Consider the following expression where we assume
\scheme|list| evaluates its arguments from left to right.

\schemedisplayspace
\begin{schemedisplay}
(let ((beta (frons 1 (ints-bottom 2))))
  (let ((alpha (frons bottom beta)))
    (list (fcar alpha) (fcadr beta) (fcadr alpha))))
\end{schemedisplay}
\nspace
\begin{schemeresponse}
~> `(1 2 2)
\end{schemeresponse}

\wspace

\noindent Figure~\ref{fig:sk} shows the data structures involved in
evaluating the expression. 

\schemedisplayspace

\vspace{-5pt}

\begin{figure}[h]
\begin{schemeregion}
\begin{picture}(40,50)(-40,-30)
\Draw\PenSize(1pt)
\namefig(a,15,35)
\namebox(\scheme|alpha|)
\renewcommand{\boxtext}{\scheme|bottom|}
\efronsbox
\renewcommand{\boxtext}{\scheme|beta|}
\rengine
\EndDraw
\end{picture}
\end{schemeregion}

\begin{schemeregion}
\begin{picture}(80,0)(-120,-40)   %%% 
\Draw\PenSize(1pt)
\namefig(b,35,35)
\namebox(\scheme|alpha|)
\renewcommand{\egap}{\qbxwd}
\renewcommand{\boxtext}{\scheme|bottom|}
\efronsbox
\rightsolid
\namebox(\scheme|beta|)
\renewcommand{\boxtext}{\schemeresult|one|}
\fronsbox
\renewcommand{\boxtext}{$\iota_2$}
\rengine
\EndDraw
\end{picture}
\end{schemeregion}

\begin{schemeregion}
\begin{picture}(80,0)(-240,-50)
\Draw\PenSize(1pt)
\namefig(c,35,35)
\namebox(\scheme|alpha|)
\renewcommand{\boxtext}{\schemeresult|one|}
\consbox\downsolid
\renewcommand{\boxtext}{\scheme|bottom|}
\renewcommand{\egap}{6}
\efronsbox\rightunright\blankup\Move(\hbxwd,-\qbxwd)
\namebox(\scheme|beta|)
\renewcommand{\boxtext}{\schemeresult|one|}
\fronsbox
\renewcommand{\boxtext}{$\iota_2$}
\renewcommand{\egap}{8}
\rengine
\EndDraw
\end{picture}
\end{schemeregion}

\begin{schemeregion}
\begin{picture}(80,80)(-40,-50)
\Draw\PenSize(1pt)
\namefig(d,40,60)
\namebox(\scheme|alpha|)
\renewcommand{\boxtext}{\schemeresult|one|}
\consbox\downsolid
\renewcommand{\boxtext}{\scheme|bottom|}
\renewcommand{\egap}{3}
\efronsbox
\renewcommand{\boxtext}{\scheme|gamma|}
\rengine
\blankup\Move(\hbxwd,-\qbxwd)
\namebox(\scheme|beta|)
\renewcommand{\boxtext}{\schemeresult|one|}
\consbox
\rightsolid
\namebox(\scheme|gamma|)
\renewcommand{\boxtext}{\schemeresult|two|}
\fronsbox
\renewcommand{\egap}{6}
\renewcommand{\boxtext}{$\iota_3$}
\rengine
\EndDraw
\end{picture}
\end{schemeregion}

\begin{schemeregion}
\begin{picture}(80,0)(-180,-60)
\Draw\PenSize(1pt)
\namefig(e,40,60)
\namebox(\scheme|alpha|)
\renewcommand{\boxtext}{\schemeresult|one|}
\consbox\downsolid
\renewcommand{\boxtext}{\schemeresult|two|}
\consbox\downsolid
\renewcommand{\boxtext}{\scheme|bottom|}
\renewcommand{\egap}{6}
\efronsbox \rightupdoublelonger
\blankup \blankleft
\Move(-\qbxwd,0)
\namebox(\scheme|beta|)
\renewcommand{\boxtext}{\schemeresult|one|}
\consbox \rightsolid
\namebox(\scheme|gamma|)
\renewcommand{\boxtext}{\schemeresult|two|}
\fronsbox
\renewcommand{\boxtext}{$\iota_3$}
\lengine
\EndDraw
\end{picture}
\end{schemeregion}

\vspace{25pt}
\caption{Fern $\alpha$ after construction (a); after $\beta$ in the cdr of $\alpha$ has been evaluated (b); after $1$ from the car of $\beta$ has been promoted to the car of $\alpha$, resulting in a shared tagged engine (c); after the shared engine is run, while evaluating ({\it{cadr$_\perp$} $\beta$}), to produce a fern $\gamma$ (d); after $2$ from the car of $\gamma$ has been promoted to the cadr of \mbox{$\alpha$ (e)}.\label{fig:sk}}
\end{figure}

\noindent Figure~\ref{fig:sk}a shows \scheme|alpha| immediately after
it has been constructed, with engines delaying evaluation of
\scheme|bottom| and \scheme|beta|. In evaluating \mbox{\scheme|(fcar alpha)|}, 
the engine for \scheme|beta| finishes, resulting in
Figure~\ref{fig:sk}b. \scheme|beta| can now participate in the race
for \scheme|(fcar alpha)|. Suppose the value \scheme|1| found in the
car of \scheme|beta| is chosen and promoted. The result is Figure~\ref{fig:sk}c, in which the engine
delaying \mbox{\scheme|(ints-bottom 2)|} is shared by both \scheme|beta| and
the cdr of \scheme|alpha|. \scheme|(fcadr beta)| forces calculation of
\mbox{\scheme|(ints-bottom 2)|}, which results in a fern, \scheme|gamma|, whose
first value (in this example) is \scheme|2|. Figure~\ref{fig:sk}d now
shows why \scheme|coax-d| updates the current pair (\scheme|beta|)
and creates a new dummy engine with the calculated value (\scheme|gamma|):
the cddr of \scheme|alpha| needs the new engine to avoid recalculation
of \mbox{\scheme|(ints-bottom 2)|}. In Figure~\ref{fig:sk}e when \mbox{\scheme|(fcadr alpha)|}
is evaluated, the value \scheme|2|, calculated already by
\mbox{\scheme|(fcadr beta)|}, is promoted and the engine delaying
\mbox{\scheme|(ints-bottom 3)|} is shared by both \scheme|alpha| and \scheme|beta|.
